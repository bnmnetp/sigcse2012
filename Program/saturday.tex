
\addcontentsline{toc}{subsection}{Saturday}
\cfoot{\colorbox[gray]{0.45}{\color{white}\textsf{Saturday 08:30 - 09:45}}}
\noindent
\framebox[5in][c]{{\Large\sffamily\textbf{Saturday,  8:30 to 9:45}}}
\begin{longtable}[l]{@{}l@{}l@{}r}
    \parbox[t]{1in}{\sffamily\large\textbf{PANEL}} & 
    \parbox[t]{3in}{\sffamily\raggedright\large\textbf{Nifty Assignments}} & 
    \parbox[t]{1in}{\sffamily\raggedleft\large\textbf{301AB}} \\
% row 2    
    Chair: & 
    Nick Parlante, \textit{Stanford University}  \\[0.5em]
% row 3
    Participants: & 
    \multicolumn{2}{@{}l}{\parbox{3.75in}{Julie Zelenski, \textit{Stanford University}; Daniel Zingaro, \textit{University of Toronto}; Kevin Wayne, \textit{Princeton University}; Joshua Guerin, \textit{University of Kentucky}; Stephen Davies, \textit{University of Mary Washington}; Dave O'Hallaron, \textit{Carnegie Mellon University} }} \\[2em]
% row 4
    \multicolumn{3}{@{}p{5in}}{\small I can worry about the strategy of my syllabus, and I can fret over my lectures. Nonetheless, I am always struck that what my students really learn and enjoy in the course depends very much on the assignments. Great assignments are hard to dream up and time-consuming to develop. With that in mind, the Nifty Assignments session is all about promoting and sharing the ideas and concrete materials of successful assignments.}
\end{longtable}
\begin{longtable}[l]{@{}l@{}l@{}r}
    \parbox[t]{1in}{\sffamily\large\textbf{PANEL}} & 
    \parbox[t]{3in}{\sffamily\raggedright\large\textbf{Update on the CS Principles Project}} & 
    \parbox[t]{1in}{\sffamily\raggedleft\large\textbf{305B}} \\
% row 2    
    Chair: & 
    Amy Briggs, \textit{Middlebury College}  \\[0.5em]
% row 3
    Participants: & 
    \multicolumn{2}{@{}l}{\parbox{3.75in}{Owen Astrachan, \textit{Duke University}; Jan Cuny, \textit{National Science Foundation}; Lien Diaz, \textit{College Board}; Chris Stephenson, \textit{Computer Science Teachers Association} }} \\[2em]
% row 4
    \multicolumn{3}{@{}p{5in}}{\small The CS Principles Project is a collaborative effort to develop a new introductory course in computer science, accessible to all students. Computer Science educators at all levels have worked together on the development of the new curriculum under the direction of the College Board with support from the National Science Foundation. This special session provides an opportunity for the CS Principles project leaders to report on recent updates and new directions, and to engage in discussion on all aspects of the project with SIGCSE participants.}
\end{longtable}
\begin{longtable}[l]{@{}l@{}l@{}r}
    \parbox[t]{1in}{\sffamily\large\textbf{PANEL}} & 
    \parbox[t]{3in}{\sffamily\raggedright\large\textbf{Implementing Evidence-Based Practices makes a Difference in Female Undergraduate Enrollments}} & 
    \parbox[t]{1in}{\sffamily\raggedleft\large\textbf{306C}} \\
% row 2    
    Chair: & 
    Wendy DuBow \textit{University of Colorado}  \\[0.5em]
% row 3
    Participants: & 
    \multicolumn{2}{@{}l}{\parbox{3.75in}{Wendy DuBow, \textit{University of Colorado}; Elizabeth Litzler, \textit{University of Washington}; Maureen Biggers, \textit{Indiana University}; Mike Erlinger, \textit{Harvey Mudd College} }} \\[2em]
% row 4
    \multicolumn{3}{@{}p{5in}}{\small While many departments are aware of promising and best practices for recruiting and retaining female students in undergraduate computing majors, there seems to be a drive to try novel approaches instead of evidence-based approaches. Developing a diverse student body requires active recruitment, inclusive pedagogy, meaningful curriculum, evaluation of progress, as well as student, faculty and institutional support. Given the intrinsic challenges of enacting change, departments could make it easier on themselves - and likely achieve better results - if they intentionally and systematically used practices that have been shown to be effective. This panel will present the rationale for implementing evidence-based practices and share the successes some departments have achieved by doing so.}
\end{longtable}
\vspace{0.5em}
\noindent\rule{5in}{0.02cm}
\vspace{0.5em}
\cfoot{\colorbox[gray]{0.45}{\color{white}\textsf{Saturday 08:30 - 10:10}}}
\noindent
\framebox[5in][c]{{\Large\sffamily\textbf{Saturday,  8:30 to 10:10}}}
\newpage
\begin{longtable}{@{}p{0.75in}@{}p{3.25in}@{}r}
   {\sffamily\large\textbf{PAPERS}} &
   {\raggedright\sffamily\large\textbf{High School Collaborations}} & 
   {\sffamily\large\textbf{302A }} \\
%row 2
   Chair:  & 
   {\raggedright Tim Bell \textit{University of Canterbury}} & \\ \\
{\sffamily \large 8:30}& 
\multicolumn{2}{@{}p{3.75in}}{\sffamily\raggedright\textbf{Life Two Years After a Game Programming Course: Longitudinal Viewpoints on K-12 Outreach}} \\
& \multicolumn{2}{@{}p{3.75in}}{\raggedright Antti-Jussi Lakanen, Ville Isomöttönen and Vesa Lappalainen, \textit{Department of Mathematical Information Technology, University of Jyvaskyla}} \\ \\
\multicolumn{3}{@{}p{5in}}{\small In our faculty we have run week-long K-12 game programming courses now for three summers. In this paper we investigate what programming-related activities students do after they take a course, and what factors in the students' background relate to post-course programming. We also investigate a possible change in the students' interest towards higher education science studies. We find that most students continue programming after the course and that their interest towards science studies keeps increasing. In student background we observed some indicative trends, but did not find reliable explaining factors related to post-course programming or increased interest towards science studies.} \\ \\
{\sffamily \large 8:55}& 
\multicolumn{2}{@{}p{3.75in}}{\sffamily\raggedright\textbf{Reflections on Outreach Programs in CS Classes: Learning Objectives for “Unplugged” Activities}} \\
& \multicolumn{2}{@{}p{3.75in}}{\raggedright Renate Thies, \textit{Cusanus-Gymnasium Erkelenz and Technische Universität Dortmund}; Jan Vahrenhold, \textit{Technische Universität Dortmund}} \\ \\
\multicolumn{3}{@{}p{5in}}{\small To provide a unified view of any scientific field, outreach programs need to realistically portray the subject in question. Consequently, topics and methods actually taught in Computer Science courses should to be touched upon in Computer Science outreach programs or, conversely, elements from successful Computer Science outreach programs should be used to enrich established courses in Computer Science.

We follow up on the latter aspect and extract and classify learning objectives from the activities of the well-received Computer Science Unplugged program. Based upon this classification, we comment on where and to which extent these activities can be used to enrich teaching Computer Science in secondary education.} \\ \\
{\sffamily \large 9:20}& 
\multicolumn{2}{@{}p{3.75in}}{\sffamily\raggedright\textbf{Weaving a Tapestry Satellite Workshop to Support HS CS Teachers in Attracting and Engaging Students}} \\
& \multicolumn{2}{@{}p{3.75in}}{\raggedright Ambareen Siraj, Martha Kosa and Summer Olmstead, \textit{Tennessee Tech University}} \\ \\
\multicolumn{3}{@{}p{5in}}{\small In this paper, we describe the Tennessee Tech University (TTU) Tapestry Workshop for high school (HS) teachers. The Tapestry Workshop initiative, a collaborative partnership between TTU, the University of Virginia (UVA) and HS teachers to share strategies, practices, and innovative ideas for teaching Computer Science (CS) effectively. This three-day professional development workshop utilized informational, technical, networking, activity-based, and discussion-oriented sessions geared towards attracting and engaging CS students. The workshop was a worthwhile professional development activity for both the organizers and attendees and contributed towards initiation of HS CS program locally.} \\ \\
{\sffamily \large 9:45}& 
\multicolumn{2}{@{}p{3.75in}}{\sffamily\raggedright\textbf{Who AM I? Understanding High School Computer Science Teachers’ Professional Identity}} \\
& \multicolumn{2}{@{}p{3.75in}}{\raggedright Lijun Ni and Mark Guzdial, \textit{Georgia Institute of Technology}} \\ \\
\multicolumn{3}{@{}p{5in}}{\small We need committed, quality CS teachers for quality secondary computing education. Teacher education literature suggests that teachers’ sense of commitment and (other aspects of) teaching profession is tightly linked with their teacher identity. However, the current educational system does not provide typical contexts for teachers to build a sense of identity as CS teachers. This study is intended to gain an initial understanding of CS teachers’ perceptions about their professional identity and potential factors that might contribute to these perceptions. Our findings indicate that current HS teachers teaching CS courses do not necessarily identify themselves as CS teachers. They have different perceptions related to CS teaching. Four kinds of factors can contribute to these perceptions.} \\ \\
\end{longtable}


\newpage
\begin{longtable}{@{}p{0.75in}@{}p{3.25in}@{}r}
   {\sffamily\large\textbf{PAPERS}} &
   {\raggedright\sffamily\large\textbf{Parallelism and Concurrency}} & 
   {\sffamily\large\textbf{302B }} \\
%row 2
   Chair:  & 
   {\raggedright Jodi Tims \textit{Baldwin-Wallace College}} & \\ \\
{\sffamily \large 8:30}& 
\multicolumn{2}{@{}p{3.75in}}{\sffamily\raggedright\textbf{Introducing Parallelism and Concurrency in the Data Structures Course}} \\
& \multicolumn{2}{@{}p{3.75in}}{\raggedright Ruth E. Anderson and Dan Grossman, \textit{University of Washington - Seattle}} \\ \\
\multicolumn{3}{@{}p{5in}}{\small We report on our experience integrating a 3-week introduction to multithreading in a required data structures course for 2nd-year computer science majors. We emphasize a distinction between parallelism and concurrency that teaches students to use extra processors effectively and enforce mutual exclusion correctly. The material fits naturally in the data structures course by having the same mix of algorithms, programming, and asymptotic analysis as the rest of the course. Our department has used this unit for 1.5 years and we report feedback from students, multiple instructors for the course, and students in a later course that uses threads. We developed a full set of course materials that have been adapted for use by instructors in various courses at five other institutions so far.} \\ \\
{\sffamily \large 8:55}& 
\multicolumn{2}{@{}p{3.75in}}{\sffamily\raggedright\textbf{Exploring Concurrency Using The Parallel Analysis Tool}} \\
& \multicolumn{2}{@{}p{3.75in}}{\raggedright Brian Rague, \textit{Weber State University}} \\ \\
\multicolumn{3}{@{}p{5in}}{\small One area of investigation that has become increasingly important across all levels of CS instruction is parallel computing.  This paper describes the initial version of the Parallel Analysis Tool (PAT), a pedagogical tool designed to assist undergraduate students in visualizing concurrency and effectively connecting parallel processing to coding strategies. The PAT is a complete Java development environment, with an emphasis on (1) helping students to identify appropriate code locations where parallelization can be applied and (2) allowing students to subsequently examine the practical performance tradeoffs of these parallelization decisions in a laboratory setting. The Parallel Quotient supports the analysis of the relative benefits of employing various parallel programming strategies.} \\ \\
{\sffamily \large 9:20}& 
\multicolumn{2}{@{}p{3.75in}}{\sffamily\raggedright\textbf{Virtual Clusters for Parallel and Distributed Education}} \\
& \multicolumn{2}{@{}p{3.75in}}{\raggedright Elizabeth Shoop, Eric Biggers, Malcom Kane, Devry Lin and Maura Warner, \textit{Macalester College}; Richard Brown, \textit{St. Olaf College}} \\ \\
\multicolumn{3}{@{}p{5in}}{\small The reality of multicore machines as a standard and the prevalence of distributed cloud computing has signaled a need for parallel and distributed computing to become integrated into the computer science curriculum.  At the same time, operating system virtualization has become a common technique with open standard tools available to any practitioners.  Virtual machines (VMs) installed on available computer lab resources can be used to simulate high-performance cluster computing environments.  This paper describes two such virtual clusters in use at small colleges, reports on their effectiveness, and provides information about how to obtain the VMs for use in an educational lab setting.} \\ \\
{\sffamily \large 9:45}& 
\multicolumn{2}{@{}p{3.75in}}{\sffamily\raggedright\textbf{Cross Teaching Parallelism and Ray Tracing: A Project-based Approach to Teaching Applied Parallel Computing}} \\
& \multicolumn{2}{@{}p{3.75in}}{\raggedright Chris Lupo and Zoe Wood, \textit{Cal Poly State University}; Christine Victorino, \textit{University of California, Santa Barbara}} \\ \\
\multicolumn{3}{@{}p{5in}}{\small This paper describes the integration of two undergraduate computer science courses to enhance student learning in parallel computing concepts. In this cross teaching experience we structured the integration of the two courses such that students studying parallel computing worked with students studying advanced rendering for approximately 30\% of the quarter long courses. Working in teams, both groups of students saw the application of parallelization to an existing software project early in the curriculum of both courses. Motivating projects and performance gains are discussed, as well as student survey data on the effectiveness of the learning outcomes. Performance and survey data indicate a positive gain from the cross teaching experience.} \\ \\
\end{longtable}


\newpage
\begin{longtable}{@{}p{0.75in}@{}p{3.25in}@{}r}
   {\sffamily\large\textbf{PAPERS}} &
   {\raggedright\sffamily\large\textbf{Mobile Computing}} & 
   {\sffamily\large\textbf{306A }} \\
%row 2
   Chair:  & 
   {\raggedright Cyndi Rader \textit{Colorado School of Mines}} & \\ \\
{\sffamily \large 8:30}& 
\multicolumn{2}{@{}p{3.75in}}{\sffamily\raggedright\textbf{Cabana: A Cross-platform Mobile Development System}} \\
& \multicolumn{2}{@{}p{3.75in}}{\raggedright Paul E. Dickson, \textit{Hampshire College}} \\ \\
\multicolumn{3}{@{}p{5in}}{\small Mobile application development is a hot topic in computer science education, and debate rages over which platform to develop on and what software to use for development. Cabana is a web-based application designed to enable development on multiple mobile platforms and to make application development easier. It uses an approach to application programming based on a wiring diagram that is supplemented with the ability to program directly using JavaScript. It is an ideal choice for application development in introductory computer science courses and for upper-level courses where the focus is on application design and not application programming. This paper introduces Cabana and describes its use in two different computer science courses.} \\ \\
{\sffamily \large 8:55}& 
\multicolumn{2}{@{}p{3.75in}}{\sffamily\raggedright\textbf{Mobile Apps for the Greater Good: A Socially Relevant Approach to Software Engineering}} \\
& \multicolumn{2}{@{}p{3.75in}}{\raggedright Victor Pauca, \textit{Wake Forest University}; Richard Guy, \textit{University of Toronto}} \\ \\
\multicolumn{3}{@{}p{5in}}{\small Socially relevant computing has recently been proposed as a way to reinvigorate interest in computer science. By appealing to students' interest in helping others, it aims to give students life-changing experiential learning not typically achieved in the classroom, while providing software that benefits society at large. 
For the last two years, we have been using mobile device programming, agile methods, and real-world, socially relevant projects for teaching software engineering in a liberal arts Computer Science curricula. We report on teaching methods, student experiences, and products  delivered by this approach. In particular, one of these products, Verbal Victor, is now a commercial and social entrepreneurship success in assistive technology for communication disabilities.} \\ \\
{\sffamily \large 9:20}& 
\multicolumn{2}{@{}p{3.75in}}{\sffamily\raggedright\textbf{Using Mobile Phone Programming to Teach Java and Advanced Programming to Computer Scientists}} \\
& \multicolumn{2}{@{}p{3.75in}}{\raggedright Derek Riley, \textit{University of Wisconsin- Parkside}} \\ \\
\multicolumn{3}{@{}p{5in}}{\small In this work the approach employing the Android mobile phone platform in an upper division computer science course to teach Java programming and other advanced computer science topics is presented.  Mobile phones are growing influences in the computing market, but their strengths and popularity are rarely exploited in computer science classrooms.  
The aim of the course is to harness this enthusiasm
to improve fluency in the Java language to afford an opportunity to learn how to work on large, complex projects and to enhance the students’ preparedness for the job market.  The ideas presented in this work could be adapted for improving learning in many courses across the computing curriculum.} \\ \\
{\sffamily \large 9:45}& 
\multicolumn{2}{@{}p{3.75in}}{\sffamily\raggedright\textbf{RoboLIFT: Engaging CS2 Students with Testable, Automatically Evaluated Android Applications}} \\
& \multicolumn{2}{@{}p{3.75in}}{\raggedright Anthony Allevato and Stephen H. Edwards, \textit{Virginia Tech}} \\ \\
\multicolumn{3}{@{}p{5in}}{\small Making computer science assignments interesting and relevant is a constant challenge for instructors of introductory courses. Android has become popular in these courses to take advantage of the increasing popularity of smartphones and mobile “apps.” This has been shown to increase student engagement but it is only the first step, and we must continue to provide support for teaching methodologies that we have used in the past, such as test-driven development and automated assessment. We have developed RoboLIFT, a library that makes unit testing of Android applications approachable for students. Furthermore, by supporting existing automated grading techniques, we are able to sustain large student enrollments, and we evaluate the effects that using Android has had on student performance.} \\ \\
\end{longtable}


\newpage
\begin{longtable}{@{}p{0.75in}@{}p{3.25in}@{}r}
   {\sffamily\large\textbf{PAPERS}} &
   {\raggedright\sffamily\large\textbf{Visualization}} & 
   {\sffamily\large\textbf{306B }} \\
%row 2
   Chair:  & 
   {\raggedright Demian Lessa \textit{State University of New York at Buffalo}} & \\ \\
{\sffamily \large 8:30}& 
\multicolumn{2}{@{}p{3.75in}}{\sffamily\raggedright\textbf{Highway Data and Map Visualizations for Educational Use}} \\
& \multicolumn{2}{@{}p{3.75in}}{\raggedright James Teresco, \textit{Siena College}} \\ \\
\multicolumn{3}{@{}p{5in}}{\small It is often a challenge to find interesting and appropriate data sets to use as examples to demonstrate graph data structures and algorithms.  The data should include examples small enough to work through manually, but some large enough to demonstrate important behaviors of a structure or algorithm.  It should be freely available in a convenient format and have some real-world relevance.  Visualization of the data and results computed from it is helpful.
This paper describes a collection of graph data sets generated from the Clinched Highway Mapping Project's highway data and some examples of their use in courses.  The source data, the process used to convert the data into a more useful format, some examples of its use, and a visualization tool using the Google Maps API, are described.} \\ \\
{\sffamily \large 8:55}& 
\multicolumn{2}{@{}p{3.75in}}{\sffamily\raggedright\textbf{Experiments with Algorithm Visualization Tool Development}} \\
& \multicolumn{2}{@{}p{3.75in}}{\raggedright Michael Orsega, \textit{University of West Georgia}; Bradley Vander Zanden and Christopher Skinner, \textit{University of Tennessee}} \\ \\
\multicolumn{3}{@{}p{5in}}{\small This paper presents the initial stages of a teaching tool named iSketchmate, intended for instructor use during lecture. iSketchmate allows users to create and manipulate splay trees through an animated GUI. It improves upon existing tools by providing (1) the ability to begin with any user-defined tree, (2) a history mechanism so tree operations can be repeated or changed, and (3) finer-grained animation within each operation so instructors may give further descriptions at intermediate steps within any given operation. Experiments showed iSketchmate users could produce significantly more diagrams and these diagrams were significantly more accurate than those made with pencil and paper.} \\ \\
{\sffamily \large 9:20}& 
\multicolumn{2}{@{}p{3.75in}}{\sffamily\raggedright\textbf{CSTutor:  A Pen-Based Tool for Visualizing Data Structures}} \\
& \multicolumn{2}{@{}p{3.75in}}{\raggedright Sarah Buchanan, Brandon Ochs and Joseph LaViola, \textit{University of Central Florida}} \\ \\
\multicolumn{3}{@{}p{5in}}{\small We present CSTutor, a pen-based application for data structure visualization which allows the user to manipulate data structures through the recognition of handwritten symbols and gestures as well as edit the corresponding code. The UI consists of a sketching area where the user can draw a data structure in a way that is as natural as pen and paper.  Running in parallel with the visualization is a code view window where the user can make changes to the source code and add functions which manipulate the data structure on the canvas in real time.  We  also  present  the results  of a perceived  usefulness survey.  The results of the study indicate that the majority of students would find CSTutor helpful for learning data structures.} \\ \\
{\sffamily \large 9:45}& 
\multicolumn{2}{@{}p{3.75in}}{\sffamily\raggedright\textbf{ECVisual: A Visualization Tool for Elliptic Curve Based Ciphers}} \\
& \multicolumn{2}{@{}p{3.75in}}{\raggedright Jean Mayo, Jun Tao, Jun Ma, Melissa Keranen and Ching-Kuang Shene, \textit{Michigan Technological University}} \\ \\
\multicolumn{3}{@{}p{5in}}{\small This paper describes a visualization tool ECvisual that helps students understand and instructors teach elliptic curve based ciphers. This tool permits a user to visualize elliptic curves over the real field and over a finite field of prime order, perform arithmetic operations, do encryption and decryption, and convert plaintext to a point. The demo mode of ECvisual can be used for classroom presentation and self-study. With the practice mode, a user may go through steps in finite field computations, encryption, decryption and plaintext conversion. She may compute, and then check, the answer to each operation herself. The opportunity for self-study provides an instructor greater flexibility  in selecting a lecture pace for this detail-filled material. Classroom evaluation was positive.} \\ \\
\end{longtable}


\begin{longtable}[l]{@{}p{1in}@{}p{3in}@{}r}
    {\sffamily\large\textbf{Presentations}} & 
    {\sffamily\large\textbf{Student Research Competition - Graduate}} & 
    {\sffamily\large\textbf{302C}} \\
\end{longtable}    
\begin{longtable}[l]{@{}p{1in}@{}p{3in}@{}r}
    {\sffamily\large\textbf{Presentations}} & 
    {\sffamily\large\textbf{Student Research Competition - Undergraduate}} & 
    {\sffamily\large\textbf{305A}} \\
\end{longtable}    
\vspace{0.5em}
\noindent\rule{5in}{0.02cm}
\vspace{0.5em}
\cfoot{\colorbox[gray]{0.45}{\color{white}\textsf{Saturday 09:00 - 12:00}}}
\noindent
\framebox[5in][c]{{\Large\sffamily\textbf{Saturday,  9:00 to 12:00}}}
\begin{longtable}[l]{@{}p{1in}@{}p{3in}@{}r}
    {\sffamily\large\textbf{Social}} & 
    {\sffamily\large\textbf{K-12 Teachers Room}} & 
    {\sffamily\large\textbf{202}} \\
\end{longtable}    
\begin{longtable}[l]{@{}p{1in}@{}p{3in}@{}r}
    {\sffamily\large\textbf{Social}} & 
    {\sffamily\large\textbf{CS Education Research Room}} & 
    {\sffamily\large\textbf{203}} \\
\end{longtable}    
\vspace{0.5em}
\noindent\rule{5in}{0.02cm}
\vspace{0.5em}
\cfoot{\colorbox[gray]{0.45}{\color{white}\textsf{Saturday 09:30 - 12:00}}}
\noindent
\framebox[5in][c]{{\Large\sffamily\textbf{Saturday,  9:30 to 12:00}}}
\begin{longtable}[l]{@{}p{1in}@{}p{3in}@{}r}
    {\sffamily\large\textbf{Exhibits}} & 
    {\sffamily\large\textbf{Exhibits}} & 
    {\sffamily\large\textbf{Exhibit Hall A}} \\
\end{longtable}    
\vspace{0.5em}
\noindent\rule{5in}{0.02cm}
\vspace{0.5em}
\cfoot{\colorbox[gray]{0.45}{\color{white}\textsf{Saturday 10:00 - 11:30}}}
\noindent
\framebox[5in][c]{{\Large\sffamily\textbf{Saturday,  10:00 to 11:30}}}
\begin{longtable}[l]{@{}p{1in}@{}p{3in}@{}r}
    {\sffamily\large\textbf{Project Showcase}} & 
    {\sffamily\large\textbf{NSF Showcase \#5}} & 
    {\sffamily\large\textbf{Exhibit Hall A}} \\
\end{longtable}    
\vspace{0.5em}
\noindent\rule{5in}{0.02cm}
\vspace{0.5em}
\cfoot{\colorbox[gray]{0.45}{\color{white}\textsf{Saturday 10:10 - 10:55}}}
\noindent
\framebox[5in][c]{{\Large\sffamily\textbf{Saturday,  10:10 to 10:55}}}
\begin{longtable}[l]{@{}p{1in}@{}p{3in}@{}r}
    {\sffamily\large\textbf{None}} & 
    {\sffamily\large\textbf{Break and Exhibits}} & 
    {\sffamily\large\textbf{Exhibit Hall A}} \\
\end{longtable}    
\vspace{0.5em}
\noindent\rule{5in}{0.02cm}
\vspace{0.5em}
\cfoot{\colorbox[gray]{0.45}{\color{white}\textsf{Saturday 10:55 - 12:10}}}
\noindent
\framebox[5in][c]{{\Large\sffamily\textbf{Saturday,  10:55 to 12:10}}}
\begin{longtable}[l]{@{}l@{}l@{}r}
    \parbox[t]{1in}{\sffamily\large\textbf{PANEL}} & 
    \parbox[t]{3in}{\sffamily\raggedright\large\textbf{Rediscovering the Passion, Beauty, Joy, and Awe:  Making Computing Fun Again}} & 
    \parbox[t]{1in}{\sffamily\raggedleft\large\textbf{301AB}} \\
% row 2    
    Chair: & 
    Daniel Garcia \textit{UC Berkeley}  \\[0.5em]
% row 3
    Participants: & 
    \multicolumn{2}{@{}l}{\parbox{3.75in}{Barbara Ericson, \textit{Georgia Institute of Technology}; Joanna Goode, \textit{University of Oregon}; Colleen Lewis, \textit{UC Berkeley} }} \\[2em]
% row 4
    \multicolumn{3}{@{}p{5in}}{\small In his SIGCSE 2007 keynote, Grady Booch exhorted us to share the “passion, beauty, joy and awe” (PBJA) of computing. This led to a series of SIGCSE sessions that provided a forum for sharing:
• What we’ve done: Highlighting successful PBJA initiatives the presenters have undertaken or seen and wish to celebrate
• What we should do (curriculum): Pointing out where our curriculum is lacking in PBJA, and how to fix it
• How we should do it (pedagogy): Sharing how a change in attitude / focus / etc. can make strides to improving PBJA
This year we’ve invited 3 educators who have worked tirelessly to broaden participation of computing to underrepresented groups.  The hope with this panel is to be able to explore best practices in outreach, in terms of extolling the PBJA of computing.}
\end{longtable}
\begin{longtable}[l]{@{}l@{}l@{}r}
    \parbox[t]{1in}{\sffamily\large\textbf{PANEL}} & 
    \parbox[t]{3in}{\sffamily\raggedright\large\textbf{Promoting Student-Centered Learning with POGIL}} & 
    \parbox[t]{1in}{\sffamily\raggedleft\large\textbf{305B}} \\
% row 2    
    Chair: & 
    Helen Hu, \textit{Westminster College}  \\[0.5em]
% row 3
    Participants: & 
    \multicolumn{2}{@{}l}{\parbox{3.75in}{Clifton Kussmaul, \textit{Muhlenberg College} }} \\[2em]
% row 4
    \multicolumn{3}{@{}p{5in}}{\small POGIL (Process Oriented Guided Inquiry Learning) is a type of learning based on the principle that students learn more when they construct their own understanding. Rather than teaching by telling, POGIL instructors provide activities that guide students to discover concepts on their own. Students work in groups, encouraging them to discuss their findings with their peers. Not only do students learn the material better, but the process of discovery teaches them to be better problem solvers. This special session will provide SIGCSE attendees the opportunity to experience a POGIL activity. The presenters will share guided inquiry activities. We will discuss ways that POGIL may be used to transform computer science classes at all levels, from small schools to large universities.}
\end{longtable}
\begin{longtable}[l]{@{}l@{}l@{}r}
    \parbox[t]{1in}{\sffamily\large\textbf{PANEL}} & 
    \parbox[t]{3in}{\sffamily\raggedright\large\textbf{Teaching Secure Coding - Report from Summit on Education in Secure Software}} & 
    \parbox[t]{1in}{\sffamily\raggedleft\large\textbf{306C}} \\
% row 2    
    Chair: & 
    Blair Taylor, \textit{Towson University}  \\[0.5em]
% row 3
    Participants: & 
    \multicolumn{2}{@{}l}{\parbox{3.75in}{Matt Bishop, \textit{University of California Davis}; Diana Burley, \textit{George Washington University}; Steve Cooper, \textit{Stanford University}; Ron Dodge, \textit{United States Military Academy} }} \\[2em]
% row 4
    \multicolumn{3}{@{}p{5in}}{\small To help education design curricula that integrates principles and practices of secure programming, the National Science Foundation Directorates of Computer and Information Science and Engineering (CISE) and Education and Human Resources (EHR) jointly sponsored the Summit on Education in Secure Software (SESS), held in Washington, DC in October, 2010. The goal of the summit was to develop roadmaps showing how best to educate students and current professionals on robust, secure programming concepts and practices, and to identify both the resources required and the problems that had to be overcome. The goal of this session is to share some of the key findings and challenges identified by the summit and to actively engage the community in the discussions.}
\end{longtable}
\newpage
\begin{longtable}{@{}p{0.75in}@{}p{3.25in}@{}r}
   {\sffamily\large\textbf{PAPERS}} &
   {\raggedright\sffamily\large\textbf{Attracting Majors}} & 
   {\sffamily\large\textbf{302A }} \\
%row 2
   Chair:  & 
   {\raggedright Nanette Veilleux \textit{Simmons College}} & \\ \\
{\sffamily \large 10:55}& 
\multicolumn{2}{@{}p{3.75in}}{\sffamily\raggedright\textbf{Starting with ubicomp:   using the SenseBoard to introduce computing}} \\
& \multicolumn{2}{@{}p{3.75in}}{\raggedright Marian Petre, Mike Richards and Arosha Bandara, \textit{The Open University}} \\ \\
\multicolumn{3}{@{}p{5in}}{\small In this paper, we describe a new undergraduate module for novice students conducted entirely through distance learning: My Digital Life (TU100). The course has been designed to lower the barriers to creating programs that interact with the world; TU100’s materials have been designed to excite, encourage, reassure and support learners who explore the novel topic of ubiquitous computing through playful experimentation. It introduces the fundamentals of computing by giving students the capability for programming a device, the SenseBoard, which has built-in input/output and sensors.  Programming is done in Sense, an extension of Scratch, which scaffolds programming and reduces the syntax burden.} \\ \\
{\sffamily \large 11:20}& 
\multicolumn{2}{@{}p{3.75in}}{\sffamily\raggedright\textbf{Improving First-Year Success and Retention through Interest-Based CS0 Courses}} \\
& \multicolumn{2}{@{}p{3.75in}}{\raggedright Michael Haungs, Christopher Clark, John Clements and David Janzen, \textit{California Polytechnic State Unviersity, San Luis Obispo}} \\ \\
\multicolumn{3}{@{}p{5in}}{\small Many computer science programs suffer from low student retention rates. At Cal Poly, academic performance and retention rates among first-year computer science students are among the lowest on campus.
     In order to remedy this, we have developed a new CS0 course featuring different "tracks'' that students can choose from. This allows students to learn the basics of programming, teamwork, and college-level study in a domain that is of personal interest. In addition, the course relies on classic Project-based Learning (PBL) approaches as well as a focus on both academic and non-academic factors shown to increase student retention.
     Initial assessment demonstrates positive results in the form of
increased academic performance in post CS0 courses and student retention.} \\ \\
{\sffamily \large 11:45}& 
\multicolumn{2}{@{}p{3.75in}}{\sffamily\raggedright\textbf{Reshaping The Image Of Computer Science In Only Fifteen Minutes (Of Class) A Week}} \\
& \multicolumn{2}{@{}p{3.75in}}{\raggedright Sara Sprenkle, \textit{Washington and Lee University}; Shannon Duvall, \textit{Elon University}} \\ \\
\multicolumn{3}{@{}p{5in}}{\small Low undergraduate enrollments in computer science will not meet the future demand of employers.  Some reasons for the low enrollments are computer science's nerdy image, lack of understanding of the field, and low motivation for learning programming.  We propose to change the image of computer science by exposing students to applications of computing and its impact on their lives through reading and discussing recent news articles in 15 minutes of class.  We call this component of our courses the Broader Issues in computer science.} \\ \\
\end{longtable}


\newpage
\begin{longtable}{@{}p{0.75in}@{}p{3.25in}@{}r}
   {\sffamily\large\textbf{PAPERS}} &
   {\raggedright\sffamily\large\textbf{OS and Distributed Computing}} & 
   {\sffamily\large\textbf{302B }} \\
%row 2
   Chair:  & 
   {\raggedright William Mongan \textit{Drexel University}} & \\ \\
{\sffamily \large 10:55}& 
\multicolumn{2}{@{}p{3.75in}}{\sffamily\raggedright\textbf{Experiences Teaching MapReduce in the Cloud}} \\
& \multicolumn{2}{@{}p{3.75in}}{\raggedright Ariel Rabkin, Charles Reiss, Randy Katz and David Patterson, \textit{UC Berkeley}} \\ \\
\multicolumn{3}{@{}p{5in}}{\small We describe our experiences teaching MapReduce in a large undergraduate lecture course using public cloud services. Using the cloud, every student could carry out scalability benchmarking assignments on realistic hardware, which would have been impossible otherwise. Over two semesters, over 500 students took our course. We believe this is the first large-scale demonstration that it is feasible to use pay-as-you-go billing in the Cloud for a large undergraduate course. Modest instructor effort was sufficient to prevent students from overspending. Average per-pupil expenses in the Cloud were under \$45, less than half our available grant funding. Students were excited by the assignment: 90\% said they thought it should be retained in future course offerings.} \\ \\
{\sffamily \large 11:20}& 
\multicolumn{2}{@{}p{3.75in}}{\sffamily\raggedright\textbf{Developing Microlabs Using Google Web Toolkit}} \\
& \multicolumn{2}{@{}p{3.75in}}{\raggedright Barry Kurtz, James Fenwick and Philip Meznar, \textit{Appalachian State University}} \\ \\
\multicolumn{3}{@{}p{5in}}{\small Closed labs provide hands-on experience in a supervised setting.  Microlabs extend this approach into lecture with very short hands-on activities occurring in lecture.  Programming microlabs were developed for a distributed computing course.  This paper describes our logical microlabs where students solve conceptual problems that do not involve programming.  These two microlab approaches are integrated into the Microlab Learning Cycle.  Microlab activities should be usable with a wide variety of computing devices, including tablets. After experimenting with different development environments we have adopted the Google Web Toolkit. After presenting the current status of our activities, we discuss future directions for microlab development.} \\ \\
{\sffamily \large 11:45}& 
\multicolumn{2}{@{}p{3.75in}}{\sffamily\raggedright\textbf{Teaching Operating Systems Using Android}} \\
& \multicolumn{2}{@{}p{3.75in}}{\raggedright Jeremy Andrus and Jason Nieh, \textit{Columbia University}} \\ \\
\multicolumn{3}{@{}p{5in}}{\small The computing landscape is shifting towards mobile and embedded devices. To learn about operating systems, it is increasingly important for students to gain hands-on kernel programming experience in these environments, which are quite different from desktops and servers. We present our work to teach operating systems by leveraging Android, an open, commercially supported software platform increasingly used on mobile and embedded devices. We introduce a series of 5 Android kernel programming projects, and an Android virtual lab which gives students hands-on Android experience with minimal computing infrastructure. We used these projects and virtual lab to teach an introductory operating systems course. Over 80\% of students surveyed enjoyed applying operating systems concepts to Android.} \\ \\
\end{longtable}


\newpage
\begin{longtable}{@{}p{0.75in}@{}p{3.25in}@{}r}
   {\sffamily\large\textbf{PAPERS}} &
   {\raggedright\sffamily\large\textbf{Curricular Innovations and Research}} & 
   {\sffamily\large\textbf{306A }} \\
%row 2
   Chair:  & 
   {\raggedright Michael Hewner \textit{Duke University}} & \\ \\
{\sffamily \large 10:55}& 
\multicolumn{2}{@{}p{3.75in}}{\sffamily\raggedright\textbf{Open Educational Resources in Computer Science Teaching}} \\
& \multicolumn{2}{@{}p{3.75in}}{\raggedright Christo Dichev and Darina Dicheva, \textit{Winston Salem State University}} \\ \\
\multicolumn{3}{@{}p{5in}}{\small Open content and open access to resources are important factors in the innovation of Computer Science education. This paper presents a study aimed at gaining an understanding of the needs of Computer Science educators in terms of Open Educational Resources (OER): what kind of resources they need, when they need them, how they use them, and what are the barriers and the enablers for using OER The results of the study are compared and analyzed in the context of the popular OER sites. The work contributes to the research on OER utilization and discovery.} \\ \\
{\sffamily \large 11:20}& 
\multicolumn{2}{@{}p{3.75in}}{\sffamily\raggedright\textbf{Emergent Themes in a UI Design Hybrid-Studio Course}} \\
& \multicolumn{2}{@{}p{3.75in}}{\raggedright Yolanda Reimer, \textit{University of Montana}; Katherine Cennamo, \textit{Virginia Tech}; Sarah Douglas, \textit{University of Oregon}} \\ \\
\multicolumn{3}{@{}p{5in}}{\small The goal of our research and teaching collaboration has been to learn more about how key aspects of pedagogy commonly incorporated in architecture and industrial design classes might positively impact the teaching of user interface (UI) design within a standard computer science curriculum. Toward that end, we studied a number of studio design courses, developed a set of curriculum guidelines, and analyzed the effectiveness of these guidelines as implemented in a UI design course. We discovered three emergent themes: 1) students need early and constant reminders that design is an iterative process involving user feedback and testing; 2) instructor modeling is critical; and 3) technology needs to be carefully managed at critical junctures throughout the class.} \\ \\
{\sffamily \large 11:45}& 
\multicolumn{2}{@{}p{3.75in}}{\sffamily\raggedright\textbf{A Multilevel, Multidimensional Undergraduate Course and Lab Experience on Embedded Multimedia Systems}} \\
& \multicolumn{2}{@{}p{3.75in}}{\raggedright Dimitrios Charalampidis, \textit{University of New Orleans}; James Haralambides, \textit{Barry University}} \\ \\
\multicolumn{3}{@{}p{5in}}{\small Traditional curricular structures can be fragmented in that course inter\&\#8208;relationships or links between theories, methodologies, and practices, are not immediately recognized by the students. The completion of the course puzzle and the integration of course knowledge usually become evident only after graduation. This paper describes a course/lab implementation that offered students a unique opportunity to experience the full spectrum of course elements, namely, fundamentals of theory, algorithmic/hardware design and simulation, and implementation and testing on FPGAs all within a single framework. The course/lab design is a collaborative effort between the U. of New Orleans and Barry U. to ensure that the course/lab can be implemented successfully in diverse environments.} \\ \\
\end{longtable}


\newpage
\begin{longtable}{@{}p{0.75in}@{}p{3.25in}@{}r}
   {\sffamily\large\textbf{PAPERS}} &
   {\raggedright\sffamily\large\textbf{CS Education Research}} & 
   {\sffamily\large\textbf{306B }} \\
%row 2
   Chair:  & 
   {\raggedright Yana Kortsarts \textit{Widener University, Chester}} & \\ \\
{\sffamily \large 10:55}& 
\multicolumn{2}{@{}p{3.75in}}{\sffamily\raggedright\textbf{Effective Closed Labs in Early CS Courses: Lessons from Seven Terms of Action Research}} \\
& \multicolumn{2}{@{}p{3.75in}}{\raggedright Elizabeth Patitsas, \textit{University of Toronto}; Steve Wolfman, \textit{University of British Columbia}} \\ \\
\multicolumn{3}{@{}p{5in}}{\small We report on best practices we have established to teach first-year computer science students in closed laboratories, founded on over three years of action research in a large introductory discrete mathematics and digital logic course. Our practices have resulted in statistically significant improvements in student and teaching assistant perception of the labs. Specifically, we discuss our practices of streamlining labs to reduce load on students that is extraneous to the lab's learning goals; establishing a positive first impression for students and TAs in the early weeks of the term; and effectively managing the teaching staff, including weekly preparation meetings for TAs using and a gradual, iterative curriculum development cycle that engages all stakeholders in the course.} \\ \\
{\sffamily \large 11:20}& 
\multicolumn{2}{@{}p{3.75in}}{\sffamily\raggedright\textbf{What Do Students Learn About Programming  From Game, Music Video, And Storytelling Projects?}} \\
& \multicolumn{2}{@{}p{3.75in}}{\raggedright Joel Adams and Andrew Webster, \textit{Calvin College}} \\ \\
\multicolumn{3}{@{}p{5in}}{\small Drag-and-drop learning environments like Alice and Scratch eliminate syntax errors, making them attractive as ways to introduce programming concepts to students.  Having had students create games, music videos, and storytelling projects, we began to wonder: What programming constructs do students actually use and hence learn well enough to be able to apply when creating different kinds of projects?  We conducted a quantitative analysis of a collection of over 300 different student projects, and found significant differences in how frequently the students creating those projects used variables, if statements, loops, and dialog constructs.} \\ \\
{\sffamily \large 11:45}& 
\multicolumn{2}{@{}p{3.75in}}{\sffamily\raggedright\textbf{Bayesian Network Analysis of Computer Science Grade Distributions}} \\
& \multicolumn{2}{@{}p{3.75in}}{\raggedright Adam Anthony and Mitchell Raney, \textit{Baldwin-Wallace College}} \\ \\
\multicolumn{3}{@{}p{5in}}{\small Time to completion is a major factor in determining the total cost of a college degree.  In an effort to reduce the number of students taking more than four years to complete a degree, we propose the use of Bayesian networks to predict student grades, given past performance prerequisite courses.  This is an intuitive approach because the necessary structure of any Bayesian network must be a directed acyclic graph, which is also the case for prerequisite graphs.  We demonstrate that building a Bayesian network directly from the prerequisite graph results in effective predictions, and demonstrate a few applications of the resulting network in areas of identifying struggling students and deciding upon which courses a department should allocate tutoring resources.} \\ \\
\end{longtable}


\begin{longtable}[l]{@{}p{1in}@{}p{3in}@{}r}
    {\sffamily\large\textbf{SupporterSession}} & 
    {\sffamily\large\textbf{TBA}} & 
    {\sffamily\large\textbf{302C}} \\
\end{longtable}    
\begin{longtable}[l]{@{}p{1in}@{}p{3in}@{}r}
    {\sffamily\large\textbf{SupporterSession}} & 
    {\sffamily\large\textbf{TBA}} & 
    {\sffamily\large\textbf{305A}} \\
\end{longtable}    
\vspace{0.5em}
\noindent\rule{5in}{0.02cm}
\vspace{0.5em}
\cfoot{\colorbox[gray]{0.45}{\color{white}\textsf{Saturday 12:30 - 14:30}}}
\noindent
\framebox[5in][c]{{\Large\sffamily\textbf{Saturday,  12:30 to 14:30}}}
\begin{longtable}[l]{@{}p{1in}@{}p{3in}@{}r}
    {\sffamily\large\textbf{Plenary Session}} & 
    {\sffamily\large\textbf{SIGCSE Luncheon: Fernanda Viégas and Martin Wattenberg}} & 
    {\sffamily\large\textbf{Ballroom AB}} \\
    \multicolumn{3}{@{}p{5in}}{\sffamily\large\textbf Through the Looking Glass: Talking about the World with Visualization} \\
    \multicolumn{3}{@{}p{5in}}{\small Data visualization has historically been accessible only to the elite in academia, business, and government. It was ``serious'' technology, created by experts for experts. In recent years, however, web-based visualizations--ranging from political art projects to news stories--have reached audiences of millions.
What will this new era of data transparency look like--and what are the implications for technologists who work with data? To help answer this question, we report on recent research into public data analysis and visualization. Some of our results come from Many Eyes, a ``living laboratory'' web site where people may upload their own data, create interactive visualizations, and carry on conversations. We'll also show how the art world has embraced visualization. We'll discuss the future of visual literacy and what it means for a world where visualizations are a part of political discussions, citizen activism, religious discussions, game playing, and educational exchanges.
}    
\end{longtable}    
\vspace{0.5em}
\noindent\rule{5in}{0.02cm}
\vspace{0.5em}
\cfoot{\colorbox[gray]{0.45}{\color{white}\textsf{Saturday 15:00 - 18:00}}}
\noindent
\framebox[5in][c]{{\Large\sffamily\textbf{Saturday,  15:00 to 18:00}}}
\begin{longtable}[l]{@{}l@{}l@{}r}
    \parbox[t]{0.25in}{\sffamily\large\textbf{28.}} & 
    \parbox[t]{3.75in}{\raggedright\sffamily\large\textbf{Snap! (Build Your Own Blocks)}} & 
    {\sffamily\large\textbf{301A}} \\[1.5em]
% row 3
    \multicolumn{3}{@{}l}{\parbox{5in}{Brian Harvey, Daniel Garcia and Luke Segars, \textit{University of California, Berkeley}; Josh Paley, \textit{Henry M. Gunn High School} }} \\[1.5em]
% row 4
    \multicolumn{3}{@{}p{5in}}{\small This workshop is for high school 
and college teachers of general-
interest ("CS 0") computer 
science courses.  It presents the 
programming environment used 
in two of the five initial AP CS 
Principles pilot courses.

Snap! (Build Your Own Blocks) is 
a free, graphical, drag-and-
drop extension to the Scratch 
programming language.  
Scratch, designed for 8-14 year 
olds, models programs as 
"scripts" without names, 
arguments, or return values. 
Snap! supports older learners 
(14-20) by adding named 
procedures (thus recursion), 
procedures as data (thus higher 
order functions) structured lists, 
and sprites as first class objects 
with inheritance.

Participants will learn Snap! 
through discussion, 
programming exercises, and 
exploration.  See 
http://snap.berkeley.edu for 
details.  Laptop required.}
\end{longtable}
\begin{longtable}[l]{@{}l@{}l@{}r}
    \parbox[t]{0.25in}{\sffamily\large\textbf{29.}} & 
    \parbox[t]{3.75in}{\raggedright\sffamily\large\textbf{Circuits and Microcontrollers in Computer Organization Laboratories}} & 
    {\sffamily\large\textbf{301B}} \\[1.5em]
% row 3
    \multicolumn{3}{@{}l}{\parbox{5in}{Marge Coahran, \textit{Dickinson College}; Janet Davis, \textit{Grinnell College} }} \\[1.5em]
% row 4
    \multicolumn{3}{@{}p{5in}}{\small This workshop will introduce a 
set of hands-on laboratory 
activities appropriate for a first 
Computer Organization course. 
Participants will work with real 
equipment: first implementing 
elementary digital circuits with 
TTL logic chips, and then 
programming AVR 
microcontrollers in assembly to 
drive fun accessories such as 
LEDs and speakers. Participants 
will not take equipment home 
afterwards, but will receive parts 
lists and vendor information. 
The workshop is intended for 
educators with little electronics 
background who are interested 
in incorporating electronics 
laboratories into their courses. 
Laptops (Linux, Mac, or 
Windows) will provide the 
programming environment for 
the AVRs. Free software will be 
available before the workshop. 
Participants will work in pairs. 
Laptop recommended.}
\end{longtable}
\begin{longtable}[l]{@{}l@{}l@{}r}
    \parbox[t]{0.25in}{\sffamily\large\textbf{30.}} & 
    \parbox[t]{3.75in}{\raggedright\sffamily\large\textbf{Web Development with Python and Django}} & 
    {\sffamily\large\textbf{302A}} \\[1.5em]
% row 3
    \multicolumn{3}{@{}l}{\parbox{5in}{Ariel Ortiz, \textit{Tecnologico de Monterrey, Campus Estado de Mexico} }} \\[1.5em]
% row 4
    \multicolumn{3}{@{}p{5in}}{\small Many instructors have already 
discovered the joy of teaching 
programming using Python. 
Now it's time to take Python to 
the next level. This workshop 
will introduce Django, an open 
source Python web framework 
that saves you time and makes 
web development fun. It's aimed 
at CS instructors who want to 
teach how to build elegant web 
applications with minimal fuss. 
Django is Python's equivalent to 
the popular Ruby on Rails 
framework. Topics that will be 
covered include: setup and 
configuration, template 
language, and database 
integration through object-
relational mapping. Participants 
should have some familiarity 
with Python, HTML and SQL. 
More information at: 
http://webcem01.cem.itesm.mx
:8005/django/ Laptop Required}
\end{longtable}
\begin{longtable}[l]{@{}l@{}l@{}r}
    \parbox[t]{0.25in}{\sffamily\large\textbf{31.}} & 
    \parbox[t]{3.75in}{\raggedright\sffamily\large\textbf{Improving the Accessibility of Computing Enrichment Programs}} & 
    {\sffamily\large\textbf{302B}} \\[1.5em]
% row 3
    \multicolumn{3}{@{}l}{\parbox{5in}{Richard Ladner, \textit{University of Washington}; Karen Alkoby, \textit{Gallaudet University}; Jeff Bigham, \textit{University of Rochester}; Stephanie Ludi, \textit{Rochester Institute of Technology}; Daniela Marghitu, \textit{Auburn University}; Andreas Stefik, \textit{University of Southern Illinois, Edwardsville} }} \\[1.5em]
% row 4
    \multicolumn{3}{@{}p{5in}}{\small Many wonderful enrichment 
programs have been created to 
introduce young people to 
computing, but with little 
attention to making them 
accessible to students with 
disabilities.  In this workshop 
participants will learn from 
practitioners who have 
introduced computing and 
programming to young people 
with disabilities.  They will also 
learn first-hand from students 
with disabilities about their 
needs in learning programming. 
There will be breakout sessions 
for participants to apply what 
they have learned to improve 
existing enrichment programs 
such as Alice, Arduino, Scratch, 
Kodu, App Inventor, Greenfoot, 
Lego Mindstorms, Processing, 
and Computer Science 
Unplugged. Laptop 
Recommended.}
\end{longtable}
\begin{longtable}[l]{@{}l@{}l@{}r}
    \parbox[t]{0.25in}{\sffamily\large\textbf{32.}} & 
    \parbox[t]{3.75in}{\raggedright\sffamily\large\textbf{Enhancing Student Interest by Extending Graphics Applications}} & 
    {\sffamily\large\textbf{302C}} \\[1.5em]
% row 3
    \multicolumn{3}{@{}l}{\parbox{5in}{Samuel Rebelsky, \textit{Grinnell College} }} \\[1.5em]
% row 4
    \multicolumn{3}{@{}p{5in}}{\small Computer science teachers 
strive for new examples and 
problems to interest millenials. 
The Media Computing approach 
has proven successful in 
attracting students in contexts 
from community colleges to R1 
universities – students are 
clearly excited by writing 
programs that make images.

In this workshop, we show how 
to go a step further and have 
write scripts and plug-ins in 
Python for open-source 
graphics programs such as GIMP 
and Inkscape.  Students not only 
make images, they write filters 
and features that they can share 
with others, even non-
programmers.  E.g., students 
have written filters that 
“fractalize” vector graphics or 
that turn images into stained 
glass.  

Further information can be 
found at
http://www.cs.grinnell.edu/~reb
elsky/Workshops/SIGCSE2012/

Laptop required.}
\end{longtable}
\begin{longtable}[l]{@{}l@{}l@{}r}
    \parbox[t]{0.25in}{\sffamily\large\textbf{33.}} & 
    \parbox[t]{3.75in}{\raggedright\sffamily\large\textbf{Engage Your Students by Teaching Programming Using Only Mobile Devices with TouchDevelop}} & 
    {\sffamily\large\textbf{305A}} \\[1.5em]
% row 3
    \multicolumn{3}{@{}l}{\parbox{5in}{Nikolai Tillmann, Michal Moskal, Jonathan de Halleux and Manuel Fahndrich, \textit{Microsoft Research}; Tao Xie, \textit{North Carolina State University} }} \\[1.5em]
% row 4
    \multicolumn{3}{@{}p{5in}}{\small The world experiences a 
technology shift: Powerful and 
easy-to-use mobile devices like 
smartphones and tablets are 
becoming more prevalent than 
traditional PCs and laptops. We 
propose to reflect this new 
reality by adapting how 
programming is taught. 
Students should develop 
software directly on 
smartphones. In this workshop, 
we introduce TouchDevelop on 
Windows Phone 7, a novel 
application creation 
environment from Microsoft 
Research. Its typed, structured 
programming language is built 
around the idea of only using a 
touchscreen as the input device 
to author code. Easy access to 
the rich sensor and personal 
data available on a mobile 
device results in an engaging 
programming experience for 
students who learn 
programming by creating fun 
games and applications. Laptop 
Optional.}
\end{longtable}
\begin{longtable}[l]{@{}l@{}l@{}r}
    \parbox[t]{0.25in}{\sffamily\large\textbf{34.}} & 
    \parbox[t]{3.75in}{\raggedright\sffamily\large\textbf{CS in Parallel: Modules for Adding Parallel Computing to CS Courses, from CS2 to Theory of Computation}} & 
    {\sffamily\large\textbf{306A}} \\[1.5em]
% row 3
    \multicolumn{3}{@{}l}{\parbox{5in}{Richard Brown, \textit{St. Olaf College}; Elizabeth Shoop, \textit{Macalester College} }} \\[1.5em]
% row 4
    \multicolumn{3}{@{}p{5in}}{\small Parallel computing with more 
and more cores is here to stay.  
This workshop presents four 
independent, class-tested, 
primarily hands-on modules for 
incrementally adding parallelism 
in undergraduate CS courses, 
each requiring 1 to 3 class days 
and versatile for diverse courses 
and curricula: parallelizing loops 
and sharing memory on Intel's 
Manycore Testing lab (for a 
second CS course or for 
computer organization);  
parallel web crawler in Java or 
C++ (second CS course);  
parallel sorting (algorithms);  \&\#960;-
calculus theory for 
communicating sequential 
processes (theory of 
computation).  Workshop 
materials provided, drawn from 
CSinParallel.org.  Intended 
audience: CS instructors.  
Laptop recommended (Windows, 
Mac, Linux).}
\end{longtable}
\begin{longtable}[l]{@{}l@{}l@{}r}
    \parbox[t]{0.25in}{\sffamily\large\textbf{35.}} & 
    \parbox[t]{3.75in}{\raggedright\sffamily\large\textbf{Listening to Linked Lists: Using Multimedia to Learn Data Structures}} & 
    {\sffamily\large\textbf{306B}} \\[1.5em]
% row 3
    \multicolumn{3}{@{}l}{\parbox{5in}{Mark Guzdial and Barbara Ericson, \textit{Georgia Institute of Technology} }} \\[1.5em]
% row 4
    \multicolumn{3}{@{}p{5in}}{\small Everybody teaches linked lists, 
with homework like 
implementing duplicate, weave, 
and reverse. When those nodes 
contain strings or numbers, 
these are pretty boring 
assignments. When these nodes 
contain music (MIDI), these 
operations are composing 
music, which can then be 
played. This workshop shows 
how to use music, images, and 
sounds to teach the basic data 
structures, including linked 
lists, circular linked lists, stacks, 
queues, and trees. These pieces 
can then be tied together 
through the use of simulations 
to generate animated movies.  
We will be using Java, though 
many of the methods can also 
be used in Python. Laptop 
Recommended.}
\end{longtable}
\begin{longtable}[l]{@{}l@{}l@{}r}
    \parbox[t]{0.25in}{\sffamily\large\textbf{36.}} & 
    \parbox[t]{3.75in}{\raggedright\sffamily\large\textbf{Puzzle-Based Learning: Introducing Creative Thinking and Problem Solving for Computer Science and Engineering}} & 
    {\sffamily\large\textbf{306C}} \\[1.5em]
% row 3
    \multicolumn{3}{@{}l}{\parbox{5in}{Raja Sooriamurthi, \textit{Carnegie Mellon University}; Nick Falkner and Zbigniew Michalewicz, \textit{University of Adelaide} }} \\[1.5em]
% row 4
    \multicolumn{3}{@{}p{5in}}{\small Puzzle-based learning (PBL) is 
an emerging model of teaching 
critical thinking and problem 
solving.  Today’s market place 
needs skilled graduates capable 
of solving real problems of 
innovation in a changing 
environment. While solving 
puzzles is innately fun, 
companies such as Google and 
Yahoo also use puzzles to 
assess the creative problem 
solving skills of potential 
employees.  In this interactive 
workshop we will examine a 
range of puzzles, games, and 
general problem solving 
strategies. Participants will 
emerge with the needed 
pedagogical foundation to offer 
a full course on PBL or to 
include it as part of another 
course.   Currently 40+ 
institutions around the world 
are offering courses based on 
PBL. More details are available at 
www.PuzzleBasedLearning.edu.a
u.  Laptop optional.}
\end{longtable}
\vspace{0.5em}
\noindent\rule{5in}{0.02cm}
\vspace{0.5em}
