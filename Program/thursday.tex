%!TEX root = program.tex
\fancyhead[RO,LE]{}
\newpage
\addcontentsline{toc}{subsection}{Thursday}
\cfoot{\colorbox[gray]{0.45}{\color{white}\textsf{Thursday 08:30 - 10:00}}}
\noindent
\framebox[5in][c]{{\Large\sffamily\textbf{Thursday,  8:30 to 10:00}}}
%
% Fred Brooks Keynote
\begin{tabular}{@{}p{3.750in}r@{}}
%    \multicolumn{2}{@{}p{4in}}{\sffamily\large\textbf{Plenary Session and Keynote }} & 
    {\sffamily\large\textbf{Plenary Session and Keynote }} &
    {\raggedleft\sffamily\large\textbf{Ballroom BC}} \\ \\
\end{tabular}
\vspace{-\baselineskip}
\begin{longtable}[l]{@{}p{1in}@{}p{3in}@{}r}
    {\sffamily\large\textbf{Welcome}} & & \\
    & \multicolumn{2}{@{}l}{Laurie Smith King, Symposium Co-Chair, \textit{College of the Holy Cross}} \\
    & \multicolumn{2}{@{}l}{Dave Musicant, Symposium Co-Chair, \textit{Carleton College}} \\ \\
    \multicolumn{3}{@{}p{5in}}{\raggedright\sffamily\large\textbf{2012 SIGCSE Lifetime Service Award}} \\
    & Jane Prey, \textit{Microsoft Research} \\ \\
    \multicolumn{3}{@{}p{5in}}{\raggedright\sffamily\large\textbf{2012 SIGCSE Outstanding Contributions to Computer Science Education:}} \\
    & Harold (Hal) Abelson, \textit{Massachusetts Institute of Technology} \\ \\
    \multicolumn{3}{@{}p{5in}}{\raggedright\sffamily\large\textbf{Keynote Address: The Teacher's Job is to Design Learning Experiences; not Primarily to Impart Information}} \\ \\
    \multicolumn{3}{@{}p{5in}}{\hspace*{1in}Frederick P. Brooks, Jr.  \textit{UNC Chapel Hill}} \\
    \multicolumn{3}{@{}p{5in}}{\small The primary job of the teacher is to make learning happen; that is a design task. Most of us learned most of what we know by what we did, not by what we heard or read. A corollary is that the careful designing of exercises, assignments, projects, even quizzes, makes more difference than the construction of lectures. A second corollary is that project courses that go deeply into narrow aspects of a subject seem to stick longer and deeper than approaches aiming at comprehensive coverage. How to strike a balance? I've taught a first software engineering laboratory course 22 times, and an advanced computer architecture course about ten times. Here are some techniques that work for me.}
\end{longtable} 


\cfoot{\colorbox[gray]{0.45}{\color{white}\textsf{Thursday 10:00 - 10:45}}}
\noindent
\framebox[5in][c]{{\Large\sffamily\textbf{Thursday,  10:00 to 10:45}}}

\vspace{\baselineskip}
\noindent
\begin{tabular*}{5in}[l]{@{}p{3.9in}@{}r@{}}
    {\sffamily\large\textbf{Break and Exhibits}} & 
    {\raggedright\sffamily\large\textbf{Exhibit Hall A}} 
\end{tabular*}    


\cfoot{\colorbox[gray]{0.45}{\color{white}\textsf{Thursday 10:00 - 11:30}}}

\vspace{2.0em}
\noindent
\framebox[5in][c]{{\Large\sffamily\textbf{Thursday,  10:00 to 11:30}}}
\begin{tabular*}{5in}[l]{@{}p{3.9in}@{}r}
    {\sffamily\large\textbf{NSF Showcase \#1}} & 
    {\raggedright\sffamily\large\textbf{Exhibit Hall A}} 
\end{tabular*}    
\begin{itemize}
%     \setlength{\itemsep}{0.5\baselineskip}
     \item {{\sffamily\textbf{Tools and Best Practices for Improving the Quality of Students' Commenting Skills}} Peter DePasquale, Michael Locasto, and Lisa Kaczmarczyk} 
     \item {{\sffamily\textbf{Student Participation in Humanitarian FOSS}} Heidi Ellis and Gregory \\ Hislop } 
     \item {{\sffamily\textbf{Computing in the Arts: A Model Curriculum}} Bill Manaris and Renee \\ McCauley } 
     \item {{\sffamily\textbf{Integrating Ethics Into Computer Science Courses: Looking for \\ Statistically Significant Effects}}  Keith W. Miller, Michael Loui, Mary Tracy, and Ken Urban} 
\end{itemize}
\newpage
\cfoot{\colorbox[gray]{0.45}{\color{white}\textsf{Thursday 10:45 - 12:00}}}
\noindent
\framebox[5in][c]{{\Large\sffamily\textbf{Thursday,  10:45 to 12:00}}}
\begin{longtable}[l]{@{}l@{}l@{}r}
    \parbox[t]{1in}{\sffamily\large\textbf{PANEL}} & 
    \parbox[t]{3in}{\sffamily\raggedright\large\textbf{Computer Curricula 2013:  Update}} & 
    \parbox[t]{1in}{\sffamily\raggedleft\large\textbf{301AB}} \\
% row 2    
    Chair: & 
    Mehran Sahami, \textit{Stanford University}  \\[1.0em]
% row 3
    Participants: & 
    \multicolumn{2}{@{}l}{\parbox{3.75in}{Steve Roach, \textit{University of Texas at El Paso}; Ernesto Cuadros-Vargas, \textit{San Pablo Catholic University}; David Reed, \textit{Creighton University} }} \\[1em]
% row 4
    \multicolumn{3}{@{}p{5in}}{\small Beginning over 40 years ago with the publication of Curriculum 68, the major professional societies in computing--ACM and IEEE-Computer Society--have sponsored various efforts to establish international curricular guidelines for undergraduate programs in computing.  In the Fall of 2010, work on the next volume in the series, Computer Science 2013 (CS2013), began.  Considerable work on the new volume has already been completed and a first draft of the CS2013 report (known as the Strawman report) will be complete by the beginning of 2012.  This panel seeks to update and engage the SIGCSE community in providing feedback on the Strawman report, which will be available shortly prior to the SIGCSE conference.}
\end{longtable}
\vspace{-\baselineskip}
\begin{longtable}[l]{@{}l@{}l@{}r}
    \parbox[t]{1in}{\sffamily\large\textbf{PANEL}} & 
    \parbox[t]{3in}{\sffamily\raggedright\large\textbf{Scrum Across the CS/SE Curricula}} & 
    \parbox[t]{1in}{\sffamily\raggedleft\large\textbf{305B}} \\
% row 2    
    Chair: & 
    Mark Hoffman, \textit{Quinnipiac University}  \\[1.0em]
% row 3
    Participants: & 
    \multicolumn{2}{@{}l}{\parbox{3.75in}{Charles Wallace, \textit{Michigan Technological University}; Douglas Troy, \textit{Miami University}; Sriram Mohan, \textit{Rose-Hulman Institute of Technology} }} \\[1em]
% row 4
    \multicolumn{3}{@{}p{5in}}{\small Scrum is one of the many agile approaches to software development that have been widely adopted over the past decade. Key agile features of Scrum are a flexible, adaptive schedule; democratic, collaborative teams; and frequent, iterative project and process reviews. Scrum is not only a software development strategy but a general learning strategy. Panel participants will describe how their students learn and practice Scrum in a software development, how they use it to manage student projects in non-software development contexts, and how Scrum provides opportunities to integrate communication skills into the CS and SE curricula. As participants in the CPATH II project, panelists have developed Scrum-based assignments that exercise the skills of reading, writing, speaking and teaming.}
\end{longtable}
\vspace{-\baselineskip}
\begin{longtable}[l]{@{}p{1in}@{}p{3in}@{}r}
    {\sffamily\large\textbf{PANEL}} & 
    {\sffamily\raggedright\large\bfseries Role of Interdisciplinary Computing in Higher Education, Research and \\Industry} & 
    {\sffamily\raggedleft\large\textbf{306C}} \\
% row 2    
    Chair: & 
    \multicolumn{2}{@{}p{4in}}{Ursula Wolz, \textit{Franklin W. Olin College of Engineering}}  \\[1.0em]
% row 3
    Participants: & 
    \multicolumn{2}{@{}l}{\parbox{3.75in}{Lillian (Boots) Cassel, \textit{Villanova University}; Guy-Alain Amoussou, \textit{National Science Foundation} }} \\[1em]
% row 4
    \multicolumn{3}{@{}p{5in}}{\small At SIGCSE 2010 NSF directors held a panel on the potential for Interdisciplinary Computing.  This session is a direct outgrowth. Via an NSF grant individuals drawn from a range of universities and industry met three times in 2011 to discuss the nature of interdisciplinary computing, its importance both for computing and for other disciplines, obstacles to the further emergence of interdisciplinary computing and ways in which these obstacles might be overcome. This session provides an opportunity for the SIGCSE community to  learn about the potential, promise and pitfalls of existing interdisciplinary computing activities. They will be asked to contribute their insights and experiences via structured small-group discussion, and will make connections with others with similar interests.}
\end{longtable}
\newpage
\begin{longtable}{@{}p{0.75in}@{}p{3.25in}@{}r}
   {\sffamily\large\textbf{PAPERS}} &
   {\raggedright\sffamily\large\textbf{Data Structures and Algorithms}} & 
   {\sffamily\large\textbf{302A }} \\
%row 2
   Chair:  & 
   {\raggedright Ivona Bezakova, \textit{Rochester Institute of Technology}} & \\ \\
{\sffamily 10:45}& 
\multicolumn{2}{@{}p{3.75in}}{\sffamily\raggedright\textbf{Sustainability Themed Problem Solving In Data Structures And Algorithms}} \\
& \multicolumn{2}{@{}p{3.75in}}{\raggedright Ali Erkan, Tom Pfaff, Jason Hamilton and Michael Rogers, \textit{Ithaca College}} \\ \\
\multicolumn{3}{@{}p{5in}}{\small During the past two years, we have been creating curricular material centered around complex problems rooted in sustainability. Since multi-disciplinary learning is one of our primary goals, these projects are most meaningful when they connect students from different disciplines working toward a common understanding. However, strong disciplinary components present in their solutions also allow us to frame these projects from strictly disciplinary perspectives. In this paper, we show how they can be used for increased engagement in the context of data structures and algorithms. We review two new ones to explore (i) the structural characteristics of the western part of the U.S. power-grid, and (ii) the effects of over-harvesting on fish stocks.} \\ \\
{\sffamily 11:10}& 
\multicolumn{2}{@{}p{3.75in}}{\sffamily\raggedright\textbf{Metaphors and Analogies for Teaching Algorithms}} \\
& \multicolumn{2}{@{}p{3.75in}}{\raggedright Monika Steinova, \textit{ETH Zurich}; Michal Forisek, \textit{Comenius University}} \\ \\
\multicolumn{3}{@{}p{5in}}{\small In this paper we explore the topic of using metaphors and analogies in teaching algorithms. We argue their importance in the teaching process. We present a selection of metaphors we successfully used when teaching algorithms to secondary school students. We also discuss the suitability of several commonly used metaphors, and in several cases we show significant weaknesses of these metaphors.} \\ \\
{\sffamily 11:35}& 
\multicolumn{2}{@{}p{3.75in}}{\sffamily\raggedright\textbf{Detecting and Understanding Students’ Misconceptions Related to Algorithms and Data Structures}} \\
& \multicolumn{2}{@{}p{3.75in}}{\raggedright Holger Danielsiek, Wolfgang Paul and Jan Vahrenhold, \textit{Technische Universität Dortmund}} \\ \\
\multicolumn{3}{@{}p{5in}}{\small We describe the first results of our work towards a concept inventory for Algorithms and Data structures. Based on expert interviews and the analysis of 400 exams we were able to identify several core topics which are prone to error. In a pilot study, we verified misconceptions known from the literature and identified previously unknown misconceptions related to Algorithms and Data Structures. In addition to this, we report on methodological issues and point out the importance of a two-pronged approach to data collection.} \\ \\
\end{longtable}


\newpage
\begin{longtable}{@{}p{0.75in}@{}p{3.25in}@{}r}
   {\sffamily\large\textbf{PAPERS}} &
   {\raggedright\sffamily\large\textbf{Robots}} & 
   {\sffamily\large\textbf{302B }} \\
%row 2
   Chair:  & 
   {\raggedright Sherri Goings, \textit{Carleton College}} & \\ \\
{\sffamily 10:45}& 
\multicolumn{2}{@{}p{3.75in}}{\sffamily\raggedright\textbf{A C-based Introductory Course Using Robots}} \\
& \multicolumn{2}{@{}p{3.75in}}{\raggedright Henry Walker, David Cowden, April O'Neill, Erik Opavsky and Dilan Ustek, \textit{Grinnell College}} \\ \\
\multicolumn{3}{@{}p{5in}}{\small Using robots in introductory computer science classes has recently become a popular method of increasing student interest in computer science.  This paper describes the development of a new  curriculum for a CS 2 course, Imperative Problem Solving and Data Structures, based upon Scribbler 2 robots with standard C.  The curriculum contains eight distinct modules with a primary topic theme, readings, labs, and project at the end. Each module resulted from collaboration among former CS 2 students and a faculty member, utilizing an iterative process with revisions.  Each lab includes a survey to obtain student feedback that will allow the course to evolve and better fit the needs of future CS 2 students.  All materials discussed here are available online for use by others.} \\ \\
{\sffamily 11:10}& 
\multicolumn{2}{@{}p{3.75in}}{\sffamily\raggedright\textbf{dLife: A Java Library for Multiplatform Robotics, AI and Vision in Undergraduate CS and Research}} \\
& \multicolumn{2}{@{}p{3.75in}}{\raggedright Grant Braught, \textit{Dickinson College}} \\ \\
\multicolumn{3}{@{}p{5in}}{\small dLife is a free and open-source Java library that supports undergraduate education and research involving robotics, artificial intelligence, machine learning and computer vision.  The design of dLife addresses many concerns raised by experience reports in the CS education literature including a shortened code/test/debug cycle, ready access to robot sensor information and close integration with a robotic simulation system. Support is currently provided for a variety of popular educational and research robots. Easily extensible packages for neural networks, genetic algorithms, reinforcement learning and computer vision support both classroom and research applications.} \\ \\
{\sffamily 11:35}& 
\multicolumn{2}{@{}p{3.75in}}{\sffamily\raggedright\textbf{Seven Big Ideas in Robotics, and How To Teach Them}} \\
& \multicolumn{2}{@{}p{3.75in}}{\raggedright David S. Touretzky, \textit{Carnegie Mellon University}} \\ \\
\multicolumn{3}{@{}p{5in}}{\small Following the curriculum design principles of Wiggins and McTighe (Understanding by Design, 2nd Ed., 2005), I present seven big ideas in robotics that can fit together in a one semester undergraduate course.  Each is introduced with an essential question, such as ``How do robots see the world?''  The answers expose students to deep concepts in computer science in a context where they can be immediately put to the test.  Hands-on demonstrations and labs using the Tekkotsu open source software framework and robots costing under \$1,000 facilitate mastery of these important ideas.  Courses based on parts of an early version of this curriculum are being offered at Carnegie Mellon and several other universities.} \\ \\
\end{longtable}


\newpage
\begin{longtable}{@{}p{0.75in}@{}p{3.25in}@{}r}
   {\sffamily\large\textbf{PAPERS}} &
   {\raggedright\sffamily\large\textbf{K-6 Collaborations}} & 
   {\sffamily\large\textbf{306A }} \\
%row 2
   Chair:  & 
   {\raggedright Sheila Casta\~neda, \textit{Clarke University}} & \\ \\
{\sffamily 10:45}& 
\multicolumn{2}{@{}p{3.75in}}{\sffamily\raggedright\textbf{Design and Evaluation of a Braided Teaching Course in Sixth Grade Computer Science Education}} \\
& \multicolumn{2}{@{}p{3.75in}}{\raggedright Arno Pasternak, \textit{Fritz-Steinhoff-Gesamtschule Hagen and Technische Universität Dortmund}; Jan Vahrenhold, \textit{Technische Universität Dortmund}} \\ \\
\multicolumn{3}{@{}p{5in}}{\small We report on the design and evaluation of the first year of a CS course in lower secondary education that implements the concept of braided teaching. Besides being a proof-of-concept, our study demonstrates that students can indeed be taught CS (as opposed to ICT) as early as in sixth grade while at the same time not falling behind with respect to IT Literacy. We present quantitative and qualitative results and argue that Computer Science can be taught just like any other Science worth full curriculum credit.} \\ \\
{\sffamily 11:10}& 
\multicolumn{2}{@{}p{3.75in}}{\sffamily\raggedright\textbf{Parallel Programming in Elementary School}} \\
& \multicolumn{2}{@{}p{3.75in}}{\raggedright Chris Gregg, Luther Tychonievich, Kim Hazelwood and James Cohoon, \textit{University of Virginia}} \\ \\
\multicolumn{3}{@{}p{5in}}{\small Traditional introductory programming classes focus on teaching sequential programming skills using conventional programming languages and single-threaded applications. Students rarely learn about parallel programming until much later in their careers. Today, there is a greater need for programmers who are not only proficient in parallel programming, but who are not burdened by previously learned sequential programming habits, with parallelism tacked on as an afterthought.  We present an introductory parallel programming course we  taught to a group of primary school students using a novel parallel programming language.  We provide examples of student-written code and we describe the overall course goal and specific lesson plans geared towards teaching students how to ``think parallel.''} \\ \\
{\sffamily 11:35}& 
\multicolumn{2}{@{}p{3.75in}}{\sffamily\raggedright\textbf{Building Upon and Enriching Grade Four Mathematics Standards with Programming Curriculum}} \\
& \multicolumn{2}{@{}p{3.75in}}{\raggedright Colleen M. Lewis and Niral Shah, \textit{University of California, Berkeley}} \\ \\
\multicolumn{3}{@{}p{5in}}{\small We found that fifth grade students’ performance on Scratch programming quizzes in a summer enrichment course was highly correlated with their scores on a standardized test for Mathematics. We identify ways in which the programming curriculum builds upon target skills from the California state Mathematics standards to help understand opportunities for building upon and enriching Mathematics content through programming curriculum.} \\ \\
\end{longtable}


\newpage
\begin{longtable}{@{}p{0.75in}@{}p{3.25in}@{}r}
   {\sffamily\large\textbf{PAPERS}} &
   {\raggedright\sffamily\large\textbf{Tools}} & 
   {\sffamily\large\textbf{306B }} \\
%row 2
   Chair:  & 
   {\raggedright Sage Miller, \textit{Webster Central School District}} & \\ \\
{\sffamily 10:45}& 
\multicolumn{2}{@{}p{3.75in}}{\sffamily\raggedright\textbf{Calico: A Multi-Programming-Language, Multi-Context Framework Designed for Computer Science Education}} \\
& \multicolumn{2}{@{}p{3.75in}}{\raggedright James Marshall, \textit{Sarah Lawrence College}; Douglas Blank and Mark Russo, \textit{Bryn Mawr College}; Jennifer S. Kay, \textit{Rowan University}; Keith O'Hara, \textit{Bard College}} \\ \\
\multicolumn{3}{@{}p{5in}}{\small The Calico project is a multi-language, multi-context programming framework and learning environment for computing education. This environment is designed to support several interoperable programming languages (including Python, Scheme, and a visual programming language), a variety of pedagogical contexts (including scientific visualization, robotics, and art), and an assortment of physical devices (including different educational robotics platforms and a variety of physical sensors).  In addition, the environment is designed to support collaboration and modern, interactive learning.  In this paper we describe the Calico project, its design and goals, our prototype system, and its current use.} \\ \\
{\sffamily 11:10}& 
\multicolumn{2}{@{}p{3.75in}}{\sffamily\raggedright\textbf{How a Language-based GUI Generator Can Influence the Teaching of Object-Oriented Programming}} \\
& \multicolumn{2}{@{}p{3.75in}}{\raggedright Prasun Dewan, \textit{University of North Carolina}} \\ \\
\multicolumn{3}{@{}p{5in}}{\small We have built a language-based direct-manipulation user-interface generator that can change, and we argue, improve the lectures and assignments on programming conventions, methods, state, constructors, preconditions, MVC, polymorphism, graphics, structured objects, loops, concurrency, and annotations. Our generator has several novel features for teaching such as interactive instantiation of a class, interactive invocation of methods and constructors that take arbitrary arguments, visualization of objects representing records, sequences, table and graphics, use of preconditions to disable/enable user-interface elements, and automatic generation of model threads.} \\ \\
{\sffamily 11:35}& 
\multicolumn{2}{@{}p{3.75in}}{\sffamily\raggedright\textbf{CodeWave: A Real-Time, Collaborative IDE for Enhanced Learning in Computer Science}} \\
& \multicolumn{2}{@{}p{3.75in}}{\raggedright Jason Vandeventer and Benjamin Barbour, \textit{University of North Carolina Wilmington}} \\ \\
\multicolumn{3}{@{}p{5in}}{\small Computer science instructors often rely on the final version of a program for assessment and feedback. This ignores the process the student used to arrive at the final program. When the instructor has the ability to observe real-time development progress of each student, they are better equipped to provide appropriate and timely feedback. CodeWave, a software program developed at the University of North Carolina Wilmington looks to alleviate these issues.
CodeWave is a real-time, collaborative Integrated Development Environment with traditional features such as syntax highlighting and non-traditional features such as integrated messaging and logged playback.  CodeWave enhances productivity by integrating many common tools students and instructors use during the programming process.} \\ \\
\end{longtable}


% \begin{longtable}[l]{@{}p{1in}@{}p{3in}@{}r}
%     {\sffamily\large\textbf{SupporterSession}} & 
%     {\sffamily\large\textbf{TBA}} & 
%     {\sffamily\large\textbf{302C}} \\
% \end{longtable}    
\newpage
\begin{longtable}[l]{@{}p{1.25in}@{}p{2.75in}@{}r}
    {\sffamily\large\textbf{SUPPORTER SESSION}} & 
    {\sffamily\large\textbf{Intel}} & 
    {\sffamily\large\textbf{302C}} \\
    \multicolumn{3}{@{}p{5in}}{\raggedright\sffamily\large\textbf{Teaching Parallelism: Lightning Rounds}} \\
\end{longtable}    
\begin{longtable}[l]{@{}p{1.25in}@{}p{2.0in}@{}r}
    {\sffamily\large\textbf{SUPPORTER SESSION}} & 
    {\sffamily\large\textbf{Microsoft}} & 
    {\sffamily\large\textbf{305A}} \\ \\
    \multicolumn{3}{@{}p{5in}}{\raggedright\sffamily\large\textbf{Empowering Students: Teaching Software Development with Windows Phone}} \\
    & \multicolumn{2}{@{}p{3.75in}}{Rob Miles, \textit{University of Hull}} \\
    \multicolumn{3}{@{}p{5in}}{With Windows Phone it is really easy to make publishing applications and games part of the learning experience. Students love being able to share their work with friends, family and even future employers. In this session you’ll discover the wealth of Windows Phone based teaching resources available and how they can be used to give students a head start in creating useful applications (including use of Cloud) and entertaining gameplay for the Windows Phone platform, while they are at the same time learning software development techniques.} \\
\end{longtable}    
\cfoot{\colorbox[gray]{0.45}{\color{white}\textsf{Thursday 12:00 - 13:45}}}
\noindent
\framebox[5in][c]{{\Large\sffamily\textbf{Thursday,  12:00 to 13:45}}}
\begin{longtable}[l]{@{}p{1in}@{}p{3in}@{}p{1in}}
%    {\sffamily\large\textbf{}} & 
    \multicolumn{2}{@{}p{4in}}{\sffamily\large\textbf{First Timer's Lunch}} & 
    {\raggedright\sffamily\large\textbf{Marriott State CDEF}} \\ \\
    \multicolumn{3}{@{}p{5in}}{\raggedright\sffamily\textbf{Lunch Speaker: Jane Prey, Winner of 2012 SIGCSE Award for Lifetime Service}} \\
\end{longtable}    
\begin{longtable}[l]{@{}p{1in}@{}p{3in}@{}r}
%    {\sffamily\large\textbf{}} & 
    \multicolumn{2}{@{}p{4in}}{\sffamily\large\textbf{Lunch: On Your Own}} & 
    {\sffamily\large\textbf{}} \\
\end{longtable}    
\newpage
\cfoot{\colorbox[gray]{0.45}{\color{white}\textsf{Thursday 13:45 - 15:00}}}
\noindent
\framebox[5in][c]{{\Large\sffamily\textbf{Thursday,  13:45 to 15:00}}}
\begin{longtable}[l]{@{}l@{}l@{}r}
    \parbox[t]{1in}{\sffamily\large\textbf{PANEL}} & 
    \parbox[t]{3in}{\sffamily\raggedright\large\textbf{A Stratified View of Programming Language Parallelism for Undergraduate CS Education}} & 
    \parbox[t]{1in}{\sffamily\raggedleft\large\textbf{301AB}} \\
% row 2    
    Chair: & 
    Richard Brown, \textit{St. Olaf College}  \\[0.5em]
% row 3
    Participants: & 
    \multicolumn{2}{@{}l}{\parbox{3.75in}{Joel Adams, \textit{Calvin College}; David Bunde, \textit{Knox College}; Jens Mache, \textit{Lewis \& Clark College}; Elizabeth Shoop, \textit{Macalester College} }} \\[2em]
% row 4
    \multicolumn{3}{@{}p{5in}}{\small It is no longer news that undergraduates in computer science need to learn more about parallelism. The range of options for parallel programming is truly staggering, involving hundreds of languages. How can a CS instructor make informed choices among all the options?  This panel provides a guided introduction to parallelism in programming languages and their potential for undergraduate CS education, organized into four progressive categories:  low-level libraries and; higher-level libraries and features; programming languages that incorporate parallelism; and frameworks for productive parallel programming. The four panelists will present representative examples in their categories, then present viewpoints on how those categories relate to coursework, curriculum, and trends in parallelism.}
\end{longtable}
\begin{longtable}[l]{@{}l@{}l@{}r}
    \parbox[t]{1in}{\sffamily\large\textbf{SPECIAL SESSION}} & 
    \parbox[t]{3in}{\sffamily\raggedright\large\textbf{Demystifying Computing with Magic}} & 
    \parbox[t]{1in}{\sffamily\raggedleft\large\textbf{305B}} \\ \\
% row 2    
    Chair: & 
    Daniel Garcia, \textit{UC Berkeley}  \\[0.5em]
% row 3
    Participants: & 
    \multicolumn{2}{@{}l}{\parbox{3.75in}{David Ginat, \textit{Tel-Aviv University} }} \\[2em]
% row 4
    \multicolumn{3}{@{}p{5in}}{\small One man’s ``magic'' is another man’s engineering. -- Robert A. Heinlein.
Many novice students have fuzzy mental models of how the computer works, or worse, sincerely believe that the computer works unpredictably, ``by magic''.  We seek to demystify computing by showing them that even magic itself isn’t necessarily mystical; it could just be clever computation. In this session, we will share a variety of magic tricks whose answer is grounded in computer science: modulo arithmetic, permutations, algorithms, binary encoding, etc. For each trick, we will have an interactive discussion of its underlying computing fundamentals, and tips for successful showmanship. Audience participation will be critical, for helping us perform the magic, discussing the solution, and contributing other magic tricks.}
\end{longtable}
\newpage
\begin{longtable}[l]{@{}l@{}l@{}r}
    \parbox[t]{1in}{\sffamily\large\textbf{PANEL}} & 
    \parbox[t]{3in}{\sffamily\raggedright\large\textbf{Community-Based Projects for Computing Majors:  Opportunities, Challenges and Best Practices}} & 
    \parbox[t]{1in}{\sffamily\raggedleft\large\textbf{306C}} \\ \\
% row 2    
    Chair: & 
    Jeffrey Stone, \textit{Pennsylvania State University}  \\[0.5em]
% row 3
    Participants: & 
    \multicolumn{2}{@{}l}{\parbox{3.75in}{Jeffrey Stone and Elinor Madigan, \textit{Pennsylvania State University}; Janice Pearce, \textit{Berea College}; Bonnie MacKellar, \textit{St. John's University} }} \\[2em]
% row 4
    \multicolumn{3}{@{}p{5in}}{\small The use of community-based projects has been recognized as having pedagogical and experiential value for computing majors. Community-based projects can be valuable learning experiences for computing majors as well as for faculty and community partners. However, these types of projects present challenges for faculty and should be aligned with desired course outcomes. This panel will discuss the use of community-based projects from multiple perspectives. The expectation is that the panel will serve as a forum for participants to share the opportunities, challenges, pedagogical motivations, and best practices obtained from prior experience. Exemplar projects will be highlighted, and audience members will have an opportunity to share their own experiences with community-based projects.}
\end{longtable}
\newpage
\begin{longtable}{@{}p{0.75in}@{}p{3.25in}@{}r}
   {\sffamily\large\textbf{PAPERS}} &
   {\raggedright\sffamily\large\textbf{Games}} & 
   {\sffamily\large\textbf{302A }} \\
%row 2
   Chair:  & 
   {\raggedright Adrienne Decker, \textit{Rochester Institute of Technology}} & \\ \\
{\sffamily 13:45}& 
\multicolumn{2}{@{}p{3.75in}}{\sffamily\raggedright\textbf{The Five Year Evolution of a Game Programming Course}} \\
& \multicolumn{2}{@{}p{3.75in}}{\raggedright Gillian Smith and Anne Sullivan, \textit{UC Santa Cruz}} \\ \\
\multicolumn{3}{@{}p{5in}}{\small This paper presents lessons learned from five years of teaching a game design and programming outreach course. This class is taught over the course of a month to high school students participating in the California Summer School for Mathematics and Science (COSMOS) at the University of California, Santa Cruz. Over these five years we have changed everything in the course, from the overall project structure to the programming language used in the class. In this paper we discuss our successes and failures, and offer recommendations to instructors offering similar courses.} \\ \\
{\sffamily 14:10}& 
\multicolumn{2}{@{}p{3.75in}}{\sffamily\raggedright\textbf{Programming, PWNed: Using Digital Game Development to Enhance Learners’ Competency and Self-Efficacy in a High School Computing Science Course}} \\
& \multicolumn{2}{@{}p{3.75in}}{\raggedright Katie Seaborn, \textit{York University}; Magy Seif El-Nasr, \textit{Northeastern University}; David Milam, \textit{Simon Fraser University}; Darren Yung, \textit{Frank Hurt Secondary School}} \\ \\
\multicolumn{3}{@{}p{5in}}{\small The popularity and inherent engagement of games has caused many educators to start thinking of ways to use game-based techniques to enhance education, particularly to promote STEM (Science Technology, Engineering and Mathematics) concepts to middle and high school students. We report on the design and evaluation of a high school game construction-based curriculum that replaced a traditional computer science class. We collected students’ overall impressions, and evaluated students’ technical competency and self-efficacy at the start and end of the semester. Our findings show that the curriculum had a positive, statistically significant effect on concept comprehension, which suggests that the curriculum was effective for understanding computer science and game design concepts.} \\ \\
{\sffamily 14:35}& 
\multicolumn{2}{@{}p{3.75in}}{\sffamily\raggedright\textbf{A Learning Objective Focused Methodology for the Design and Evaluation of Game-based Tutors}} \\
& \multicolumn{2}{@{}p{3.75in}}{\raggedright Michael Eagle and Tiffany Barnes, \textit{University of North Carolina at Charlotte}} \\ \\
\multicolumn{3}{@{}p{5in}}{\small We present a methodology for the design and evaluation of educational games with a focus on well defined learning objectives	and	empirical	verification.Combining practices from educational design, intelligent tutoring systems, classical test-theory, and game design, this methodology guides researchers through the steps of the design process, including identification of specific learning objectives, translation of learning activities to game mechanics, and the empirical evaluation of the final product. This methodology is particularly useful for young researchers and educators are encouraged to promote this methodology for use in student research experiences or serious games courses.} \\ \\
\end{longtable}


\newpage
\begin{longtable}{@{}p{0.75in}@{}p{3.25in}@{}r}
   {\sffamily\large\textbf{PAPERS}} &
   {\raggedright\sffamily\large\textbf{Professional Experiences}} & 
   {\sffamily\large\textbf{302B }} \\
%row 2
   Chair:  & 
   {\raggedright Sarah Heckman, \textit{North Carolina State University}} & \\ \\
{\sffamily 13:45}& 
\multicolumn{2}{@{}p{3.75in}}{\sffamily\raggedright\textbf{Course Guides: A Model for Bringing  Professionals into the Classroom}} \\
& \multicolumn{2}{@{}p{3.75in}}{\raggedright Thomas Gibbons, \textit{The College of St. Scholastica}} \\ \\
\multicolumn{3}{@{}p{5in}}{\small A new model, professional course guides, describes how practicing professionals can be brought into the classroom as student mentors and integrated into the course material. This new model is compared to existing models for student interactions with practicing professionals including guest speakers, adjunct faculty, and program mentors.} \\ \\
{\sffamily 14:10}& 
\multicolumn{2}{@{}p{3.75in}}{\sffamily\raggedright\textbf{Towards a Better Capstone Experience}} \\
& \multicolumn{2}{@{}p{3.75in}}{\raggedright Sriram Mohan, Stephen Chenoweth and Shawn Bohner, \textit{Rose-Hulman Institute of Technology}} \\ \\
\multicolumn{3}{@{}p{5in}}{\small The capstone experience is designed to bridge the gap from university expectations to those of industry. Yet trying to solve this problem with a single course sequence, even one spanning the senior year, has some shortcomings, in terms of learning outcomes which can be achieved, and also instructional strategies that can be employed. We describe a plan which provides a junior year of practice on a client-based project integrated with learning design and other related topics, followed by a senior year in which students can work more independently to hone these skills on a harder year-long project with another client. This sequence, with scaffolding provided at first that is gradually removed, has proven to be especially effective in preparing undergraduates for a career in the software industry.} \\ \\
{\sffamily 14:35}& 
\multicolumn{2}{@{}p{3.75in}}{\sffamily\raggedright\textbf{An Open Co-op Model for Global Enterprise Technology Education: Integrating the Internship and Course Work}} \\
& \multicolumn{2}{@{}p{3.75in}}{\raggedright Jeffrey Saltz, \textit{JP Morgan Chase}; Jae Oh, \textit{Syracuse University}} \\ \\
\multicolumn{3}{@{}p{5in}}{\small We present an open co-op program called Global Enterprise Technology Immersion Experience (GET IE). The program provides a global enterprise focus integrated with hands-on experiential work-based learning. GET IE includes a two-semester internship that can be seamlessly incorporated within an existing computer science curriculum. 
The internship's  unique pedagogical innovation is to simultaneously provide the students academic course work  that is integrated within a students extended internship and provides relevant problems in  global enterprise technology. The curricula is ``open'' in the sense that other institutions and companies can join the consortium to enrich choices for the students and foster cross-fertilization of curricula activities.} \\ \\
\end{longtable}


\newpage
\begin{longtable}{@{}p{0.75in}@{}p{3.25in}@{}r}
   {\sffamily\large\textbf{PAPERS}} &
   {\raggedright\sffamily\large\textbf{A Session with a View}} & 
   {\sffamily\large\textbf{306A }} \\
%row 2
   Chair:  & 
   {\raggedright Don Goelman, \textit{Villanova University}} & \\ \\
{\sffamily 13:45}& 
\multicolumn{2}{@{}p{3.75in}}{\sffamily\raggedright\textbf{Integrating Video Components in CS1}} \\
& \multicolumn{2}{@{}p{3.75in}}{\raggedright Tamar Vilner, Ela Zur and Ronit Sagi, \textit{The Open University of Israel}} \\ \\
\multicolumn{3}{@{}p{5in}}{\small The Open University of Israel (OUI) is a distance learning university. Our CS1 course is taught through video-taped lectures that cover the study material. In addition, students may participate in face-to-face group meetings in study centers located all over the country and taught by tutors. There is a special group called Ofek, in which the tutor is located in a studio and the lesson is broadcast over the internet. Students enrolled in this group participate from their home PCs. The taped Ofek sessions as well as the lectures are stored on the course website, and students can watch them whenever convenient. We conducted a study to investigate students’ viewing habits and the relationship between viewing and the success rate in the course.} \\ \\
{\sffamily 14:10}& 
\multicolumn{2}{@{}p{3.75in}}{\sffamily\raggedright\textbf{Development and Evaluation of Indexed Captioned Searchable Videos for STEM Coursework}} \\
& \multicolumn{2}{@{}p{3.75in}}{\raggedright Tayfun Tuna, Jaspal Subhlok, Varun Varghese, Olin Johnson and Shishir Shah, \textit{University of Houston - Computer Science Department}; Lecia Barker, \textit{University of Texas-School of Information}} \\ \\
\multicolumn{3}{@{}p{5in}}{\small Videos of classroom lectures have proven to be a popular and versatile learning resource. This paper reports on ICS videos featuring Indexing, Captioning, and Search capability. The goal is to allow a user to rapidly zoom in on a topic of interest, a key shortcoming of the standard video format. A lecture is automatically divided into logical indexed video segments by analyzing video frames. Text is automatically identified with OCR technology and image transformations to drive keyword search. Captions can be added to videos. ICS video player integrates indexing, search, and captioning in video playback and is used by dozens of courses and 1000s of students. The paper reports on development and evaluation of ICS videos framework and assessment of its value as an academic learning resource.} \\ \\
{\sffamily 14:35}& 
\multicolumn{2}{@{}p{3.75in}}{\sffamily\raggedright\textbf{Metaview: A Tool for Learning About Viewing in 3D}} \\
& \multicolumn{2}{@{}p{3.75in}}{\raggedright James Miller, \textit{University of Kansas}} \\ \\
\multicolumn{3}{@{}p{5in}}{\small Metaview is an interactive tool that helps to teach concepts related to nested 3D coordinate systems, especially in the context of defining and establishing views of 3D scenes in common graphics APIs like OpenGL and Direct3D. We describe the context in which nested coordinate systems arise in the study of graphics programming, then we relate the common conceptual difficulties students typically experience when studying and trying to put this material into practice. We then describe the role that metaview plays in helping to overcome these problems. Metaview is packaged with a set of built-in 3D models used to demonstrate major concepts. In addition, external and/or student-programmed models are easily imported into the tool. Metaview can be run anywhere, anytime using Java Web Start.} \\ \\
\end{longtable}


\newpage
\begin{longtable}{@{}p{0.75in}@{}p{3.25in}@{}r}
   {\sffamily\large\textbf{PAPERS}} &
   {\raggedright\sffamily\large\textbf{Pedagogy:  Programming}} & 
   {\sffamily\large\textbf{306B }} \\
%row 2
   Chair:  & 
   {\raggedright Saquib Razak, \textit{Carnegie Mellon University}} & \\ \\
{\sffamily 13:45}& 
\multicolumn{2}{@{}p{3.75in}}{\sffamily\raggedright\textbf{Mediated Transfer: Alice 3 to Java}} \\
& \multicolumn{2}{@{}p{3.75in}}{\raggedright Wanda Dann, Dennis Cosgrove, Don Slater and Dave Culyba, \textit{Carnegie Mellon University}; Steve Cooper, \textit{Stanford University}} \\ \\
\multicolumn{3}{@{}p{5in}}{\small In this paper, we describe a pedagogy for an undergraduate programming course using Alice 3 and Java. We applied the educational theory of mediated transfer to develop a new version of the Alice system and accompanying instructional materials. The pedagogy was implemented and tested over two years. Student test scores in experimental, treatment course sections showed a dramatic increase in scores over comparable, non-treatment sections.} \\ \\
{\sffamily 14:10}& 
\multicolumn{2}{@{}p{3.75in}}{\sffamily\raggedright\textbf{Over-Confidence and Confusion in Using Bloom for Programming Fundamentals Assessment}} \\
& \multicolumn{2}{@{}p{3.75in}}{\raggedright Richard Gluga, Judy Kay, Sabina Kleitman and Tim Lever, \textit{University of Sydney}; Raymond Lister, \textit{University of Technology Sydney}} \\ \\
\multicolumn{3}{@{}p{5in}}{\small A computer science student is required to progress from a novice to an expert through the CS1/2 programming fundamentals sequence. The key contribution is a web-based interactive tutorial that enables computer science educators to practice applying the Bloom Taxonomy in classifying programming exam questions, to define this learning progression. The results of an evaluation with ten participants were analyzed for consistency and accuracy in the application of Bloom. Confidence and self-explanation measures were used to identify problem areas in the application of Bloom to programming fundamentals. The tutorial and findings are valuable contributions to future ACM/IEEE CS curriculum revisions, which are expected to have a continued emphasis on Bloom.} \\ \\
{\sffamily 14:35}& 
\multicolumn{2}{@{}p{3.75in}}{\sffamily\raggedright\textbf{Modeling How Students Learn to Program}} \\
& \multicolumn{2}{@{}p{3.75in}}{\raggedright Stephen Cooper, Chris Piech, Mehran Sahami, Daphne Koller and Paulo Blikstein, \textit{Stanford University}} \\ \\
\multicolumn{3}{@{}p{5in}}{\small Despite the potential wealth of educational indicators expressed in a student’s approach to homework assignments, how students arrive at their final solution is largely overlooked in university courses. In this paper we present a methodology which uses machine learning techniques to autonomously create a graphical model of how students in an introductory programming course progress through a homework assignment. We subsequently show that this model is predictive of which students will struggle with material presented later in the class.} \\ \\
\end{longtable}

\newpage
\begin{longtable}[l]{@{}p{1in}@{}p{3in}@{}r}
    {\sffamily\large\textbf{Supporter Session}} & 
    {\sffamily\large\textbf{Microsoft}} & 
    {\sffamily\large\textbf{305A}} \\ \\
    \multicolumn{3}{@{}p{5in}}{\raggedright\sffamily\large\textbf{Creative Uses for Kinect in Teaching – with Curriculum Materials}} \\
    & Rob Miles, \textit{University of Hull} \\
    \multicolumn{3}{@{}p{5in}}{\small The Kinect sensor is the ``Fastest Selling Consumer Electronics Gadget in History''. It is a great way to add a new dimension to Xbox 360 gameplay, able to read its environment and track the body movement of players. It is also a great teaching tool and a genuinely creative device. In this session Rob Miles will show how you can harness this creativity and get students enjoying themselves while writing programs that make use of the unique abilities of this amazing sensor and its accompanying Kinect for Windows software. He will also have curriculum materials to share with you that you can use freely in your own classes.} \\
\end{longtable}    
\vspace{0.5em}
\vspace{0.5em}
\cfoot{\colorbox[gray]{0.45}{\color{white}\textsf{Thursday 13:45 - 17:15}}}
\noindent
\vspace{0.5\baselineskip}
\framebox[5in][c]{{\Large\sffamily\textbf{Thursday,  13:45 to 17:15}}}
\begin{tabular*}{5in}[l]{@{}p{3.9in}@{}r@{}}
    {\sffamily\large\textbf{Student Research Poster Session}} & 
    {\raggedright\sffamily\large\textbf{Exhibit Hall A}} 
\end{tabular*}    

\cfoot{\colorbox[gray]{0.45}{\color{white}\textsf{Thursday 15:00 - 15:45}}}

\vspace{2.0em}
\noindent
\vspace{0.5\baselineskip}
\framebox[5in][c]{{\Large\sffamily\textbf{Thursday,  15:00 to 15:45}}}
\begin{tabular*}{5in}[l]{@{}p{3.9in}@{}r@{}}
    {\sffamily\large\textbf{Break and Exhibits}} & 
    {\raggedright\sffamily\large\textbf{Exhibit Hall A}} 
\end{tabular*}    

\vspace{2.0em}
\cfoot{\colorbox[gray]{0.45}{\color{white}\textsf{Thursday 15:00 - 16:30}}}
\noindent
\vspace{0.5\baselineskip}
\framebox[5in][c]{{\Large\sffamily\textbf{Thursday,  15:00 to 16:30}}}
\begin{tabular*}{5in}[l]{@{}p{3.9in}@{}r}
    {\sffamily\large\textbf{NSF Showcase \#2}} & 
    {\raggedright\sffamily\large\textbf{Exhibit Hall A}} 
\end{tabular*}    
\begin{itemize}   
  \setlength{\itemsep}{2pt}
     \item {{\sffamily\textbf{Engaging African Americans in Computing through the Collaborative Creation of Musical Remixes,}} Brian Magerko and Jason Freeman } 
     \item {{\sffamily\textbf{Building a K-12 Computing Pipeline in Alabama to Address Participation Diversity,}} Jeff Gray, Mike Wyss, Shelia Cotten, and Shaundra Daily } 
     \item {{\sffamily\textbf{Computational Thinking in IT - A Scenario-based Approach,}} Deborah \\ Boisvert, Paula Velluto, Irene Bruno, and Charles Winer }
     \item {{\sffamily\textbf{Process-Oriented Guided Inquiry Learning in Computer Science,}}   Clifton Kussmaul, R. Libby, and Carl Salter }
\end{itemize}     
\newpage
\cfoot{\colorbox[gray]{0.45}{\color{white}\textsf{Thursday 15:45 - 17:00}}}
\noindent
\framebox[5in][c]{{\Large\sffamily\textbf{Thursday,  15:45 to 17:00}}}
\begin{longtable}[l]{@{}p{1in}@{}p{3in}@{}r}
    {\sffamily\large\textbf{PANEL}} & 
    {\sffamily\raggedright\large\textbf{Science Fiction in Computer Science Education}} & 
    \parbox[t]{1in}{\sffamily\raggedleft\large\textbf{301AB}} \\
% row 2    
    Chair: & 
    \multicolumn{2}{@{}l}{Rebecca Bates, \textit{Minnesota State University, Mankato}}  \\[1.0em]
% row 3
    Participants: & 
    \multicolumn{2}{@{}l}{\parbox{3.75in}{Judy Goldsmith, \textit{University of Kentucky}; Rosalyn Berne, \textit{University of Virginia}; Valerie Summet, \textit{Emory University}; Nanette Veilleux, \textit{Simmons College} }} \\[1em]
% row 4
    \multicolumn{3}{@{}p{5in}}{\small The use of science fiction to engage students in computer science learning is becoming more popular, with ample material available to help students make connections between technical content and human experience, from Star Trek to The Hitchhiker’s Guide to the Galaxy to 2001 to I, Robot to …  Fiction can be included in technical courses or used to draw students into the field in introductory classes. The panelists, who represent a range of schools, perspectives and classes, will present brief overviews of how they have used science fiction to engage students in technical topics as well as related ethical and societal issues. After the overviews, there will be plenty of time for discussion of examples and ways to make connections between science fiction and particular classes or topics.}
\end{longtable}
\begin{longtable}[l]{@{}l@{}l@{}r}
    \parbox[t]{1in}{\sffamily\large\textbf{PANEL}} & 
    \parbox[t]{3in}{\sffamily\raggedright\large\textbf{Diversity Initiatives to Support Systemic Change in Undergraduate Computing}} & 
    \parbox[t]{1in}{\sffamily\raggedleft\large\textbf{305B}} \\
% row 2    
    Chair: & 
    Leisa D. Thompson, \textit{University of Virginia}  \\[0.2em]
% row 3
    Participants: & 
    \multicolumn{2}{@{}l}{\parbox{3.75in}{Lecia Barker, \textit{University of Texas}; Rita Manco Powell, \textit{University of Pennsylvania}; Catherine Brawner, \textit{Research Triangle Educational Consultants}; Tom McKlin, \textit{The Findings Group, LLC} }} \\[1em]
% row 4
    \multicolumn{3}{@{}p{5in}}{\small The National Center for Women \& Information Technology (NCWIT) Extension Services for Undergraduate Programs (ES-UP) has created a large group of trained consultants (ESCs) and clients who are passionate about women’s participation in computing. This panel will describe how our ESCs and clients have worked together to effect change and will show outcomes from our activities over the past three years.}
\end{longtable}
\begin{longtable}[l]{@{}p{1in}@{}p{3in}@{}r}
    {\sffamily\large\textbf{SPECIAL SESSION}} & 
    {\sffamily\raggedright\large\textbf{Transforming the CS Classroom with Studio-Based Learning}} & 
    \parbox[t]{1in}{\sffamily\raggedleft\large\textbf{306C}} \\
% row 2    
    Chair: & 
    \multicolumn{2}{@{}l}{Christopher Hundhausen, \textit{Washington State University}}  \\[0.2em]
% row 3
    Participants: & 
    \multicolumn{2}{@{}l}{\parbox{3.75in}{N. Hari Narayanan and Dean Hendrix, \textit{Auburn University}}} \\[1em]
% row 4
    \multicolumn{3}{@{}p{5in}}{\small The studio-based learning (SBL) model has been the centerpiece of architecture and fine arts education for over a century. Over the past five years, we have been adapting SBL for computing education and empirically evaluating its impact. This effort now involves 26 computing courses at 15 institutions in seven states. To our knowledge, this is the largest implementation and evaluation of a pedagogy for computing education to date. This special session will introduce SBL to a general audience and facilitate a discussion and exchange of ideas. In addition to oral and poster presentations of the SBL model and its evaluation results, the session will feature ``war stories'' from teachers who have used SBL in their courses, and hands-on activities to help attendees apply SBL to their courses.}
\end{longtable}
\newpage
\begin{longtable}{@{}p{0.75in}@{}p{3.25in}@{}r}
   {\sffamily\large\textbf{PAPERS}} &
   {\raggedright\sffamily\large\textbf{Broadening Participation}} & 
   {\sffamily\large\textbf{302A }} \\
%row 2
   Chair:  & 
   {\raggedright Kristine Nagel, \textit{Georgia Gwinnett College}} & \\ \\
{\sffamily 15:45}& 
\multicolumn{2}{@{}p{3.75in}}{\sffamily\raggedright\textbf{Making Turing Machines Accessible to Blind Students}} \\
& \multicolumn{2}{@{}p{3.75in}}{\raggedright Pierluigi Crescenzi, Leonardo Rossi and Gianluca Apollaro, \textit{University of Florence}} \\ \\
\multicolumn{3}{@{}p{5in}}{\small In this paper we describe how we tried to make the JFLAP Turing machine simulator accessible to blind students. Software accessibility is an important topic for both legal and ethical reasons: in our case, however, we also wanted to make the accessible software usable by blind students in cooperation with the other students, in order to encourage the integration of the blind students within the rest of the class. For this reason, the accessible version of the JFLAP Turing machine simulator that we developed is as much similar as possible to and fully compatible with the original one. In the paper, we also report some very satisfactory preliminary validation results that indicate how the new software can really make Turing machines accessible to blind students.} \\ \\
{\sffamily 16:10}& 
\multicolumn{2}{@{}p{3.75in}}{\sffamily\raggedright\textbf{Toward an Emergent Theory of Broadening Participation in Computer Science Education}} \\
& \multicolumn{2}{@{}p{3.75in}}{\raggedright David Webb, Alexander Repenning and Kyu Han Koh, \textit{University of Colorado at Boulder}} \\ \\
\multicolumn{3}{@{}p{5in}}{\small A fundamental challenge to computer science education is the difficulty of broadening participation of women and underserved communities. The idea of game design and game programming as an activity to introduce children at an early age to computational thinking in a motivational way is quickly gaining momentum. A pedagogical approach called Project First has allowed the Scalable Game Design project to reach over 4,000 middle schools students including a large percentage of female (45\%) and underrepresented (48\%) students. Our analysis of student motivation data, gender ratios and pedagogical approaches employed by teachers such as mediation and scaffolding suggests strong gender effects based on gender ratios and classroom scaffolding approaches.} \\ \\
{\sffamily 16:35}& 
\multicolumn{2}{@{}p{3.75in}}{\sffamily\raggedright\textbf{Exploring Formal Learning Groups and their Impact on Recruitment of Women in Undergraduate CS}} \\
& \multicolumn{2}{@{}p{3.75in}}{\raggedright Julie Krause, Irene Polycarpou and Keith Hellman, \textit{Colorado School of Mines}} \\ \\
\multicolumn{3}{@{}p{5in}}{\small As percentages of women in computing jobs and university programs decline, recruiting and retaining women in the field of Computer Science (CS) becomes increasingly important. Undergraduate CS programs, and more specifically, introductory-level CS courses, offer an opportunity to introduce women to CS studies. Furthermore, learning experiences in introductory CS courses can be pivotal in shaping female students’ perceptions of CS. Collaborative learning, in various forms, is an instructional construct that has been shown to be effective in recruiting and retaining women in undergraduate CS programs. In this paper we present an exploratory study on formal learning groups and their potential to attract and maintain students’ interest in CS studies.} \\ \\
\end{longtable}


\newpage
\begin{longtable}{@{}p{0.75in}@{}p{3.25in}@{}r}
   {\sffamily\large\textbf{PAPERS}} &
   {\raggedright\sffamily\large\textbf{Online Collaboration}} & 
   {\sffamily\large\textbf{302B }} \\
%row 2
   Chair:  & 
   {\raggedright Charles Leska, \textit{Randolph-Macon College}} & \\ \\
{\sffamily 15:45}& 
\multicolumn{2}{@{}p{3.75in}}{\sffamily\raggedright\textbf{Perspectives on Active Learning and Collaboration: JavaWIDE in the Classroom}} \\
& \multicolumn{2}{@{}p{3.75in}}{\raggedright Jam Jenkins, \textit{Valdosta State University}; Evelyn Brannock and Sonal Dekhane, \textit{Georgia Gwinnett College}; Thomas Cooper, \textit{The Walker School}; Mark Hall, \textit{University of Wisconsin -- Marathon County}; Michael Nguyen, \textit{Emory University}} \\ \\
\multicolumn{3}{@{}p{5in}}{\small JavaWIDE is an innovative environment that promotes active learning and collaboration in programming courses. This paper discusses where and how JavaWIDE has been used to promote active and collaborative learning in both traditional and synchronous distance education courses in four different environments: high school, summer enrichment, and at two- and four-year colleges. After discussing the educational atmosphere and how active learning and collaboration are used in the courses, student responses to the experience are summarized. This collection of case studies illustrates how the concurrent editing, shared environment awareness and other features of JavaWIDE can be used to promote active learning and collaboration within a heterogeneous set of teaching and learning environments.} \\ \\
{\sffamily 16:10}& 
\multicolumn{2}{@{}p{3.75in}}{\sffamily\raggedright\textbf{How Well Do Online Forums Facilitate Discussion and Collaboration Among Novice Animation Programmers?}} \\
& \multicolumn{2}{@{}p{3.75in}}{\raggedright Christopher Scaffidi, Aniket Dahotre and Yan Zhang, \textit{Oregon State University}} \\ \\
\multicolumn{3}{@{}p{5in}}{\small Animation programming is a widely-respected approach for helping students to learn programming skills, and online forums are a widely-used approach for helping students to interact with one another. But in what ways, if any, does combining animation programming with online forums lead to useful discussion and collaboration among learners? To answer this question, we analyzed online forum discussions among people who were learning to create animation programs using the Scratch programming environment. We discovered that specific kinds of online posts were more likely than others to be followed by discussion, and we found that the ensuing collaboration often involved the exchange of design ideas and feedback within small groups of users.} \\ \\
{\sffamily 16:35}& 
\multicolumn{2}{@{}p{3.75in}}{\sffamily\raggedright\textbf{Classroom Salon: A Tool for Social Collaboration}} \\
& \multicolumn{2}{@{}p{3.75in}}{\raggedright John Barr, \textit{Ithaca College}; Ananda Gunawardena, \textit{Carnegie Mellon University}} \\ \\
\multicolumn{3}{@{}p{5in}}{\small Classroom Salon is an on-line social collaboration tool that allows instructors to create, manage, and analyze social networks (called Salons) to enhance student learning. Students in a Salon can cooperatively create, comment on, and modify documents. Classroom Salon provides tools that allow the instructor to monitor the social networks and gauge both student participation and individual effectiveness. This paper describes Classroom Salon, provides some use cases that we have developed for introductory computer science classes and presents some preliminary observations of using this tool in several computer science courses at Carnegie Mellon University.} \\ \\
\end{longtable}


\newpage
\begin{longtable}{@{}p{0.75in}@{}p{3.25in}@{}r}
   {\sffamily\large\textbf{PAPERS}} &
   {\raggedright\sffamily\large\textbf{Middle School Collaborations}} & 
   {\sffamily\large\textbf{306A }} \\
%row 2
   Chair:  & 
   {\raggedright Catherine Lang, \textit{Swinburne University of Technology}} & \\ \\
{\sffamily 15:45}& 
\multicolumn{2}{@{}p{3.75in}}{\sffamily\raggedright\textbf{Bringing The Breadth of Computer Science to Middle Schools}} \\
& \multicolumn{2}{@{}p{3.75in}}{\raggedright Elizabeth Carter and Glenn Blank, \textit{Lehigh University}; Jennifer Walz, \textit{Harrison Morton Middle School}} \\ \\
\multicolumn{3}{@{}p{5in}}{\small In order to garner more student interest in the pursuit of computer science as both a major and a career path, K-12 students need to be made aware of what computer science is and what it is about earlier in their education.  Although students in many high schools can pursue introductory programming, high school is arguably too late to interest students who may have developed ill-informed attitudes about computer science.  This paper documents curricular items developed for and taught to an audience of mixed ability 6th through 8th graders taking a local Technology Education class that attempts to showcase some of the more interesting, less stereotypical, aspects of computer science using a breadth approach in an effort to encourage interest in the field.} \\ \\
{\sffamily 16:10}& 
\multicolumn{2}{@{}p{3.75in}}{\sffamily\raggedright\textbf{Integrating Hard and Soft Skills: Software Engineers Serving Middle School Teachers}} \\
& \multicolumn{2}{@{}p{3.75in}}{\raggedright Richard Burns, Lori Pollock and Terry Harvey, \textit{University of Delaware}} \\ \\
\multicolumn{3}{@{}p{5in}}{\small We have developed and implemented, in four instances, a model for engaging computer science majors in integrating computing into teaching at a K-8 school in an underserved community.  This paper describes the design of the service learning course particularly focused on interweaving  software engineering practice, service learning, and development of soft skills. 
CS student teams partner with middle school teacher teams to create learning games, 
and conduct classroom instruction and observation.  We report on our results from evaluating the impact of the course experience on the CS students and middle school teachers through pre-post surveys, evaluator observation of student demo presentations and classroom instruction, focus groups, and student reflective journals.} \\ \\
{\sffamily 16:35}& 
\multicolumn{2}{@{}p{3.75in}}{\sffamily\raggedright\textbf{The Fairy Performance Assessment: Measuring Computational Thinking in Middle School}} \\
& \multicolumn{2}{@{}p{3.75in}}{\raggedright Linda Werner, \textit{University of California, Santa Cruz}; Jill Denner and Shannon Campe, \textit{ETR Associates}; Damon Chizuru Kawamoto, \textit{Brown University}} \\ \\
\multicolumn{3}{@{}p{5in}}{\small Computational thinking (CT) has been described as an essential capacity to prepare students for computer science, as well as to be productive members of society. But efforts to engage K-12 students in CT are hampered by a lack of definition and assessment tools. In this paper, we describe the first results of a newly created performance assessment tool for measuring CT in middle school. We briefly describe the context for the performance assessment (game-programming courses), the aspects of CT that are measured, the results, and the factors that are associated with performance. We see the development of assessment tools as a critical step in efforts to bring CT to K-12, and to strengthen the use of game programming in middle school. We discuss problems and implications of our results.} \\ \\
\end{longtable}


\newpage
\begin{longtable}{@{}p{0.75in}@{}p{3.25in}@{}r}
   {\sffamily\large\textbf{PAPERS}} &
   {\raggedright\sffamily\large\textbf{New Tricks for the Classroom}} & 
   {\sffamily\large\textbf{306B }} \\
%row 2
   Chair:  & 
   {\raggedright Julian Mason, \textit{Duke University}} & \\ \\
{\sffamily 15:45}& 
\multicolumn{2}{@{}p{3.75in}}{\sffamily\raggedright\textbf{Running Students’ Software Tests Against Each Others’ Code: New Life for an Old “Gimmick”}} \\
& \multicolumn{2}{@{}p{3.75in}}{\raggedright Stephen Edwards, Zalia Shams, Michael Cogswell and Robert Senkbeil, \textit{Virginia Tech, Department of Computer Science}} \\ \\
\multicolumn{3}{@{}p{5in}}{\small At SIGCSE'02, Goldwasser suggested including testing in assignments and then running every student’s tests against every other student’s program.  This provides more insight into the quality of a student's tests as well as her solution.  Software testing is more common now, but the all-pairs model of executing tests is still rare.  This is because student-written tests, such as JUnit tests, take the form of program code and may depend on any aspect of their author’s own solution, and these dependencies can keep tests from compiling against other programs.  We discusse this problem and present a Java solution using bytecode rewriting and reflection.  Results of applying this technique to two assignments involving 147 students and 240,158 individual test case runs demonstrates feasibility.} \\ \\
{\sffamily 16:10}& 
\multicolumn{2}{@{}p{3.75in}}{\sffamily\raggedright\textbf{Group Note-Taking in a Large Lecture Class: Design, Implementation, and Evaluation of a Low-Cost Universal Design Practice}} \\
& \multicolumn{2}{@{}p{3.75in}}{\raggedright Christopher Plaue, Sal LaMarca and Shelby H. Funk, \textit{The University of Georgia}} \\ \\
\multicolumn{3}{@{}p{5in}}{\small We created a group note-taking system in our large intro computer science course to increase interaction amongst students, promote good note-taking strategies, and provide study resources to all students while also fulfilling the role of accommodating for students with learning disabilities. We show that the section of the course taught with our intervention performed significantly better on their final examination compared to a course taught without the intervention. We report that students enjoyed increased interactions with their peers, and that a third of the class self-reported an increase in their note-taking skills. Furthermore, our intervention only required minimal cost to the institution, and no financial cost to students, and is easily implemented in any size class.} \\ \\
{\sffamily 16:35}& 
\multicolumn{2}{@{}p{3.75in}}{\sffamily\raggedright\textbf{Following a Thread: Knitting Patterns and Program Tracing}} \\
& \multicolumn{2}{@{}p{3.75in}}{\raggedright Michelle Craig, \textit{University of Toronto}; Sarah Petersen; Andrew Petersen, \textit{University of Toronto Mississauga}} \\ \\
\multicolumn{3}{@{}p{5in}}{\small This paper presents observations about teaching program tracing to novices drawn from a study of knitting patterns. Knitting patterns have evolved from vague, chatty discourse written for experts to precise, line-by-line procedures akin to programs. The knitting community has developed conventions for articulating iteration, conditions, and design decisions. ``Executing'' one of these patterns is analogous to tracing, so we argue that conventions adopted by knitters to make their patterns comprehensible to non-experts provide insights about teaching tracing to novices. Our observations suggest that using ``until'' instead of ``while'' and partially unrolling loops may help beginners understand code and that some structures, like parameters, may be unfamiliar.} \\ \\
\end{longtable}


% \begin{longtable}[l]{@{}p{1in}@{}p{3in}@{}r}
%     {\sffamily\large\textbf{Supporter Session}} & 
%     {\sffamily\large\textbf{TBA}} & 
%     {\sffamily\large\textbf{302C}} \\
% \end{longtable}    
\begin{longtable}[l]{@{}p{1.25in}@{}p{2.5in}@{}r}
    {\sffamily\large\textbf{SUPPORTER SESSION}} & 
    {\sffamily\large\textbf{Google}} & 
    {\sffamily\large\textbf{305A}} \\ \\
    \multicolumn{3}{@{}p{5in}}{\raggedright\sffamily\large\textbf{All Things Google and Education}} \\
    & Margaret Johnson, \textit{Director of University Relations and Education, Google Inc.} \\
    \multicolumn{3}{@{}p{5in}}{\small Google believes that all students should have the opportunity to become active creators of tomorrow's technology.  Through our diverse set of education efforts, we invest in the next generation of computer scientists and engineers, providing opportunities for all students to engage more directly in technology.  Google’s mission is to organize the world's information and make it universally accessible and useful. With regard to education, our goal is to leverage Google's strengths and infrastructure to increase access to high-quality, open educational content and technology. 

During this session, you will learn about all of Google's education initiatives with a focus on those related to Computer Science.
}
\end{longtable}    
\cfoot{\colorbox[gray]{0.45}{\color{white}\textsf{Thursday 17:10 - 18:00}}}
\addcontentsline{toc}{subsubsection}{Birds of a Feather Flock I}
\vspace{2em}
\noindent
\framebox[5in][c]{{\Large\sffamily\textbf{Thursday,  17:10 to 18:00}}}
\begin{longtable}[l]{@{}l@{}l@{}r}
    \parbox[t]{1in}{\sffamily\large\textbf{BOF}} & 
    \parbox[t]{3in}{\sffamily\raggedright\large\textbf{CS Unplugged, Outreach and CS Kinesthetic Activities}} & 
    \parbox[t]{1in}{\sffamily\raggedleft\large\textbf{201}} \\ \\
% row 2    
    % Chair: & 
    %   \\[0.5em]
% row 3
     & 
    \multicolumn{2}{@{}l}{\parbox{3.75in}{Lynn Lambert, \textit{Christopher Newport University;} Tim Bell, \emph{University of Canterbury;} Daniela Marghitu, \emph{Auburn} }} \\[2em]
% row 4
    \multicolumn{3}{@{}p{5in}}{\small Outreach activities including Computer Science Unplugged demonstrate computer science concepts at schools and public venues based around kinesthetic activities rather than hands-on computer use. Computer Science Unplugged is a global project with many such activities for children to adults using no technology, including how binary numbers represent words, images and sound, routing and deadlock, public/private key encryption, and others. Effective outreach programs such as this combats the idea that computer science = programming or, worse, keyboarding; and can educate the public, interest students, and recruit majors.  Many people have used these activities, and adapted them for their own culture or outreach purposes. Come share your outreach ideas and experiences with such activities.}
\end{longtable}
\newpage
\fancyhead[RO,LE]{\colorbox[gray]{0.45}{\color{white}\textsf{BoFs}}}
\begin{longtable}[l]{@{}l@{}p{3in}@{}r}
    \parbox[t]{1in}{\sffamily\large\textbf{BOF}} & 
    {\raggedright\sffamily\large\bfseries Infusing Software Assurance and \\ Secure Coding into Introductory CS courses} & 
    \parbox[t]{1in}{\sffamily\raggedleft\large\textbf{205}} \\
% row 2    
% row 3
     &
    \multicolumn{2}{@{}l}{\parbox{3.75in}{\raggedright Elizabeth Hawthorne, \textit{Union County College;} Carol Sledge and Nancy Mead, \emph{Carnegie Mellon University;} Mark Ardis, \emph{Stevens Institute of Technology} }} \\[2em]
% row 4
    \multicolumn{3}{@{}p{5in}}{\small Nearly every facet of modern society depends heavily on highly complex software systems. The business, energy, transportation, education, communication, government, and defense communities rely on software to function, and software is an intrinsic part of our personal lives. Software assurance is an important discipline to ensure that software systems and services function dependably and are secure. So, where are the resources to assist computer science educators with this instructional material? Session leaders will share materials from the Software Assurance Curriculum Project at the Software Engineering Institute of Carnegie Mellon University, and will facilitate discussion centered on infusing software assurance into introductory computer science courses at different types of colleges.}
\end{longtable}
\begin{longtable}[l]{@{}l@{}p{3in}@{}r}
    \parbox[t]{1in}{\sffamily\large\textbf{BOF}} & 
    {\sffamily\raggedright\large\textbf{Web-CAT User Group}} & 
    \parbox[t]{1in}{\sffamily\raggedleft\large\textbf{206}} \\
% row 2    
     & 
    Stephen Edwards, \textit{Virginia Tech, Department of Computer Science}  \\[0.5em]
% row 3
% row 4
    \multicolumn{3}{@{}p{5in}}{\small Web-CAT is the most widely used open-source automated grading system, with about 10,000 users at over 65 institutions worldwide.  Its plug-in architecture supports extensibility, with plug-ins for Java (including Objectdraw, JTF, Swing, and Android), C++, Python, Haskell, and more.  It is also a powerful tool for educational research data collection. It supports a wide variety of assessment strategies, but is famous for “grading students on how well they test their own code.”  Web-CAT won the 2006 Premier Award, recognizing high-quality, non-commercial courseware for engineering education. This BOF will allow existing users and new adopters to meet, share experiences, and talk about what works and what doesn’t.  Information on getting started quickly with Web-CAT will also be provided.}
\end{longtable}
\begin{longtable}[l]{@{}l@{}p{3in}@{}r}
    \parbox[t]{1in}{\sffamily\large\textbf{BOF}} & 
    {\sffamily\raggedright\large\textbf{Teaching Open Source}} & 
    \parbox[t]{1in}{\sffamily\raggedleft\large\textbf{301AB}} \\
% row 2    
% row 3
     & 
    \multicolumn{2}{@{}l}{\parbox{3.75in}{\raggedright Sebastian Dziallas, \textit{Franklin W. Olin College of Engineering;} Heidi Ellis, \emph{Western New England University;} Mel Chua, \emph{Purdue University;} Steven Huss-Lederman, \emph{Beloit College} and Karl Wurst, \emph{Worcester State University} }} \\[2em]
% row 4
    \multicolumn{3}{@{}p{5in}}{\small Involving students from a wide range of backgrounds in Free and Open Source Software project communities gets them a hands-on, portfolio-building experience in the creation of a real-world project while simultaneously building their institution's public profile. The Teaching Open Source (http://teachingopensource.org) community is an emergent (3 year old) group working on scaffolding to bridge the cultural differences between academic and FOSS communities of practice. Join us to share questions, challenges, and triumphs of incorporating FOSS participation into existing and new curricula as well support resources for doing so. Alumni and current members of the POSSE (Professors' Open Source Summer Experience http://communityleadershipteam.org/posse) will attend in mentorship roles.}
\end{longtable}
\begin{longtable}[l]{@{}l@{}l@{}r}
    \parbox[t]{1in}{\sffamily\large\textbf{BOF}} & 
    \parbox[t]{3in}{\sffamily\raggedright\large\textbf{AP CS Principles and the `Beauty and Joy of Computing' Curriculum}} & 
    \parbox[t]{1in}{\sffamily\raggedleft\large\textbf{302A}} \\
% row 2    
% row 3
     & 
    \multicolumn{2}{@{}l}{\parbox{3.75in}{\raggedright Brian Harvey and Luke Segars, \textit{University of California, Berkeley}  Tiffany Barnes, \emph{University of North Carolina, Charlotte;} }} \\[2em]
% row 4
    \multicolumn{3}{@{}p{5in}}{\small The College Board's guidelines for the coming AP CS Principles course are broad enough to allow many different interpretations.  In particular, different courses have different levels of technical depth.  The ``Beauty and Joy of Computing'' curriculum, used by two of the initial five pilot sites, aims high, with recursion and higher order functions included in the programming half of the course.  This session is for high school or college level instructors considering teaching an AP CS Principles course and interested in using the BJC curriculum, and/or the Snap! (formerly BYOB) visual programming language used in the curriculum.  See http://bjc.berkeley.edu for the curriculum and http://snap.berkeley.edu for the language.}
\end{longtable}
\begin{longtable}[l]{@{}l@{}l@{}r}
    \parbox[t]{1in}{\sffamily\large\textbf{BOF}} & 
    \parbox[t]{3in}{\sffamily\raggedright\large\textbf{Teaching Track Faculty in CS}} & 
    \parbox[t]{1in}{\sffamily\raggedleft\large\textbf{302B}} \\
% row 2    
% row 3
     & 
    \multicolumn{2}{@{}l}{\parbox{3.75in}{Daniel Garcia, \textit{UC Berkeley;}  Jody Paul, \emph{Metropolitan State College at Denver;} Mark S. Sherriff, \emph{University of Virginia} }} \\[2em]
% row 4
    \multicolumn{3}{@{}p{5in}}{\small Many computer science departments have chosen to hire faculty to teach in a teaching-track position that parallels the standard tenure-track position, providing the possibility of promotion, longer-term contracts, and higher pay for excellence in teaching and service. This birds-of-a-feather is designed to gather educators who are currently in such a position to share their experiences as members of the faculty of their departments and schools, and to provide opportunities for schools considering such positions to gather information.}
\end{longtable}
\begin{longtable}[l]{@{}l@{}l@{}r}
    \parbox[t]{1in}{\sffamily\large\textbf{BOF}} & 
    \parbox[t]{3in}{\sffamily\raggedright\large\textbf{A Town Meeting:  SIGCSE Committee on Expanding the Women-in-Computing Community}} & 
    \parbox[t]{1in}{\sffamily\raggedleft\large\textbf{302C}} \\
% row 2    
     & 
    Gloria Townsend, \textit{DePauw Universtiy}  \\[0.5em]
% row 3
    % Participants: & 
    % \multicolumn{2}{@{}l}{\parbox{3.75in}{ }} \\[2em]
% row 4
    \multicolumn{3}{@{}p{5in}}{\small In January 2004, we organized 
the second SIGCSE Committee 
(``Expanding the Women-in-Computing Community'').  Our 
annual Town Meeting provides 
dissemination of information 
concerning successful gender 
issues projects, along with 
group discussion and 
brainstorming, in order to 
create committee goals for the 
coming year. We select projects 
to highlight through listserv 
communication and through our 
connections with NCWIT, ABI, 
ACM-W, CRA-W, etc.  This year 
we will highlight the new NSF 
Broadening Participation in 
Computing grant – a grant that 
encompasses projects we 
presented in previous BOFs and 
a grant that builds on an 
alliance among ACM-W, ABI and 
NCWIT.}
\end{longtable}
\newpage
\begin{longtable}[l]{@{}l@{}l@{}r}
    \parbox[t]{1in}{\sffamily\large\textbf{BOF}} & 
    \parbox[t]{3in}{\sffamily\raggedright\large\textbf{Sharing Incremental Approaches for Adding Parallelism to CS Curricula}} & 
    \parbox[t]{1in}{\sffamily\raggedleft\large\textbf{305A}} \\ \\
% row 2    
% row 3
    & 
    \multicolumn{2}{@{}l}{\parbox{3.75in}{\raggedright Richard Brown, \textit{St. Olaf College;} Elizabeth Shoop, \emph{Macalester College;} Joel Adams, \emph{Calvin College;} David Bunde, \emph{Knox College;} Jens Mache, \emph{Lewis \& Clark College;} Paul Steinberg and Michael Wrinn, \emph{Intel Corporation;} Matthew Wolf, \emph{Georgia Tech} }} \\[1em]
% row 4
    \multicolumn{3}{@{}p{5in}}{\small Recent industry changes, including multi-core processors, cloud computing, and GPU programming, increase the need to teach parallelism to CS undergraduates. But few CS programs can afford to add new courses or greatly alter syllabi, and the large parallelism body of knowledge relates to many courses. Participants in this BOF will share incremental approaches for adding parallelism to undergraduate CS curricula, where students study parallel computing in brief units. This networking event/brainstorming session/swap meet will bring together: people with sharable parallelism expository readings, hands-on exercises, tech support ideas, etc.; people wishing to include such materials in their courses; and people curious about incremental approaches to teaching parallel computing.}
\end{longtable}
\begin{longtable}[l]{@{}l@{}l@{}r}
    \parbox[t]{1in}{\sffamily\large\textbf{BOF}} & 
    \parbox[t]{3in}{\sffamily\raggedright\large\textbf{Computer Science: Small Department Initiative}} & 
    \parbox[t]{1in}{\sffamily\raggedleft\large\textbf{305B}} \\ \\
% row 2    
% row 3
    & 
    \multicolumn{2}{@{}l}{\parbox{3.75in}{\raggedright James Jerkofsky, \textit{Walsh University;}  Cathy Bareiss, \emph{Olivet Nazarene University} }} \\[1em]
% row 4
    \multicolumn{3}{@{}p{5in}}{\small Faculty in small departments (perhaps 3 FTE, perhaps only 1 or 2,…) face special situations – both challenges and strengths.  In this BOF, members will have a chance to talk about both.  Challenges include maintaining a well-rounded curriculum and attracting students.   Strengths include a close relationship with other members of the department and majors.  These and other topics are open for discussion; the specific topics will be based upon the composition and interests of the group assembled.}
\end{longtable}
\begin{longtable}[l]{@{}l@{}l@{}r}
    \parbox[t]{1in}{\sffamily\large\textbf{BOF}} & 
    \parbox[t]{3in}{\sffamily\raggedright\large\textbf{Teaching with Alice}} & 
    \parbox[t]{1in}{\sffamily\raggedleft\large\textbf{306A}} \\ \\
% row 2    
% row 3
    & 
    \multicolumn{2}{@{}l}{\parbox{3.75in}{\raggedright Donald Slater and Wanda Dann, \textit{Carnegie Mellon University;} Steve Cooper, \emph{Stanford University} }} \\[1em]
% row 4
    \multicolumn{3}{@{}p{5in}}{\small This session is for anyone currently using Alice 2.2 and / or thinking about using Alice 3, or exploring the possibility of using Alice in his or her curriculum. The discussion leaders and experienced Alice instructors will share teaching strategies, tips, and tricks with each other and those new to Alice. The session provides an arena for sharing Alice instructional materials and ideas for courses at all educational levels. This is an opportunity to share assignments and pointers to web sites where collections of instructional materials, such as syllabi, student projects, exams, and other resources are available.}
\end{longtable}
\newpage
\begin{longtable}[l]{@{}l@{}l@{}r}
    \parbox[t]{1in}{\sffamily\large\textbf{BOF}} & 
    \parbox[t]{3in}{\sffamily\raggedright\large\textbf{Identifying Effective Pedagogical Practices for Commenting Computer Source Code}} & 
    \parbox[t]{1in}{\sffamily\raggedleft\large\textbf{306B}} \\ \\
% row 2    
% row 3
     & 
    \multicolumn{2}{@{}l}{\parbox{3.75in}{\raggedright Peter DePasquale, \textit{The College of New Jersey;} Michael Locasto, \emph{University of Calgary;} Lisa Kaczmarczyk, Independent Consultant }} \\[2em]
% row 4
    \multicolumn{3}{@{}p{5in}}{\small Few, if any, pedagogical practices exist for helping students embrace best practices in writing software documentation, particularly source code comments. Although instructors often stress the importance of good commenting, two problems exist. First, it can be difficult to actually define these best practices, and second, it can be difficult to grade or assess students’ application of such methods/practices. This BoF focuses on capturing for dissemination a concrete list of code commenting best practices used by the BoF attendees as they teach their classes.}
\end{longtable}
\begin{longtable}[l]{@{}l@{}l@{}r}
    \parbox[t]{1in}{\sffamily\large\textbf{BOF}} & 
    \parbox[t]{3in}{\sffamily\raggedright\large\textbf{Design of a Computer Security Teaching and Research Laboratory}} & 
    \parbox[t]{1in}{\sffamily\raggedleft\large\textbf{306C}} \\ \\
% row 2    
% row 3
    & 
    \multicolumn{2}{@{}l}{\parbox{3.75in}{\raggedright Jeffrey Duffany, \textit{Universidad del Turabo;} Alfredo Cruz, \emph{Politechnic University of Puerto Rico} }} \\[1em]
% row 4
    \multicolumn{3}{@{}p{5in}}{\small To engage students and enhance the learning process a certain amout of hands-on experience is desirable to supplement the theory portion of computer security-related courses.  This includes courses in information assurance, database security, network security, computer
security, computer forensics among others. This BOF will include the opinion of professors that are actually delivering these courses to
graduate and undergraduate students. They will tell us what kind of hardware and software is needed to develop a computer security lab
or to enhance a classroom environment, with an emphasis on free and open source software, operating systems and the use of virtual
machines to perform virus research.}
\end{longtable}
\begin{longtable}[l]{@{}l@{}l@{}r}
    \parbox[t]{1in}{\sffamily\large\textbf{BOF}} & 
    \parbox[t]{3in}{\sffamily\raggedright\large\textbf{Student ICTD Research and Service Learning Abroad}} & 
    \parbox[t]{1in}{\sffamily\raggedleft\large\textbf{307}} \\ \\
% row 2    
% row 3
     & 
    \multicolumn{2}{@{}l}{\parbox{3.75in}{\raggedright Joseph Mertz, \textit{Carnegie Mellon University;} Ralph Morelli, \emph{Trinity College;} Ruth Anderson, \emph{University of Washington} }} \\[1em]
% row 4
    \multicolumn{3}{@{}p{5in}}{\small This BOF is a chance for information sharing among faculty interested in involving students in ICTD research and/or service learning toward cultural and economic development globally.
It takes a lot to get students out into the field.  Challenges include developing partnerships, negotiating agreements, vetting the safety of destinations, identifying sources of funding, navigating the logistics of immunizations, visas, accommodations and flights to less-traveled places, reassuring parents as to the wisdom of their child's participation, managing development partner expectations, advising students' activities, and many more. This BOF will provide a venue for sharing experiences, information, and identifying potential new collaborations.}
\end{longtable}
\newpage
\begin{longtable}[l]{@{}l@{}l@{}r}
    \parbox[t]{1in}{\sffamily\large\textbf{BOF}} & 
    \parbox[t]{3in}{\sffamily\raggedright\large\textbf{Imaging College Educators}} & 
    \parbox[t]{1in}{\sffamily\raggedleft\large\textbf{Marriott University A}} \\
% row 2    
% row 3
  & 
    \multicolumn{2}{@{}l}{\parbox{3.75in}{\raggedright Jerod Weinman, \textit{Grinnell College;} Ellen Walker, \emph{Hiram College} }} \\[2em]
% row 4
    \multicolumn{3}{@{}p{5in}}{\small Within computing, the imaging field includes computer vision, image understanding, and image processing. While much research and teaching is done at the graduate level, the typical imaging educator at an undergraduate institution is the only specialist in his or her department. This BOF brings together educators who currently teach imaging courses or may be interested in expanding curricular offerings. We will emphasize sharing best practices, ideas, and resources as well as building a network for continued cooperation. Discussion topics may include course organization, assignments and projects, and lecture aids or other materials. Our network will include a mailing list for participants to ask questions and share ideas about imaging pedagogy and other means of sharing course materials.}
\end{longtable}
\begin{longtable}[l]{@{}l@{}l@{}r}
    \parbox[t]{1in}{\sffamily\large\textbf{BOF}} & 
    \parbox[t]{3in}{\sffamily\raggedright\large\textbf{Let's Talk Social Media}} & 
    \parbox[t]{1in}{\sffamily\raggedleft\large\textbf{Marriott University B}} \\
% row 2    
    & 
    Kimberly Voll, \textit{University of British Columbia}  \\[0.5em]
% row 3
    % Participants: & 
    % \multicolumn{2}{@{}l}{\parbox{3.75in}{ }} \\[2em]
% row 4
    \multicolumn{3}{@{}p{5in}}{\small Our students have Facebook, G+, and even Twitter accounts as a matter of course, and are used to rich, highly integrated environments. In contrast, CS education is via themed modalities: lectures, textbooks, labs, discussions, et cetera, that share no active or social connection (you cannot +1 a lecture, for example, share a passage of a text with a classmate, or pull up a view that truly integrates a course and its community). But we now have the technology to create learning environments that share the same rich, multimedia experience as the popular social media sites. What should this look like? How do we start? What have you tried? We’ll open with a brief overview of the leading social media tools for those unfamiliar, then proceed straight to an open discussion.}
\end{longtable}
\begin{longtable}[l]{@{}l@{}l@{}r}
    \parbox[t]{1in}{\sffamily\large\textbf{BOF}} & 
    \parbox[t]{3in}{\sffamily\raggedright\large\textbf{Program by Design: TeachScheme/ReachJava}} & 
    \parbox[t]{1in}{\sffamily\raggedleft\large\textbf{Marriott University C}} \\ \\
% row 2    
% row 3
     & 
    \multicolumn{2}{@{}l}{\parbox{3.75in}{\raggedright Viera Proulx, \textit{Northeastern University;} Stephen Bloch, \emph{Adelphi University} }} \\[2em]
% row 4
    \multicolumn{3}{@{}p{5in}}{\small Program by Design is a new name for the comprehensive introduction to programming at all levels that began with TeachScheme/ReachJava. This unconventional introductory computing curriculum covers both functional and the object-oriented program design in a systematic design-based style, enforcing test-first design from the beginning. The Bootstrap curriculum makes programming and algebra exciting for children ages 11-15. Special libraries support the design of interactive graphics-based games, musical explorations, client-server and mobile computing.
We invite you to come and meet those who have used the curriculum, learn about new additions, libraries, bring in your experiences with the curriculum, show your projects, or ask questions about how it works and how you can use it.}
\end{longtable}
\newpage
\begin{longtable}[l]{@{}l@{}p{3in}@{}r}
    \parbox[t]{1in}{\sffamily\large\textbf{BOF}} & 
    {\raggedright\sffamily\large\textbf{CSTA Chapters: Supporting Your \\ Local Computer Science Educators}} & 
    \parbox[t]{1in}{\sffamily\raggedleft\large\textbf{Marriott Chancellor}} \\ \\
% row 3
     & 
    \multicolumn{2}{@{}l}{\parbox{3.75in}{\raggedright Frances P. Trees, \textit{Rutgers, The State University of New Jersey;} Helen Hu, \emph{Westminster College;} Chinma Uche, \emph{Greater Hartford Academy of Mathematics and Science} }} \\[2em]
% row 4
    \multicolumn{3}{@{}p{5in}}{\small As part of its commitment to developing a strong community of computer science educators, the Computer Science Teachers Association (CSTA) supports the development of regional CSTA chapters. A CSTA chapter is a local branch of CSTA designed to facilitate discussion of local issues, provision of member services at the local level, and to promote CSTA membership on the national level.  This BOF will provide a platform for the discussion of CSTA chapter formation and for the sharing of successful chapter activities.}
\end{longtable}
\begin{longtable}[l]{@{}l@{}l@{}r}
    \parbox[t]{1in}{\sffamily\large\textbf{BOF}} & 
    \parbox[t]{3in}{\sffamily\raggedright\large\textbf{Revitalizing Computing Camp and Outreach:  How Do We Engage Teenagers in “Cool” Technology?}} & 
    \parbox[t]{1in}{\sffamily\raggedleft\large\textbf{Marriott Alumni}} \\ \\
% row 2    
% row 3
 & 
    \multicolumn{2}{@{}l}{\parbox{3.75in}{\raggedright  Kristine Nagel, \textit{Georgia Gwinnett College;} Evelyn Brannock Georgia and Robert Lutz, \emph{Georgia Gwinnett College} }} \\[2em]
% row 4
    \multicolumn{3}{@{}p{5in}}{\small Tech Camps are popular outreach tools to interest teens in computing programs and technology careers. One of the biggest obstacles is how to make Tech Camp “cool” and inviting for teenagers. How do we grab the attention of students to enroll? Once at camp, how do we engage teens with computing as a creative tool with relevancy to their lives?  It is summer; subject areas must be entertaining and relevant. Can we stay ahead of the tech-savvy teens with our budget constraints? Robots and storytelling have long been used; how do we innovate and spark interest, throughout the year? The purpose of this BOF is to share ideas, such as App Inventor for Android to create apps, including text messaging, encouraging students to incorporate their own creative graphics, and using tablet devices.}
\end{longtable}
\cfoot{\colorbox[gray]{0.45}{\color{white}\textsf{Thursday 18:10 - 19:00}}}
\noindent
\framebox[5in][c]{{\Large\sffamily\textbf{Thursday,  18:10 to 19:00}}}
\addcontentsline{toc}{subsubsection}{Birds of a Feather Flock II}
\begin{longtable}[l]{@{}l@{}l@{}r}
    \parbox[t]{1in}{\sffamily\large\textbf{BOF}} & 
    \parbox[t]{3in}{\sffamily\raggedright\large\textbf{Active eTextbooks for CS: What Should They Be?}} & 
    \parbox[t]{1in}{\sffamily\raggedleft\large\textbf{201}} \\
% row 2    
    & 
    Cliff Shaffer, \textit{Virginia Tech}  \\[0.5em]
% row 3
    % Participants: & 
    % \multicolumn{2}{@{}l}{\parbox{3.75in}{ }} \\[2em]
% row 4
    \multicolumn{3}{@{}p{5in}}{\small What should the textbook of tomorrow look like in a world of ubiquitous access to computing? Hypertextbooks have proved difficult to create and been fundamentally passive experiences. Commercial eBooks are merely books printed on an electronic screen instead of paper. New technologies such as HTML5 make it feasible to develop interactive applications that integrate with web services to provide a rich, pedagogically effective learning environment compatible with a range of computing platforms. We seek to generate discussion by participants to describe what they hope to see in online textbooks in the near future, and what resources and support would be required for them to adopt such a thing into their own courses.}
\end{longtable}
\newpage
\begin{longtable}[l]{@{}l@{}l@{}r}
    \parbox[t]{1in}{\sffamily\large\textbf{BOF}} & 
    \parbox[t]{3in}{\sffamily\raggedright\large\textbf{Enriching Computing Instruction with Studio-Based Learning}} & 
    \parbox[t]{1in}{\sffamily\raggedleft\large\textbf{205}} \\ \\
% row 2    
% row 3
     & 
    \multicolumn{2}{@{}l}{\parbox{3.75in}{\raggedright N. Hari Narayanan, \textit{Auburn University;} Martha Crosby, \emph{University of Hawaii at Manoa;} Dean Hendrix, \emph{Auburn University} and Christopher Hundhausen, \emph{Washington State University} }} \\[1em]
% row 4
    \multicolumn{3}{@{}p{5in}}{\small This BOF is related to the Special Session Transforming the CS Classroom with Studio-Based Learning (SBL). SBL promotes learning in a collaborative context by having students construct, present, review and refine their work with the guidance of peers and teachers. A team of CS educators and education experts have been implementing and evaluating SBL in CS courses over the past five years. The BOF will introduce SBL to the SIGCSE audience, and engage them in a discussion of the potential of, evidence for, and practical advice regarding SBL as an instructional approach that can motivate as well as teach students. Discussions will include ``war stories'' from teachers who have adopted the approach in their courses and hands-on activities to help participants apply SBL to their courses.}
\end{longtable}
\begin{longtable}[l]{@{}l@{}l@{}r}
    \parbox[t]{1in}{\sffamily\large\textbf{BOF}} & 
    \parbox[t]{3in}{\sffamily\raggedright\large\textbf{AP CS A - Sharing Teaching Strategies and Curricular Ideas}} & 
    \parbox[t]{1in}{\sffamily\raggedleft\large\textbf{206}} \\ \\
% row 2    
% row 3
     & 
    \multicolumn{2}{@{}l}{\parbox{3.75in}{\raggedright Lester Wainwright, \textit{Charlottesville High School;}Renee Ciezki, \emph{Estrella Mountain Community College;} Robert Glen Martin, \emph{TAG Magnet High School} }} \\[2em]
% row 4
    \multicolumn{3}{@{}p{5in}}{\small This BOF will provide an opportunity for high school and college faculty to discuss the AP CS A curriculum and to explore possibilities for collaborations and outreach activities between high schools and colleges.}
\end{longtable}
\begin{longtable}[l]{@{}l@{}l@{}r}
    \parbox[t]{1in}{\sffamily\large\textbf{BOF}} & 
    \parbox[t]{3in}{\sffamily\raggedright\large\textbf{Regional Celebrations of Women in Computing (WiC) -- Best Practices}} & 
    \parbox[t]{1in}{\sffamily\raggedleft\large\textbf{301AB}} \\ \\
% row 2    
% row 3
     & 
    \multicolumn{2}{@{}l}{\parbox{3.75in}{\raggedright Jodi Tims, \textit{Baldwin-Wallace College;}  Ellen Walker, \emph{Hiram College;} Rachelle Kristof Hippler, \emph{Bowling Green State University Firelands College} }} \\[1em]
% row 4
    \multicolumn{3}{@{}p{5in}}{\small Regional celebrations are locally organized, professional conferences modeled after the Grace Hopper Celebration of Women in Computing (GHC). This BOF allows people who have organized or would like to organize such a conference to get together to share successes and challenges.  Attendees that have hosted a regional celebration are invited to bring a un-poster (i.e. 8.5 x 11 flyer, 30 copies) that highlights their conference features and/or shares lessons learned. The leaders plan to divide the time between the 5 major areas of conference planning:  program, sponsorship, publicity/communications, registration, and site/logistics.}
\end{longtable}
\newpage
\begin{longtable}[l]{@{}l@{}l@{}r}
    \parbox[t]{1in}{\sffamily\large\textbf{BOF}} & 
    \parbox[t]{3in}{\sffamily\raggedright\large\textbf{Hacking and the Security Curriculum}} & 
    \parbox[t]{1in}{\sffamily\raggedleft\large\textbf{302A}} \\ \\
% row 2    
% row 3
     & 
    \multicolumn{2}{@{}l}{\parbox{3.75in}{\raggedright Richard Weiss, \textit{The Evergreen State College;} Michael Locasto, \emph{University of Calgary} and Jens Mache, \emph{Lewis \& Clark College }}} \\[1em]
% row 4
    \multicolumn{3}{@{}p{5in}}{\small Incorporating information security into the undergraduate curriculum continues to be a topic of interest to SIGCSE attendees. The purpose of this BOF is to help sustain the existing community of educators and researchers interested in bringing ethical hacking skills and an understanding of security into the classroom and relating these topics to the foundations of Computer Science. We would like to bring our colleagues together to share pedagogical practices, stories of hacking and how to use them to inspire our students and communicate complex concepts in computer science and security. We also plan to discuss our own experiences, practices and ongoing efforts (e.g., our infosec teaching experiences, the SISMAT program, EDURange and the dissemination of infosec interactive exercises).}
\end{longtable}
\begin{longtable}[l]{@{}l@{}l@{}r}
    \parbox[t]{1in}{\sffamily\large\textbf{BOF}} & 
    \parbox[t]{3in}{\sffamily\raggedright\large\textbf{Flipping the Classroom}} & 
    \parbox[t]{1in}{\sffamily\raggedleft\large\textbf{302B}} \\ \\
% row 2    
    & 
    Barry Brown, \textit{Sierra College}  \\[1em]
% row 3
    % Participants: & 
    % \multicolumn{2}{@{}l}{\parbox{3.75in}{ }} \\[2em]
% row 4
    \multicolumn{3}{@{}p{5in}}{\small In a flipped classroom, students watch or listen to the lecture at home and do homework in the classroom. The classroom becomes much more interactive and the educator has ample opportunity to provide individualized guidance when it's most needed. The watch-at-home content can include recorded lectures, demonstration videos, adaptive quizzes, or anything in between. Come share your experiences developing ``flip'' material, learn from others what's involved, and find out whether it's working to improve success and retention.}
\end{longtable}
\begin{longtable}[l]{@{}l@{}p{3in}@{}r}
    \parbox[t]{1in}{\sffamily\large\textbf{BOF}} & 
    {\sffamily\raggedright\large\textbf{Using Social Networks to Engage Computer Science Students}} & 
    \parbox[t]{1in}{\sffamily\raggedleft\large\textbf{302C}} \\ \\
% row 2    
% row 3
     & 
    \multicolumn{2}{@{}l}{\parbox{3.75in}{\raggedright Semmy Purewal, \textit{University of North Carolina at Asheville;} Owen Astrachan, \emph{Duke University;} David Brown, \emph{Pellissippi State Community College} and Jeffrey Forbes, \emph{Duke University} }} \\[2em]
% row 4
    \multicolumn{3}{@{}p{5in}}{\small Social Networking continues to be a popular past-time among high school and college students. In this birds of a feather session, we will share ideas on integrating social networking topics into computer science courses at the introductory and non-major levels. Additionally we will discuss approaches to integrating social network programming into upper level courses. Finally we will attempt to address the following questions: will social networking draw new students into the computing disciplines the way that video games did in the previous generation? Will it attract new types of students with different expectations? Is social networking just a fad that will have no effect on Computer Science programs? Or is social networking a topic that is better left to other academic disciplines?}
\end{longtable}
\newpage
\begin{longtable}[l]{@{}l@{}l@{}r}
    \parbox[t]{1in}{\sffamily\large\textbf{BOF}} & 
    \parbox[t]{3in}{\sffamily\raggedright\large\textbf{Digital Humanities: Reaching Out to the Other Culture}} & 
    \parbox[t]{1in}{\sffamily\raggedleft\large\textbf{305A}} \\ \\
% row 2    
     & 
    Robert Beck, \textit{Villanova University}  \\[0.5em]
% row 3
    % Participants: & 
    % \multicolumn{2}{@{}l}{\parbox{3.75in}{ }} \\[2em]
% row 4
    \multicolumn{3}{@{}p{5in}}{\small This discussion will connect instructors who are reaching out to their colleagues in the humanities to discover areas of collaboration. It focuses on what these disciplines have to contribute to our knowledge of computing and how computational thinking informs these disciplines. One goal is to lay the foundation for a more general program of study in digital humanities that would reach students who would like to see how computing could enhance their work in history, literature, anthropology, or philosophy, for example.}
\end{longtable}
\begin{longtable}[l]{@{}l@{}l@{}r}
    \parbox[t]{1in}{\sffamily\large\textbf{BOF}} & 
    \parbox[t]{3in}{\sffamily\raggedright\large\textbf{A Multimedia and Liberal Arts Approach to a First Course in Programming and its Crossover Potential for Computer Science and the Arts}} & 
    \parbox[t]{1in}{\sffamily\raggedleft\large\textbf{305B}} \\ \\
% row 2    
% row 3
     & 
    \multicolumn{2}{@{}l}{\parbox{3.75in}{\raggedright Trish Cornez, \textit{University of Redlands;} Richard Cornez, \emph{University of Redlands} }} \\[1.5em]
% row 4
    \multicolumn{3}{@{}p{5in}}{\small Students are acculturated in a visual, interactive, and interdisciplinary world. 
This BOF will provide a platform for a discussion on how multimedia can be integrated in a CS1 course. Discussions will focus on attributes of conventional and unconventional first languages and explore a liberal arts approach to integrate disciplines both scientific and artistic. We envision discussions relevant to:
 Mathematicians visualizing processes using multimedia and algorithms; Physicists using game programming to deconstruct and explore physical environments and re-assembling them as virtual worlds; Computer scientists and behavioral scientists collaborating on responsive systems to explore philosophical underpinnings of media; Musicians and computer scientists creating computational art.}
\end{longtable}
\begin{longtable}[l]{@{}l@{}l@{}r}
    \parbox[t]{1in}{\sffamily\large\textbf{BOF}} & 
    \parbox[t]{3in}{\sffamily\raggedright\large\textbf{Teaching with App Inventor for Android}} & 
    \parbox[t]{1in}{\sffamily\raggedleft\large\textbf{306A}} \\ \\
% row 2    
% row 3
     & 
    \multicolumn{2}{@{}l}{\parbox{3.75in}{\raggedright Jeff Gray, \textit{University of Alabama;} Harold Abelson, \emph{MIT;} Ralph Morelli, \emph{Trinity College;} Jeff Gray, \emph{University of Alabama} and Chinma Uche, \emph{Greater Hartford Academy of Math and Science} }} \\[2em]
% row 4
    \multicolumn{3}{@{}p{5in}}{\small App Inventor for Android is a visual blocks language for building mobile apps. Like Scratch, the language’s drag-and-drop blocks interface significantly lowers the barrier to entry. Beginners can immediately build apps that interface with mobile technology (e.g., GPS, Text-to-speech, SMS Texting) and build apps that have a real-world impact. In this BoF, hosted by App Inventor creator Hal Abelson and experienced teachers and authors, we’ll discuss the language, its future in K-12 and university education, and its new home at the MIT Center for Mobile Learning.}
\end{longtable}
\newpage
\begin{longtable}[l]{@{}l@{}l@{}r}
    \parbox[t]{1in}{\sffamily\large\textbf{BOF}} & 
    \parbox[t]{3in}{\sffamily\raggedright\large\textbf{Technology that Educators of Computing Hail (TECH): Come, Share your Favorites!}} & 
    \parbox[t]{1in}{\sffamily\raggedleft\large\textbf{306B}} \\ \\
% row 2    
% row 3
     & 
    \multicolumn{2}{@{}l}{\parbox{3.75in}{Daniel Garcia and Luke Segars, \emph{UC Berkeley} }} \\[2em]
% row 4
    \multicolumn{3}{@{}p{5in}}{\small The pace of technology for use in computing education is staggering.  In the last five years, the following tools/websites have completely transformed our teaching: Piazza, Google Docs, YouTube, Doodle and whenisgood.net, Skype and Google Hangout, and Khan Academy among others.  Hardware has also played a part – we love our Zoom H2 digital voice recorder (for recording CD-quality lecture audio), Blue Yeti USB mike (for audio/videoconferences), and iClickers (for engaging students in class).  Do you wish you could easily share your favorites?  Want to find out what the others know that you don’t?  Have a tool you’ve built and want to get some users?  Come to this BOF!  We’ll also show the TECH website we’ve built that attempts to collect all of these tools in one place.}
\end{longtable}
\begin{longtable}[l]{@{}l@{}l@{}r}
    \parbox[t]{1in}{\sffamily\large\textbf{BOF}} & 
    \parbox[t]{3in}{\sffamily\raggedright\large\textbf{Motivating CS1/2 Students with the Android Platform}} & 
    \parbox[t]{1in}{\sffamily\raggedleft\large\textbf{306C}} \\ \\
% row 2    
% row 3
 & 
    \multicolumn{2}{@{}l}{\parbox{3.75in}{\raggedright John Lewis, \textit{Virginia Tech;} Anthony Allevato and Stephen H. Edwards, \emph{Virginia Tech} }} \\[2em]
% row 4
    \multicolumn{3}{@{}p{5in}}{\small The use of Android in computing courses is growing. Students find it engaging because they can develop Java apps for mobile devices. Android also offers challenges in the classroom, especially in CS1 and CS2. As a professional- level platform, it uses design idioms that may require students to learn advanced language features earlier. It also adds logistical complications to setting up projects and development tools. Existing approaches to software testing and automated grading need adaptation. This BOF is for sharing assignments, resources, techniques, and experiences with others, focusing on issues that arise when balancing the teaching of fundamental concepts with the complexities required to accomplish basic tasks on the Android platform.}
\end{longtable}
\begin{longtable}[l]{@{}l@{}l@{}r}
    \parbox[t]{1in}{\sffamily\large\textbf{BOF}} & 
    \parbox[t]{3in}{\sffamily\raggedright\large\textbf{Interdisciplinary Database Collaborations}} & 
    \parbox[t]{1in}{\sffamily\raggedleft\large\textbf{307}} \\ \\
% row 2    
% row 3
     & 
    \multicolumn{2}{@{}l}{\parbox{3.75in}{\raggedright Suzanne Dietrich, \textit{Arizona State University;}  Don Goelman, \emph{Villanova University} }} \\[2em]
% row 4
    \multicolumn{3}{@{}p{5in}}{\small Databases play a major role across many disciplines for the storage and retrieval of information. Many database educators are establishing collaborations with colleagues representing a diverse spectrum of interests, for both research and pedagogical purposes. Further, the range of cooperating disciplines is expanding, as evidenced by the emergence of new fields such as computational journalism, as well as by the proliferation of discipline-specific dialects of XML. The goal of this Birds-of-a-Feather session is to bring database educators together to share their experiences on interdisciplinary collaborations in an open dialogue that is fostered by this format.}
\end{longtable}
\newpage
\begin{longtable}[l]{@{}l@{}l@{}r}
    \parbox[t]{1in}{\sffamily\large\textbf{BOF}} & 
    \parbox[t]{3in}{\sffamily\raggedright\large\textbf{Google Summer of Code and Google Code-in BoF}} & 
    \parbox[t]{1in}{\sffamily\raggedleft\large\textbf{Marriott University A}} \\ \\
% row 2    
     & 
    Carol Smith, \textit{Google, Inc.}  \\[0.5em]
% row 3
    % Participants: & 
    % \multicolumn{2}{@{}l}{\parbox{3.75in}{ }} \\[2em]
% row 4
    \multicolumn{3}{@{}p{5in}}{\small Google Summer of Code is the outreach program aimed at getting university students involved in a 3-month online internship working in open source software development. 

Google Code-in is the contest aimed at involving 13-18 year olds in open source software development, documentation translation, outreach, research, and more. I will be discussing both programs at this BoF and encouraging students and teachers to get involved. 

We'll open the forum for discussion amongst the attendees about how to participate, how to get the word out, and answer any questions they may have.}
\end{longtable}
\begin{longtable}[l]{@{}l@{}p{3in}@{}r}
    \parbox[t]{1in}{\sffamily\large\textbf{BOF}} & 
    {\sffamily\raggedright\large\textbf{Building Partnerships Across the CS Education Spectrum}} & 
    \parbox[t]{1in}{\sffamily\raggedleft\large\textbf{Marriott University B}} \\ \\
% row 2    
% row 3
     & 
    \multicolumn{2}{@{}l}{\parbox{3.75in}{\raggedright Chris Stephenson, \textit{Computer Science Teachers Association;} Steve Cooper, \emph{Stanford University;} Don Yanek, \emph{Northside College Prep High School} and Jeff Gray, \emph{University of Alabama} }} \\[2em]
% row 4
    \multicolumn{3}{@{}p{5in}}{\small Over the last five years, CSTA has built a solid outreach and teacher support network through the work of its chapters and Leadership Cohort. This network has also become a major source of active partnerships between K-12 teachers, their schools, and colleagues from colleges, universities, and industry. The goal of this BOF is to provide concrete examples and suggestions for SIGCSE members interested in building these kinds of partnerships.}
\end{longtable}
\begin{longtable}[l]{@{}l@{}l@{}r}
    \parbox[t]{1in}{\sffamily\large\textbf{BOF}} & 
    \parbox[t]{3in}{\sffamily\raggedright\large\textbf{Engaging The Community With Mobile App Projects}} & 
    \parbox[t]{1in}{\sffamily\raggedleft\large\textbf{Marriott University C}} \\ \\
% row 2    
% row 3
     & 
    \multicolumn{2}{@{}l}{\parbox{3.75in}{William Turkett and Paul Pauca, \emph{Wake Forest University} and Joel Hollingsworth, \emph{Elon University} }} \\[2em]
% row 4
    \multicolumn{3}{@{}p{5in}}{\small As the popularity of mobile devices surges, more and more organizations are looking to exploit the novel interaction methods of mobile devices to re-deploy legacy software or to develop innovative new  applications. Many organizations are looking to nearby universities for expertise in this area. At the same time, mobile computing has become increasingly integrated within courses in CS departments. Historically, capstone courses and other advanced electives have resulted in the production of non-trivial software artifacts. This  BOF will provide a platform for discussion of how the use of mobile app platforms in such courses can allow for the development of meaningful software projects that engage with and give back to the community and provide rich opportunities for service learning.}
\end{longtable}
\newpage
\begin{longtable}[l]{@{}l@{}l@{}r}
    \parbox[t]{1in}{\sffamily\large\textbf{BOF}} & 
    \parbox[t]{3in}{\sffamily\raggedright\large\textbf{Have Class, Will Travel}} & 
    \parbox[t]{1in}{\sffamily\raggedleft\large\textbf{Marriott Chancellor}} \\ \\
% row 2    
     & 
    Paige Meeker, \textit{Presbyterian College}  \\[0.5em]
% row 3
    % Participants: & 
    % \multicolumn{2}{@{}l}{\parbox{3.75in}{ }} \\[2em]
% row 4
    \multicolumn{3}{@{}p{5in}}{\small At many schools, various disciplines offer travel courses (to other lands or to locations within the USA) to give students an experiential component to their learning. How can we introduce such courses to computer science departments? This BOF will provide a time of sharing ideas for such courses and welcomes discussion of travel courses that have been successfully taught. In addition to normal course preparation, these courses also involve travel arrangements, payment schedules, and careful scheduling to provide maximum benefit to the student. Our group will share ideas for locations of travel, topics of courses, and collaboration with other disciplines, as well as the additional overhead such a course entails, such as cost/payment schedule, insurance, itinerary, safety, etc.}
\end{longtable}
\begin{longtable}[l]{@{}l@{}l@{}r}
    \parbox[t]{1in}{\sffamily\large\textbf{BOF}} & 
    \parbox[t]{3in}{\sffamily\raggedright\large\textbf{Integration of Experiential Learning and Teaching: Beyond the Walls of the Classroom, Techniques, Challenges and Merits.}} & 
    \parbox[t]{1in}{\sffamily\raggedleft\large\textbf{Marriott Alumni}} \\ \\
% row 2    
% row 3
 & 
    \multicolumn{2}{@{}l}{\parbox{3.75in}{\raggedright Arshia Khan, Tamara Lichtenberg, Rishika Dhody, Joel Pouale, \emph{The College of St. Scholastica} and John Woosley, \emph{Southeastern Louisiana University} }} \\[2em]
% row 4
    \multicolumn{3}{@{}p{5in}}{\small Integration of experiential learning is critical in the field of computer science. With technology evolving over night, job requirements are extremely volatile. Educators have a challenging task of staying abreast with the technology and market needs while self learning the new technologies. One solution is to rely on the businesses for input on what should be taught and using them to extend the learning into the real world through experiential learning(not just internships). Talking points:  Filling gaps between academia and industry; Faculty can share their methods of experiential learning; Applying the practical skills to theoretical knowledge -- turning theory into practice; Opportunities to bring real world clients in the classroom}
\end{longtable}
\cfoot{\colorbox[gray]{0.45}{\color{white}\textsf{Thursday 19:00 - 20:00}}}
\vspace{1em}
\noindent
\framebox[5in][c]{{\Large\sffamily\textbf{Thursday,  19:00 to 20:00}}}
\begin{longtable}[l]{@{}p{1in}@{}p{3in}@{}p{1in}}
    {\sffamily\large\textbf{}} & 
    {\sffamily\large\textbf{SIGCSE Reception}} & 
    {\sffamily\large\textbf{Registration Foyer}} \\
\end{longtable}    
\vspace{2em}

