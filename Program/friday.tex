%!TEX root = program.tex
\newpage
\fancyhead[RO,LE]{}
\addcontentsline{toc}{subsection}{Friday}
\cfoot{\colorbox[gray]{0.45}{\color{white}\textsf{Friday 07:15 - 08:15}}}
\noindent
\framebox[5in][c]{{\Large\sffamily\textbf{Friday,  7:15 to 8:15}}}
\begin{longtable}[l]{@{}p{1in}@{}p{3in}@{}r}
    {\sffamily\large\textbf{}} & 
    {\sffamily\large\textbf{Alice Breakfast}} & 
    {\sffamily\large\textbf{302B}} \\
\end{longtable}    

\cfoot{\colorbox[gray]{0.45}{\color{white}\textsf{Friday 08:30 - 10:00}}}

\noindent
\framebox[5in][c]{{\Large\sffamily\textbf{Friday,  8:30 to 10:00}}}
\begin{longtable}[l]{@{}p{1in}@{}p{3in}@{}r}
    \multicolumn{2}{@{}p{3.75in}}{\sffamily\large\textbf{SIGCSE Keynote Address}} & 
    {\sffamily\large\textbf{Ballroom BC}} \\ \\
    \multicolumn{3}{@{}p{5in}}{\sffamily\large\textbf{From Computational Thinking to Computational Values} }\\
    & \multicolumn{2}{@{}p{4in}}{Hal Abelson, \textit{Massachusetts Institute of Technology}} \\
    \multicolumn{3}{@{}p{5in}}{\small SIGCSE members love the beauty of computational thinking. They know the joy of bringing those ideas to young people. That love for computational thinking entails respect for the computational values that empower people in the digital world. For academics, those values have been central to the flowering of computing as an intellectual endeavor.

Today, those values are increasingly threatened by stresses from both within and outside academia: squabbles over who owns academic work, increasingly stringent and overreaching intellectual property laws, and the replacement of open computing platforms by closed applications and walled-garden application markets.

In this talk I'll describe some things we've done at MIT to support computational values, like open publication of all our course materials, our faculty policy on open publication of academic research, and our recently announced initiative for open online instruction based on non-proprietary software platforms. I'll discuss Creative Commons licensing and Free Software, and the importance of tinkerability for empowering citizens in an information society. And I'll describe App Inventor for Android, a new programming tool motivated by the vision that all of us us can experience mobile computing as creators using tools that we can control and reshape, rather than only as consumers of packaged applications. } \\

\end{longtable}    

\cfoot{\colorbox[gray]{0.45}{\color{white}\textsf{Friday 10:00 - 10:45}}}
\noindent
\vspace{\baselineskip}
\framebox[5in][c]{{\Large\sffamily\textbf{Friday,  10:00 to 10:45}}}
\begin{tabular*}{5in}[l]{@{}p{3.9in}@{}r@{}}
    {\sffamily\large\textbf{Break and Exhibits}} & 
    {\raggedright\sffamily\large\textbf{Exhibit Hall A}} 
\end{tabular*}    

\cfoot{\colorbox[gray]{0.45}{\color{white}\textsf{Friday 10:00 - 11:30}}}
\vspace{2em}
\noindent
\vspace{0.5\baselineskip}
\framebox[5in][c]{{\Large\sffamily\textbf{Friday,  10:00 to 11:30}}}
\begin{tabular*}{5in}[l]{@{}p{3.9in}@{}r}
    {\sffamily\large\textbf{NSF Showcase \#3}} & 
    {\raggedright\sffamily\large\textbf{Exhibit Hall A}} 
\end{tabular*}
\begin{itemize}
\addtolength{\itemsep}{-2mm}
     \item {{\sffamily\textbf{Databases for Many Majors: A Student-Centered Approach,}} Suzanne W. Dietrich and Don Goelman }
     \item{{\sffamily\textbf{Establishing the Exploring Computer Science Course in Silicon Valley High Schools,}} Dan Lewis, Ruth Davis, Pedro Hernandez-Ramos, and Craig Blackburn } 
     \item {{\sffamily\textbf{Integrating Security in the Computing Curriculum: Security Injections \@ Towson,}} Siddharth Kaza and Blair Taylor } 
     \item {{\sffamily\textbf{How to Run a Successful REU Site Program,}}  Susan D. Urban, Michael Shin, and Mohan Sridharan } 
\end{itemize}
\newpage
\cfoot{\colorbox[gray]{0.45}{\color{white}\textsf{Friday 10:00 - 12:00}}}
\vspace{2em}
\noindent
\framebox[5in][c]{{\Large\sffamily\textbf{Friday,  10:00 to 12:00}}}
\addcontentsline{toc}{subsubsection}{Poster Session I}
\begin{longtable}{@{}p{0.75in}@{}p{3.25in}@{}r}
   {\sffamily\large\textbf{POSTER}} &
   {\raggedright\sffamily\large\textbf{Poster Session I}} & 
   {\sffamily\large\textbf{Exhibit Hall A }} \\ \\
\multicolumn{3}{@{}p{5in}}{\sffamily\raggedright\textbf{Using Reflection to Enhance Feedback for Automated Grading}} \\
\multicolumn{3}{@{}p{5in}}{\raggedright Carl Alphonce, \textit{University at Buffalo}; Joseph LeGasse, \textit{Meritain Health}} \\ \\
\multicolumn{3}{@{}p{5in}}{\sffamily\raggedright\textbf{The Cross-Curriculum Mobile Computing Labware for CS}} \\
\multicolumn{3}{@{}p{5in}}{\raggedright Liang Hong, \textit{Tennessee State University}; Kai Qian and Dan Lo, \textit{Southern Polytechnic State University}; Yi Pan, Yanqing Zhang and Xiaolin Hu, \textit{Georgia State University}} \\ \\
\multicolumn{3}{@{}p{5in}}{\sffamily\raggedright\textbf{Merging Healthcare and Technology:  A Multi-disciplinary Health Information Technology (HIT) Curriculum}} \\
\multicolumn{3}{@{}p{5in}}{\raggedright Elizabeth Howard, Donna Evans and Marilyn Anderson, \textit{Miami University - Middletwon}; Jill Courte, \textit{Miami University - Hamilton}} \\ \\
\multicolumn{3}{@{}p{5in}}{\sffamily\raggedright\textbf{An Integrated Introduction to Network Protocols and Cryptography to High School Students}} \\
\multicolumn{3}{@{}p{5in}}{\raggedright William Mongan, \textit{Drexel University}} \\ 
\multicolumn{3}{@{}p{5in}}{\sffamily\raggedright\textbf{A PC Robot for Learning Computer Vision and Advanced Programming}} \\
\multicolumn{3}{@{}p{5in}}{\raggedright Xuzhou Chen and Nadimpalli V.R. Mahadev, \textit{Fitchburg State University}} \\ \\
\multicolumn{3}{@{}p{5in}}{\sffamily\raggedright\textbf{Girls Gather for Computer Science (G2CS)}} \\
\multicolumn{3}{@{}p{5in}}{\raggedright Shereen Khoja, Juliet Brosing, Camille Wainwright and Jeffrey Barlow, \textit{Pacific University}} \\ \\
\multicolumn{3}{@{}p{5in}}{\sffamily\raggedright\textbf{Debuggems to Assess Student Learning in E-Textiles}} \\
\multicolumn{3}{@{}p{5in}}{\raggedright Deborah Fields, Kristin Searle, Yasmin Kafai and Hannah Min, \textit{University of Pennsylvania}} \\ \\
\multicolumn{3}{@{}p{5in}}{\sffamily\raggedright\textbf{Mediascripting – Teaching Introductory CS by Through Interactive Graphics Scripting}} \\
\multicolumn{3}{@{}p{5in}}{\raggedright Samuel Rebelsky, Janet Davis and Jerod Weinman, \textit{Grinnell College}} \\ \\
\multicolumn{3}{@{}p{5in}}{\sffamily\raggedright\textbf{Do Faculty Recognize the Difference Between Computer Science and Information Technology?  A Survey of Liberal Arts Faculty}} \\
\multicolumn{3}{@{}p{5in}}{\raggedright Jaime Spacco and Hannah Fidoten, \textit{Knox College}} \\ \\
\multicolumn{3}{@{}p{5in}}{\sffamily\raggedright\textbf{Interdisciplinary Travel Courses in Computer Science}} \\
\multicolumn{3}{@{}p{5in}}{\raggedright Paige Meeker, \textit{Presbyterian College}} \\ \\
\multicolumn{3}{@{}p{5in}}{\sffamily\raggedright\textbf{User type clustering to refine search and browse for educational resources}} \\
\multicolumn{3}{@{}p{5in}}{\raggedright Monika Akbar and Clifford A. Shaffer, \textit{Virginia Tech}} \\ \\
\multicolumn{3}{@{}p{5in}}{\sffamily\raggedright\textbf{A Comprehensive CS Curriculum Revision, Implementation and Analysis}} \\
\multicolumn{3}{@{}p{5in}}{\raggedright Steven Huss-Lederman, \textit{Beloit College}} \\ \\
\multicolumn{3}{@{}p{5in}}{\sffamily\raggedright\textbf{Developing an Interdisciplinary Health Informatics Security and Privacy Program}} \\
\multicolumn{3}{@{}p{5in}}{\raggedright Xiaohong Yuan, Jinsheng Xu, Hong Wang and Kossi Edoh, \textit{North Carolina A\&T State University}} \\ \\
\multicolumn{3}{@{}p{5in}}{\sffamily\raggedright\textbf{A Team Software Development Course Featuring iPad Programming}} \\
\multicolumn{3}{@{}p{5in}}{\raggedright Robert England, \textit{Transylvania University}} \\ \\
\multicolumn{3}{@{}p{5in}}{\sffamily\raggedright\textbf{The Role of Belonging in Computer Science Student Engagement}} \\
\multicolumn{3}{@{}p{5in}}{\raggedright Nanette Veilleux, \textit{Simmons College}; Rebecca Bates, \textit{Computer Science, Minnesota State University, Mankato}; Cheryl Allendoerfer, Diane Jones and Joy Crawford, \textit{University of Washington}} \\ \\
\multicolumn{3}{@{}p{5in}}{\sffamily\raggedright\textbf{Streamlining Project Setup in Eclipse for Both Time-Constrained and Large-Scale Assignments}} \\
\multicolumn{3}{@{}p{5in}}{\raggedright Ellen Boyd and Anthony Allevato, \textit{Virginia Tech}} \\ \\
\multicolumn{3}{@{}p{5in}}{\sffamily\raggedright\textbf{A Customizable Platform for Classroom Collaboration Using Mobile Devices}} \\
\multicolumn{3}{@{}p{5in}}{\raggedright Stephen Hughes, Ben Schafer, Aaron Mangel and Sean Fredericksen, \textit{University of Northern Iowa}} \\ \\
\multicolumn{3}{@{}p{5in}}{\sffamily\raggedright\textbf{Explaining the Dynamic Structure and Behavior of Java Programs using a Visual Debugger}} \\
\multicolumn{3}{@{}p{5in}}{\raggedright Demian Lessa and Bharat Jayaraman, \textit{SUNY at Buffalo}} \\ \\
\multicolumn{3}{@{}p{5in}}{\sffamily\raggedright\textbf{Using FPGA Systems Across the Computer Science Curriculum}} \\
\multicolumn{3}{@{}p{5in}}{\raggedright D. Brian Larkins, H. Erin Rickard and William M. Jones, \textit{Coastal Carolina University}} \\ \\
\multicolumn{3}{@{}p{5in}}{\sffamily\raggedright\textbf{Maximizing Content Learning for Deaf Students and English as a Second Language Students}} \\
\multicolumn{3}{@{}p{5in}}{\raggedright Raja Kushalnagar and Joseph Stanislow, \textit{Rochester Institute of Technology}} \\ \\
\multicolumn{3}{@{}p{5in}}{\sffamily\raggedright\textbf{All-In-One Virtualized Laboratory}} \\
\multicolumn{3}{@{}p{5in}}{\raggedright Shamsi Moussavi and Giuseppe Sena, \textit{MassBay Community College}} \\ \\
\multicolumn{3}{@{}p{5in}}{\sffamily\raggedright\textbf{Recursive Thinkers and Doers in CS1}} \\
\multicolumn{3}{@{}p{5in}}{\raggedright Suzanne Menzel and Joseph Cottam, \textit{Indiana University}} \\ \\
\multicolumn{3}{@{}p{5in}}{\sffamily\raggedright\textbf{Computing in Context: Video Scenarios for Recognizing and Utilizing Basic Computing Constructs}} \\
\multicolumn{3}{@{}p{5in}}{\raggedright Madalene Spezialetti, \textit{Trinity College}} \\ \\
\multicolumn{3}{@{}p{5in}}{\sffamily\raggedright\textbf{Programming board-game strategies in the introductory CS sequence}} \\
\multicolumn{3}{@{}p{5in}}{\raggedright Ivona Bezakova, James Heliotis, Sean Strout, Adam Oest and Paul Solt, \textit{Rochester Institute of Technology}} \\ \\
\end{longtable}


\cfoot{\colorbox[gray]{0.45}{\color{white}\textsf{Friday 10:45 - 12:00}}}
\vspace{2em}
\noindent
\framebox[5in][c]{{\Large\sffamily\textbf{Friday,  10:45 to 12:00}}}
\begin{longtable}[l]{@{}l@{}l@{}r}
    \parbox[t]{1in}{\sffamily\large\textbf{PANEL}} & 
    \parbox[t]{3in}{\sffamily\raggedright\large\textbf{Teaching Mathematical Reasoning Across the Curriculum}} & 
    \parbox[t]{1in}{\sffamily\raggedleft\large\textbf{301AB}} \\ \\
% row 2    
    Chair: & 
    Joan Krone, \textit{Denison University}  \\[0.5em]
% row 3
    Participants: & 
    \multicolumn{2}{@{}l}{\parbox{3.75in}{Doug Baldwin, \textit{SUNY Geneseo}; Jeff Carver, \textit{University of Alabama}; Joseph Hollingsworth, \textit{Indiana University Southeast}; Amruth Kumar, \textit{Ramapo College of New Jersey} }} \\[2em]
% row 4
    \multicolumn{3}{@{}p{5in}}{\small We discuss ways in which the panel members have incorporated mathematical reasoning into a variety of courses, encouraging and supporting students to apply and enhance their reasoning skills in productive ways across the entire CS curriculum.}
\end{longtable}

\begin{longtable}[l]{@{}l@{}l@{}r}
    \parbox[t]{1in}{\sffamily\large\textbf{SPECIAL SESSION}} & 
    \parbox[t]{3in}{\sffamily\raggedright\large\textbf{Teaching HS Computer Science as if the Rest of the World Existed}} & 
    \parbox[t]{1in}{\sffamily\raggedleft\large\textbf{305B}} \\ \\
% row 2    
    Chair: & 
    Scott Portnoff, \textit{Downtown Magnets High School, Los Angeles}  \\[0.5em]
% row 3
    % Participants: & 
    % \multicolumn{2}{@{}l}{\parbox{3.75in}{ }} \\[2em]
% row 4
    \multicolumn{3}{@{}p{5in}}{\small This session discusses the design, implementation and rationale for a HS pre-APCS curriculum of Interdisciplinary Central-Problem-Based (ICPB) units that model real-world applications. In a typical multi-week unit, students begin by solving a problem using a complex software application, such as SDSC Biology Workbench.  Students then build a small-scale version of the program, focusing on 1 or 2 algorithms, using Processing, Excel, BYOB or Alice. This approach affords students both context and practical potential for their work.  Unit topics come from the fields of Astronomy (Galileo), Bioinformatics (Evolution), Molecular Modeling (DNA Double Helix), Political Science (Women's Suffrage/ Electoral Process), Environmental Science, Music, and Holocaust Studies (Hollerith Machine Technology).}
\end{longtable}
\begin{longtable}[l]{@{}l@{}l@{}r}
    \parbox[t]{1in}{\sffamily\large\textbf{SPECIAL SESSION}} & 
    \parbox[t]{3in}{\sffamily\raggedright\large\textbf{Funding the Challenges in Computing}} & 
    \parbox[t]{1in}{\sffamily\raggedleft\large\textbf{306C}} \\ \\
% row 2    
    Chair: & 
    Guy-Alain Amoussou, \textit{NSF}  \\[0.5em]
% row 3
    Participants: & 
    \multicolumn{2}{@{}l}{\parbox{3.75in}{Scott Grissom, \textit{Grand Valley State University} }} \\[2em]
% row 4
    \multicolumn{3}{@{}p{5in}}{\small What are the perceived challenges by the computing education and research communities? After small groups discuss this question, we will compare and contrast these perceived challenges to the current trend in proposals submitted and funded by the National Science Foundation’s (NSF) Transforming Undergraduate Education in STEM (TUES) program. The intention is to create awareness for all involved and to broaden the range of proposals submitted to NSF.}
\end{longtable}
\begin{longtable}[l]{@{}p{1in}@{}p{3in}@{}r}
    {\sffamily\large\textbf{Supporter Session}} & 
    {\sffamily\large\textbf{Intel}} & 
    {\sffamily\large\textbf{305A}} \\
\end{longtable}    

\newpage
\begin{longtable}{@{}p{0.75in}@{}p{3.25in}@{}r}
   {\sffamily\large\textbf{PAPERS}} &
   {\raggedright\sffamily\large\textbf{CS1:  New Ideas}} & 
   {\sffamily\large\textbf{302A }} \\
%row 2
   Chair:  & 
   {\raggedright Lori Carter, \textit{Point Loma Nazarene University}} & \\ \\
{\sffamily 10:45}& 
\multicolumn{2}{@{}p{3.75in}}{\sffamily\raggedright\textbf{Creative Coding and Visual Portfolios for CS1}} \\
& \multicolumn{2}{@{}p{3.75in}}{\raggedright Dianna Xu and Deepak Kumar, \textit{Bryn Mawr College}; Ira Greenberg, \textit{Southern Methodist University}} \\ \\
\multicolumn{3}{@{}p{5in}}{\small In this paper, we present the design and development of a new approach to teaching the college-level introductory computing course (CS1) using the context of art and creative coding. Over the course of a semester, students create a portfolio of aesthetic visual designs that employ basic computing structures typically taught in traditional CS1 courses using the Processing programming language. The goal of this approach is to bring the excitement, creativity, and innovation fostered by the context of creative coding. We also present results from a comparative study involving two offerings of the new course at two different institutions. Additionally, we compare our results with another successful approach that uses personal robots to teach CS1.} \\ \\
{\sffamily 11:10}& 
\multicolumn{2}{@{}p{3.75in}}{\sffamily\raggedright\textbf{Stepping Up to Integrative Questions on CS1 Exams}} \\
& \multicolumn{2}{@{}p{3.75in}}{\raggedright Daniel Zingaro and Michelle Craig, \textit{University of Toronto}; Andrew Petersen, \textit{University of Toronto Mississauga}} \\ \\
\multicolumn{3}{@{}p{5in}}{\small We explore the use of sequences of small code writing questions (``concept questions'') designed to incrementally evaluate single programming concepts. We report on student performance on a CS1 final examination that included a traditional code-writing question and four corresponding concept questions. We find that the concept questions are significant predictors of performance on both the corresponding code-writing question and the final exam as a whole. We argue that concept questions provide more accurate formative feedback and simplify marking by reducing the number of variants that must be considered. An analysis of the student responses categorized by the students' previous programming experience suggests inexperienced students have the most to gain from the use of concept questions.} \\ \\
{\sffamily 11:35}& 
\multicolumn{2}{@{}p{3.75in}}{\sffamily\raggedright\textbf{Using Reflective Blogs for Pedagogical Feedback in CS1}} \\
& \multicolumn{2}{@{}p{3.75in}}{\raggedright Jeffrey Stone, \textit{Pennsylvania State University}} \\ \\
\multicolumn{3}{@{}p{5in}}{\small The use of weekly, reflective student blogs can be one method for collecting ongoing feedback about a CS1 course. Reflective blogs permit a continuous feedback loop that can be used for both formative and summative assessment of pedagogical innovations. This paper reports on a two-year qualitative study involving the use of reflective blogging in six sections of two CS1 courses. Reflective blogs were used as a low stakes, non-intimidating vehicle whereby concerns, requests, and other course-related issues could be voiced by students. The posts were used as an assessment and feedback mechanism for pedagogical transformation of the participating courses. This study demonstrates that reflective blogs in CS1 can be a useful tool for faculty course development.} \\ \\
\end{longtable}


\newpage
\begin{longtable}{@{}p{0.75in}@{}p{3.25in}@{}r}
   {\sffamily\large\textbf{PAPERS}} &
   {\raggedright\sffamily\large\textbf{Team Work}} & 
   {\sffamily\large\textbf{302B }} \\
%row 2
   Chair:  & 
   {\raggedright Jody Paul, \textit{Metropolitan State College of Denver}} & \\ \\
{\sffamily 10:45}& 
\multicolumn{2}{@{}p{3.75in}}{\sffamily\raggedright\textbf{Participation Patterns in Student Teams}} \\
& \multicolumn{2}{@{}p{3.75in}}{\raggedright Vreda Pieterse, Lisa Thompson and Linda Marshall, \textit{University of Pretoria}; Dina Venter, \textit{Olrac-SPS}} \\ \\
\multicolumn{3}{@{}p{5in}}{\small We describe a process for teaching teamwork in a Software Engineering module. Our objective was to create opportunities for our students to experience some problems of working in a group before they formed teams in which they had to work for the rest of the year.  The process entails expecting students to work on well defined assignments for short periods in teams where risk factors were induced. Through experiencing these short bursts of team tribulation students are prepared to handle difficult events and situations in their teams.  We describe the design and implementation of this process.  We report on changes in the levels of participation of the students during the process.  We offer our explanation of possible factors that may have caused the observed variations.} \\ \\
{\sffamily 11:10}& 
\multicolumn{2}{@{}p{3.75in}}{\sffamily\raggedright\textbf{Application of Non-programming Focused Treisman-style Workshops in Introductory Computer Science}} \\
& \multicolumn{2}{@{}p{3.75in}}{\raggedright Lindsay Jamieson, Alan Jamieson and Angela Johnson, \textit{St. Mary's College of Maryland}} \\ \\
\multicolumn{3}{@{}p{5in}}{\small In the 60s and 70s, Uri Treisman developed a specific style of workshops to encourage the retention of underrepresented minority students in Calculus courses. Since that time, workshops based on the Treisman model have been successful across the US and have resulted in more underrepresented minority students successfully completing Calculus. Some attempts have been made to translate the Treisman model to CS1, but all previous attempts have been focused on programming skills.  However, one of the student assumptions that deter underrepresented minorities from attempting a major or minor in CS is that a computer scientist is a solitary programmer. In this paper, we discuss a specific two year pilot program of non-programming focused Treisman-style workshops in conjunction with a CS1 course.} \\ \\
{\sffamily 11:35}& 
\multicolumn{2}{@{}p{3.75in}}{\sffamily\raggedright\textbf{Collaboration Across the Curriculum: A Disciplined Approach to Developing Team Skills}} \\
& \multicolumn{2}{@{}p{3.75in}}{\raggedright Matthew Lang and Ben Coleman, \textit{Moravian College}} \\ \\
\multicolumn{3}{@{}p{5in}}{\small Increasing the communication and collaborative skills of computer science
students has been a priority in the community for some time.  We present our
philosophy, collaboration across the curriculum, which moves beyond existing
individual courses or course units to teach collaboration skills in a pervasive
manner.  In our approach, concepts are introduced and skills are developed
throughout the computer science curriculum---from CS1 to a capstone experience.
Students are provided with opportunities to exercise skills in reflective environments
that eventually mirror real-world experiences, and technical course content is
not compromised. We argue for this system and provide details about how collaboration
across the curriculum is accomplished at a small liberal arts college.}
\end{longtable}


\newpage
\begin{longtable}{@{}p{0.75in}@{}p{3.25in}@{}r}
   {\sffamily\large\textbf{PAPERS}} &
   {\raggedright\sffamily\large\textbf{Summer Experiences}} & 
   {\sffamily\large\textbf{306A }} \\
%row 2
   Chair:  & 
   {\raggedright Kinnis Gosha, \textit{Morehouse College}} & \\ \\
{\sffamily 10:45}& 
\multicolumn{2}{@{}p{3.75in}}{\sffamily\raggedright\textbf{App Inventor for Android: Report from a Summer Camp}} \\
& \multicolumn{2}{@{}p{3.75in}}{\raggedright Krishnendu Roy, \textit{Valdosta State University}} \\ \\
\multicolumn{3}{@{}p{5in}}{\small Google's App Inventor for Android (AIA) is the newest visual programming language designed to introduce students to programming through creation of mobile apps. AIA opens up the world of mobile apps to novice programmers. Success stories of using AIA to introduce college students to programming exist. We used AIA in computing summer camps for high school students that we offer at our university. This paper is an experience report about using AIA in our camps. We provide a detailed description of designing our camps with AIA including the process of selecting and setting-up an Android device and instructional materials that we designed and made available to everyone. We also share evaluation results of using AIA in our camps and our impression of AIA as a programming-introduction tool.} \\ \\
{\sffamily 11:10}& 
\multicolumn{2}{@{}p{3.75in}}{\sffamily\raggedright\textbf{Sustainable and Effective Computing Summer Camps}} \\
& \multicolumn{2}{@{}p{3.75in}}{\raggedright Barbara Ericson, \textit{Georgia Institute of Technology}; Tom McKlin, \textit{The Findings Group}} \\ \\
\multicolumn{3}{@{}p{5in}}{\small Summer camps are a popular form of outreach for colleges and universities. But, it isn't enough to offer computing summer camps and hope students like them.  The camps should be effective by some measure, such as broadening participation by underrepresented groups and/or increasing learning. Summer camps should also be sustainable, so that institutions can continue to offer them regularly.  The summer camps at Georgia Tech have evolved to the point where they are sustainable and effective. This paper presents the rationale for our camps, the business model that makes them sustainable, and the evaluation results that demonstrate positive attitude changes and increases in learning.} \\ \\
{\sffamily 11:35}& 
\multicolumn{2}{@{}p{3.75in}}{\sffamily\raggedright\textbf{A Summer Science Experience with Computer Graphics for Secondary Students}} \\
& \multicolumn{2}{@{}p{3.75in}}{\raggedright Timothy Davis, \textit{Clemson University}} \\ \\
\multicolumn{3}{@{}p{5in}}{\small This paper describes the principles, implementation, and results of a weeklong summer science course for junior high and high school students interested in computer science.  To motivate and foster interest and creativity in students, while providing adequate complexity to introduce programming concepts and techniques, we used programming projects in computer graphics as the main learning tool.  Included in our discussion are experiences across three course offerings, as well as detailed course assignments.} \\ \\
\end{longtable}


\newpage
\begin{longtable}{@{}p{0.75in}@{}p{3.25in}@{}r}
   {\sffamily\large\textbf{PAPERS}} &
   {\raggedright\sffamily\large\textbf{Software Engineering}} & 
   {\sffamily\large\textbf{306B }} \\
%row 2
   Chair:  & 
   {\raggedright Ariel Ortiz, \textit{Tecnológico de Monterrey, Campus Estado de México}} & \\ \\
{\sffamily 10:45}& 
\multicolumn{2}{@{}p{3.75in}}{\sffamily\raggedright\textbf{Integrating UX with Scrum in an Undergraduate Software Development Project}} \\
& \multicolumn{2}{@{}p{3.75in}}{\raggedright Janet Davis, Chase Felker and Radka Slamova, \textit{Grinnell College}} \\ \\
\multicolumn{3}{@{}p{5in}}{\small We report our experiences using the Scrum agile software development method in an undergraduate user-centered web development project. Our chief contributions are to report on using Scrum in a summer research setting as distinct from academic-year coursework and to consider the integration of Scrum and user experience (UX) development methods in a non-professional, learning environment. Our experience with combining Scrum and UX was positive: this methodology gave our project a clear structure, kept us motivated, and focused us on developing a usable final product. We discuss our adaptations of Scrum to UX development and to the summer research setting, along with challenges we faced and lessons learned, to inform students and faculty who wish to apply such methods in future projects.} \\ \\
{\sffamily 11:10}& 
\multicolumn{2}{@{}p{3.75in}}{\sffamily\raggedright\textbf{Using WReSTT in SE Courses: An Empirical Study}} \\
& \multicolumn{2}{@{}p{3.75in}}{\raggedright Peter J. Clarke, Jairo Pava, Debra Davis and Frank Hernandez, \textit{Florida International University}; Tariq M. King, \textit{North Dakota State University}} \\ \\
\multicolumn{3}{@{}p{5in}}{\small There continues to be a lack of adequate training for students in software testing techniques and tools at most academic institutions. Several educators and researchers have investigated innovative approaches to integrate testing into programming and software engineering (SE) courses with some success.  The main problem is getting other educators to adopt their approaches and getting students to continue to use the techniques.   In this paper we present a study that evaluates a non-intrusive approach to integrating software testing techniques and tools in SE courses using a Web-Based Repository of Software Testing Tools (WReSTT).  The results of the study show that students who use WReSTT in the classroom can improve their understanding and use of testing techniques and tools.} \\ \\
{\sffamily 11:35}& 
\multicolumn{2}{@{}p{3.75in}}{\sffamily\raggedright\textbf{Understanding the Tenets of Agile Software Engineering: Lecturing, Exploration and Critical Thinking}} \\
& \multicolumn{2}{@{}p{3.75in}}{\raggedright Shvetha Soundararajan, Amine Chigani and Arthur James, \textit{Virginia Tech}} \\ \\
\multicolumn{3}{@{}p{5in}}{\small In our quest to develop better software products, it is imperative that we strive to learn and understand the application of agile methods to the software development enterprise. Unfortunately, students have only limited exposure to the agile philosophy, principles and practices at the graduate and undergraduate levels of education. In an effort to address this concern, we offered an advanced graduate-level course entitled “Agile Software Engineering” in the Department of Computer Science at Virginia Tech. The primary objectives of the course were to introduce the values and principles and practices underlying the agile philosophy, and to do so in an atmosphere that encourages debate and critical thinking. This paper describes our experiences during the offering of that course.} \\ \\
\end{longtable}


% \begin{longtable}[l]{@{}p{1in}@{}p{3in}@{}r}
%     {\sffamily\large\textbf{Supporter Session}} & 
%     {\sffamily\large\textbf{TBA}} & 
%     {\sffamily\large\textbf{302C}} \\
% \end{longtable}    
\cfoot{\colorbox[gray]{0.45}{\color{white}\textsf{Friday 12:00 - 13:45}}}
\noindent
\framebox[5in][c]{{\Large\sffamily\textbf{Friday,  12:00 to 13:45}}}
\begin{longtable}[l]{@{}p{1in}@{}p{3in}@{}r}
    {\sffamily\large\textbf{}} & 
    {\sffamily\large\textbf{Lunch Break}} & 
    {\sffamily\large\textbf{On your own}} \\
\end{longtable}    

\cfoot{\colorbox[gray]{0.45}{\color{white}\textsf{Friday 12:10 - 13:35}}}
\vspace{2em}
\noindent
\framebox[5in][c]{{\Large\sffamily\textbf{Friday,  12:10 to 13:35}}}
\begin{longtable}[l]{@{}p{1in}@{}p{3in}@{}r}
    {\sffamily\large\textbf{}} & 
    {\sffamily\large\textbf{Snap! Lunch}} & 
    {\sffamily\large\textbf{301AB}} \\
\end{longtable}    
\begin{longtable}[l]{@{}p{1in}@{}p{3in}@{}r}
    {\sffamily\large\textbf{}} & 
    {\sffamily\large\textbf{UPE National Meeting}} & 
    {\sffamily\large\textbf{302A}} \\
\end{longtable}    
\cfoot{\colorbox[gray]{0.45}{\color{white}\textsf{Friday 13:45 - 15:00}}}
\noindent
\framebox[5in][c]{{\Large\sffamily\textbf{Friday,  13:45 to 15:00}}}
\begin{longtable}[l]{@{}l@{}l@{}r}
    \parbox[t]{1in}{\sffamily\large\textbf{SPECIAL SESSION}} & 
    \parbox[t]{3in}{\sffamily\raggedright\large\textbf{CS Principles:  Piloting a National Course}} & 
    \parbox[t]{1in}{\sffamily\raggedleft\large\textbf{301AB}} \\ \\
% row 2    
    Chair: & 
    Owen Astrachan, \textit{Duke University}  \\[0.5em]
% row 3
    Participants: & 
    \multicolumn{2}{@{}l}{\parbox{3.75in}{Ralph Morelli, \textit{Trinity College}; Dwight Barnette, \textit{Virginia Tech}; Jeff Gray, \textit{University of Alabama}; Chinma Uche, \textit{Hartford Academy of Math and Science} }} \\[2em]
% row 4
    \multicolumn{3}{@{}p{5in}}{\small The CS Principles course has been designed to be taught nationwide at both the secondary and post-secondary levels. As part of this joint NSF/College Board project, 11 high schools are partnered with 10 colleges to teach the course, be part of a national initiative to test assessment items, and to help validate the curriculum framework that is the basis for the course and project. 
This special session is a report of the second stage of pilot that is designed to lead to a national standard and a new, additional AP exam in the next five years. This course will not replace the traditional, CS1-oriented AP exam, but will be a new national introduction to Computer Science.}
\end{longtable}
\begin{longtable}[l]{@{}l@{}l@{}r}
    \parbox[t]{1in}{\sffamily\large\textbf{SPECIAL SESSION}} & 
    \parbox[t]{3in}{\sffamily\raggedright\large\textbf{Fun, Phone, and the Future - Microsoft XNA Game Studio, Windows Phone, and Kinect SDK}} & 
    \parbox[t]{1in}{\sffamily\raggedleft\large\textbf{305B}} \\ \\
% row 2    
    Chair: & 
    Pat Yongpradit, \textit{Springbrook High School}  \\[0.5em]
% row 4
    \multicolumn{3}{@{}p{5in}}{\small Microsoft XNA Game Studio and C\# provide the basis of an advanced high school or introductory post-high school game development computer science course. Game development is serious computer science. The curriculum tools enable students to create games, simulations, and applications for the PC, Xbox 360, Windows Phone, and Kinect that expands students’ skills in complex logic, object oriented programming (OOP), advanced algorithms, and data structures. See and participate in demonstrations of student projects from the new Game Development with XNA course curriculum.}
\end{longtable}
\begin{longtable}[l]{@{}l@{}l@{}r}
    \parbox[t]{1in}{\sffamily\large\textbf{SPECIAL SESSION}} & 
    \parbox[t]{3in}{\sffamily\raggedright\large\textbf{Building an Open, Large-Scale Research Repository of Initial Programming Student Behavior}} & 
    \parbox[t]{1in}{\sffamily\raggedleft\large\textbf{306C}} \\ \\
% row 2    
    Chair: & 
    Michael Kölling, \textit{University of Kent}  \\[0.5em]
% row 3
    Participants: & 
    \multicolumn{2}{@{}l}{\parbox{3.75in}{Ian Utting, \textit{University of Kent} }} \\[2em]
% row 4
    \multicolumn{3}{@{}p{5in}}{\small Many initiatives in improving learning of programming are based on gut instinct or localised experience. Gathering data as a basis for interventions, especially on a large scale, is hard. The BlueJ environment is being instrumented to collect data useful to a variety of educational programming researchers. BlueJ is ideally placed to collect such data: Users number in the millions, situated all over the world. This volume and diversity is unique in the history of such investigations and presents a significant opportunity for researchers. The data will be open to interested research groups, which will enable a wide variety of investigations that were previously impractical.
This session presents work to date and solicits input from researchers about the design of the data collection.}
\end{longtable}
\newpage
\begin{longtable}{@{}p{0.75in}@{}p{3.25in}@{}r}
   {\sffamily\large\textbf{PAPERS}} &
   {\raggedright\sffamily\large\textbf{Collaborative Learning}} & 
   {\sffamily\large\textbf{302A }} \\
%row 2
   Chair:  & 
   {\raggedright Adrian German, \textit{Indiana University School of Informatics and Computing}} & \\ \\
{\sffamily 13:45}& 
\multicolumn{2}{@{}p{3.75in}}{\sffamily\raggedright\textbf{Assigning Student Programming Pairs Based on their Mental Model Consistency: An Initial Investigation}} \\
& \multicolumn{2}{@{}p{3.75in}}{\raggedright Alex Radermacher, Gursimran Walia and Richard Rummelt, \textit{North Dakota State University}} \\ \\
\multicolumn{3}{@{}p{5in}}{\small Pair Programming has been shown to be beneficial to student learning. This paper reports results of research investigating the effectiveness of pairing students based on their mental models. Prior research has found a correlation between mental model consistency and performance in computer programming courses. Students’ mental models helps to provide insights into how students approach problem solving and may indicate how to effectively pair students to improve their programming ability and learning. Results indicate that mental model consistency is a predictor of student success in an introductory programming course. Future goals of this research are to fully evaluate all pairing arrangements and to produce tests to evaluate mental model consistency for other computer science concepts.} \\ \\
{\sffamily 14:10}& 
\multicolumn{2}{@{}p{3.75in}}{\sffamily\raggedright\textbf{Group Whiteboards and Modeler/Customer Teams: Getting Closer to Industrial-Style Collaboration in a Classroom}} \\
& \multicolumn{2}{@{}p{3.75in}}{\raggedright David Socha, \textit{University of Washington Bothell}} \\ \\
\multicolumn{3}{@{}p{5in}}{\small This paper reports on two simple innovations that help create a more authentic and engaging modeling experience in an undergraduate analysis and design course: (a) having each team of students act both as modelers for another team, and as customers for another team, and (b) providing each team with their own whiteboard. The results from their use throughout the course, and for a single use of the whiteboards in a Computing Technology and Public Policy course, were quite positive. They resulted in a qualitatively different experience noticeable both to the instructors and the students. While some students were initially reluctant to use the whiteboards, by the end of the course most students were enthusiastic about their use.} \\ \\
{\sffamily 14:35}& 
\multicolumn{2}{@{}p{3.75in}}{\sffamily\raggedright\textbf{Is There Service in Computing Service Learning?}} \\
& \multicolumn{2}{@{}p{3.75in}}{\raggedright Randy Connolly, \textit{Mount Royal University}} \\ \\
\multicolumn{3}{@{}p{5in}}{\small While service learning projects in post-secondary computing can achieve important disciplinary outcomes for the students, the benefit of these projects for the service recipients and their community has been under-examined. This paper argues that since these projects are meant to benefit both student donors and community recipients, we must examine more carefully how computing service projects interact with all the social actors affected by the projects. Taking such an approach will require recognizing that ICT by itself will not increase democracy, equality, or any other social good; indeed some service learning projects may actually do more harm than good. The paper concludes with some sample computer learning projects that are oriented towards achieving true service for the recipients.} \\ \\
\end{longtable}


\newpage
\begin{longtable}{@{}p{0.75in}@{}p{3.25in}@{}r}
   {\sffamily\large\textbf{PAPERS}} &
   {\raggedright\sffamily\large\textbf{Curriculum Issues}} & 
   {\sffamily\large\textbf{302B }} \\
%row 2
   Chair:  & 
   {\raggedright Colleen Lewis, \textit{University of California, Berkeley}} & \\ \\
{\sffamily 13:45}& 
\multicolumn{2}{@{}p{3.75in}}{\sffamily\raggedright\textbf{Computer Science in NZ High Schools: The first Year of the New Standards}} \\
& \multicolumn{2}{@{}p{3.75in}}{\raggedright Tim Bell, \textit{University of Canterbury}; Peter Andreae, \textit{Victoria University of Wellington}; Anthony Robins, \textit{University of Otago}} \\ \\
\multicolumn{3}{@{}p{5in}}{\small Computer science became available as a nationally assessed topic in NZ schools for the first time in 2011. We review the introduction of computer science as a formal topic, including the level of adoption, issues that have arisen in the process of introducing it, and work that has been undertaken to address those issues.} \\ \\
{\sffamily 14:10}& 
\multicolumn{2}{@{}p{3.75in}}{\sffamily\raggedright\textbf{Web Science: expanding the notion of Computer Science}} \\
& \multicolumn{2}{@{}p{3.75in}}{\raggedright Su White, \textit{University of Southampton}; Michalis Vafopoulos, \textit{Aristotle University of Thessaloniki}} \\ \\
\multicolumn{3}{@{}p{5in}}{\small Academic disciplines which experience rapid change face problems maintaining teaching programs. Web Science: the science of decentralized information systems is fundamentally interdisciplinary, encompassing the technologies and engineering of the Web alongside associated emerging human, social and organizational practices. As work on the Computer Curricula 2013 is underway, it seems timely to ask what place Web Science may have in the curriculum landscape. This paper discusses the role and place of Web Science in the computing disciplines. It provides an account of work towards defining a curriculum for Web Science utilizing novel methods to support and elaborate curriculum definition and review. The findings of a desk study of existing related curriculum recommendations are presented.} \\ \\
{\sffamily 14:35}& 
\multicolumn{2}{@{}p{3.75in}}{\sffamily\raggedright\textbf{Educating the Educator Through Computation: What GIS Can Do For Computer Science}} \\
& \multicolumn{2}{@{}p{3.75in}}{\raggedright John Barr and Ali Erkan, \textit{Ithaca College}} \\ \\
\multicolumn{3}{@{}p{5in}}{\small We designed a system where non-computational faculty members (along with undergraduates) enroll in an introductory, multidisciplinary, open source Geographic Information System (GIS) course to experience integrative learning as students. The faculty participants are subsequently required to integrate their newly acquired expertise with their own disciplinary teaching and research; the necessary time commitment is compensated by a three-credit teaching load reallocation. Our hypothesis is that increasing the general faculty's appreciation of computation (in the context of integrative learning) is an indirect yet effective and scalable way to reach a wider group of students to convey our fundamental disciplinary message: computing is more than programming and computing empowers people.} \\ \\
\end{longtable}


\newpage
\begin{longtable}{@{}p{0.75in}@{}p{3.25in}@{}r}
   {\sffamily\large\textbf{PAPERS}} &
   {\raggedright\sffamily\large\textbf{Active Learning I}} & 
   {\sffamily\large\textbf{306A }} \\
%row 2
   Chair:  & 
   {\raggedright Robert England, \textit{Transylvania University}} & \\ \\
{\sffamily 13:45}& 
\multicolumn{2}{@{}p{3.75in}}{\sffamily\raggedright\textbf{An Experience Report: On The Use Of Multimedia Pre-Instruction And Just-In-Time Teaching In A CS1 Course}} \\
& \multicolumn{2}{@{}p{3.75in}}{\raggedright Paul Carter, \textit{University of British Columbia}} \\ \\
\multicolumn{3}{@{}p{5in}}{\small We describe an experience using online multimedia instruction and just-in-time teaching in an introductory programming course. Survey data has shown that students are strongly in favour of the approach. A series of screencasts was developed to replace the traditional lecture component of the course. Students were asked to review a small number of screencasts before each class and were assessed on their comprehension at the start of class using a series of “clicker” questions. A just-in-time mini-lecture was provided in response to the initial assessment, on an as-needed basis. The remaining class time was devoted to small-group exercises.} \\ \\
{\sffamily 14:10}& 
\multicolumn{2}{@{}p{3.75in}}{\sffamily\raggedright\textbf{Using JiTT in a Database Course}} \\
& \multicolumn{2}{@{}p{3.75in}}{\raggedright Alexandra Martinez, \textit{Universidad de Costa Rica}} \\ \\
\multicolumn{3}{@{}p{5in}}{\small This paper describes our experience using the Just-in-Time Teaching (JiTT) technique in an undergraduate database course for computer science majors during two semesters. JiTT was implemented by giving the students reading assignments and asking them to complete web-based reading tests the day before class, so that the instructor could detect weaknesses in students' understanding of the material and adjust the lesson plan just in time for the next day class. Based on surveys as well as on exams and course grades, we found a significant improvement on the students interest in the course and learning of the material.} \\ \\
{\sffamily 14:35}& 
\multicolumn{2}{@{}p{3.75in}}{\sffamily\raggedright\textbf{Process Oriented Guided Inquiry Learning (POGIL) for Computer Science}} \\
& \multicolumn{2}{@{}p{3.75in}}{\raggedright Clifton Kussmaul, \textit{Muhlenberg College}} \\ \\
\multicolumn{3}{@{}p{5in}}{\small This paper describes an ongoing project to develop activities for computer science (CS) using process oriented guided inquiry learning (POGIL). First, it reviews relevant background on effective learning and POGIL, compares POGIL to other forms of active learning, and describes the potential of POGIL for CS. Second, it describes a sample POGIL activity, including the structure and contents, student and facilitator actions during the activity, and how activities are designed. Third, it summarizes current progress and plans for a NSF TUES project to development POGIL materials for CS. Finally, it discusses student feedback and future directions.} \\ \\
\end{longtable}


\newpage
\begin{longtable}{@{}p{0.75in}@{}p{3.25in}@{}r}
   {\sffamily\large\textbf{PAPERS}} &
   {\raggedright\sffamily\large\textbf{Communication Skills}} & 
   {\sffamily\large\textbf{306B }} \\
%row 2
   Chair:  & 
   {\raggedright James Early, \textit{SUNY Oswego}} & \\ \\
{\sffamily 13:45}& 
\multicolumn{2}{@{}p{3.75in}}{\sffamily\raggedright\textbf{Integrating Communication Skills into the Computer Science Curriculum}} \\
& \multicolumn{2}{@{}p{3.75in}}{\raggedright Katrina Falkner and Nickolas Falkner, \textit{University of Adelaide}} \\ \\
\multicolumn{3}{@{}p{5in}}{\small Computer Science majors must be able to communicate effectively.  There is considerable work in the area of communication skills development, positioned in terms of curriculum guidelines for effective communication skills development, and example communication skills activities. However, this research is deficient in detailed, contextualised methodologies and frameworks for the development of communication skills within the Computer Science curriculum.  We present a new methodology, building upon well established theoretical frameworks, designed to assist academics in the development of communication skills activities integrated with discipline content across the curriculum.} \\ \\
{\sffamily 14:10}& 
\multicolumn{2}{@{}p{3.75in}}{\sffamily\raggedright\textbf{`Explain in Plain English' Questions: Implications for Teaching}} \\
& \multicolumn{2}{@{}p{3.75in}}{\raggedright Laurie Murphy, \textit{Pacific Lutheran University}; Renée McCauley, \textit{College of Charleston}; Sue Fitzgerald, \textit{Metropolitan State University}} \\ \\
\multicolumn{3}{@{}p{5in}}{\small This paper reports on a replication of work by Corney, Lister and Teague who performed a longitudinal study of novice programmers, looking for relationships between ability to `explain in plain English' the meaning of a code segment and success in writing code later in the semester.  The study extends the work of Corney, Lister and Teague by qualitatively evaluating ‘explain in plain English’ responses to gain a deeper understanding of student misconceptions.  Statistical results from this study are similar to those of Corney, Lister and Teague.  Results highlight students’ fragile knowledge and suggest the need for assessment and instruction of basic concepts later into the term than instructors are likely to expect.} \\ \\
{\sffamily 14:35}& 
\multicolumn{2}{@{}p{3.75in}}{\sffamily\raggedright\textbf{The Impact of Question Generation Activities on Performance}} \\
& \multicolumn{2}{@{}p{3.75in}}{\raggedright Andrew Luxton-Reilly, Daniel Bertinshaw, Paul Denny, Beryl Plimmer and Robert Sheehan, \textit{The University of Auckland}} \\ \\
\multicolumn{3}{@{}p{5in}}{\small Recent interest in student-centric pedagogies have resulted in the development of numerous tools that support student-generated questions.  Previous evaluations of such tools have reported strong correlations between student participation and exam performance, yet the level of student engagement with other learning activities in the course is a potential confounding factor.  We show such correlations may be explained by other factors, and we undertake a deeper analysis that reveals evidence of the positive impact question-generation activities have on student performance.} \\ \\
\end{longtable}


% \begin{longtable}[l]{@{}p{1in}@{}p{3in}@{}r}
%     {\sffamily\large\textbf{Supporter Session}} & 
%     {\sffamily\large\textbf{TBA}} & 
%     {\sffamily\large\textbf{302C}} \\
% \end{longtable}    
\newpage
\begin{longtable}[l]{@{}p{1in}@{}p{3in}@{}r}
    {\sffamily\large\textbf{Supporter Session}} & 
    {\sffamily\large\textbf{Google}} & 
    {\sffamily\large\textbf{305A}} \\
    \multicolumn{3}{@{}p{5in}}{\raggedright\sffamily\large\textbf{The MIT Center for Mobile Learning and the Future of App Inventor}} \\
    & Hal Abelson, Professor of Computer Science and Engineering, \emph{MIT} \\
    & Mark Friedman, Former App Inventor Project Lead, \emph{Google} \\
    \multicolumn{3}{@{}p{5in}}{\small The MIT Media Lab applies an unorthodox research approach to envision the impact of emerging technologies on everyday life—including technologies used in education.  Through a generous grant from Google, the Media Lab recently expanded on this work by establishing the MIT Center for Mobile Learning
led by Hal Abelson, Mitch Resnick, and Eric Kopfler.  The Center’s work revolves around the principle that mobile technology can fulfill its potential to enhance education only if  teachers and learners can create new mobile technologies, not merely experience them as consumers.  This session will discuss the Center's three new initiatives:  Scratch, TaleBlazer, and App Inventor.
}
\end{longtable}    
\cfoot{\colorbox[gray]{0.45}{\color{white}\textsf{Friday 15:00 - 15:45}}}
\noindent
\vspace{0.5\baselineskip}
\framebox[5in][c]{{\Large\sffamily\textbf{Friday,  15:00 to 15:45}}}
\noindent
\begin{tabular*}{5in}[l]{@{}p{3.9in}@{}r@{}}
    {\sffamily\large\textbf{Break and Exhibits}} & 
    {\raggedright\sffamily\large\textbf{Exhibit Hall A}} 
\end{tabular*}    

\cfoot{\colorbox[gray]{0.45}{\color{white}\textsf{Friday 15:00 - 16:30}}}
\vspace{2em}
\noindent
\vspace{0.5\baselineskip}
\framebox[5in][c]{{\Large\sffamily\textbf{Friday,  15:00 to 16:30}}}
\vspace{1em}
\noindent
\begin{tabular*}{5in}[l]{@{}p{3.9in}@{}r}
    {\sffamily\large\textbf{NSF Showcase \#4}} & 
    {\raggedright\sffamily\large\textbf{Exhibit Hall A}} 
\end{tabular*}    
\begin{itemize}
     \item {{\sffamily\textbf{FRABJOUS CS - Framing a Rigorous Approach to Beauty and Joy for Outreach to Underrepresented Students in Computing at Scale, }} Daniel Garcia, Tiffany Barnes, and Brian Harvey } 
     \item {{\sffamily\textbf{Exploring Computer Science - An Equitable Learning Model for Democratizing K-12 Computer Science Education, }} Joanna Goode and Gail Chapman} 
     \item {{\sffamily\textbf{How to Broaden Participation and Scaleb Up Computational Thinking by Bringing Game Design into Middle Schools, }} Alexander Repenning, Kris Gutierrez, Jeffrey Kidder, and David Webb} 
     \item {{\sffamily\textbf{Cybersecurity Laboratory: Enhancing the Hands-on Experience in Cybersecurity Education, }}  Susanne Wetzel} 
\end{itemize}    


\newpage
\cfoot{\colorbox[gray]{0.45}{\color{white}\textsf{Friday 15:00 - 17:00}}}
\addcontentsline{toc}{subsubsection}{Poster Session II}
\noindent
\framebox[5in][c]{{\Large\sffamily\textbf{Friday,  15:00 to 17:00}}}
\begin{longtable}{@{}p{0.75in}@{}p{3.25in}@{}r}
   {\sffamily\large\textbf{POSTER}} &
   {\raggedright\sffamily\large\textbf{Poster Session II}} & 
   {\sffamily\large\textbf{Exhibit Hall A}} \\ \\
\multicolumn{3}{@{}p{5in}}{\sffamily\raggedright\textbf{Implementing and Assessing a Blended CS1 Course}} \\
\multicolumn{3}{@{}p{5in}}{\raggedright John Wright, \textit{Juniata College}} \\[0.5em]
\multicolumn{3}{@{}p{5in}}{\sffamily\raggedright\textbf{Designing with Projects in Mind: An Approach for Creating Authentic (and Manageable) Programming Projects}} \\
\multicolumn{3}{@{}p{5in}}{\raggedright Scott Turner, \textit{UNC Pembroke}} \\[0.5em]
\multicolumn{3}{@{}p{5in}}{\sffamily\raggedright\textbf{Integrating Elementary Computational Modeling and Programming Principles}} \\
\multicolumn{3}{@{}p{5in}}{\raggedright Jose Garrido, \textit{Kennesaw State University}} \\[0.5em]
\multicolumn{3}{@{}p{5in}}{\sffamily\raggedright\textbf{RoboLIFT: Simple GUI-Based Unit Testing of Student-Written Android Applications}} \\
\multicolumn{3}{@{}p{5in}}{\raggedright Anthony Allevato and Stephen Edwards, \textit{Virginia Tech}} \\[0.5em]
\multicolumn{3}{@{}p{5in}}{\sffamily\raggedright\textbf{OpenDSA: A Creative Commons Active-eBook}} \\
\multicolumn{3}{@{}p{5in}}{\raggedright Eric Fouh, Maoyuan Sun and Clifford Shaffer, \textit{Virginia Tech}} \\[0.5em]
\multicolumn{3}{@{}p{5in}}{\sffamily\raggedright\textbf{Active Learning in Computer Science Education Using Meta-Cognition}} \\
\multicolumn{3}{@{}p{5in}}{\raggedright Murali Mani and Quamrul Mazumder, \textit{University of Michigan, Flint}} \\[0.5em]
\multicolumn{3}{@{}p{5in}}{\sffamily\raggedright\textbf{Dynamic Programming Across the CS Curriculum}} \\
\multicolumn{3}{@{}p{5in}}{\raggedright Yana Kortsarts, \textit{Widener University}; Vasily Kolchenko, \textit{New York City College of Technology  The City University of New York}} \\ \\
\multicolumn{3}{@{}p{5in}}{\sffamily\raggedright\textbf{50 Ways to be a FOSSer:  Simple ways to involve students \& faculty}} \\
\multicolumn{3}{@{}p{5in}}{\raggedright Clifton Kussmaul, \textit{Muhlenberg College}; Heidi Ellis, \textit{Western New England University}; Greg Hislop, \textit{Drexel University}} \\ \\
\multicolumn{3}{@{}p{5in}}{\sffamily\raggedright\textbf{Teaching Computer Science and programming concepts using LEGO NXT and TETRIX Robotics, and Computer Science Unplugged activities}} \\
\multicolumn{3}{@{}p{5in}}{\raggedright Daniela Marghitu, Taha Ben Brahim and John Weaver, \textit{Auburn University}} \\[0.5em]
\multicolumn{3}{@{}p{5in}}{\sffamily\raggedright\textbf{Using POGIL to Teach Students To Be Better Problem Solvers}} \\
\multicolumn{3}{@{}p{5in}}{\raggedright Helen Hu, \textit{Westminster College}} \\ \\
\multicolumn{3}{@{}p{5in}}{\sffamily\raggedright\textbf{Developing a Gaming Concentration in the Computer Science Curriculum  at an HBCU}} \\
\multicolumn{3}{@{}p{5in}}{\raggedright Jinghua Zhang and Elva Jones, \textit{Winston-Salem State University}} \\[0.5em]
\multicolumn{3}{@{}p{5in}}{\sffamily\raggedright\textbf{OSSIE: An Open Source Software Defined Radio (SDR) Toolset for Education and Research}} \\
\multicolumn{3}{@{}p{5in}}{\raggedright Jason Snyder, \textit{Virginia Tech}} \\ \\
\multicolumn{3}{@{}p{5in}}{\sffamily\raggedright\textbf{Implementing a Communication Intensive Core Course in the CS Curriculum:  A Survey of Methods}} \\
\multicolumn{3}{@{}p{5in}}{\raggedright Jean French, \textit{Coastal Carolina University}} \\ \\
\multicolumn{3}{@{}p{5in}}{\sffamily\raggedright\textbf{The Reflective Mentor: Charting Undergraduates' Responses to Computer Science Service Learning}} \\
\multicolumn{3}{@{}p{5in}}{\raggedright Quinn Burke, Yasmin Kafai, Jean Griffin, Rita Powell, Michele Grab, Susan Davidson and Joseph Sun, \textit{University of Pennsylvania}} \\ \\
\multicolumn{3}{@{}p{5in}}{\sffamily\raggedright\textbf{Teaching Cryptography Using Hands-on Labs}} \\
\multicolumn{3}{@{}p{5in}}{\raggedright Li Yang, \textit{University of Tennessee at Chattanooga}; Joseph Kizza, \textit{Univeristy of Tennessee at Chattanooga}; Andy Wang, \textit{Southern Polytechnic State University}; Chung-Han Chen, \textit{Tuskegee University}} \\ \\
\multicolumn{3}{@{}p{5in}}{\sffamily\raggedright\textbf{From Drawing to Programming Attracting Middle-School Students to Programming through Self-Disclosing Code}} \\
\multicolumn{3}{@{}p{5in}}{\raggedright Jennelle Nystrom, Pelle Hall, Andrew Hirakawa and Samuel Rebelsky, \textit{Grinnell College}} \\ \\
\multicolumn{3}{@{}p{5in}}{\sffamily\raggedright\textbf{Proposed Revisions to the Social and Professional Knowledge Area for CS2013}} \\
\multicolumn{3}{@{}p{5in}}{\raggedright Carol Spradling, \textit{Northwest Missouri State University}; Florence Appel, \textit{Saint Xavier University}; Elizabeth Hawthorne, \textit{Union County College}} \\ \\
\multicolumn{3}{@{}p{5in}}{\sffamily\raggedright\textbf{A Better API for Java Reflection}} \\
\multicolumn{3}{@{}p{5in}}{\raggedright Zalia Shams, \textit{Virginia Tech}} \\ \\
\multicolumn{3}{@{}p{5in}}{\sffamily\raggedright\textbf{Hands-on Labs for a Mini-Course on Mobile Application Development}} \\
\multicolumn{3}{@{}p{5in}}{\raggedright Qusay H. Mahmoud, Nicholas Mair, Younis Mohamed and Sunny Dhillon, \textit{University of Guelph}} \\ \\
\multicolumn{3}{@{}p{5in}}{\sffamily\raggedright\textbf{CEOHP Evaluation, Evolution, and Archival}} \\
\multicolumn{3}{@{}p{5in}}{\raggedright Vicki Almstrum, \textit{Strayer University}; Barbara Owens, \textit{Southwestern University}; Mary Last, \textit{CEOHP}; Deepa Muralidhar, \textit{North Gwinnett High School}} \\ \\
\multicolumn{3}{@{}p{5in}}{\sffamily\raggedright\textbf{CodeTrainer Teacher Authoring System: Facilitating User-Created Content in an Intelligent Tutoring System}} \\
\multicolumn{3}{@{}p{5in}}{\raggedright Christy McGuire, Thomas Harris and Jonathan Steinhart, \textit{Tutor Technologies, Inc}; Leigh Ann Sudol-DeLyser, \textit{Carnegie Mellon University}} \\ \\
\multicolumn{3}{@{}p{5in}}{\sffamily\raggedright\textbf{Comparing Feature Sets within Visual and Command Line Environments and their effect on Novice Programming}} \\
\multicolumn{3}{@{}p{5in}}{\raggedright Edward Dillon, Jr., Monica Anderson-Herzog and Marcus Brown, \textit{University of Alabama}} \\[1em]
\multicolumn{3}{@{}p{5in}}{\sffamily\raggedright\textbf{Exploring Connected Worlds}} \\
\multicolumn{3}{@{}p{5in}}{\raggedright Jeffrey Forbes, \textit{Duke University}} \\[1em]
\multicolumn{3}{@{}p{5in}}{\sffamily\raggedright\textbf{Teaching parallel computing with higher-level languages and compelling examples}} \\
\multicolumn{3}{@{}p{5in}}{\raggedright Jens Mache, Christopher T. Mitchell and Julian H. Dale, \textit{Lewis \& Clark College}; David P. Bunde, Casey Samoore, Sung Joo Lee and Johnathan Ebbers, \textit{Knox College}} 
\end{longtable}
\newpage
\cfoot{\colorbox[gray]{0.45}{\color{white}\textsf{Friday 15:45 - 17:00}}}
\vspace{-\baselineskip}
\noindent
\framebox[5in][c]{{\Large\sffamily\textbf{Friday,  15:45 to 17:00}}}
\begin{longtable}[l]{@{}l@{}l@{}r}
    \parbox[t]{1in}{\sffamily\large\textbf{SPECIAL SESSION}} & 
    \parbox[t]{3in}{\sffamily\raggedright\large\textbf{Understanding NSF Funding Opportunities}} & 
    \parbox[t]{1in}{\sffamily\raggedleft\large\textbf{301AB}} \\
% row 2    
    Chair: & 
    Suzanne Westbrook, \textit{National Science Foundation}  \\[0.5em]
% row 3
    Participants: & 
    \multicolumn{2}{@{}l}{\parbox{3.75in}{Victor Piotrowski, Jeff Forbes, Harriet Taylor and Mimi McClure, \textit{National Science Foundation} }} \\[2em]
% row 4
    \multicolumn{3}{@{}p{5in}}{\small This session highlights programs in the National Science Foundation’s Division of Undergraduate Education, Office of Cyberinfrastructure and Directorate of Computer and Information Science and Engineering. The focus is on providing descriptions of several programs of interest to college faculty and discussing the requirements and guidelines for programs in these areas. It includes a description of the proposal and review processes as well as strategies for writing competitive proposals. Participants are encouraged to discuss procedural issues with the presenters.}
\end{longtable}
\vspace{-\baselineskip}
\begin{longtable}[l]{@{}l@{}l@{}r}
    \parbox[t]{1in}{\sffamily\large\textbf{PANEL}} & 
    \parbox[t]{3in}{\sffamily\raggedright\large\textbf{Teaching Outside the Text}} & 
    \parbox[t]{1in}{\sffamily\raggedleft\large\textbf{305B}} \\
% row 2    
    Chair: & 
    Lester Wainwright, \textit{Charlottesville High School}  \\[0.5em]
% row 3
    Participants: & 
    \multicolumn{2}{@{}l}{\parbox{3.75in}{Renee Ciezki, \textit{Estrella Mountain Community College}; Barbara Ericson, \textit{Georgia Institute of Technology}; Glen Martin, \textit{TAG Magnet High School} }} \\[2em]
% row 4
    \multicolumn{3}{@{}p{5in}}{\small We know that students bring diverse experiences and an assortment of learning styles into our classrooms.  We greet them and hand out a syllabus listing the required book(s).  One size does not fit all when it comes to textbooks.  In this session, participants will discover teaching activities that can be used to supplement any text:   hands-on, interesting and fun activities that help students understand CS topics. Members of the AP Computer Science-A Development Committee will share these resources and lead a discussion of proven strategies and lesson ideas for teaching outside the textbook.}
\end{longtable}
\vspace{-\baselineskip}
\begin{longtable}[l]{@{}l@{}l@{}r}
    \parbox[t]{1in}{\sffamily\large\textbf{SPECIAL SESSION}} & 
    \parbox[t]{3in}{\sffamily\raggedright\large\textbf{Computing Engineering Review Task Force Report}} & 
    \parbox[t]{1in}{\sffamily\raggedleft\large\textbf{306C}} \\
% row 2    
    Chair: & 
    John Impagliazzo, \textit{Hofstra University}  \\[0.5em]
% row 3
    Participants: & 
    \multicolumn{2}{@{}l}{\parbox{3.75in}{Susan Conry, \textit{Clarkson University}; Eric Durant, \textit{Milwaukee School of Engineering}; Andrew McGettrick, \textit{University of Strathclyde}; Timothy Wilson, \textit{Embry-Riddle Aeronautical University}; Mitch Thornton, \textit{Southern Methodist University} }} \\[2em]
% row 4
    \multicolumn{3}{@{}p{5in}}{\small The ACM and the IEEE Computer Society created the CE2004 Review Task Force (RTF) and charged it with the task of reviewing and determining the extent to which the CE2004 document required revisions. The RTF recommended keeping the structure and the vast majority of the content of the original CE2004 document. It also recommended that contemporary topics should be strengthened or added while de-emphasizing other topics. Additionally, the RTF recommended that the two societies form a joint special-purpose committee to update and edit the earlier document and to seek input and review from the computer engineering industrial and academic communities. The presentation will provide insights in the RTF findings and thoughts on how a future computer engineering report might evolve.}
\end{longtable}
\newpage
\begin{longtable}{@{}p{0.75in}@{}p{3.25in}@{}r}
   {\sffamily\large\textbf{PAPERS}} &
   {\raggedright\sffamily\large\textbf{Projects}} & 
   {\sffamily\large\textbf{302A }} \\
%row 2
   Chair:  & 
   {\raggedright Jeff Gray, \textit{University of Alabma}} & \\ \\
{\sffamily 15:45}& 
\multicolumn{2}{@{}p{3.75in}}{\sffamily\raggedright\textbf{Social Sensitivity and Classroom Team Projects: An Empirical Investigation}} \\
& \multicolumn{2}{@{}p{3.75in}}{\raggedright Lisa Bender, Gursimran Walia, Krishna Kambhampaty and Kendall E. Nygard, \textit{North Dakota State University}; Travis E. Nygard, \textit{Ripon College}} \\ \\
\multicolumn{3}{@{}p{5in}}{\small Team work is the norm in major development projects and industry is continually striving to improve team effectiveness.  Researchers have established that teams with high levels of social sensitivity tend to perform well when completing a variety of specific collaborative tasks. Our claim is that, the social sensitivity can be a key component in predicting the performance of teams that carry out major projects. This paper reports the results from an empirical study that investigates whether social sensitivity is correlated with the performance of student teams on large semester-long projects. The overall result supports our claim. It suggests, therefore, that educators in computer-related disciplines should take the concept of social sensitivity seriously as an aid to productivity.} \\ \\
{\sffamily 16:10}& 
\multicolumn{2}{@{}p{3.75in}}{\sffamily\raggedright\textbf{Taming Complexity in Large Scale Systems Projects}} \\
& \multicolumn{2}{@{}p{3.75in}}{\raggedright Shimon Schocken, \textit{IDC Herzliya}} \\ \\
\multicolumn{3}{@{}p{5in}}{\small Engaging students in large software development projects is an important objective, since it exposes design and programming challenges that come to play only with scale. Alas, large scale projects can be monstrously complex – to the extent of being infeasible in academic settings. We describe a framework and a set of principles that enable students to develop large scale systems – e.g. a complete hardware platform or a compiler – in several semester weeks.} \\ \\
{\sffamily 16:35}& 
\multicolumn{2}{@{}p{3.75in}}{\sffamily\raggedright\textbf{An Approach for Evaluating FOSS Projects for Student Participation}} \\
& \multicolumn{2}{@{}p{3.75in}}{\raggedright Heidi Ellis, \textit{Western New England University}; Michelle Purcell and Gregory Hislop, \textit{Drexel University}} \\ \\
\multicolumn{3}{@{}p{5in}}{\small Free and Open Source Software (FOSS) offers a transparent development environment and community in which to involve students. Students can learn much about software development and professionalism by contributing to an on-going project. However, the number of FOSS projects is very large and there is a wide range of size, complexity, domains, and communities, making selection of an ideal project for students difficult. This paper addresses the need for guidance when selecting a FOSS project for student involvement by presenting an approach for FOSS project selection based on clearly identified criteria. The approach is based on several years of experience involving students in FOSS projects.} \\ \\
\end{longtable}


\newpage
\begin{longtable}{@{}p{0.75in}@{}p{3.25in}@{}r}
   {\sffamily\large\textbf{PAPERS}} &
   {\raggedright\sffamily\large\textbf{Alice and Scratch}} & 
   {\sffamily\large\textbf{302B }} \\
%row 2
   Chair:  & 
   {\raggedright Kelly Powers, \textit{Advanced Math \& Science Academy Charter School}} & \\ \\
{\sffamily 15:45}& 
\multicolumn{2}{@{}p{3.75in}}{\sffamily\raggedright\textbf{Integrating Computing into Middle School Disciplines Through Projects}} \\
& \multicolumn{2}{@{}p{3.75in}}{\raggedright Susan Rodger, Melissa Dalis, Peggy Li, Liz Liang and Wenhui Zhang, \textit{Duke University}; Chitra Gadwal, \textit{UMBC}; Francine Wolfe, \textit{Benedictine College}} \\ \\
\multicolumn{3}{@{}p{5in}}{\small For four years we have been integrating computing into a variety of middle 
school disciplines via the Alice programing language.  This paper describes our efforts over
the past two years in creating model projects for students to build in all
disciplines, and our most recent focus on science and mathematics
projects. For science we have introduced experiments in Alice and the tools
needed for them. In mathematics we have created projects to increase their
understanding of programming and to use the projects to increase their 
understanding of mathematics. We also discuss our efforts in workshops to 
teach K-12 teachers Alice and an analysis of the teachers' lesson plans and 
worlds developed in the most recent workshop.} \\ \\
{\sffamily 16:10}& 
\multicolumn{2}{@{}p{3.75in}}{\sffamily\raggedright\textbf{Children Learning Computer Science Concepts via Alice Game-Programming}} \\
& \multicolumn{2}{@{}p{3.75in}}{\raggedright Linda Werner, \textit{University of California, Santa Cruz}; Shannon Campe and Jill Denner, \textit{ETR Associates}} \\ \\
\multicolumn{3}{@{}p{5in}}{\small Programming environments that incorporate drag-and-drop methods and many pre-defined objects and operations are being widely used in K-12 settings. But can students as young as those in middle school learn complex computer science concepts using these programming environments when computer science is not the focus of the course? In this paper, we describe a semester-long game-programming course where 325 middle school students used Alice. We report on our analysis of 225 final games where we measured the frequency of successful execution of programming constructs. Our results show that many games exhibit successful uses of high level computer science concepts such as student-created abstractions, concurrent execution, and event handlers.} \\ \\
{\sffamily 16:35}& 
\multicolumn{2}{@{}p{3.75in}}{\sffamily\raggedright\textbf{The Writers’ Workshop for Youth Programmers: Digital Storytelling with Scratch in Middle School Classrooms}} \\
& \multicolumn{2}{@{}p{3.75in}}{\raggedright Quinn Burke and Yasmin B. Kafai, \textit{University of Pennsylvania}} \\ \\
\multicolumn{3}{@{}p{5in}}{\small This study investigates the potential to introduce basic programming concepts to middle school children within the context of a classroom writing-workshop. In this paper we describe how students drafted, revised, and published their own digital stories using the introductory programming language Scratch and in the process learned fundamental CS concepts as well as the wider connection between programming and writing as interrelated processes of composition.} \\ \\
\end{longtable}


\newpage
\begin{longtable}{@{}p{0.75in}@{}p{3.25in}@{}r}
   {\sffamily\large\textbf{PAPERS}} &
   {\raggedright\sffamily\large\textbf{Active Learning II}} & 
   {\sffamily\large\textbf{306A }} \\
%row 2
   Chair:  & 
   {\raggedright Douglas Kranch, \textit{North Central State College}} & \\ \\
{\sffamily 15:45}& 
\multicolumn{2}{@{}p{3.75in}}{\sffamily\raggedright\textbf{A Software Craftsman’s Approach to Data Structures}} \\
& \multicolumn{2}{@{}p{3.75in}}{\raggedright Arto Vihavainen, Matti Luukkainen and Thomas Vikberg, \textit{University of Helsinki}} \\ \\
\multicolumn{3}{@{}p{5in}}{\small Data Structures (CS2) courses and course books do not usually put much emphasis in the process of how a data structure is engineered or invented. Instead, algorithms are readily given, and the main focus is in the mathematical complexity analysis of the algorithms. We present an alternative approach on presenting data structures using worked examples, i.e., by explicitly displaying the process that leads to the invention and creation of a data stucture and its algorithms. Our approach is heavily backed up by some of the best programming practices advocated by the Agile and Software Craftmanship communities and it brings the often mathematically oriented CS2 course closer to modern software engineering and practical problem solving, without a need for compromise in proofs and analysis.} \\ \\
{\sffamily 16:10}& 
\multicolumn{2}{@{}p{3.75in}}{\sffamily\raggedright\textbf{Jutge.org: An Educational Programming Judge}} \\
& \multicolumn{2}{@{}p{3.75in}}{\raggedright Petit Jordi and Roura Salvador, \textit{Universitat Politècnica de Catalunya}; Giménez Omer, \textit{Google}} \\ \\
\multicolumn{3}{@{}p{5in}}{\small Jutge.org is an educational programming judge where students can solve more than 800 problems using 22 programming languages. The verdict of their solutions is computed using exhaustive test sets run under time, memory and security restrictions. By contrast to many popular online judges, Jutge.org is designed for students and 
instructors: On one hand, the problem repository is mainly aimed to beginners, with a clear organization and gradding. On the other 
hand, the system is designed as a virtual learning environment where instructors can administer their own courses, manage their roster of students and tutors, add problems, attach documents, create lists of problems, assignments, contests and exams. This paper presents Jutge.org and offers some case studies of courses using it.} \\ \\
{\sffamily 16:35}& 
\multicolumn{2}{@{}p{3.75in}}{\sffamily\raggedright\textbf{Integrating Formal Verification in an Online Judge for e-Learning Logic Circuit Design}} \\
& \multicolumn{2}{@{}p{3.75in}}{\raggedright Javier De San Pedro, Josep Carmona, Jordi Cortadella and Jordi Petit, \textit{Universitat Politècnica de Catalunya}} \\ \\
\multicolumn{3}{@{}p{5in}}{\small This paper investigates the use of formal verification techniques to create online judges that can assist in teaching logic circuit design. 
Formal verification not only contributes to give an exact assessment about correctness, but also saves the instructor the tedious task of designing test cases.
The paper explains how formal verification has been integrated in an online judge. It also describes the courseware created for a course on logic circuits and the successful experience of using it in a one-week summer course with students from secondary and high school.} \\ \\
\end{longtable}


\newpage
\begin{longtable}{@{}p{0.75in}@{}p{3.25in}@{}r}
   {\sffamily\large\textbf{PAPERS}} &
   {\raggedright\sffamily\large\textbf{Non-majors}} & 
   {\sffamily\large\textbf{306B }} \\
%row 2
   Chair:  & 
   {\raggedright Derek Schuurman, \textit{Redeemer University College}} & \\ \\
{\sffamily 15:45}& 
\multicolumn{2}{@{}p{3.75in}}{\sffamily\raggedright\textbf{Computing For STEM Majors:  Enhancing Non CS Majors’ Computing Skills}} \\
& \multicolumn{2}{@{}p{3.75in}}{\raggedright Joel Adams and Randall Pruim, \textit{Calvin College}} \\ \\
\multicolumn{3}{@{}p{5in}}{\small One of the challenges facing the U.S. technological workforce is that fewer college graduates are being prepared for computing careers.  Besides trying to attract more CS majors, another approach is to (i) design a computing curriculum that appeals to students and faculty from non-CS disciplines, (ii) use special scholarships to attract students to that curriculum, and (iii) sponsor faculty development workshops for non-CS departments.  In this paper, we detail this approach, using a new introductory course oriented to science majors, and scholarships funded by the National Science Foundation Scholarships for Science, Technology, Engineering, and Mathematics (NSF S-STEM) program.  We also present several success stories that this approach has produced in its first two years.} \\ \\
{\sffamily 16:10}& 
\multicolumn{2}{@{}p{3.75in}}{\sffamily\raggedright\textbf{Operations Research: Broadening Computer Science In A Liberal Arts College}} \\
& \multicolumn{2}{@{}p{3.75in}}{\raggedright Barbara Anthony, \textit{Southwestern University}} \\ \\
\multicolumn{3}{@{}p{5in}}{\small Operations research, while not traditionally taught at many small or liberal arts colleges, can be a significant asset to the offerings of a computer science department. Often seen as a discipline at the intersection of mathematics, computer science, business, and engineering, it has great interdisciplinary potential and practical appeal, allowing for recruitment of students who may not consider taking a CS0 or CS1 course. Offering this course not only benefited computer science majors who appreciated the applications and different perspectives, but it provided a means for the department to serve a wider population, increased interdisciplinary education, and resulted in a filled-to-capacity upper-level course in computer science for the first time in recent memory.} \\ \\
{\sffamily 16:35}& 
\multicolumn{2}{@{}p{3.75in}}{\sffamily\raggedright\textbf{Beyond Competency: A Context-Driven CS0 Course}} \\
& \multicolumn{2}{@{}p{3.75in}}{\raggedright Jeff Cramer and Bill Toll, \textit{Taylor University}} \\ \\
\multicolumn{3}{@{}p{5in}}{\small In the process of revising our general education course we attempted to answer the question “What should a graduate of a liberal arts university understand about computational technology?” University students may know more about narrow areas of technology but the true impact on their lives cannot be understood without an appreciation for the nature and limitations of the technology.  This paper presents a set of assumptions about the impact of technology on individuals and society and describes elements of a computing context designed to enable students to critically evaluate the technology that has such an impact on their lives.  Assessment of the approach indicates that students are more aware of the impact of technology and the importance of an understanding of the technology.} \\ \\
\end{longtable}

\newpage
% \begin{longtable}[l]{@{}p{1in}@{}p{3in}@{}r}
%     {\sffamily\large\textbf{Supporter Session}} & 
%     {\sffamily\large\textbf{TBA}} & 
%     {\sffamily\large\textbf{302C}} \\
% \end{longtable}    
\begin{longtable}[l]{@{}p{1in}@{}p{3in}@{}r}
    {\sffamily\large\textbf{Supporter Session}} & 
    {\sffamily\large\textbf{Microsoft}} & 
    {\sffamily\large\textbf{305A}} \\ \\
    \multicolumn{3}{@{}p{5in}}{\raggedright\sffamily\large\textbf{ Cloud in a Classroom:  Faculty Experiences}} \\
    & Nilajan Banerjee, \emph{University of Arkansas} \\
    & Chia-Chi Teng, \emph{Brigham Young, University} \\
    & Alexander Schmidt, \emph{University of Potsdam} \\
    & Moderator:  Arkady Retik, \emph{Microsoft} \\
    \multicolumn{3}{@{}p{5in}}{\small Cloud computing introduces new and exciting opportunities for computing industry.  To realize the potential of cloud computing in higher education, one must think about the cloud as a holistic platform for creating new services, new experiences, and new methods for research and teaching.  Pursuing these goals in the current set of the CS courses presents a broad range of interesting questions.  Come along to hear about the cloud-based teaching resources available and how they have been used in universities world-wide.  This panel will provide an opportunity for SIGCSE attendees to hear from faculty who have been teaching CS courses using Windows Azure ask questions and discuss and share their own experiences.}
\end{longtable}    
\cfoot{\colorbox[gray]{0.45}{\color{white}\textsf{Friday 17:10 - 17:55}}}
\vspace{2em}
\noindent
\framebox[5in][c]{{\Large\sffamily\textbf{Friday,  17:10 to 17:55}}}
\begin{longtable}[l]{@{}p{1in}@{}p{3in}@{}r}
    {\sffamily\large\textbf{}} & 
    {\sffamily\large\textbf{SIGCSE Business Meeting}} & 
    {\sffamily\large\textbf{302A}} \\
\end{longtable}    
\cfoot{\colorbox[gray]{0.45}{\color{white}\textsf{Friday 18:00 - 18:45}}}
\vspace{2em}
\noindent
\framebox[5in][c]{{\Large\sffamily\textbf{Friday,  18:00 to 18:45}}}
\begin{longtable}[l]{@{}p{1in}@{}p{3in}@{}r}
    {\sffamily\large\textbf{}} & 
    {\sffamily\large\textbf{Going Greenfoot}} & 
    {\sffamily\large\textbf{302C}} \\
% \end{longtable}    
% \begin{longtable}[l]{@{}p{1in}@{}p{3in}@{}r}
    {\sffamily\large\textbf{}} & 
    {\sffamily\large\textbf{CCSC Business Meeting}} & 
    {\sffamily\large\textbf{305B}} \\
\end{longtable}    
\noindent
\framebox[5in][c]{{\Large\sffamily\textbf{Friday,  18:00 to 18:50}}}
\begin{longtable}[l]{@{}p{1in}@{}p{3in}@{}r}
    {\sffamily\large\textbf{}} & 
    {\sffamily\large\textbf{\raggedright NCWIT Academic Alliance Reception}} & 
    {\sffamily\large\textbf{305A}} \\
\end{longtable}    

\newpage
\fancyhead[RO,LE]{\colorbox[gray]{0.45}{\color{white}\textsf{Workshops}}}
\cfoot{\colorbox[gray]{0.45}{\color{white}\textsf{Friday 19:00 - 22:00}}}
\noindent
\framebox[5in][c]{{\Large\sffamily\textbf{Friday,  19:00 to 22:00}}}
\addcontentsline{toc}{subsubsection}{Workshops}
\begin{longtable}[l]{@{}l@{}l@{}r}
    \parbox[t]{0.25in}{\sffamily\large\textbf{16.}} & 
    \parbox[t]{3.75in}{\raggedright\sffamily\large\textbf{Intellectual Property Law Basics for Computer Science Instructors}} & 
    {\sffamily\large\textbf{205}} \\[1.5em]
% row 3
    & \multicolumn{2}{@{}l}{\parbox{4.75in}{David G. Kay, \textit{UC Irvine} }} \\[1.5em]
% row 4
    \multicolumn{3}{@{}p{5in}}{\small Increasingly the practice of 
computing involves legal issues.  
Patenting algorithms, domain 
name poaching, downloading 
music, and ``re-using'' HTML and 
graphics from web sites all raise 
questions of intellectual 
property (IP) law (which includes 
patents, copyrights, trade 
secrets, and trademarks).  In the 
classroom, computer science 
educators often confront 
questions that have legal 
ramifications. The presenter, 
who is both a computer 
scientist and a lawyer, will 
introduce the basics of 
intellectual property law to give 
instructors a framework for 
recognizing the issues, 
answering students' questions, 
debunking the most egregious 
misconceptions about IP, and 
understanding generally how 
the law and computing interact.  
All CS educators are welcome; 
no computer is required.}
\end{longtable}
\begin{longtable}[l]{@{}l@{}l@{}r}
    \parbox[t]{0.25in}{\sffamily\large\textbf{17.}} & 
    \parbox[t]{3.75in}{\raggedright\sffamily\large\textbf{Teaching and Learning Computing via Social Gaming with Pex4Fun}} & 
    {\sffamily\large\textbf{301A}} \\[1.5em]
% row 3
    & \multicolumn{2}{@{}l}{\parbox{4.75in}{Nikolai Tillmann, Jonathan de Halleux and Judith Bishop, \textit{Microsoft Research}; Tao Xie, \textit{North Carolina State University} }} \\[1.5em]
% row 4
    \multicolumn{3}{@{}p{5in}}{\small Pex4Fun (pexforfun.com) is a 
web-based serious gaming 
environment for teaching 
computing at many levels, from 
high school all the way through 
graduate courses. Unique to the 
Pex4Fun experience is a cloud-based program evaluation 
engine based on dynamic 
symbolic execution and SMT-solving, which provides 
customized feedback to the 
student and automated grading 
for the teacher. Thus, Pex4Fun 
connects teachers, curriculum 
authors, and students in a 
social experience, tracking and 
streaming progress updates in 
real time. This workshop 
involves creating and teaching 
course materials at Pex4Fun. 
Participants should bring a 
laptop computer. The intended 
audience includes all levels of 
CS educators who are interested 
in integrating educational 
technology in their teaching 
environments.}
\end{longtable}
\begin{longtable}[l]{@{}l@{}l@{}r}
    \parbox[t]{0.25in}{\sffamily\large\textbf{18.}} & 
    \parbox[t]{3.75in}{\raggedright\sffamily\large\textbf{Welcome to Makerland: A First Cultural Immersion into Open Source Communities}} & 
    {\sffamily\large\textbf{301B}} \\[1.5em]
% row 3
    & \multicolumn{2}{@{}l}{\parbox{4.75in}{Mel Chua, \textit{Purdue University}; Sebastian Dziallas, \textit{Franklin W. Olin College of Engineering}; Heidi Ellis, \textit{Western New England University}; Greg Hislop, \textit{Drexel University}; Karl Wurst, \textit{Worcester State University} }} \\[1.5em]
% row 4
    \multicolumn{3}{@{}p{5in}}{\small Participating in free and open 
source (FOSS) software 
communities provides students 
with authentic learning while 
supplying instructors with a 
wide variety of educational 
opportunities including coding, 
testing, documentation, 
professionalism and more. 
However, instructors may be 
unfamiliar with how FOSS 
communities work and therefore 
may be reluctant to involve 
students. This workshop is a 
subset of material used in Red 
Hat's Professors' Open Source 
Summer Experience 
(http://communityleadershipteam.org/posse), now in its third 
year of successfully providing a 
ramp to FOSS projects for 
instructors. These instructors 
have demonstrated success in 
involving their students in FOSS 
communities where students 
have contributed code, interface 
design, and more. Laptop 
Required.}
\end{longtable}
\begin{longtable}[l]{@{}l@{}l@{}r}
    \parbox[t]{0.25in}{\sffamily\large\textbf{19.}} & 
    \parbox[t]{3.75in}{\raggedright\sffamily\large\textbf{Computational Art and Creative Coding: Teaching CS1 with Processing}} & 
    {\sffamily\large\textbf{302A}} \\[1.5em]
% row 3
    & \multicolumn{2}{@{}l}{\parbox{4.75in}{Dianna Xu and Deepak Kumar, \textit{Bryn Mawr College}; Ira Greenberg, \textit{Southern Methodist University} }} \\[1.5em]
% row 4
    \multicolumn{3}{@{}p{5in}}{\small This workshop showcases a new 
approach to teaching CS1 using 
computational art as a context. 
Participants will be introduced 
to the Processing programming 
language and environment, 
designed for the construction of 
2D and 3D visual forms. Its IDE 
is light-weight, but well-suited 
for the rapid proto-typing 
needed for dynamic visual work. 
We hope to bring the 
excitement, creativity, and 
innovation fostered by 
Processing into the computer 
science education community. 
Instructors of all experience 
levels are welcome. Hands-on 
portion of the workshop will 
enable participants to explore 
Processing and create visual 
effects on the fly. Course 
materials and handouts 
detailing the software and 
teaching 
resources will be given out.
Laptop Required.}
\end{longtable}
\begin{longtable}[l]{@{}l@{}l@{}r}
    \parbox[t]{0.25in}{\sffamily\large\textbf{20.}} & 
    \parbox[t]{3.75in}{\raggedright\sffamily\large\textbf{AP CS Principles and The Beauty and Joy of Computing Curriculum}} & 
    {\sffamily\large\textbf{302B}} \\[1.5em]
% row 3
    & \multicolumn{2}{@{}l}{\parbox{4.75in}{Daniel Garcia, Brian Harvey, Nathaniel Titterton and Luke Segars, \textit{UC Berkeley}; Tiffany Barnes, \textit{University of North Carolina, Charlotte;} Eugene Lemon, \textit{Ralph J Bunche High School}; Sean Morris, \textit{Albany High School}; Josh Paley, \textit{Henry M. Gunn High School} }} \\ \\
% row 4
    \multicolumn{3}{@{}p{5in}}{\small The Beauty and Joy of 
Computing (BJC) is an 
introductory computer science 
curriculum developed at UC 
Berkeley (and adapted at UNC
Charlotte) intended for high 
school juniors through 
university non-majors. It was 
used in two of the five initial 
pilot programs for the AP CS 
Principles course being 
developed by the College Board 
and the NSF. Our overall goal is 
to support the CS10K project by 
preparing instructors to teach 
the AP CS Principles course 
through the BJC curriculum.
We will share our 
experiences as instructors of the 
course at the university and 
high school level, provide a 
glimpse into a typical week of 
the course, and share details of 
NSF-funded summer 
professional development 
opportunities.
Laptop Required.}
\end{longtable}
\begin{longtable}[l]{@{}l@{}l@{}r}
    \parbox[t]{0.25in}{\sffamily\large\textbf{21.}} & 
    \parbox[t]{3.75in}{\raggedright\sffamily\large\textbf{Peer Instruction in the CS Classroom: A Hands-On Introduction}} & 
    {\sffamily\large\textbf{302C}} \\[1.5em]
% row 3
    & \multicolumn{2}{@{}l}{\parbox{4.75in}{Daniel Zingaro, \textit{University of Toronto}; Cynthia Bailey-Lee and Beth Simon, \textit{University of California, San Diego}; John Glick, \textit{University of San Diego}; Leo Porter, \textit{Skidmore College} }} \\[1.5em]
% row 4
    \multicolumn{3}{@{}p{5in}}{\small We introduce participants to 
Peer Instruction (PI): an active 
learning technique applicable to 
the teaching of many subjects, 
including CS. In PI, students 
work together to exchange 
perspectives and answer 
challenging conceptual 
questions, and are supported by 
short teaching segments. We 
will introduce and motivate PI, 
demonstrate its use  in 
combination with a clicker 
system, and show that PI is 
much more than the use of 
clickers.  Participants will work in groups 
to develop new PI questions 
addressing challenges to their 
students' learning, and discuss 
numerous pedagogical benefits 
conferred through PI. Instructors interested in 
increasing engagement in any 
CS course may attend. 
Participants are encouraged to 
bring current lecture materials. 
Laptop Optional.}
\end{longtable}
\begin{longtable}[l]{@{}l@{}l@{}r}
    \parbox[t]{0.25in}{\sffamily\large\textbf{22.}} & 
    \parbox[t]{3.75in}{\raggedright\sffamily\large\textbf{Incorporating Software Architecture in the Computer Science Curriculum}} & 
    {\sffamily\large\textbf{307}} \\[1.5em]
% row 3
    & \multicolumn{2}{@{}l}{\parbox{4.75in}{Martin Barrett, \textit{East Tennessee State University}; Steve Chenoweth, \textit{Rose-Hulman Institute of Technology}; Larry Jones, \textit{Software Engineering Institute}; Amine Chigani, \textit{Virginia Tech}; Ayse Bener, \textit{Ryerson University}; Mei-Huei Tang, \textit{Gannon University} }} \\ \\
% row 4
    \multicolumn{3}{@{}p{5in}}{\small This workshop introduces and 
incorporates 
software architecture concepts 
into CS and SE curricula.  
Participants will learn 
techniques used in industry to 
specify quality attributes critical 
to architecture and use those 
attributes to drive the system 
structure using common 
architectural styles.  Exercises 
will demonstrate pedagogical 
uses of the 
techniques in CS and SE classes. 
Sample computer science 
curricula with courses that 
integrate workshop material will 
be presented. Presenters will 
lead a brainstorming session to 
help participants develop 
practical methods for using the 
material in their courses. 
Participants will become part of 
a community of educators 
sharing educational resources in 
software architecture.
Laptop Optional.}
\end{longtable}
\begin{longtable}[l]{@{}l@{}l@{}r}
    \parbox[t]{0.25in}{\sffamily\large\textbf{23.}} & 
    \parbox[t]{3.75in}{\raggedright\sffamily\large\textbf{Parallelism and Concurrency for Data-Structures \& Algorithms Courses}} & 
    {\sffamily\large\textbf{305A}} \\[1.5em]
% row 3
    & \multicolumn{2}{@{}l}{\parbox{4.75in}{Robert Chesebrough, \textit{Intel Corporation}; Johnnie Baker, \textit{Kent State University} }} \\[1.5em]
% row 4
    \multicolumn{3}{@{}p{5in}}{\small This workshop is inspired by 
Dan Grossman’s SIGCSE 2011 
workshop on Data Abstractions.  
We review C/C++ conversions 
of the original Java-based 
materials and will include 
material from the Parallel 
Algorithms course at Kent State.  
The workshop will appeal to 
data-structure and algorithms 
course instructors. Workshop 
topics will include divide and 
conquer approaches, work 
sharing concepts, and a scoped 
locking scheme in OpenMP for 
C++ classes. This material is 
driven via core data-structure 
examples (queues, sorting, 
reductions, etc.) and using a 
Fork/Join Framework found in 
OpenMP and Intel® Cilk Plus and 
Intel® Threading Building 
Blocks.  Participants will write 
parallel programs and test them 
on the Intel® Many-core 
Testing Lab. Laptop Required.}
\end{longtable}
\begin{longtable}[l]{@{}l@{}l@{}r}
    \parbox[t]{0.25in}{\sffamily\large\textbf{24.}} & 
    \parbox[t]{3.75in}{\raggedright\sffamily\large\textbf{ARTSI Robotics Roadshow-in-a-Box: Turnkey Solution for Providing Robotics Workshops to Middle and High School Students}} & 
    {\sffamily\large\textbf{305B}} \\[1.5em]
% row 3
    & \multicolumn{2}{@{}l}{\parbox{4.75in}{Monica Anderson, \textit{The University of Alabama}; Dave Touretzky, \textit{Carnegie Mellon University}; Chutima Boonthum-Denecke, \textit{Hampton University} }} \\[1.5em]
% row 4
    \multicolumn{3}{@{}p{5in}}{\small In this half-day tutorial, we will 
introduce the ARTSI “Robotics 
Roadshow-in-a-Box (RRIB)”, a 
single point resource for those 
getting started in 
robotics outreach. The RRIB is a 
kit which contains robots, 
software 
and prepared materials for 
providing robotics workshops 
for middle and high school 
students that focuses on 
showing computer scientists as 
problem solvers and not just 
programmers through activities 
with a larger context.  The RRIB 
fills a need for materials that 
are accessible to those who may 
have limited knowledge of 
robotics or limited experience in 
middle school outreach, 
whether that is undergraduate 
students or faculty researchers 
who might have limited 
outreach experience or 
preparation time. Laptop 
Required.}
\end{longtable}
\begin{longtable}[l]{@{}l@{}l@{}r}
    \parbox[t]{0.25in}{\sffamily\large\textbf{25.}} & 
    \parbox[t]{3.75in}{\raggedright\sffamily\large\textbf{Program by Design: From Animations to Data Structures}} & 
    {\sffamily\large\textbf{306A}} \\[1.5em]
% row 3
    & \multicolumn{2}{@{}l}{\parbox{4.75in}{Kathi Fisler, \textit{WPI}; Stephen Bloch, \textit{Adelphi University} }} \\[1.5em]
% row 4
    \multicolumn{3}{@{}p{5in}}{\small We present the Program by 
Design introductory CS 
curriculum through the lenses 
of graphics, animations, 
algebra, and data structures.  
Animations programming is 
popular for CS1, but many such 
curricula lack clean paths into 
CS2.  Program by Design is 
different.  Using and reinforcing 
concepts from algebra, students 
learn to write animations 
(including standard topics such 
as model/view separation and 
event-handling), then move 
seamlessly into working with 
structured data, lists, trees, and 
objects.  The curriculum 
emphasizes design, testing, and 
writing maintainable programs, 
without losing the engagement 
of animations.  The workshop 
uses lectures and hands-on 
exercises to provide high-
school and college teachers an 
overview of the approach.  See 
www.programbydesign.org.   
Laptop Optional.}
\end{longtable}
\begin{longtable}[l]{@{}l@{}l@{}r}
    \parbox[t]{0.25in}{\sffamily\large\textbf{26.}} & 
    \parbox[t]{3.75in}{\raggedright\sffamily\large\textbf{CS Outreach with App Inventor}} & 
    {\sffamily\large\textbf{306B}} \\[1.5em]
% row 3
    & \multicolumn{2}{@{}l}{\parbox{4.75in}{Michelle Friend, \textit{Stanford University}; Jeff Gray, \textit{University of Alabama} }} \\[1.5em]
% row 4
    \multicolumn{3}{@{}p{5in}}{\small Mobile phone programming can 
provide teens an authentic and 
engaging hook into computer 
science. With App Inventor, 
developed by Google and moved 
to MIT, programming Android 
apps is as easy as clicking 
blocks together. App Inventor 
has been used successfully in 
after school programs, 
roadshows, summer camps, 
teacher workshops, and 
computer science classrooms 
from middle school through 
college. Participants will get an 
overview of App Inventor 
including project ideas and 
sample student code, hear 
outreach planning suggestions, 
write programs, develop 
outreach plans, and see how the 
Java Bridge helps transition from 
App Inventor to Java. Even the 
most time-stretched professor 
or teacher can encourage 
students in computer science 
with App Inventor. Laptop 
Required.}
\end{longtable}
\begin{longtable}[l]{@{}l@{}l@{}r}
    \parbox[t]{0.25in}{\sffamily\large\textbf{27.}} & 
    \parbox[t]{3.75in}{\raggedright\sffamily\large\textbf{Making Mathematical Reasoning Fun: Tool-Assisted, Collaborative Techniques}} & 
    {\sffamily\large\textbf{306C}} \\[1.5em]
% row 3
    & \multicolumn{2}{@{}l}{\parbox{4.75in}{Jason Hallstrom and Murali Sitaraman, \textit{Clemson University}; Joe Hollingsworth, \textit{Indiana University Southeast}; Joan Krone, \textit{Denison University} }} \\[1.5em]
% row 4
    \multicolumn{3}{@{}p{5in}}{\small Is it possible to excite students 
about learning the mathematical 
principles that underly high-quality software? Can we teach 
them to apply these principles 
using modern software tools? 
Can this be accomplished 
without displacing existing 
content? Yes, with the right 
pedagogical principles, teaching 
tools, and classroom exercises. 
This hands-on laboratory will 
introduce a set of principles, 
tools, and exercises that have 
proven to work. By adopting one 
content module at a time, 
educators will better prepare 
students to reason rigorously 
about the software they develop 
and maintain. Fees for this 
workshop will be covered for a 
limited number of attendees 
through an NSF award; limited 
travel support is also available.
Laptop Required.}
\end{longtable}
