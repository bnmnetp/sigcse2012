%!TEX root = program.tex

\fancyhead[RO,LE]{\colorbox[gray]{0.45}{\color{white}\textsf{Workshops}}}
\fancyfoot[LE,RO]{\thepage}
\addcontentsline{toc}{subsection}{Wednesday}
\addcontentsline{toc}{subsubsection}{Workshops}
\cfoot{\colorbox[gray]{0.45}{\color{white}\textsf{Wednesday 19:00 - 22:00}}}
\noindent
\framebox[5in][c]{{\Large\sffamily\textbf{Wednesday,  19:00 to 22:00}}}
\begin{longtable}[l]{@{}l@{}l@{}r}
    \parbox[t]{0.25in}{\sffamily\large\textbf{1.}} & 
    \parbox[t]{3.75in}{\raggedright\sffamily\large\textbf{Using Social Networking to Improve Student Learning Through Classroom Salon}} & 
    {\sffamily\large\textbf{201}} \\[1.5em]
% row 3
    & \multicolumn{2}{@{}l}{\parbox{4.75in}{John Barr, \textit{Ithaca College}; Ananda Gunawardena, \textit{Carnegie Mellon University} }} \\[1.5em]
% row 4
    \multicolumn{3}{@{}p{5in}}{\small This workshop introduces an 
innovative social collaboration 
tool called Classroom Salon 
(CLS). Developed at Carnegie 
Mellon University, CLS is a 
combination of electronic 
books, social networks, and 
analytic tools.  It enables 
students to learn by 
participating in social networks 
and allows instructors to easily 
analyze student participation. 
The workshop covers extant 
social networks, introduces CLS 
web-based software (nothing to 
install) and demonstrates the 
use of CLS to help students 
master critical skills such as 
code review, debugging, and 
reading documentation.  
Participants will create Salons, 
learn how to use them in their 
courses, and learn how to use 
the built-in tools to analyze 
student activities.  See 
http://classroomsalon.org.  
Laptop (with wifi) required.}
\end{longtable}

\begin{longtable}[l]{@{}l@{}l@{}r}
    \parbox[t]{0.25in}{\sffamily\large\textbf{2.}} & 
    \parbox[t]{3.75in}{\raggedright\sffamily\large\textbf{Challenges and Opportunities in Conducting Educational Research in the Computer Science Classroom}} & 
    {\sffamily\large\textbf{204}} \\[3em]
% row 3
    & \multicolumn{2}{@{}l}{\parbox{4.75in}{Aman Yadav and Tim Korb, \textit{Purdue University} }} \\[1.5em]
% row 4
    \multicolumn{3}{@{}p{5in}}{\small This workshop will provide CS 
educators with tools to conduct 
educational research. Primary 
objectives of this workshop 
are: (1) learn basic principles of 
research design; (2) learn about 
various types of research 
designs: qualitative vs. 
quantitative; experimental vs. 
quasi-experimental; case 
studies, survey; (3) to practice 
designing research. This 
workshop will help participants 
make informed decisions when 
faced with limitations of 
educational research and collect 
empirical evidence about what 
works in the classroom. In 
addition, we will also discuss 
how to develop robust student 
outcome measures, such as 
surveys and tests. The 
workshop will be beneficial to 
participants who have not yet 
done all of these activities as 
well as those who have some 
background in educational 
research.}
\end{longtable}
\begin{longtable}[l]{@{}l@{}l@{}r}
    \parbox[t]{0.25in}{\sffamily\large\textbf{3.}} & 
    \parbox[t]{3.75in}{\raggedright\sffamily\large\textbf{C++11 in Parallel}} & 
    {\sffamily\large\textbf{205}} \\[1.5em]
% row 3
    & \multicolumn{2}{@{}l}{\parbox{4.75in}{Joe Hummel, \textit{U. of California, Irvine} }} \\[1.5em]
% row 4
    \multicolumn{3}{@{}p{5in}}{\small As hardware designers turn to 
multi-core CPUs and GPUs, 
software developers must 
embrace parallel programming 
to increase performance. No 
single approach has yet 
established itself as the “right 
way” to develop parallel 
software. However, C++ has 
long been used for 
performance-oriented work, 
and it’s a safe bet that any 
viable approach involves C++. 
This position has been 
strengthened by ratification of 
the new C++0x standard, 
officially referred to as 
“C++11”. This workshop will 
introduce the new features of 
C++11 related to parallel 
programming, including type 
inference, lambda expressions, 
closures, multithreading 
support, and thread-local 
storage. It will close with 
discussion of other 
technologies, including Intel 
TBB, ArBB, Cilk Plus, and 
Microsoft PPL, AAL, AMP.  
Laptop Optional.}
\end{longtable}
\begin{longtable}[l]{@{}l@{}l@{}r}
    \parbox[t]{0.25in}{\sffamily\large\textbf{4.}} & 
    \parbox[t]{3.75in}{\raggedright\sffamily\large\textbf{The Absolute Beginner’s Guide to JUnit in the Classroom}} & 
    {\sffamily\large\textbf{206}} \\[1.5em]
% row 3
    & \multicolumn{2}{@{}l}{\parbox{4.75in}{Stephen Edwards and Manuel Pérez-Qui\~nones, \textit{Virginia Tech} }} \\[1.5em]
% row 4
    \multicolumn{3}{@{}p{5in}}{\small Software testing has become 
popular in introductory courses, 
but many educators are 
unfamiliar with how to write 
software tests or how they 
might be used in the classroom.  
This workshop provides a 
practical introduction to JUnit 
for educators.  JUnit is the Java 
testing framework that is most 
commonly used in the 
classroom.  Participants will 
learn how to write and run JUnit 
test cases; how-to’s for 
common classroom uses (as a 
behavioral addition to an 
assignment specification, as 
part of manual grading, as part 
of automated grading, as a 
student-written activity, etc.); 
and common solutions to tricky 
classroom problems (testing 
standard input/output, 
randomness, main programs, 
assignments with lots of design 
freedom, assertions, and code 
that calls exit()). Laptop 
Recommended.}
\end{longtable}
\begin{longtable}[l]{@{}l@{}l@{}r}
    \parbox[t]{0.25in}{\sffamily\large\textbf{5.}} & 
    \parbox[t]{3.75in}{\raggedright\sffamily\large\textbf{Student Scrums}} & 
    {\sffamily\large\textbf{301A}} \\[1.5em]
% row 3
    & \multicolumn{2}{@{}l}{\parbox{4.75in}{Thomas Reichlmayr, \textit{Rochester Institutue of Technology} }} \\[1.5em]
% row 4
    \multicolumn{3}{@{}p{5in}}{\small Our students are entering the 
workforce into an increasing 
number of companies using 
Agile processes and practices in 
the development of their 
products and services, with 
Scrum being the most widely 
used Agile project management 
framework. Selecting Scrum as 
the framework for student team 
projects has the advantage of 
introducing software process at 
a level of ceremony that 
captures foundational software 
engineering practices and is 
manageable within the 
constraints of a class or 
capstone project. This workshop 
will introduce participants to the 
components of the Scrum 
framework with activities 
designed to demonstrate the 
flexibility of Scrum to support a 
diverse set of course learning 
outcomes at all levels of the 
curriculum. Laptop Optional.}
\end{longtable}
\begin{longtable}[l]{@{}l@{}l@{}r}
    \parbox[t]{0.25in}{\sffamily\large\textbf{6.}} & 
    \parbox[t]{3.75in}{\raggedright\sffamily\large\textbf{Reviewing NSF Proposals: Learn about Effective Proposal Writing via the Review Process}} & 
    {\sffamily\large\textbf{301B}} \\ \\
% row 3
    & \multicolumn{2}{@{}l}{\parbox{4.75in}{Sue Fitzgerald and Guy-Alain Amoussou, \textit{National Science Foundation} }} \\[1.5em]
% row 4
    \multicolumn{3}{@{}p{5in}}{\small This interactive workshop 
focuses on the National Science 
Foundation grant proposal 
review process.  Via close 
examination of the review 
process, participants gain an 
understanding of how to write 
good reviews and how to 
improve their own proposal 
writing. The workshop topics 
include: the proposal review 
process from submission to 
award or decline; elements of a 
good review; NSF merit criteria 
(intellectual merit and broader 
impacts); elements of good 
proposals; how to volunteer to 
review.  Faculty who wish to understand 
the NSF review process or seek 
funding in support of 
undergraduate education are 
encouraged to attend.  
Participants will include novice 
proposal writers and those with 
more experience who seek to 
improve their proposal writing 
and reviewing skills. Laptop 
optional.}
\end{longtable}
\newpage
\begin{longtable}[l]{@{}l@{}l@{}r}
    \parbox[t]{0.25in}{\sffamily\large\textbf{7.}} & 
    \parbox[t]{3.75in}{\raggedright\sffamily\large\textbf{A Hands-On Comparison of iOS vs. Android}} & 
    {\sffamily\large\textbf{302A}} \\[1.5em]
% row 3
    & \multicolumn{2}{@{}l}{\parbox{4.75in}{Michael Rogers, \textit{Northwest Missouri State University}; Mark Goadrich, \textit{Centenary College of Louisiana} }} \\[1.5em]
% row 4
    \multicolumn{3}{@{}p{5in}}{\small This workshop is designed for 
faculty, considering teaching a 
course in mobile app 
development, who are unsure as 
to whether they should use iOS, 
Android, or both.  To help them 
make an educated decision, in 
this workshop participants will 
build one app, to implement the 
game Pig, in both platforms.  
By so doing, they will be able to 
make a head-to-head 
comparison of the respective 
development environments, 
languages, and frameworks, 
guided by experienced 
instructors.  Participants will need to bring 
(or share) a recent-vintage 
MacBook Pro / MacBook Air, 
with Xcode, Eclipse, and 
appropriate SDKs, installed prior 
to the workshop.  Details, 
including installation 
instructions, may be found at 
androidios.goadrich.com.
Laptop Required.}
\end{longtable}
\begin{longtable}[l]{@{}l@{}l@{}r}
    \parbox[t]{0.25in}{\sffamily\large\textbf{8.}} & 
    \parbox[t]{3.75in}{\raggedright\sffamily\large\textbf{Killing 3 Birds with One Course: Service Learning, Professional Writing, and Project Management}} & 
    {\sffamily\large\textbf{302B}} \\ \\
% row 3
    & \multicolumn{2}{@{}l}{\parbox{4.75in}{Joseph Mertz, \textit{Carnegie Mellon University}; Scott McElfresh, \textit{Wake Forest University}; Steven Andrianoff and Jennifer Dempsey, \textit{St. Bonaventure University} }} \\[1.5em]
% row 4
    \multicolumn{3}{@{}p{5in}}{\small Service learning is a great idea, 
but can be fraught with 
problems. We present an 
alternative to the project-
course approach. Instead of 
team-based system-
development, we use a student-
consultant model. Students 
individually consult with a 
nonprofit. Each student leads a 
small technology project that 
brings about sustainable change 
in an organization, while 
developing analysis, planning, 
and communication skills. One 
instructor can manage 30 
clients a semester, and we have 
had nearly 400 to date. Our 
clients are happy and recruit 
others. In this session we will 
share our tricks: managing a 
large number of partnerships, 
helping students develop 
leadership and communication 
skills, and assessing their 
performance. A student 
presenter will describe her 
consulting experience. Laptop 
Optional}
\end{longtable}
\begin{longtable}[l]{@{}l@{}l@{}r}
    \parbox[t]{0.25in}{\sffamily\large\textbf{9.}} & 
    \parbox[t]{3.75in}{\raggedright\sffamily\large\textbf{Computer Science Unplugged, Robotics, and Outreach Activities}} & 
    {\sffamily\large\textbf{302C}} \\[1.5em]
% row 3
    & \multicolumn{2}{@{}l}{\parbox{4.75in}{Tim Bell, \textit{University of Canterbury}; Daniela Marghitu, \textit{Auburn University}; Lynn Lambert, \textit{Christopher Newport University} }} \\[1.5em]
% row 4
    \multicolumn{3}{@{}p{5in}}{\small You've been asked to talk to an 
elementary or high school class 
about Computer Science, but 
how can you ensure that the 
talk is engaging? Perhaps you’re 
trying to introduce a concept 
from Computer Science to a 
school group, but you want 
a fun way to get the class 
engaged. This workshop is a 
hands-on introduction to 
Computer Science Unplugged 
(www.csunplugged.org), a 
widely used set of kinesthetic, 
fun activities that cover many 
core areas of computer science 
without using high technology. 
We will explore how to use the 
activities in a variety of 
situations, including combining 
them with robotics activities, 
and explore some novel 
applications. Attendees will 
receive a CD with a copy of a 
handbook for teachers and a 
collection of videos 
demonstrating the activities. 
Laptop Optional.}
\end{longtable}
\begin{longtable}[l]{@{}l@{}l@{}r}
    \parbox[t]{0.25in}{\sffamily\large\textbf{10.}} & 
    \parbox[t]{3.75in}{\raggedright\sffamily\large\textbf{Introduction to Using FPGAs in the Computer Science Curriculum}} & 
    {\sffamily\large\textbf{307}} \\[1.5em]
% row 3
    & \multicolumn{2}{@{}l}{\parbox{4.75in}{William Jones and Brian Larkins, \textit{Coastal Carolina University} }} \\[1.5em]
% row 4
    \multicolumn{3}{@{}p{5in}}{\small One of the challenges in 
modern curriculum design is 
balancing between breadth and 
depth of topics while 
simultaneously reinforcing the 
interconnections
among topics in the field. We 
have integrated
field-programming gate arrays 
(FPGA) systems first used in our 
hardware-based courses into 
several higher-level systems 
and applications courses. This 
allows us
to leverage student familiarity 
with a hands-on, hardware 
platform and also strengthen 
the relationships between 
different subfields within 
computer
science.  In this workshop, we 
present participants with guided 
hands-on activities for making 
use of FPGAs in common 
computer science courses. 
Laptop Required.}
\end{longtable}
\begin{longtable}[l]{@{}l@{}l@{}r}
    \parbox[t]{0.25in}{\sffamily\large\textbf{11.}} & 
    \parbox[t]{3.75in}{\raggedright\sffamily\large\textbf{Helping Students Become Better Communicators}} & 
    {\sffamily\large\textbf{305A}} \\[1.5em]
% row 3
    & \multicolumn{2}{@{}l}{\parbox{4.75in}{Janet Burge, Paul Anderson, and Gerald Gannod, \textit{Miami University}}} \\[1.5em]
% row 4
    \multicolumn{3}{@{}p{5in}}{\small To be successful, CS and SE 
graduates need strong 
communication skills (writing, 
speaking, and teaming), 
particularly within their 
discipline. Students exercise 
these skills during their classes 
but are not always given explicit 
domain-specific instruction on 
these skills, instead relying on 
instruction provided outside the 
program. CS and SE faculty are 
not always comfortable in 
evaluating these aspects of their 
assignments and are often 
unhappy with the results.  In 
this workshop we will lead 
sessions on teaching writing, 
speaking, and teaming; situating 
assignments in workplace-
scenarios; and writing 
communication rubrics that 
convey faculty expectations to 
students and support evaluation 
of student work. For more 
information, see 
www.muohio.edu/sigcse\_workshop11. Laptop Recommended.}
\end{longtable}
\begin{longtable}[l]{@{}l@{}l@{}r}
    \parbox[t]{0.25in}{\sffamily\large\textbf{12.}} & 
    \parbox[t]{3.75in}{\raggedright\sffamily\large\textbf{ROS for Educators: Teaching with the Robot Operating System and Microsoft Kinect}} & 
    {\sffamily\large\textbf{305B}} \\[1.5em]
% row 3
    & \multicolumn{2}{@{}l}{\parbox{4.75in}{Michael Ferguson, \textit{Willow Garage, Inc.}; Julian Mason, \textit{Duke University}; Sharon Gower Small, \textit{Siena College}; Zachary Dodds, \textit{Harvey Mudd College} }} \\[1.5em]
% row 4
    \multicolumn{3}{@{}p{5in}}{\small The Microsoft Kinect and Willow 
Garage's Robot Operating 
System (ROS) are changing the 
way robots are developed. 
Together, these tools can enable 
today's CS educators to provide 
richer and more research-representative experiences with 
robots and perception. This 
hands-on workshop will 
introduce ROS and showcase 
two pilot courses taught using 
ROS and the Kinect.  Four 20-
minute talks will intersperse 
with participants' hands-on 
development of Python 
programs on low-cost Kinect-equipped robots and the 
ARDrone quadcopter. This 
workshop is intended for all 
college-level CS educators 
interested in robotics or 
embodied AI. First-time 
ROS/Kinect users are 
particularly welcome! Laptops 
and robots will be provided. See 
http://www.ros.org/wiki/Courses/sigcse2012. Laptops optional.}
\end{longtable}
\newpage
\begin{longtable}[l]{@{}l@{}l@{}r}
    \parbox[t]{0.25in}{\sffamily\large\textbf{13.}} & 
    \parbox[t]{3.75in}{\raggedright\sffamily\large\textbf{Board Game Project Ideas for CS 1 and CS 2}} & 
    {\sffamily\large\textbf{306A}} \\[1.5em]
% row 3
    & \multicolumn{2}{@{}l}{\parbox{4.75in}{Zachary Kurmas, \textit{Grand Valley State University}; James Vanderhyde, \textit{Benedictine College} }} \\[1.5em]
% row 4
    \multicolumn{3}{@{}p{5in}}{\small Participants will have fun 
learning and playing relatively 
unknown
board games that are especially 
suitable for programming 
projects. We
will present games where (1) all 
players can view the same 
screen, (2)
the board is reasonably simple 
to program, and (3) there are 
several
elements of the game that 
relate strongly to a common CS 
1, CS 2, or
discrete math topic. After we 
explain the rules and highlight 
the
CS-related elements of the 
games, participants will have 
the
opportunity to play the games, 
ask questions, and suggest rule
variations that will improve the 
resulting programming project.
See 
\url{http://www.cis.gvsu.edu/~kurmasz/GamesWorkshop/} for more 
details and
a list of games that may be 
presented.  Laptop Optional.}
\end{longtable}
\begin{longtable}[l]{@{}l@{}l@{}r}
    \parbox[t]{0.25in}{\sffamily\large\textbf{14.}} & 
    \parbox[t]{3.75in}{\raggedright\sffamily\large\textbf{A Taste of Linked Data and the Semantic Web}} & 
    {\sffamily\large\textbf{306B}} \\[1.5em]
% row 3
    & \multicolumn{2}{@{}l}{\parbox{4.75in}{Marsha Zaidman and David Hyland-Wood, \textit{University of Mary Washington} }} \\[1.5em]
% row 4
    \multicolumn{3}{@{}p{5in}}{\small The Web has created a global 
information space of linked 
documents.  The Semantic Web 
creates an information space of 
linked data from multiple 
sources.  Information can be 
mined from the interlinking of 
available datasets by a 
distributed query language 
known as SPARQL, the SQL 
equivalent for the Semantic 
Web.  Participants will 
understand and appreciate the 
role of linked data on the 
Semantic Web; be able to model, 
represent, and interpret simple 
linked data applications; 
complete exercises that create 
simple Linked Data models; 
appreciate the benefits of 
Linked Data over relational 
database modeling; be aware of 
successful commercial 
applications of linked Data; be 
directed to resources that 
facilitate incorporation of this 
material into their courses.  
WiFi/Laptop Required.}
\end{longtable}
\begin{longtable}[l]{@{}l@{}l@{}r}
    \parbox[t]{0.25in}{\sffamily\large\textbf{15.}} & 
    \parbox[t]{3.75in}{\raggedright\sffamily\large\textbf{Teaching with Greenfoot and the Kinect – A Novel Way to Engage Beginners}} & 
    {\sffamily\large\textbf{306C}} \\[1.5em]
% row 3
    & \multicolumn{2}{@{}l}{\parbox{4.75in}{Michael Kölling and Neil Brown, \textit{University of Kent} }} \\[1.5em]
% row 4
    \multicolumn{3}{@{}p{5in}}{\small The Microsoft Kinect is a sensor 
module that allows accurate 
tracking of humans moving in 
front of it. Greenfoot is an 
introductory Java programming 
environment that makes it easy 
to create animated graphical 
projects. By combining 
Greenfoot and the Kinect 
students can write programs 
where the user’s body is used 
for input. Users interact with 
games by waving their hands, 
jumping, running, dancing, …. 
These kinds of programs are 
incredibly good fun and engage 
target groups who would not 
normally be interested in 
programming. The workshop is 
aimed at teachers of 
introductory programming 
courses (high school/university) 
who have some programming 
experience and want to 
incorporate new kinds of 
projects into their teaching. 
Laptop recommended but not 
required. Kinect hardware will 
be provided.}
\end{longtable}
