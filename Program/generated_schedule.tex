\fancyfoot[LE,RO]{\thepage}
\addcontentsline{toc}{subsection}{Wednesday}
\cfoot{\colorbox[gray]{0.45}{\color{white}\textsf{Wednesday 19:00 - 22:00}}}
\noindent
\framebox[5in][c]{{\Large\sffamily\textbf{Wednesday,  19:00 to 22:00}}}
\begin{longtable}[l]{@{}l@{}l@{}r}
    \parbox[t]{0.25in}{\sffamily\large\textbf{1.}} & 
    \parbox[t]{3.75in}{\raggedright\sffamily\large\textbf{Using Social Networking to Improve Student Learning Through Classroom Salon}} & 
    {\sffamily\large\textbf{201}} \\[1.5em]
% row 3
    \multicolumn{3}{@{}l}{\parbox{5in}{John Barr, \textit{Ithaca College}; Ananda Gunawardena, \textit{Carnegie Mellon University} }} \\[1.5em]
% row 4
    \multicolumn{3}{@{}p{5in}}{\small This workshop introduces an 
innovative social collaboration 
tool called Classroom Salon 
(CLS). Developed at Carnegie 
Mellon University, CLS is a 
combination of electronic 
books, social networks, and 
analytic tools.  It enables 
students to learn by 
participating in social networks 
and allows instructors to easily 
analyze student participation. 
The workshop covers extant 
social networks, introduces CLS 
web-based software (nothing to 
install) and demonstrates the 
use of CLS to help students 
master critical skills such as 
code review, debugging, and 
reading documentation.  
Participants will create Salons, 
learn how to use them in their 
courses, and learn how to use 
the built-in tools to analyze 
student activities.  See 
http://classroomsalon.org.  
Laptop (with wifi) required.}
\end{longtable}
\begin{longtable}[l]{@{}l@{}l@{}r}
    \parbox[t]{0.25in}{\sffamily\large\textbf{2.}} & 
    \parbox[t]{3.75in}{\raggedright\sffamily\large\textbf{Challenges and Opportunities in Conducting Educational Research in the Computer Science Classroom}} & 
    {\sffamily\large\textbf{204}} \\[1.5em]
% row 3
    \multicolumn{3}{@{}l}{\parbox{5in}{Aman Yadav and Tim Korb, \textit{Purdue University} }} \\[1.5em]
% row 4
    \multicolumn{3}{@{}p{5in}}{\small This workshop will provide CS 
educators with tools to conduct 
educational research. Primary 
objectives of this workshop 
are:(1) learn basic principles of 
research design; (2) learn about 
various types of research 
designs: qualitative vs. 
quantitative; experimental vs. 
quasi-experimental; case 
studies, survey; (3) to practice 
designing research. This 
workshop will help participants 
make informed decisions when 
faced with limitations of 
educational research and collect 
empirical evidence about what 
works in the classroom. In 
addition, we will also discuss 
how to develop robust student 
outcome measures, such as 
surveys and tests. The 
workshop will be beneficial to 
participants who have not yet 
done all of these activities as 
well as those who have some 
background in educational 
research}
\end{longtable}
\begin{longtable}[l]{@{}l@{}l@{}r}
    \parbox[t]{0.25in}{\sffamily\large\textbf{3.}} & 
    \parbox[t]{3.75in}{\raggedright\sffamily\large\textbf{C++11 in Parallel}} & 
    {\sffamily\large\textbf{205}} \\[1.5em]
% row 3
    \multicolumn{3}{@{}l}{\parbox{5in}{Joe Hummel, \textit{U. of California, Irvine} }} \\[1.5em]
% row 4
    \multicolumn{3}{@{}p{5in}}{\small As hardware designers turn to 
multi-core CPUs and GPUs, 
software developers must 
embrace parallel programming 
to increase performance. No 
single approach has yet 
established itself as the “right 
way” to develop parallel 
software. However, C++ has 
long been used for 
performance-oriented work, 
and it’s a safe bet that any 
viable approach involves C++. 
This position has been 
strengthened by ratification of 
the new C++0x standard, 
officially referred to as 
“C++11”. This workshop will 
introduce the new features of 
C++11 related to parallel 
programming, including type 
inference, lambda expressions, 
closures, multithreading 
support, and thread-local 
storage. It will close with 
discussion of other 
technologies, including Intel 
TBB, ArBB, Cilk Plus, and 
Microsoft PPL, AAL, AMP.  
Laptop optional.}
\end{longtable}
\begin{longtable}[l]{@{}l@{}l@{}r}
    \parbox[t]{0.25in}{\sffamily\large\textbf{4.}} & 
    \parbox[t]{3.75in}{\raggedright\sffamily\large\textbf{The Absolute Beginner’s Guide to JUnit in the Classroom}} & 
    {\sffamily\large\textbf{206}} \\[1.5em]
% row 3
    \multicolumn{3}{@{}l}{\parbox{5in}{Stephen Edwards and Manuel Perez-Quinones, \textit{Virginia Tech, Department of Computer Science} }} \\[1.5em]
% row 4
    \multicolumn{3}{@{}p{5in}}{\small Software testing has become 
popular in introductory courses, 
but many educators are 
unfamiliar with how to write 
software tests or how they 
might be used in the classroom.  
This workshop provides a 
practical introduction to JUnit 
for educators.  JUnit is the Java 
testing framework that is most 
commonly used in the 
classroom.  Participants will 
learn how to write and run JUnit 
test cases; how-to’s for 
common classroom uses (as a 
behavioral addition to an 
assignment specification, as 
part of manual grading, as part 
of automated grading, as a 
student-written activity, etc.); 
and common solutions to tricky 
classroom problems (testing 
standard input/output, 
randomness, main programs, 
assignments with lots of design 
freedom, assertions, and code 
that calls exit()). Laptop 
Recommended.}
\end{longtable}
\begin{longtable}[l]{@{}l@{}l@{}r}
    \parbox[t]{0.25in}{\sffamily\large\textbf{5.}} & 
    \parbox[t]{3.75in}{\raggedright\sffamily\large\textbf{Student Scrums}} & 
    {\sffamily\large\textbf{301A}} \\[1.5em]
% row 3
    \multicolumn{3}{@{}l}{\parbox{5in}{Thomas Reichlmayr, \textit{Rochester Instittue of Technology} }} \\[1.5em]
% row 4
    \multicolumn{3}{@{}p{5in}}{\small Our students are entering the 
workforce into an increasing 
number of companies using 
Agile processes and practices in 
the development of their 
products and services, with 
Scrum being the most widely 
used Agile project management 
framework. Selecting Scrum as 
the framework for student team 
projects has the advantage of 
introducing software process at 
a level of ceremony that 
captures foundational software 
engineering practices and is 
manageable within the 
constraints of a class or 
capstone project. This workshop 
will introduce participants to the 
components of the Scrum 
framework with activities 
designed to demonstrate the 
flexibility of Scrum to support a 
diverse set of course learning 
outcomes at all levels of the 
curriculum. Laptop Optional.}
\end{longtable}
\begin{longtable}[l]{@{}l@{}l@{}r}
    \parbox[t]{0.25in}{\sffamily\large\textbf{6.}} & 
    \parbox[t]{3.75in}{\raggedright\sffamily\large\textbf{Reviewing NSF Proposals: Learn about Effective Proposal Writing via the Review Process}} & 
    {\sffamily\large\textbf{301B}} \\[1.5em]
% row 3
    \multicolumn{3}{@{}l}{\parbox{5in}{Sue Fitzgerald and Guy-Alain Amoussou, \textit{National Science Foundation} }} \\[1.5em]
% row 4
    \multicolumn{3}{@{}p{5in}}{\small This interactive workshop 
focuses on the National Science 
Foundation grant proposal 
review process.  Via close 
examination of the review 
process, participants gain an 
understanding of how to write 
good reviews and how to 
improve their own proposal 
writing. The workshop topics 
include: the proposal review 
process from submission to 
award or decline; elements of a 
good review; NSF merit criteria 
(intellectual merit and broader 
impacts); elements of good 
proposals; how to volunteer to 
review.

Faculty who wish to understand 
the NSF review process or seek 
funding in support of 
undergraduate education are 
encouraged to attend.  
Participants will include novice 
proposal writers and those with 
more experience who seek to 
improve their proposal writing 
and reviewing skills. Laptop 
optional}
\end{longtable}
\begin{longtable}[l]{@{}l@{}l@{}r}
    \parbox[t]{0.25in}{\sffamily\large\textbf{7.}} & 
    \parbox[t]{3.75in}{\raggedright\sffamily\large\textbf{A Hands-On Comparison of iOS vs. Android}} & 
    {\sffamily\large\textbf{302A}} \\[1.5em]
% row 3
    \multicolumn{3}{@{}l}{\parbox{5in}{Michael Rogers, \textit{Northwest Missouri State University}; Mark Goadrich, \textit{Centenary College of Louisiana} }} \\[1.5em]
% row 4
    \multicolumn{3}{@{}p{5in}}{\small This workshop is designed for 
faculty, considering teaching a 
course in mobile app 
development, who are unsure as 
to whether they should use iOS, 
Android, or both.  To help them 
make an educated decision, in 
this workshop participants will 
build one app, to implement the 
game Pig, in both platforms.
  
By so doing, they will be able to 
make a head-to-head 
comparison of the respective 
development environments, 
languages, and frameworks, 
guided by experienced 
instructors.
  
Participants will need to bring 
(or share) a recent-vintage 
MacBook Pro / MacBook Air, 
with Xcode, Eclipse, and 
appropriate SDKs, installed prior 
to the workshop.  Details, 
including installation 
instructions, may be found at 
androidios.goadrich.com.
 
Laptop Required.}
\end{longtable}
\begin{longtable}[l]{@{}l@{}l@{}r}
    \parbox[t]{0.25in}{\sffamily\large\textbf{8.}} & 
    \parbox[t]{3.75in}{\raggedright\sffamily\large\textbf{Killing 3 Birds with One Course: Service Learning, Professional Writing, and Project Management}} & 
    {\sffamily\large\textbf{302B}} \\[1.5em]
% row 3
    \multicolumn{3}{@{}l}{\parbox{5in}{Joseph Mertz, \textit{Carnegie Mellon University}; Scott McElfresh, \textit{Wake Forest University}; Steven Andrianoff and Jennifer Dempsey, \textit{St. Bonaventure University} }} \\[1.5em]
% row 4
    \multicolumn{3}{@{}p{5in}}{\small Service learning is a great idea, 
but can be fraught with 
problems. We present an 
alternative to the project-
course approach. Instead of 
team-based system-
development, we use a student-
consultant model. Students 
individually consult with a 
nonprofit. Each student leads a 
small technology project that 
brings about sustainable change 
in an organization, while 
developing analysis, planning, 
and communication skills. One 
instructor can manage 30 
clients a semester, and we have 
had nearly 400 to date. Our 
clients are happy and recruit 
others. In this session we will 
share our tricks: managing a 
large number of partnerships, 
helping students develop 
leadership and communication 
skills, and assessing their 
performance. A student 
presenter will describe her 
consulting experience. Laptop 
Optional}
\end{longtable}
\begin{longtable}[l]{@{}l@{}l@{}r}
    \parbox[t]{0.25in}{\sffamily\large\textbf{9.}} & 
    \parbox[t]{3.75in}{\raggedright\sffamily\large\textbf{Computer Science Unplugged, Robotics, and Outreach Activities}} & 
    {\sffamily\large\textbf{302C}} \\[1.5em]
% row 3
    \multicolumn{3}{@{}l}{\parbox{5in}{Tim Bell, \textit{University of Canterbury}; Daniela Marghitu, \textit{Auburn University}; Lynn Lambert, \textit{Christopher Newport University} }} \\[1.5em]
% row 4
    \multicolumn{3}{@{}p{5in}}{\small You've been asked to talk to an 
elementary or high school class 
about Computer Science, but 
how can you ensure that the 
talk is engaging? Perhaps you’re 
trying to introduce a concept 
from Computer Science to a 
school group, but you want 
a fun way to get the class 
engaged. This workshop is a 
hands-on introduction to 
Computer Science Unplugged 
(www.csunplugged.org), a 
widely used set of kinesthetic, 
fun activities that cover many 
core areas of computer science 
without using high technology. 
We will explore how to use the 
activities in a variety of 
situations, including combining 
them with robotics activities, 
and explore some novel 
applications. Attendees will 
receive a CD with a copy of a 
handbook for teachers and a 
collection of videos 
demonstrating the activities. 
Laptop Optional.}
\end{longtable}
\begin{longtable}[l]{@{}l@{}l@{}r}
    \parbox[t]{0.25in}{\sffamily\large\textbf{10.}} & 
    \parbox[t]{3.75in}{\raggedright\sffamily\large\textbf{Introduction to Using FPGAs in the Computer Science Curriculum}} & 
    {\sffamily\large\textbf{307}} \\[1.5em]
% row 3
    \multicolumn{3}{@{}l}{\parbox{5in}{William Jones and Brian Larkins, \textit{Coastal Carolina University} }} \\[1.5em]
% row 4
    \multicolumn{3}{@{}p{5in}}{\small One of the challenges in 
modern curriculum design is 
balancing between breadth and 
depth of topics while 
simultaneously reinforcing the 
interconnections
among topics in the field. We 
have integrated
field-programming gate arrays 
(FPGA) systems first used in our 
hardware-based courses into 
several higher-level systems 
and applications courses. This 
allows us
to leverage student familiarity 
with a hands-on, hardware 
platform and also strengthen 
the relationships between 
different subfields within 
computer
science.  In this workshop, we 
present participants with guided 
hands-on activities for making 
use of FPGAs in common 
computer science courses. 
Laptop required.}
\end{longtable}
\begin{longtable}[l]{@{}l@{}l@{}r}
    \parbox[t]{0.25in}{\sffamily\large\textbf{11.}} & 
    \parbox[t]{3.75in}{\raggedright\sffamily\large\textbf{Helping Students Become Better Communicators}} & 
    {\sffamily\large\textbf{305A}} \\[1.5em]
% row 3
    \multicolumn{3}{@{}l}{\parbox{5in}{Janet Burge and Gerald Gannod, \textit{Miami University}; Paul Anderson, \textit{Miami Univerity} }} \\[1.5em]
% row 4
    \multicolumn{3}{@{}p{5in}}{\small To be successful, CS and SE 
graduates need strong 
communication skills (writing, 
speaking, and teaming), 
particularly within their 
discipline. Students exercise 
these skills during their classes 
but are not always given explicit 
domain-specific instruction on 
these skills, instead relying on 
instruction provided outside the 
program. CS and SE faculty are 
not always comfortable in 
evaluating these aspects of their 
assignments and are often 
unhappy with the results.  In 
this workshop we will lead 
sessions on teaching writing, 
speaking, and teaming; situating 
assignments in workplace-
scenarios; and writing 
communication rubrics that 
convey faculty expectations to 
students and support evaluation 
of student work. For more 
information, see 
www.muohio.edu/sigcse\_workshop11. Laptop Recommended.}
\end{longtable}
\begin{longtable}[l]{@{}l@{}l@{}r}
    \parbox[t]{0.25in}{\sffamily\large\textbf{12.}} & 
    \parbox[t]{3.75in}{\raggedright\sffamily\large\textbf{ROS for Educators: Teaching with the Robot Operating System and Microsoft Kinect}} & 
    {\sffamily\large\textbf{305B}} \\[1.5em]
% row 3
    \multicolumn{3}{@{}l}{\parbox{5in}{Michael Ferguson, \textit{Willow Garage, Inc.}; Julian Mason, \textit{Duke University}; Sharon Gower Small, \textit{Siena College}; Zachary Dodds, \textit{Harvey Mudd College} }} \\[1.5em]
% row 4
    \multicolumn{3}{@{}p{5in}}{\small The Microsoft Kinect and Willow 
Garage's Robot Operating 
System (ROS) are changing the 
way robots are developed. 
Together, these tools can enable 
today's CS educators to provide 
richer and more research-
representative experiences with 
robots and perception. This 
hands-on workshop will 
introduce ROS and showcase 
two pilot courses taught using 
ROS and the Kinect.  Four 20-
minute talks will intersperse 
with participants' hands-on 
development of Python 
programs on low-cost Kinect-
equipped robots and the 
ARDrone quadcopter. This 
workshop is intended for all 
college-level CS educators 
interested in robotics or 
embodied AI. First-time 
ROS/Kinect users are 
particularly welcome! Laptops 
and robots will be provided. See 
http://www.ros.org/wiki/Course
s/sigcse2012. Laptops optional.}
\end{longtable}
\begin{longtable}[l]{@{}l@{}l@{}r}
    \parbox[t]{0.25in}{\sffamily\large\textbf{13.}} & 
    \parbox[t]{3.75in}{\raggedright\sffamily\large\textbf{Board Game Project Ideas for CS 1 and CS 2}} & 
    {\sffamily\large\textbf{306A}} \\[1.5em]
% row 3
    \multicolumn{3}{@{}l}{\parbox{5in}{Zachary Kurmas, \textit{Grand Valley State University}; James Vanderhyde, \textit{Benedictine College} }} \\[1.5em]
% row 4
    \multicolumn{3}{@{}p{5in}}{\small Participants will have fun 
learning and playing relatively 
unknown
board games that are especially 
suitable for programming 
projects. We
will present games where (1) all 
players can view the same 
screen, (2)
the board is reasonably simple 
to program, and (3) there are 
several
elements of the game that 
relate strongly to a common CS 
1, CS 2, or
discrete math topic. After we 
explain the rules and highlight 
the
CS-related elements of the 
games, participants will have 
the
opportunity to play the games, 
ask questions, and suggest rule
variations that will improve the 
resulting programming project.
See 
http://www.cis.gvsu.edu/~kurm
asz/GamesWorkshop/ for more 
details and
a list of games that may be 
presented.  Laptop Optional.}
\end{longtable}
\begin{longtable}[l]{@{}l@{}l@{}r}
    \parbox[t]{0.25in}{\sffamily\large\textbf{14.}} & 
    \parbox[t]{3.75in}{\raggedright\sffamily\large\textbf{A Taste of Linked Data and the Semantic Web}} & 
    {\sffamily\large\textbf{306B}} \\[1.5em]
% row 3
    \multicolumn{3}{@{}l}{\parbox{5in}{Marsha Zaidman and David Hyland-Wood, \textit{University of Mary Washington} }} \\[1.5em]
% row 4
    \multicolumn{3}{@{}p{5in}}{\small The Web has created a global 
information space of linked 
documents.  The Semantic Web 
creates an information space of 
linked data from multiple 
sources.  Information can be 
mined from the interlinking of 
available datasets by a 
distributed query language 
known as SPARQL, the SQL 
equivalent for the Semantic 
Web.  Participants will 
understand and appreciate the 
role of linked data on the 
Semantic Web; be able to model, 
represent, and interpret simple 
linked data applications; 
complete exercises that create 
simple Linked Data models; 
appreciate the benefits of 
Linked Data over relational 
database modeling; be aware of 
successful commercial 
applications of linked Data;be 
directed to resources that 
facilitate incorporation of this 
material into their courses.  
WiFi/Laptop Required.}
\end{longtable}
\begin{longtable}[l]{@{}l@{}l@{}r}
    \parbox[t]{0.25in}{\sffamily\large\textbf{15.}} & 
    \parbox[t]{3.75in}{\raggedright\sffamily\large\textbf{Teaching with Greenfoot and the Kinect – A Novel Way to Engage Beginners}} & 
    {\sffamily\large\textbf{306C}} \\[1.5em]
% row 3
    \multicolumn{3}{@{}l}{\parbox{5in}{Michael Kölling and Neil Brown, \textit{University of Kent} }} \\[1.5em]
% row 4
    \multicolumn{3}{@{}p{5in}}{\small The Microsoft Kinect is a sensor 
module that allows accurate 
tracking of humans moving in 
front of it. Greenfoot is an 
introductory Java programming 
environment that makes it easy 
to create animated graphical 
projects. By combining 
Greenfoot and the Kinect 
students can write programs 
where the user’s body is used 
for input. Users interact with 
games by waving their hands, 
jumping, running, dancing, …. 
These kinds of programs are 
incredibly good fun and engage 
target groups who would not 
normally be interested in 
programming. The workshop is 
aimed at teachers of 
introductory programming 
courses (high school/university) 
who have some programming 
experience and want to 
incorporate new kinds of 
projects into their teaching. 
Laptop recommended but not 
required. Kinect hardware will 
be provided.}
\end{longtable}
\vspace{0.5em}
\noindent\rule{5in}{0.02cm}
\vspace{0.5em}
\addcontentsline{toc}{subsection}{Thursday}
\cfoot{\colorbox[gray]{0.45}{\color{white}\textsf{Thursday 08:30 - 10:00}}}
\noindent
\framebox[5in][c]{{\Large\sffamily\textbf{Thursday,  8:30 to 10:00}}}
\begin{longtable}[l]{@{}p{1in}@{}p{3in}@{}r}
    {\sffamily\large\textbf{Plenary Session}} & 
    {\sffamily\large\textbf{Plenary SessionKeynote Speaker: Frederick P. Brooks, Jr.}} & 
    {\sffamily\large\textbf{Ballroom AB}} \\
\end{longtable}    
\vspace{0.5em}
\noindent\rule{5in}{0.02cm}
\vspace{0.5em}
\cfoot{\colorbox[gray]{0.45}{\color{white}\textsf{Thursday 10:00 - 17:00}}}
\noindent
\framebox[5in][c]{{\Large\sffamily\textbf{Thursday,  10:00 to 17:00}}}
\begin{longtable}[l]{@{}p{1in}@{}p{3in}@{}r}
    {\sffamily\large\textbf{Exhibits}} & 
    {\sffamily\large\textbf{Exhibits}} & 
    {\sffamily\large\textbf{Exhibit Hall A}} \\
\end{longtable}    
\vspace{0.5em}
\noindent\rule{5in}{0.02cm}
\vspace{0.5em}
\cfoot{\colorbox[gray]{0.45}{\color{white}\textsf{Thursday 10:00 - 10:45}}}
\noindent
\framebox[5in][c]{{\Large\sffamily\textbf{Thursday,  10:00 to 10:45}}}
\begin{longtable}[l]{@{}p{1in}@{}p{3in}@{}r}
    {\sffamily\large\textbf{Tasty}} & 
    {\sffamily\large\textbf{Break \& Exhibits}} & 
    {\sffamily\large\textbf{Exhibit Hall A}} \\
\end{longtable}    
\vspace{0.5em}
\noindent\rule{5in}{0.02cm}
\vspace{0.5em}
\cfoot{\colorbox[gray]{0.45}{\color{white}\textsf{Thursday 10:00 - 11:30}}}
\noindent
\framebox[5in][c]{{\Large\sffamily\textbf{Thursday,  10:00 to 11:30}}}
\begin{longtable}[l]{@{}p{1in}@{}p{3in}@{}r}
    {\sffamily\large\textbf{Project Showcase}} & 
    {\sffamily\large\textbf{NSF Showcase \#1}} & 
    {\sffamily\large\textbf{Exhibit Hall A}} \\
\end{longtable}    
\vspace{0.5em}
\noindent\rule{5in}{0.02cm}
\vspace{0.5em}
\cfoot{\colorbox[gray]{0.45}{\color{white}\textsf{Thursday 10:00 - 17:00}}}
\noindent
\framebox[5in][c]{{\Large\sffamily\textbf{Thursday,  10:00 to 17:00}}}
\begin{longtable}[l]{@{}p{1in}@{}p{3in}@{}r}
    {\sffamily\large\textbf{Social}} & 
    {\sffamily\large\textbf{K-12 Teachers Room}} & 
    {\sffamily\large\textbf{202}} \\
\end{longtable}    
\begin{longtable}[l]{@{}p{1in}@{}p{3in}@{}r}
    {\sffamily\large\textbf{Social}} & 
    {\sffamily\large\textbf{CS Education Research Room}} & 
    {\sffamily\large\textbf{203}} \\
\end{longtable}    
\vspace{0.5em}
\noindent\rule{5in}{0.02cm}
\vspace{0.5em}
\cfoot{\colorbox[gray]{0.45}{\color{white}\textsf{Thursday 10:45 - 12:00}}}
\noindent
\framebox[5in][c]{{\Large\sffamily\textbf{Thursday,  10:45 to 12:00}}}
\begin{longtable}[l]{@{}l@{}l@{}r}
    \parbox[t]{1in}{\sffamily\large\textbf{PANEL}} & 
    \parbox[t]{3in}{\sffamily\raggedright\large\textbf{Computer Curricula 2013:  Update}} & 
    \parbox[t]{1in}{\sffamily\raggedleft\large\textbf{301AB}} \\
% row 2    
    Chair: & 
    Mehran Sahami \textit{Stanford University}  \\[0.5em]
% row 3
    Participants: & 
    \multicolumn{2}{@{}l}{\parbox{3.75in}{Steve Roach, \textit{University of Texas at El Paso}; Ernesto Cuadros-Vargas, \textit{San Pablo Catholic University}; David Reed, \textit{Creighton University} }} \\[2em]
% row 4
    \multicolumn{3}{@{}p{5in}}{\small Beginning over 40 years ago with the publication of Curriculum 68, the major professional societies in computing--ACM and IEEE-Computer Society--have sponsored various efforts to establish international curricular guidelines for undergraduate programs in computing.  In the Fall of 2010, work on the next volume in the series, Computer Science 2013 (CS2013), began.  Considerable work on the new volume has already been completed and a first draft of the CS2013 report (known as the Strawman report) will be complete by the beginning of 2012.  This panel seeks to update and engage the SIGCSE community in providing feedback on the Strawman report, which will be available shortly prior to the SIGCSE conference.}
\end{longtable}
\begin{longtable}[l]{@{}l@{}l@{}r}
    \parbox[t]{1in}{\sffamily\large\textbf{PANEL}} & 
    \parbox[t]{3in}{\sffamily\raggedright\large\textbf{Scrum Across the CS/SE Curricula}} & 
    \parbox[t]{1in}{\sffamily\raggedleft\large\textbf{305B}} \\
% row 2    
    Chair: & 
    Mark Hoffman \textit{Quinnipiac University}  \\[0.5em]
% row 3
    Participants: & 
    \multicolumn{2}{@{}l}{\parbox{3.75in}{Charles Wallace, \textit{Michigan Technological University}; Douglas Troy, \textit{Miami University}; Sriram Mohan, \textit{Rose-Hulman Institute of Technology} }} \\[2em]
% row 4
    \multicolumn{3}{@{}p{5in}}{\small Scrum is one of the many agile approaches to software development that have been widely adopted over the past decade. Key agile features of Scrum are a flexible, adaptive schedule; democratic, collaborative teams; and frequent, iterative project and process reviews. Scrum is not only a software development strategy but a general learning strategy. Panel participants will describe how their students learn and practice Scrum in a software development, how they use it to manage student projects in non-software development contexts, and how Scrum provides opportunities to integrate communication skills into the CS and SE curricula. As participants in the CPATH II project, panelists have developed Scrum-based assignments that exercise the skills of reading, writing, speaking and teaming.}
\end{longtable}
\begin{longtable}[l]{@{}l@{}l@{}r}
    \parbox[t]{1in}{\sffamily\large\textbf{PANEL}} & 
    \parbox[t]{3in}{\sffamily\raggedright\large\textbf{Role of Interdisciplinary Computing in Higher Education, Research and Industry}} & 
    \parbox[t]{1in}{\sffamily\raggedleft\large\textbf{306C}} \\
% row 2    
    Chair: & 
    Ursula Wolz, \textit{Franklin W. Olin College of Engineering}  \\[0.5em]
% row 3
    Participants: & 
    \multicolumn{2}{@{}l}{\parbox{3.75in}{Lillian (Boots) Cassel, \textit{Villanova University}; Guy-Alain Amoussou, \textit{National Science Foundation} }} \\[2em]
% row 4
    \multicolumn{3}{@{}p{5in}}{\small At SIGCSE 2010 NSF directors held a panel on the potential for Interdisciplinary Computing.  This session is a direct outgrowth. Via an NSF grant individuals drawn from a range of universities and industry met three times in 2011 to discuss the nature of interdisciplinary computing, its importance both for computing and for other disciplines, obstacles to the further emergence of interdisciplinary computing and ways in which these obstacles might be overcome. This session provides an opportunity for the SIGCSE community to  learn about the potential, promise and pitfalls of existing interdisciplinary computing activities. They will be asked to contribute their insights and experiences via structured small-group discussion, and will make connections with others with similar interests.}
\end{longtable}
\newpage
\begin{longtable}{@{}p{0.75in}@{}p{3.25in}@{}r}
   {\sffamily\large\textbf{PAPERS}} &
   {\raggedright\sffamily\large\textbf{Data Structures and Algorithms}} & 
   {\sffamily\large\textbf{302A }} \\
%row 2
   Chair:  & 
   {\raggedright Ivona Bezakova \textit{Rochester Institute of Technology}} & \\ \\
{\sffamily \large 10:45}& 
\multicolumn{2}{@{}p{3.75in}}{\sffamily\raggedright\textbf{Sustainability Themed Problem Solving In Data Structures And Algorithms}} \\
& \multicolumn{2}{@{}p{3.75in}}{\raggedright Ali Erkan, Tom Pfaff, Jason Hamilton and Michael Rogers, \textit{Ithaca College}} \\ \\
\multicolumn{3}{@{}p{5in}}{\small During the past two years, we have been creating curricular material centered around complex problems rooted in sustainability. Since multi-disciplinary learning is one of our primary goals, these projects are most meaningful when they connect students from different disciplines working toward a common understanding. However, strong disciplinary components present in their solutions also allow us to frame these projects from strictly disciplinary perspectives. In this paper, we show how they can be used for increased engagement in the context of data structures and algorithms. We review two new ones to explore (i) the structural characteristics of the western part of the U.S. power-grid, and (ii) the effects of over-harvesting on fish stocks.} \\ \\
{\sffamily \large 11:10}& 
\multicolumn{2}{@{}p{3.75in}}{\sffamily\raggedright\textbf{Metaphors and Analogies for Teaching Algorithms}} \\
& \multicolumn{2}{@{}p{3.75in}}{\raggedright Monika Steinova, \textit{ETH Zurich}; Michal Forisek, \textit{Comenius University}} \\ \\
\multicolumn{3}{@{}p{5in}}{\small In this paper we explore the topic of using metaphors and analogies in teaching algorithms. We argue their importance in the teaching process. We present a selection of metaphors we successfully used when teaching algorithms to secondary school students. We also discuss the suitability of several commonly used metaphors, and in several cases we show significant weaknesses of these metaphors.} \\ \\
{\sffamily \large 11:35}& 
\multicolumn{2}{@{}p{3.75in}}{\sffamily\raggedright\textbf{Detecting and Understanding Students’ Misconceptions Related to Algorithms and Data Structures}} \\
& \multicolumn{2}{@{}p{3.75in}}{\raggedright Holger Danielsiek, Wolfgang Paul and Jan Vahrenhold, \textit{Technische Universität Dortmund}} \\ \\
\multicolumn{3}{@{}p{5in}}{\small We describe the first results of our work towards a concept inventory for Algorithms and Data structures. Based on expert interviews and the analysis of 400 exams we were able to identify several core topics which are prone to error. In a pilot study, we verified misconceptions known from the literature and identified previously unknown misconceptions related to Algorithms and Data Structures. In addition to this, we report on methodological issues and point out the importance of a two-pronged approach to data collection.} \\ \\
\end{longtable}


\newpage
\begin{longtable}{@{}p{0.75in}@{}p{3.25in}@{}r}
   {\sffamily\large\textbf{PAPERS}} &
   {\raggedright\sffamily\large\textbf{Robots}} & 
   {\sffamily\large\textbf{302B }} \\
%row 2
   Chair:  & 
   {\raggedright Sherri Goings \textit{Carleton College}} & \\ \\
{\sffamily \large 10:45}& 
\multicolumn{2}{@{}p{3.75in}}{\sffamily\raggedright\textbf{A C-based Introductory Course Using Robots}} \\
& \multicolumn{2}{@{}p{3.75in}}{\raggedright Henry Walker, David Cowden, April O'Neill, Erik Opavsky and Dilan Ustek, \textit{Grinnell College}} \\ \\
\multicolumn{3}{@{}p{5in}}{\small Using robots in introductory computer science classes has recently become a popular method of increasing student interest in computer science.  This paper describes the development of a new  curriculum for a CS 2 course, Imperative Problem Solving and Data Structures, based upon Scribbler 2 robots with standard C.  The curriculum contains eight distinct modules with a primary topic theme, readings, labs, and project at the end. Each module resulted from collaboration among former CS 2 students and a faculty member, utilizing an iterative process with revisions.  Each lab includes a survey to obtain student feedback that will allow the course to evolve and better fit the needs of future CS 2 students.  All materials discussed here are available online for use by others.} \\ \\
{\sffamily \large 11:10}& 
\multicolumn{2}{@{}p{3.75in}}{\sffamily\raggedright\textbf{dLife: A Java Library for Multiplatform Robotics, AI and Vision in Undergraduate CS and Research}} \\
& \multicolumn{2}{@{}p{3.75in}}{\raggedright Grant Braught, \textit{Dickinson College}} \\ \\
\multicolumn{3}{@{}p{5in}}{\small dLife is a free and open-source Java library that supports undergraduate education and research involving robotics, artificial intelligence, machine learning and computer vision.  The design of dLife addresses many concerns raised by experience reports in the CS education literature including a shortened code/test/debug cycle, ready access to robot sensor information and close integration with a robotic simulation system. Support is currently provided for a variety of popular educational and research robots. Easily extensible packages for neural networks, genetic algorithms, reinforcement learning and computer vision support both classroom and research applications.} \\ \\
{\sffamily \large 11:35}& 
\multicolumn{2}{@{}p{3.75in}}{\sffamily\raggedright\textbf{Seven Big Ideas in Robotics, and How To Teach Them}} \\
& \multicolumn{2}{@{}p{3.75in}}{\raggedright David S. Touretzky, \textit{Carnegie Mellon University}} \\ \\
\multicolumn{3}{@{}p{5in}}{\small Following the curriculum design principles of Wiggins and McTighe (Understanding by Design, 2nd Ed., 2005), I present seven big ideas in robotics that can fit together in a one semester undergraduate course.  Each is introduced with an essential question, such as "How do robots see the world?"  The answers expose students to deep concepts in computer science in a context where they can be immediately put to the test.  Hands-on demonstrations and labs using the Tekkotsu open source software framework and robots costing under \$1,000 facilitate mastery of these important ideas.  Courses based on parts of an early version of this curriculum are being offered at Carnegie Mellon and several other universities.} \\ \\
\end{longtable}


\newpage
\begin{longtable}{@{}p{0.75in}@{}p{3.25in}@{}r}
   {\sffamily\large\textbf{PAPERS}} &
   {\raggedright\sffamily\large\textbf{K-6 Collaborations}} & 
   {\sffamily\large\textbf{306A }} \\
%row 2
   Chair:  & 
   {\raggedright Sheila Castaneda \textit{Clarke University}} & \\ \\
{\sffamily \large 10:45}& 
\multicolumn{2}{@{}p{3.75in}}{\sffamily\raggedright\textbf{Design and Evaluation of a Braided Teaching Course in Sixth Grade Computer Science Education}} \\
& \multicolumn{2}{@{}p{3.75in}}{\raggedright Arno Pasternak, \textit{Fritz-Steinhoff-Gesamtschule Hagen and Technische Universität Dortmund}; Jan Vahrenhold, \textit{Technische Universität Dortmund}} \\ \\
\multicolumn{3}{@{}p{5in}}{\small We report on the design and evaluation of the first year of a CS course in lower secondary education that implements the concept of braided teaching. Besides being a proof-of-concept, our study demonstrates that students can indeed be taught CS (as opposed to ICT) as early as in sixth grade while at the same time not falling behind with respect to IT Literacy. We present quantitative and qualitative results and argue that Computer Science can be taught just like any other Science worth full curriculum credit.} \\ \\
{\sffamily \large 11:10}& 
\multicolumn{2}{@{}p{3.75in}}{\sffamily\raggedright\textbf{Parallel Programming in Elementary School}} \\
& \multicolumn{2}{@{}p{3.75in}}{\raggedright Chris Gregg, Luther Tychonievich, Kim Hazelwood and James Cohoon, \textit{University of Virginia}} \\ \\
\multicolumn{3}{@{}p{5in}}{\small Traditional introductory programming classes focus on teaching sequential programming skills using conventional programming languages and single-threaded applications. Students rarely learn about parallel programming until much later in their careers. Today, there is a greater need for programmers who are not only proficient in parallel programming, but who are not burdened by previously learned sequential programming habits, with parallelism tacked on as an afterthought.

We present an introductory parallel programming course we  taught to a group of primary school students using a novel parallel programming language.  We provide examples of student-written code and we describe the overall course goal and specific lesson plans geared towards teaching students how to "think parallel."} \\ \\
{\sffamily \large 11:35}& 
\multicolumn{2}{@{}p{3.75in}}{\sffamily\raggedright\textbf{Building Upon and Enriching Grade Four Mathematics Standards with Programming Curriculum}} \\
& \multicolumn{2}{@{}p{3.75in}}{\raggedright Colleen M. Lewis and Niral Shah, \textit{University of California, Berkeley}} \\ \\
\multicolumn{3}{@{}p{5in}}{\small We found that fifth grade students’ performance on Scratch programming quizzes in a summer enrichment course was highly correlated with their scores on a standardized test for Mathematics. We identify ways in which the programming curriculum builds upon target skills from the California state Mathematics standards to help understand opportunities for building upon and enriching Mathematics content through programming curriculum.} \\ \\
\end{longtable}


\newpage
\begin{longtable}{@{}p{0.75in}@{}p{3.25in}@{}r}
   {\sffamily\large\textbf{PAPERS}} &
   {\raggedright\sffamily\large\textbf{Tools}} & 
   {\sffamily\large\textbf{306B }} \\
%row 2
   Chair:  & 
   {\raggedright Sage Miller \textit{Webster Central School District}} & \\ \\
{\sffamily \large 10:45}& 
\multicolumn{2}{@{}p{3.75in}}{\sffamily\raggedright\textbf{Calico: A Multi-Programming-Language, Multi-Context Framework Designed for Computer Science Education}} \\
& \multicolumn{2}{@{}p{3.75in}}{\raggedright James Marshall, \textit{Sarah Lawrence College}; Douglas Blank and Mark Russo, \textit{Bryn Mawr College}; Jennifer S. Kay, \textit{Rowan University}; Keith O'Hara, \textit{Bard College}} \\ \\
\multicolumn{3}{@{}p{5in}}{\small The Calico project is a multi-language, multi-context programming framework and learning environment for computing education. This environment is designed to support several interoperable programming languages (including Python, Scheme, and a visual programming language), a variety of pedagogical contexts (including scientific visualization, robotics, and art), and an assortment of physical devices (including different educational robotics platforms and a variety of physical sensors).  In addition, the environment is designed to support collaboration and modern, interactive learning.  In this paper we describe the Calico project, its design and goals, our prototype system, and its current use.} \\ \\
{\sffamily \large 11:10}& 
\multicolumn{2}{@{}p{3.75in}}{\sffamily\raggedright\textbf{How a Language-based GUI Generator Can Influence the Teaching of Object-Oriented Programming}} \\
& \multicolumn{2}{@{}p{3.75in}}{\raggedright Prasun Dewan, \textit{University of North Carolina}} \\ \\
\multicolumn{3}{@{}p{5in}}{\small We have built a language-based direct-manipulation user-interface generator that can change, and we argue, improve the lectures and assignments on programming conventions, methods, state, constructors, preconditions, MVC, polymorphism, graphics, structured objects, loops, concurrency, and annotations. Our generator has several novel features for teaching such as interactive instantiation of a class, interactive invocation of methods and constructors that take arbitrary arguments, visualization of objects representing records, sequences, table and graphics, use of preconditions to disable/enable user-interface elements, and automatic generation of model threads.} \\ \\
{\sffamily \large 11:35}& 
\multicolumn{2}{@{}p{3.75in}}{\sffamily\raggedright\textbf{CodeWave: A Real-Time, Collaborative IDE for Enhanced Learning in Computer Science}} \\
& \multicolumn{2}{@{}p{3.75in}}{\raggedright Jason Vandeventer and Benjamin Barbour, \textit{University of North Carolina Wilmington}} \\ \\
\multicolumn{3}{@{}p{5in}}{\small Computer science instructors often rely on the final version of a program for assessment and feedback. This ignores the process the student used to arrive at the final program. When the instructor has the ability to observe real-time development progress of each student, they are better equipped to provide appropriate and timely feedback. CodeWave, a software program developed at the University of North Carolina Wilmington looks to alleviate these issues.
CodeWave is a real-time, collaborative Integrated Development Environment with traditional features such as syntax highlighting and non-traditional features such as integrated messaging and logged playback.  CodeWave enhances productivity by integrating many common tools students and instructors use during the programming process.} \\ \\
\end{longtable}


\begin{longtable}[l]{@{}p{1in}@{}p{3in}@{}r}
    {\sffamily\large\textbf{SupporterSession}} & 
    {\sffamily\large\textbf{TBA}} & 
    {\sffamily\large\textbf{302C}} \\
\end{longtable}    
\begin{longtable}[l]{@{}p{1in}@{}p{3in}@{}r}
    {\sffamily\large\textbf{SupporterSession}} & 
    {\sffamily\large\textbf{Supporter Session: MicrosoftEmpowering Students: Teaching Software Development with Windows Phone}} & 
    {\sffamily\large\textbf{305A}} \\
\end{longtable}    
\vspace{0.5em}
\noindent\rule{5in}{0.02cm}
\vspace{0.5em}
\cfoot{\colorbox[gray]{0.45}{\color{white}\textsf{Thursday 12:00 - 12:15}}}
\noindent
\framebox[5in][c]{{\Large\sffamily\textbf{Thursday,  12:00 to 12:15}}}
\vspace{0.5em}
\noindent\rule{5in}{0.02cm}
\vspace{0.5em}
\cfoot{\colorbox[gray]{0.45}{\color{white}\textsf{Thursday 12:00 - 13:45}}}
\noindent
\framebox[5in][c]{{\Large\sffamily\textbf{Thursday,  12:00 to 13:45}}}
\begin{longtable}[l]{@{}p{1in}@{}p{3in}@{}r}
    {\sffamily\large\textbf{Tasty}} & 
    {\sffamily\large\textbf{First Timer's LunchSpeaker: Jane Prey, Winner of 2012 SIGCSE Award for Lifetime Service}} & 
    {\sffamily\large\textbf{Marriott State CDEF}} \\
\end{longtable}    
\begin{longtable}[l]{@{}p{1in}@{}p{3in}@{}r}
    {\sffamily\large\textbf{Tasty}} & 
    {\sffamily\large\textbf{Lunch: On Your Own}} & 
    {\sffamily\large\textbf{On your own}} \\
\end{longtable}    
\vspace{0.5em}
\noindent\rule{5in}{0.02cm}
\vspace{0.5em}
\cfoot{\colorbox[gray]{0.45}{\color{white}\textsf{Thursday 13:45 - 15:00}}}
\noindent
\framebox[5in][c]{{\Large\sffamily\textbf{Thursday,  13:45 to 15:00}}}
\begin{longtable}[l]{@{}l@{}l@{}r}
    \parbox[t]{1in}{\sffamily\large\textbf{PANEL}} & 
    \parbox[t]{3in}{\sffamily\raggedright\large\textbf{A Stratified View of Programming Language Parallelism for Undergraduate CS Education}} & 
    \parbox[t]{1in}{\sffamily\raggedleft\large\textbf{301AB}} \\
% row 2    
    Chair: & 
    Richard Brown \textit{St. Olaf College}  \\[0.5em]
% row 3
    Participants: & 
    \multicolumn{2}{@{}l}{\parbox{3.75in}{Joel Adams, \textit{Calvin College}; David Bunde, \textit{Knox College}; Jens Mache, \textit{Lewis \& Clark College}; Elizabeth Shoop, \textit{Macalester College} }} \\[2em]
% row 4
    \multicolumn{3}{@{}p{5in}}{\small It is no longer news that undergraduates in computer science need to learn more about parallelism. The range of options for parallel programming is truly staggering, involving hundreds of languages. How can a CS instructor make informed choices among all the options?  This panel provides a guided introduction to parallelism in programming languages and their potential for undergraduate CS education, organized into four progressive categories:  low-level libraries and; higher-level libraries and features; programming languages that incorporate parallelism; and frameworks for productive parallel programming. The four panelists will present representative examples in their categories, then present viewpoints on how those categories relate to coursework, curriculum, and trends in parallelism.}
\end{longtable}
\begin{longtable}[l]{@{}l@{}l@{}r}
    \parbox[t]{1in}{\sffamily\large\textbf{PANEL}} & 
    \parbox[t]{3in}{\sffamily\raggedright\large\textbf{Demystifying Computing with Magic}} & 
    \parbox[t]{1in}{\sffamily\raggedleft\large\textbf{305B}} \\
% row 2    
    Chair: & 
    Daniel Garcia, \textit{UC Berkeley}  \\[0.5em]
% row 3
    Participants: & 
    \multicolumn{2}{@{}l}{\parbox{3.75in}{David Ginat, \textit{Tel-Aviv University} }} \\[2em]
% row 4
    \multicolumn{3}{@{}p{5in}}{\small One man’s “magic” is another man’s engineering. – Robert A. Heinlein
Many novice students have fuzzy mental models of how the computer works, or worse, sincerely believe that the computer works unpredictably, “by magic”.  We seek to demystify computing by showing them that even magic itself isn’t necessarily mystical; it could just be clever computation. In this session, we will share a variety of magic tricks whose answer is grounded in computer science: modulo arithmetic, permutations, algorithms, binary encoding, etc. For each trick, we will have an interactive discussion of its underlying computing fundamentals, and tips for successful showmanship. Audience participation will be critical, for helping us perform the magic, discussing the solution, and contributing other magic tricks.}
\end{longtable}
\begin{longtable}[l]{@{}l@{}l@{}r}
    \parbox[t]{1in}{\sffamily\large\textbf{PANEL}} & 
    \parbox[t]{3in}{\sffamily\raggedright\large\textbf{Community-Based Projects for Computing Majors:  Opportunities, Challenges and Best Practices}} & 
    \parbox[t]{1in}{\sffamily\raggedleft\large\textbf{306C}} \\
% row 2    
    Chair: & 
    Jeffrey Stone \textit{Pennsylvania State University}  \\[0.5em]
% row 3
    Participants: & 
    \multicolumn{2}{@{}l}{\parbox{3.75in}{Jeffrey Stone and Elinor Madigan, \textit{Pennsylvania State University}; Janice Pearce, \textit{Berea College}; Bonnie MacKellar, \textit{St. John's University} }} \\[2em]
% row 4
    \multicolumn{3}{@{}p{5in}}{\small The use of community-based projects has been recognized as having pedagogical and experiential value for computing majors. Community-based projects can be valuable learning experiences for computing majors as well as for faculty and community partners. However, these types of projects present challenges for faculty and should be aligned with desired course outcomes. This panel will discuss the use of community-based projects from multiple perspectives. The expectation is that the panel will serve as a forum for participants to share the opportunities, challenges, pedagogical motivations, and best practices obtained from prior experience. Exemplar projects will be highlighted, and audience members will have an opportunity to share their own experiences with community-based projects.}
\end{longtable}
\newpage
\begin{longtable}{@{}p{0.75in}@{}p{3.25in}@{}r}
   {\sffamily\large\textbf{PAPERS}} &
   {\raggedright\sffamily\large\textbf{Games}} & 
   {\sffamily\large\textbf{302A }} \\
%row 2
   Chair:  & 
   {\raggedright Adrienne Decker \textit{Rochester Institute of Technology}} & \\ \\
{\sffamily \large 13:45}& 
\multicolumn{2}{@{}p{3.75in}}{\sffamily\raggedright\textbf{The Five Year Evolution of a Game Programming Course}} \\
& \multicolumn{2}{@{}p{3.75in}}{\raggedright Gillian Smith and Anne Sullivan, \textit{UC Santa Cruz}} \\ \\
\multicolumn{3}{@{}p{5in}}{\small This paper presents lessons learned from five years of teaching a game design and programming outreach course. This class is taught over the course of a month to high school students participating in the California Summer School for Mathematics and Science (COSMOS) at the University of California, Santa Cruz. Over these five years we have changed everything in the course, from the overall project structure to the programming language used in the class. In this paper we discuss our successes and failures, and offer recommendations to instructors offering similar courses.} \\ \\
{\sffamily \large 14:10}& 
\multicolumn{2}{@{}p{3.75in}}{\sffamily\raggedright\textbf{Programming, PWNed: Using Digital Game Development to Enhance Learners’ Competency and Self-Efficacy in a High School Computing Science Course}} \\
& \multicolumn{2}{@{}p{3.75in}}{\raggedright Katie Seaborn, \textit{York University}; Magy Seif El-Nasr, \textit{Northeastern University}; David Milam, \textit{Simon Fraser University}; Darren Yung, \textit{Frank Hurt Secondary School}} \\ \\
\multicolumn{3}{@{}p{5in}}{\small The popularity and inherent engagement of games has caused many educators to start thinking of ways to use game-based techniques to enhance education, particularly to promote STEM (Science Technology, Engineering and Mathematics) concepts to middle and high school students. We report on the design and evaluation of a high school game construction-based curriculum that replaced a traditional computer science class. We collected students’ overall impressions, and evaluated students’ technical competency and self-efficacy at the start and end of the semester. Our findings show that the curriculum had a positive, statistically significant effect on concept comprehension, which suggests that the curriculum was effective for understanding computer science and game design concepts.} \\ \\
{\sffamily \large 14:35}& 
\multicolumn{2}{@{}p{3.75in}}{\sffamily\raggedright\textbf{A Learning Objective Focused Methodology for the Design and Evaluation of Game-based Tutors}} \\
& \multicolumn{2}{@{}p{3.75in}}{\raggedright Michael Eagle and Tiffany Barnes, \textit{University of North Carolina at Charlotte}} \\ \\
\multicolumn{3}{@{}p{5in}}{\small We present a methodology for the design and evaluation of educational games with a focus on well defined learning objectives	and	empirical	verification.Combiningpractices from educational design, intelligent tutoring systems, classical test-theory, and game design, this methodology guides researchers through the steps of the design process, including identification of specific learning objectives, translation of learning activities to game mechanics, and the empirical evaluation of the final product. This methodology is particularly useful for young researchers and educators are encouraged to promote this methodology for use in student research experiences or serious games courses.} \\ \\
\end{longtable}


\newpage
\begin{longtable}{@{}p{0.75in}@{}p{3.25in}@{}r}
   {\sffamily\large\textbf{PAPERS}} &
   {\raggedright\sffamily\large\textbf{Professional Experiences}} & 
   {\sffamily\large\textbf{302B }} \\
%row 2
   Chair:  & 
   {\raggedright Sarah Heckman \textit{North Carolina State University}} & \\ \\
{\sffamily \large 13:45}& 
\multicolumn{2}{@{}p{3.75in}}{\sffamily\raggedright\textbf{Course Guides: A Model for Bringing  Professionals into the Classroom}} \\
& \multicolumn{2}{@{}p{3.75in}}{\raggedright Thomas Gibbons, \textit{The College of St. Scholastica}} \\ \\
\multicolumn{3}{@{}p{5in}}{\small A new model, professional course guides, describes how practicing professionals can be brought into the classroom as student mentors and integrated into the course material. This new model is compared to existing models for student interactions with practicing professionals including guest speakers, adjunct faculty, and program mentors.} \\ \\
{\sffamily \large 14:10}& 
\multicolumn{2}{@{}p{3.75in}}{\sffamily\raggedright\textbf{Towards a Better Capstone Experience}} \\
& \multicolumn{2}{@{}p{3.75in}}{\raggedright Sriram Mohan, Stephen Chenoweth and Shawn Bohner, \textit{Rose-Hulman Institute of Technology}} \\ \\
\multicolumn{3}{@{}p{5in}}{\small The capstone experience is designed to bridge the gap from university expectations to those of industry.Yet trying to solve this problem with a single course sequence,even one spanning the senior year,has some shortcomings,in terms of learning outcomes which can be achieved,and also instructional strategies that can be employed. We describe a plan which provides a junior year of practice on a client-based project integrated with learning design and other related topics, followed by a senior year in which students can work more independently to hone these skills on a harder year-long project with another client.This sequence,with scaffolding provided at first that is gradually removed,has proven to be especially effective in preparing undergraduates for a career in the software industry.} \\ \\
{\sffamily \large 14:35}& 
\multicolumn{2}{@{}p{3.75in}}{\sffamily\raggedright\textbf{An Open Co-op Model for Global Enterprise Technology Education: Integrating the Internship and Course Work}} \\
& \multicolumn{2}{@{}p{3.75in}}{\raggedright Jeffrey saltz, \textit{JP Morgan Chase}; Jae Oh, \textit{Syracuse University}} \\ \\
\multicolumn{3}{@{}p{5in}}{\small We present an open co-op program called Global Enterprise Technology Immersion Experience (GET IE). The program provides a global enterprise focus integrated with hands-on experiential work-based learning. GET IE includes a two-semester internship that can be seamlessly incorporated within an existing computer science curriculum. 
The internship's  unique pedagogical innovation is to simultaneously provide the students academic course work  that is integrated within a students extended internship and provides relevant problems in  global enterprise technology. The curricula is ``open'' in the sense that other institutions and companies can join the consortium to enrich choices for the students and foster cross-fertilization of curricula activities.} \\ \\
\end{longtable}


\newpage
\begin{longtable}{@{}p{0.75in}@{}p{3.25in}@{}r}
   {\sffamily\large\textbf{PAPERS}} &
   {\raggedright\sffamily\large\textbf{A Session with a View}} & 
   {\sffamily\large\textbf{306A }} \\
%row 2
   Chair:  & 
   {\raggedright Don Goelman \textit{Villanova University}} & \\ \\
{\sffamily \large 13:45}& 
\multicolumn{2}{@{}p{3.75in}}{\sffamily\raggedright\textbf{Integrating Video Components in CS1}} \\
& \multicolumn{2}{@{}p{3.75in}}{\raggedright Tamar Vilner, Ela Zur and Ronit Sagi, \textit{The Open University of Israel}} \\ \\
\multicolumn{3}{@{}p{5in}}{\small The Open University of Israel (OUI) is a distance learning university. Our CS1 course is taught through video-taped lectures that cover the study material. In addition, students may participate in face-to-face group meetings in study centers located all over the country and taught by tutors. There is a special group called Ofek, in which the tutor is located in a studio and the lesson is broadcast over the internet. Students enrolled in this group participate from their home PCs. The taped Ofek sessions as well as the lectures are stored on the course website, and students can watch them whenever convenient. We conducted a study to investigate students’ viewing habits and the relationship between viewing and the success rate in the course.} \\ \\
{\sffamily \large 14:10}& 
\multicolumn{2}{@{}p{3.75in}}{\sffamily\raggedright\textbf{Development and Evaluation of Indexed Captioned Searchable Videos for STEM Coursework}} \\
& \multicolumn{2}{@{}p{3.75in}}{\raggedright Tayfun Tuna, Jaspal Subhlok, Varun Varghese, Olin Johnson and Shishir Shah, \textit{University of Houston - Computer Science Department}; Lecia Barker, \textit{University of Texas-School of Information}} \\ \\
\multicolumn{3}{@{}p{5in}}{\small Videos of classroom lectures have proven to be a popular and versatile learning resource. This paper reports on ICS videos featuring Indexing, Captioning, and Search capability. The goal is to allow a user to rapidly zoom in on a topic of interest, a key shortcoming of the standard video format. A lecture is automatically divided into logical indexed video segments by analyzing video frames. Text is automatically identified with OCR technology and image transformations to drive keyword search. Captions can be added to videos. ICS video player integrates indexing, search, and captioning in video playback and is used by dozens of courses and 1000s of students. The paper reports on development and evaluation of ICS videos framework and assessment of its value as an academic learning resource.} \\ \\
{\sffamily \large 14:35}& 
\multicolumn{2}{@{}p{3.75in}}{\sffamily\raggedright\textbf{metaview: A Tool for Learning About Viewing in 3D}} \\
& \multicolumn{2}{@{}p{3.75in}}{\raggedright James Miller, \textit{University of Kansas}} \\ \\
\multicolumn{3}{@{}p{5in}}{\small Metaview is an interactive tool that helps to teach concepts related to nested 3D coordinate systems, especially in the context of defining and establishing views of 3D scenes in common graphics APIs like OpenGL and Direct3D. We describe the context in which nested coordinate systems arise in the study of graphics programming, then we relate the common conceptual difficulties students typically experience when studying and trying to put this material into practice. We then describe the role that metaview plays in helping to overcome these problems. Metaview is packaged with a set of built-in 3D models used to demonstrate major concepts. In addition, external and/or student-programmed models are easily imported into the tool. Metaview can be run anywhere, anytime using Java Web Start.} \\ \\
\end{longtable}


\newpage
\begin{longtable}{@{}p{0.75in}@{}p{3.25in}@{}r}
   {\sffamily\large\textbf{PAPERS}} &
   {\raggedright\sffamily\large\textbf{Pedagogy:  Programming}} & 
   {\sffamily\large\textbf{306B }} \\
%row 2
   Chair:  & 
   {\raggedright Saquib Razak \textit{Carnegie Mellon University}} & \\ \\
{\sffamily \large 13:45}& 
\multicolumn{2}{@{}p{3.75in}}{\sffamily\raggedright\textbf{Mediated Transfer: Alice 3 to Java}} \\
& \multicolumn{2}{@{}p{3.75in}}{\raggedright Wanda Dann, Dennis Cosgrove, Don Slater and Dave Culyba, \textit{Carnegie Mellon University}; Steve Cooper, \textit{Stanford University}} \\ \\
\multicolumn{3}{@{}p{5in}}{\small In this paper, we describe a pedagogy for an undergraduate programming course using Alice 3 and Java. We applied the educational theory of mediated transfer to develop a new version of the Alice system and accompanying instructional materials. The pedagogy was implemented and tested over two years. Student test scores in experimental, treatment course sections showed a dramatic increase in scores over comparable, non-treatment sections.} \\ \\
{\sffamily \large 14:10}& 
\multicolumn{2}{@{}p{3.75in}}{\sffamily\raggedright\textbf{Over-Confidence and Confusion in Using Bloom for Programming Fundamentals Assessment}} \\
& \multicolumn{2}{@{}p{3.75in}}{\raggedright Richard Gluga, Judy Kay, Sabina Kleitman and Tim Lever, \textit{University of Sydney}; Raymond Lister, \textit{University of Technology Sydney}} \\ \\
\multicolumn{3}{@{}p{5in}}{\small A computer science student is required to progress from a novice to an expert through the CS1/2 programming fundamentals sequence. The key contribution is a web-based interactive tutorial that enables computer science educators to practice applying the Bloom Taxonomy in classifying programming exam questions, to define this learning progression. The results of an evaluation with ten participants were analyzed for consistency and accuracy in the application of Bloom. Confidence and self-explanation measures were used to identify problem areas in the application of Bloom to programming fundamentals. The tutorial and findings are valuable contributions to future ACM/IEEE CS curriculum revisions, which are expected to have a continued emphasis on Bloom [9].} \\ \\
{\sffamily \large 14:35}& 
\multicolumn{2}{@{}p{3.75in}}{\sffamily\raggedright\textbf{Modeling How Students Learn to Program}} \\
& \multicolumn{2}{@{}p{3.75in}}{\raggedright Stephen Cooper, Chris Piech, Mehran Sahami, Daphne Koller and Paulo Blikstein, \textit{Stanford University}} \\ \\
\multicolumn{3}{@{}p{5in}}{\small Despite the potential wealth of educational indicators expressed in a student’s approach to homework assignments, how students arrive at their final solution is largely overlooked in university courses. In this paper we present a methodology which uses machine learning techniques to autonomously create a graphical model of how students in an introductory programming course progress through a homework assignment. We subsequently show that this model is predictive of which students will struggle with material presented later in the class.} \\ \\
\end{longtable}


\begin{longtable}[l]{@{}p{1in}@{}p{3in}@{}r}
    {\sffamily\large\textbf{SupporterSession}} & 
    {\sffamily\large\textbf{Supporter Session: Intel}} & 
    {\sffamily\large\textbf{302C}} \\
\end{longtable}    
\begin{longtable}[l]{@{}p{1in}@{}p{3in}@{}r}
    {\sffamily\large\textbf{SupporterSession}} & 
    {\sffamily\large\textbf{Supporter Session: MicrosoftCreative Uses for Kinect in Teaching – with Curriculum Materials}} & 
    {\sffamily\large\textbf{305A}} \\
\end{longtable}    
\vspace{0.5em}
\noindent\rule{5in}{0.02cm}
\vspace{0.5em}
\cfoot{\colorbox[gray]{0.45}{\color{white}\textsf{Thursday 13:45 - 17:15}}}
\noindent
\framebox[5in][c]{{\Large\sffamily\textbf{Thursday,  13:45 to 17:15}}}
\begin{longtable}[l]{@{}p{1in}@{}p{3in}@{}r}
    {\sffamily\large\textbf{Presentations}} & 
    {\sffamily\large\textbf{Student Research Poster Session}} & 
    {\sffamily\large\textbf{Exhibit Hall A}} \\
\end{longtable}    
\vspace{0.5em}
\noindent\rule{5in}{0.02cm}
\vspace{0.5em}
\cfoot{\colorbox[gray]{0.45}{\color{white}\textsf{Thursday 15:00 - 15:45}}}
\noindent
\framebox[5in][c]{{\Large\sffamily\textbf{Thursday,  15:00 to 15:45}}}
\begin{longtable}[l]{@{}p{1in}@{}p{3in}@{}r}
    {\sffamily\large\textbf{None}} & 
    {\sffamily\large\textbf{Break and Exhibits}} & 
    {\sffamily\large\textbf{Exhibit Hall A}} \\
\end{longtable}    
\vspace{0.5em}
\noindent\rule{5in}{0.02cm}
\vspace{0.5em}
\cfoot{\colorbox[gray]{0.45}{\color{white}\textsf{Thursday 15:00 - 16:30}}}
\noindent
\framebox[5in][c]{{\Large\sffamily\textbf{Thursday,  15:00 to 16:30}}}
\begin{longtable}[l]{@{}p{1in}@{}p{3in}@{}r}
    {\sffamily\large\textbf{Project Showcase}} & 
    {\sffamily\large\textbf{NSF Showcase \#2}} & 
    {\sffamily\large\textbf{Exhibit Hall A}} \\
\end{longtable}    
\vspace{0.5em}
\noindent\rule{5in}{0.02cm}
\vspace{0.5em}
\cfoot{\colorbox[gray]{0.45}{\color{white}\textsf{Thursday 15:45 - 17:00}}}
\noindent
\framebox[5in][c]{{\Large\sffamily\textbf{Thursday,  15:45 to 17:00}}}
\begin{longtable}[l]{@{}l@{}l@{}r}
    \parbox[t]{1in}{\sffamily\large\textbf{PANEL}} & 
    \parbox[t]{3in}{\sffamily\raggedright\large\textbf{Science Fiction in Computer Science Education}} & 
    \parbox[t]{1in}{\sffamily\raggedleft\large\textbf{301AB}} \\
% row 2    
    Chair: & 
    Rebecca Bates \textit{Minnesota State University, Mankato}  \\[0.5em]
% row 3
    Participants: & 
    \multicolumn{2}{@{}l}{\parbox{3.75in}{Judy Goldsmith, \textit{University of Kentucky}; Rosalyn Berne, \textit{University of Virginia}; Valerie Summet, \textit{Emory University}; Nanette Veilleux, \textit{Simmons College} }} \\[2em]
% row 4
    \multicolumn{3}{@{}p{5in}}{\small The use of science fiction to engage students in computer science learning is becoming more popular, with ample material available to help students make connections between technical content and human experience, from Star Trek to The Hitchhiker’s Guide to the Galaxy to 2001 to I, Robot to …  Fiction can be included in technical courses or used to draw students into the field in introductory classes. The panelists, who represent a range of schools, perspectives and classes, will present brief overviews of how they have used science fiction to engage students in technical topics as well as related ethical and societal issues. After the overviews, there will be plenty of time for discussion of examples and ways to make connections between science fiction and particular classes or topics.}
\end{longtable}
\begin{longtable}[l]{@{}l@{}l@{}r}
    \parbox[t]{1in}{\sffamily\large\textbf{PANEL}} & 
    \parbox[t]{3in}{\sffamily\raggedright\large\textbf{Diversity Initiatives to Support Systemic Change in Undergraduate Computing}} & 
    \parbox[t]{1in}{\sffamily\raggedleft\large\textbf{305B}} \\
% row 2    
    Chair: & 
    Leisa D. Thompson \textit{University of Virginia}  \\[0.5em]
% row 3
    Participants: & 
    \multicolumn{2}{@{}l}{\parbox{3.75in}{Lecia Barker, \textit{University of Texas}; Rita Manco Powell, \textit{University of Pennsylvania}; Catherine Brawner, \textit{Research Triangle Educational Consultants}; Tom McKlin, \textit{The Findings Group, LLC} }} \\[2em]
% row 4
    \multicolumn{3}{@{}p{5in}}{\small The National Center for Women \& Information Technology (NCWIT) Extension Services for Undergraduate Programs (ES-UP) has created a large group of trained consultants (ESCs) and clients who are passionate about women’s participation in computing. This panel will describe how our ESCs and clients have worked together to effect change and will show outcomes from our activities over the past three years.}
\end{longtable}
\begin{longtable}[l]{@{}l@{}l@{}r}
    \parbox[t]{1in}{\sffamily\large\textbf{PANEL}} & 
    \parbox[t]{3in}{\sffamily\raggedright\large\textbf{Transforming the CS Classroom with Studio-Based Learning}} & 
    \parbox[t]{1in}{\sffamily\raggedleft\large\textbf{306C}} \\
% row 2    
    Chair: & 
    Christopher Hundhausen, \textit{Washington State University}  \\[0.5em]
% row 3
    Participants: & 
    \multicolumn{2}{@{}l}{\parbox{3.75in}{N. Hari Narayanan and Dean Hendrix, \textit{Auburn University}; Martha Crosby, \textit{University of Hawaii - Manoa} }} \\[2em]
% row 4
    \multicolumn{3}{@{}p{5in}}{\small The studio-based learning (SBL) model has been the centerpiece of architecture and fine arts education for over a century. Over the past five years, we have been adapting SBL for computing education and empirically evaluating its impact. This effort now involves 26 computing courses at 15 institutions in seven states. To our knowledge, this is the largest implementation and evaluation of a pedagogy for computing education to date. This special session will introduce SBL to a general audience and facilitate a discussion and exchange of ideas. In addition to oral and poster presentations of the SBL model and its evaluation results, the session will feature "war stories" from teachers who have used SBL in their courses, and hands-on activities to help attendees apply SBL to their courses.}
\end{longtable}
\newpage
\begin{longtable}{@{}p{0.75in}@{}p{3.25in}@{}r}
   {\sffamily\large\textbf{PAPERS}} &
   {\raggedright\sffamily\large\textbf{Broadening Participation}} & 
   {\sffamily\large\textbf{302A }} \\
%row 2
   Chair:  & 
   {\raggedright Kristine Nagel \textit{Georgia Gwinnett College}} & \\ \\
{\sffamily \large 15:45}& 
\multicolumn{2}{@{}p{3.75in}}{\sffamily\raggedright\textbf{Making Turing Machines Accessible to Blind Students}} \\
& \multicolumn{2}{@{}p{3.75in}}{\raggedright Pierluigi Crescenzi, Leonardo Rossi and Gianluca Apollaro, \textit{University of Florence}} \\ \\
\multicolumn{3}{@{}p{5in}}{\small In this paper we describe how we tried to make the JFLAP Turing machine simulator accessible to blind students. Software accessibility is an important topic for both legal and ethical reasons: in our case, however, we also wanted to make the accessible software usable by blind students in cooperation with the other students, in order to encourage the integration of the blind students within the rest of the class. For this reason, the accessible version of the JFLAP Turing machine simulator that we developed is as much similar as possible to and fully compatible with the original one. In the paper, we also report some very satisfactory preliminary validation results that indicate how the new software can really make Turing machines accessible to blind students.} \\ \\
{\sffamily \large 16:10}& 
\multicolumn{2}{@{}p{3.75in}}{\sffamily\raggedright\textbf{Toward an Emergent Theory of Broadening Participation in Computer Science Education}} \\
& \multicolumn{2}{@{}p{3.75in}}{\raggedright David Webb, Alexander Repenning and Kyu Han Koh, \textit{University of Colorado at Boulder}} \\ \\
\multicolumn{3}{@{}p{5in}}{\small A fundamental challenge to computer science education is the difficulty of broadening participation of women and underserved communities. The idea of game design and game programming as an activity to introduce children at an early age to computational thinking in a motivational way is quickly gaining momentum. A pedagogical approach called Project First has allowed the Scalable Game Design project to reach over 4,000 middle schools students including a large percentage of female (45\%) and underrepresented (48\%) students. Our analysis of student motivation data, gender ratios and pedagogical approaches employed by teachers such as mediation and scaffolding suggests strong gender effects based on gender ratios and classroom scaffolding approaches.} \\ \\
{\sffamily \large 16:35}& 
\multicolumn{2}{@{}p{3.75in}}{\sffamily\raggedright\textbf{Exploring Formal Learning Groups and their Impact on Recruitment of Women in Undergraduate CS}} \\
& \multicolumn{2}{@{}p{3.75in}}{\raggedright Julie Krause, Irene Polycarpou and Keith Hellman, \textit{Colorado School of Mines}} \\ \\
\multicolumn{3}{@{}p{5in}}{\small As percentages of women in computing jobs and university programs decline, recruiting and retaining women in the field of Computer Science (CS) becomes increasingly important. Undergraduate CS programs, and more specifically, introductory-level CS courses, offer an opportunity to introduce women to CS studies. Furthermore, learning experiences in introductory CS courses can be pivotal in shaping female students’ perceptions of CS. Collaborative learning, in various forms, is an instructional construct that has been shown to be effective in recruiting and retaining women in undergraduate CS programs. In this paper we present an exploratory study on formal learning groups and their potential to attract and maintain students’ interest in CS studies.} \\ \\
\end{longtable}


\newpage
\begin{longtable}{@{}p{0.75in}@{}p{3.25in}@{}r}
   {\sffamily\large\textbf{PAPERS}} &
   {\raggedright\sffamily\large\textbf{Online Collaboration}} & 
   {\sffamily\large\textbf{302B }} \\
%row 2
   Chair:  & 
   {\raggedright Charles Leska \textit{Randolph-Macon College}} & \\ \\
{\sffamily \large 15:45}& 
\multicolumn{2}{@{}p{3.75in}}{\sffamily\raggedright\textbf{Perspectives on Active Learning and Collaboration: JavaWIDE in the Classroom}} \\
& \multicolumn{2}{@{}p{3.75in}}{\raggedright Jam Jenkins, \textit{Valdosta State University}; Evelyn Brannock and Sonal Dekhane, \textit{Georgia Gwinnett College}; Thomas Cooper, \textit{The Walker School}; Mark Hall, \textit{University of Wisconsin - Marathon County}; Michael Nguyen, \textit{Emory University}} \\ \\
\multicolumn{3}{@{}p{5in}}{\small JavaWIDE is an innovative environment that promotes active learning and collaboration in programming courses. This paper discusses where and how JavaWIDE has been used to promote active and collaborative learning in both traditional and synchronous distance education courses in four different environments: high school, summer enrichment, and at two- and four-year colleges. After discussing the educational atmosphere and how active learning and collaboration are used in the courses, student responses to the experience are summarized. This collection of case studies illustrates how the concurrent editing, shared environment awareness and other features of JavaWIDE can be used to promote active learning and collaboration within a heterogeneous set of teaching and learning environments.} \\ \\
{\sffamily \large 16:10}& 
\multicolumn{2}{@{}p{3.75in}}{\sffamily\raggedright\textbf{How Well Do Online Forums Facilitate Discussion and Collaboration Among Novice Animation Programmers?}} \\
& \multicolumn{2}{@{}p{3.75in}}{\raggedright Christopher Scaffidi, Aniket Dahotre and Yan Zhang, \textit{Oregon State University}} \\ \\
\multicolumn{3}{@{}p{5in}}{\small Animation programming is a widely-respected approach for helping students to learn programming skills, and online forums are a widely-used approach for helping students to interact with one another. But in what ways, if any, does combining animation programming with online forums lead to useful discussion and collaboration among learners? To answer this question, we analyzed online forum discussions among people who were learning to create animation programs using the Scratch programming environment. We discovered that specific kinds of online posts were more likely than others to be followed by discussion, and we found that the ensuing collaboration often involved the exchange of design ideas and feedback within small groups of users.} \\ \\
{\sffamily \large 16:35}& 
\multicolumn{2}{@{}p{3.75in}}{\sffamily\raggedright\textbf{Classroom Salon: A Tool for Social Collaboration}} \\
& \multicolumn{2}{@{}p{3.75in}}{\raggedright John Barr, \textit{Ithaca College}; Ananda Gunawardena, \textit{Carnegie Mellon University}} \\ \\
\multicolumn{3}{@{}p{5in}}{\small Classroom Salon is an on-line social collaboration tool that allows instructors to create, manage, and analyze social networks (called Salons) to enhance student learning. Students in a Salon can cooperatively create, comment on, and modify documents. Classroom Salon provides tools that allow the instructor to monitor the social networks and gauge both student participation and individual effectiveness. This paper describes Classroom Salon, provides some use cases that we have developed for introductory computer science classes and presents some preliminary observations of using this tool in several computer science courses at Carnegie Mellon University.} \\ \\
\end{longtable}


\newpage
\begin{longtable}{@{}p{0.75in}@{}p{3.25in}@{}r}
   {\sffamily\large\textbf{PAPERS}} &
   {\raggedright\sffamily\large\textbf{Middle School Collaborations}} & 
   {\sffamily\large\textbf{306A }} \\
%row 2
   Chair:  & 
   {\raggedright Catherine Lang \textit{Swinburne University of Technology}} & \\ \\
{\sffamily \large 15:45}& 
\multicolumn{2}{@{}p{3.75in}}{\sffamily\raggedright\textbf{Bringing The Breadth of Computer Science to Middle Schools}} \\
& \multicolumn{2}{@{}p{3.75in}}{\raggedright Elizabeth Carter and Glenn Blank, \textit{Lehigh University}; Jennifer Walz, \textit{Harrison Morton Middle School}} \\ \\
\multicolumn{3}{@{}p{5in}}{\small In order to garner more student interest in the pursuit of computer science as both a major and a career path, K-12 students need to be made aware of what computer science is and what it is about earlier in their education.  Although students in many high schools can pursue introductory programming, high school is arguably too late to interest students who may have developed ill-informed attitudes about computer science.  This paper documents curricular items developed for and taught to an audience of mixed ability 6th through 8th graders taking a local Technology Education class that attempts to showcase some of the more interesting, less stereotypical, aspects of computer science using a breadth approach in an effort to encourage interest in the field.} \\ \\
{\sffamily \large 16:10}& 
\multicolumn{2}{@{}p{3.75in}}{\sffamily\raggedright\textbf{Integrating Hard and Soft Skills: Software Engineers Serving Middle School Teachers}} \\
& \multicolumn{2}{@{}p{3.75in}}{\raggedright Richard Burns, Lori Pollock and Terry Harvey, \textit{University of Delaware}} \\ \\
\multicolumn{3}{@{}p{5in}}{\small We have developed and implemented, in four instances, a model for engaging computer science majors in integrating computing into teaching at a K-8 school in an underserved community.  This paper describes the design of the service learning course particularly focused on interweaving  software engineering practice, service learning, and development of soft skills. 
CS student teams partner with middle school teacher teams to create learning games, 
and conduct classroom instruction and observation.  We report on our results from evaluating the impact of the course experience on the CS students and middle school teachers through pre-post surveys, evaluator observation of student demo presentations and classroom instruction, focus groups, and student reflective journals.} \\ \\
{\sffamily \large 16:35}& 
\multicolumn{2}{@{}p{3.75in}}{\sffamily\raggedright\textbf{The Fairy Performance Assessment: Measuring Computational Thinking in Middle School}} \\
& \multicolumn{2}{@{}p{3.75in}}{\raggedright Linda Werner, \textit{University of California, Santa Cruz}; Jill Denner and Shannon Campe, \textit{ETR Associates}; Damon Chizuru Kawamoto, \textit{Brown University}} \\ \\
\multicolumn{3}{@{}p{5in}}{\small Computational thinking (CT) has been described as an essential capacity to prepare students for computer science, as well as to be productive members of society. But efforts to engage K-12 students in CT are hampered by a lack of definition and assessment tools. In this paper, we describe the first results of a newly created performance assessment tool for measuring CT in middle school. We briefly describe the context for the performance assessment (game-programming courses), the aspects of CT that are measured, the results, and the factors that are associated with performance. We see the development of assessment tools as a critical step in efforts to bring CT to K-12, and to strengthen the use of game programming in middle school. We discuss problems and implications of our results.} \\ \\
\end{longtable}


\newpage
\begin{longtable}{@{}p{0.75in}@{}p{3.25in}@{}r}
   {\sffamily\large\textbf{PAPERS}} &
   {\raggedright\sffamily\large\textbf{New Tricks for the Classroom}} & 
   {\sffamily\large\textbf{306B }} \\
%row 2
   Chair:  & 
   {\raggedright Julian Mason \textit{Duke University (PhD Student)}} & \\ \\
{\sffamily \large 15:45}& 
\multicolumn{2}{@{}p{3.75in}}{\sffamily\raggedright\textbf{Running Students’ Software Tests Against Each Others’ Code: New Life for an Old “Gimmick”}} \\
& \multicolumn{2}{@{}p{3.75in}}{\raggedright Stephen Edwards, Zalia Shams, Michael Cogswell and Robert Senkbeil, \textit{Virginia Tech, Department of Computer Science}} \\ \\
\multicolumn{3}{@{}p{5in}}{\small At SIGCSE'02, Goldwasser suggested including testing in assignments and then running every student’s tests against every other student’s program.  This provides more insight into the quality of a student's tests as well as her solution.  Software testing is more common now, but the all-pairs model of executing tests is still rare.  This is because student-written tests, such as JUnit tests, take the form of program code and may depend on any aspect of their author’s own solution, and these dependencies can keep tests from compiling against other programs.  We discusses this problem and present a Java solution using bytecode rewriting and reflection.  Results of applying this technique to two assignments involving 147 student and 240,158 individual test case runs demonstrates feasibility.} \\ \\
{\sffamily \large 16:10}& 
\multicolumn{2}{@{}p{3.75in}}{\sffamily\raggedright\textbf{Group Note-Taking in a Large Lecture Class: Design, Implementation, and Evaluation of a Low-Cost Universal Design Practice}} \\
& \multicolumn{2}{@{}p{3.75in}}{\raggedright Christopher Plaue, Sal LaMarca and Shelby H. Funk, \textit{The University of Georgia}} \\ \\
\multicolumn{3}{@{}p{5in}}{\small We created a group note-taking system in our large intro computer science course to increase interaction amongst students, promote good note-taking strategies, and provide study resources to all students while also fulfilling the role of accommodating for students with learning disabilities. We show that the section of the course taught with our intervention performed significantly better on their final examination compared to a course taught without the intervention. We report that students enjoyed increased interactions with their peers, and that a third of the class self-reported an increase in their note-taking skills. Furthermore, our intervention only required minimal cost to the institution, and no financial cost to students, and is easily implemented in any size class.} \\ \\
{\sffamily \large 16:35}& 
\multicolumn{2}{@{}p{3.75in}}{\sffamily\raggedright\textbf{Following a Thread: Knitting Patterns and Program Tracing}} \\
& \multicolumn{2}{@{}p{3.75in}}{\raggedright Michelle Craig, \textit{University of Toronto}; Sarah Petersen, \textit{N/A}; Andrew Petersen, \textit{University of Toronto Mississauga}} \\ \\
\multicolumn{3}{@{}p{5in}}{\small This paper presents observations about teaching program tracing to novices drawn from a study of knitting patterns. Knitting patterns have evolved from vague, chatty discourse written for experts to precise, line-by-line procedures akin to programs. The knitting community has developed conventions for articulating iteration, conditions, and design decisions. "Executing" one of these patterns is analogous to tracing, so we argue that conventions adopted by knitters to make their patterns comprehensible to non-experts provide insights about teaching tracing to novices. Our observations suggest that using "until" instead of "while" and partially unrolling loops may help beginners understand code and that some structures, like parameters, may be unfamiliar.} \\ \\
\end{longtable}


\begin{longtable}[l]{@{}p{1in}@{}p{3in}@{}r}
    {\sffamily\large\textbf{SupporterSession}} & 
    {\sffamily\large\textbf{TBA}} & 
    {\sffamily\large\textbf{302C}} \\
\end{longtable}    
\begin{longtable}[l]{@{}p{1in}@{}p{3in}@{}r}
    {\sffamily\large\textbf{SupporterSession}} & 
    {\sffamily\large\textbf{Supporter Session: GoogleAll Things Google and Education}} & 
    {\sffamily\large\textbf{305A}} \\
\end{longtable}    
\vspace{0.5em}
\noindent\rule{5in}{0.02cm}
\vspace{0.5em}
\cfoot{\colorbox[gray]{0.45}{\color{white}\textsf{Thursday 17:10 - 18:00}}}
\noindent
\framebox[5in][c]{{\Large\sffamily\textbf{Thursday,  17:10 to 18:00}}}
\begin{longtable}[l]{@{}l@{}l@{}r}
    \parbox[t]{1in}{\sffamily\large\textbf{BOF}} & 
    \parbox[t]{3in}{\sffamily\raggedright\large\textbf{CS Unplugged, Outreach and CS Kinesthetic Activities}} & 
    \parbox[t]{1in}{\sffamily\raggedleft\large\textbf{201}} \\
% row 2    
    Chair: & 
    Lynn Lambert \textit{Christopher Newport University}  \\[0.5em]
% row 3
    Participants: & 
    \multicolumn{2}{@{}l}{\parbox{3.75in}{Tim Bell University of Canterbury; Daniela Marghitu Auburn }} \\[2em]
% row 4
    \multicolumn{3}{@{}p{5in}}{\small Outreach activities including Computer Science Unplugged demonstrate computer science concepts at schools and public venues based around kinesthetic activities rather than hands-on computer use. Computer Science Unplugged is a global project with many such activities for children to adults using no technology, including how binary numbers represent words, images and sound, routing and deadlock, public/private key encryption, and others. Effective outreach programs such as this combats the idea that computer science = programming or, worse, keyboarding; and can educate the public, interest students, and recruit majors.  Many people have used these activities, and adapted them for their own culture or outreach purposes. Come share your outreach ideas and experiences with such activities.}
\end{longtable}
\begin{longtable}[l]{@{}l@{}l@{}r}
    \parbox[t]{1in}{\sffamily\large\textbf{BOF}} & 
    \parbox[t]{3in}{\sffamily\raggedright\large\textbf{Infusing Software Assurance and Secure Coding into Introductory CS courses}} & 
    \parbox[t]{1in}{\sffamily\raggedleft\large\textbf{205}} \\
% row 2    
    Chair: & 
    Elizabeth Hawthorne Hawthorne \textit{Union County College}  \\[0.5em]
% row 3
    Participants: & 
    \multicolumn{2}{@{}l}{\parbox{3.75in}{Carol Sledge Carnegie Mellon University; Mark Ardis Stevens Institute of Technology; Nancy Mead Carnegie Mellon University }} \\[2em]
% row 4
    \multicolumn{3}{@{}p{5in}}{\small Nearly every facet of modern society depends heavily on highly complex software systems. The business, energy, transportation, education, communication, government, and defense communities rely on software to function, and software is an intrinsic part of our personal lives. Software assurance is an important discipline to ensure that software systems and services function dependably and are secure. So, where are the resources to assist computer science educators with this instructional material? Session leaders will share materials from the Software Assurance Curriculum Project at the Software Engineering Institute of Carnegie Mellon University, and will facilitate discussion centered on infusing software assurance into introductory computer science courses at different types of colleges.}
\end{longtable}
\begin{longtable}[l]{@{}l@{}l@{}r}
    \parbox[t]{1in}{\sffamily\large\textbf{BOF}} & 
    \parbox[t]{3in}{\sffamily\raggedright\large\textbf{Web-CAT User Group}} & 
    \parbox[t]{1in}{\sffamily\raggedleft\large\textbf{206}} \\
% row 2    
    Chair: & 
    Stephen Edwards \textit{Virginia Tech, Department of Computer Science}  \\[0.5em]
% row 3
    Participants: & 
    \multicolumn{2}{@{}l}{\parbox{3.75in}{ }} \\[2em]
% row 4
    \multicolumn{3}{@{}p{5in}}{\small Web-CAT is the most widely used open-source automated grading system, with about 10,000 users at over 65 institutions worldwide.  Its plug-in architecture supports extensibility, with plug-ins for Java (including Objectdraw, JTF, Swing, and Android), C++, Python, Haskell, and more.  It is also a powerful tool for educational research data collection. It supports a wide variety of assessment strategies, but is famous for “grading students on how well they test their own code.”  Web-CAT won the 2006 Premier Award, recognizing high-quality, non-commercial courseware for engineering education. This BOF will allow existing users and new adopters to meet, share experiences, and talk about what works and what doesn’t.  Information on getting started quickly with Web-CAT will also be provided.}
\end{longtable}
\begin{longtable}[l]{@{}l@{}l@{}r}
    \parbox[t]{1in}{\sffamily\large\textbf{BOF}} & 
    \parbox[t]{3in}{\sffamily\raggedright\large\textbf{Teaching Open Source}} & 
    \parbox[t]{1in}{\sffamily\raggedleft\large\textbf{301AB}} \\
% row 2    
    Chair: & 
    Sebastian Dziallas \textit{Franklin W. Olin College of Engineering}  \\[0.5em]
% row 3
    Participants: & 
    \multicolumn{2}{@{}l}{\parbox{3.75in}{Heidi Ellis Western New England University; Mel Chua Purdue University; Steven Huss-Lederman Beloit College; Karl Wurst Worcester State University }} \\[2em]
% row 4
    \multicolumn{3}{@{}p{5in}}{\small Involving students from a wide range of backgrounds in Free and Open Source Software project communities gets them a hands-on, portfolio-building experience in the creation of a real-world project while simultaneously building their institution's public profile. The Teaching Open Source (http://teachingopensource.org) community is an emergent (3 year old) group working on scaffolding to bridge the cultural differences between academic and FOSS communities of practice. Join us to share questions, challenges, and triumphs of incorporating FOSS participation into existing and new curricula as well support resources for doing so. Alumni and current members of the POSSE (Professors' Open Source Summer Experience http://communityleadershipteam.org/posse) will attend in mentorship roles.}
\end{longtable}
\begin{longtable}[l]{@{}l@{}l@{}r}
    \parbox[t]{1in}{\sffamily\large\textbf{BOF}} & 
    \parbox[t]{3in}{\sffamily\raggedright\large\textbf{AP CS Principles and the 'Beauty and Joy of Computing' Curriculum}} & 
    \parbox[t]{1in}{\sffamily\raggedleft\large\textbf{302A}} \\
% row 2    
    Chair: & 
    Brian Harvey \textit{University of California, Berkeley}  \\[0.5em]
% row 3
    Participants: & 
    \multicolumn{2}{@{}l}{\parbox{3.75in}{Tiffany Barnes University of North Carolina, Charlotte; Luke Segars University of California, Berkeley }} \\[2em]
% row 4
    \multicolumn{3}{@{}p{5in}}{\small The College Board's guidelines for the coming AP CS Principles course are broad enough to allow many different interpretations.  In particular, different courses have different levels of technical depth.  The "Beauty and Joy of Computing" curriculum, used by two of the initial five pilot sites, aims high, with recursion and higher order functions included in the programming half of the course.  This session is for high school or college level instructors considering teaching an AP CS Principles course and interested in using the BJC curriculum, and/or the Snap! (formerly BYOB) visual programming language used in the curriculum.  See http://bjc.berkeley.edu for the curriculum and http://snap.berkeley.edu for the language.}
\end{longtable}
\begin{longtable}[l]{@{}l@{}l@{}r}
    \parbox[t]{1in}{\sffamily\large\textbf{BOF}} & 
    \parbox[t]{3in}{\sffamily\raggedright\large\textbf{Teaching Track Faculty in CS}} & 
    \parbox[t]{1in}{\sffamily\raggedleft\large\textbf{302B}} \\
% row 2    
    Chair: & 
    Daniel D. Garcia \textit{UC Berkeley}  \\[0.5em]
% row 3
    Participants: & 
    \multicolumn{2}{@{}l}{\parbox{3.75in}{Jody Paul Metropolitan State College at Denver; Mark S. Sherriff University of Virginia }} \\[2em]
% row 4
    \multicolumn{3}{@{}p{5in}}{\small Many computer science departments have chosen to hire faculty to teach in a teaching-track position that parallels the standard tenure-track position, providing the possibility of promotion, longer-term contracts, and higher pay for excellence in teaching and service. This birds-of-a-feather is designed to gather educators who are currently in such a position to share their experiences as members of the faculty of their departments and schools, and to provide opportunities for schools considering such positions to gather information.}
\end{longtable}
\begin{longtable}[l]{@{}l@{}l@{}r}
    \parbox[t]{1in}{\sffamily\large\textbf{BOF}} & 
    \parbox[t]{3in}{\sffamily\raggedright\large\textbf{A Town Meeting:  SIGCSE Committee on Expanding the Women-in-Computing Community}} & 
    \parbox[t]{1in}{\sffamily\raggedleft\large\textbf{302C}} \\
% row 2    
    Chair: & 
    Gloria Townsend \textit{DePauw Universtiy}  \\[0.5em]
% row 3
    Participants: & 
    \multicolumn{2}{@{}l}{\parbox{3.75in}{ }} \\[2em]
% row 4
    \multicolumn{3}{@{}p{5in}}{\small In January 2004, we organized 
the second SIGCSE Committee 
("Expanding the Women-in-
Computing Community").  Our 
annual Town Meeting provides 
dissemination of information 
concerning successful gender 
issues projects, along with 
group discussion and 
brainstorming, in order to 
create committee goals for the 
coming year. We select projects 
to highlight through listserv 
communication and through our 
connections with NCWIT, ABI, 
ACM-W, CRA-W, etc.  This year 
we will highlight the new NSF 
Broadening Participation in 
Computing grant – a grant that 
encompasses projects we 
presented in previous BOFs and 
a grant that builds on an 
alliance among ACM-W, ABI and 
NCWIT.}
\end{longtable}
\begin{longtable}[l]{@{}l@{}l@{}r}
    \parbox[t]{1in}{\sffamily\large\textbf{BOF}} & 
    \parbox[t]{3in}{\sffamily\raggedright\large\textbf{Sharing Incremental Approaches for Adding Parallelism to CS Curricula}} & 
    \parbox[t]{1in}{\sffamily\raggedleft\large\textbf{305A}} \\
% row 2    
    Chair: & 
    Richard Brown \textit{St. Olaf College}  \\[0.5em]
% row 3
    Participants: & 
    \multicolumn{2}{@{}l}{\parbox{3.75in}{Elizabeth Shoop Macalester College; Joel Adams Calvin College; David Bunde Knox College; Jens Mache Lewis \& Clark College; Paul Steinberg Intel Corporation; Matthew Wolf Georgia Tech; Michael Wrinn Intel Corporation }} \\[2em]
% row 4
    \multicolumn{3}{@{}p{5in}}{\small Recent industry changes, including multi-core processors, cloud computing, and GPU programming, increase the need to teach parallelism to CS undergraduates. But few CS programs can afford to add new courses or greatly alter syllabi, and the large parallelism body of knowledge relates to many courses. Participants in this BOF will share incremental approaches for adding parallelism to undergraduate CS curricula, where students study parallel computing in brief units. This networking event/ brainstorming session/ swap meet will bring together:
 * people with sharable parallelism expository readings, hands-on exercises, tech support ideas, etc.;
 * people wishing to include such materials in their courses; and
 * people curious about incremental approaches to teaching parallel computing.}
\end{longtable}
\begin{longtable}[l]{@{}l@{}l@{}r}
    \parbox[t]{1in}{\sffamily\large\textbf{BOF}} & 
    \parbox[t]{3in}{\sffamily\raggedright\large\textbf{Computer Science: Small Department Initiative}} & 
    \parbox[t]{1in}{\sffamily\raggedleft\large\textbf{305B}} \\
% row 2    
    Chair: & 
    James Jerkofsky \textit{Walsh University}  \\[0.5em]
% row 3
    Participants: & 
    \multicolumn{2}{@{}l}{\parbox{3.75in}{Cathy Bareiss Olivet Nazarene University }} \\[2em]
% row 4
    \multicolumn{3}{@{}p{5in}}{\small Faculty in small departments (perhaps 3 FTE, perhaps only 1 or 2,…) face special situations – both challenges and strengths.  In this BOF, members will have a chance to talk about both.  Challenges include maintaining a well-rounded curriculum and attracting students.   Strengths include a close relationship with other members of the department and majors.  These and other topics are open for discussion; the specific topics will be based upon the composition and interests of the group assembled.}
\end{longtable}
\begin{longtable}[l]{@{}l@{}l@{}r}
    \parbox[t]{1in}{\sffamily\large\textbf{BOF}} & 
    \parbox[t]{3in}{\sffamily\raggedright\large\textbf{Teaching with Alice}} & 
    \parbox[t]{1in}{\sffamily\raggedleft\large\textbf{306A}} \\
% row 2    
    Chair: & 
    Donald Slater \textit{Carnegie Mellon University}  \\[0.5em]
% row 3
    Participants: & 
    \multicolumn{2}{@{}l}{\parbox{3.75in}{Wanda Dann Carnegie Mellon University; Steve Cooper Stanford University }} \\[2em]
% row 4
    \multicolumn{3}{@{}p{5in}}{\small This session is for anyone currently using Alice 2.2 and / or thinking about using Alice 3, or exploring the possibility of using Alice in his or her curriculum. The discussion leaders and experienced Alice instructors will share teaching strategies, tips, and tricks with each other and those new to Alice. The session provides an arena for sharing Alice instructional materials and ideas for courses at all educational levels. This is an opportunity to share assignments and pointers to web sites where collections of instructional materials, such as syllabi, student projects, exams, and other resources are available.}
\end{longtable}
\begin{longtable}[l]{@{}l@{}l@{}r}
    \parbox[t]{1in}{\sffamily\large\textbf{BOF}} & 
    \parbox[t]{3in}{\sffamily\raggedright\large\textbf{Identifying Effective Pedagogical Practices for Commenting Computer Source Code}} & 
    \parbox[t]{1in}{\sffamily\raggedleft\large\textbf{306B}} \\
% row 2    
    Chair: & 
    Peter DePasquale \textit{The College of New Jersey}  \\[0.5em]
% row 3
    Participants: & 
    \multicolumn{2}{@{}l}{\parbox{3.75in}{Michael Locasto University of Calgary; Lisa Kaczmarczyk Independent Consultant }} \\[2em]
% row 4
    \multicolumn{3}{@{}p{5in}}{\small Few, if any, pedagogical practices exist for helping students embrace best practices in writing software documentation, particularly source code comments. Although instructors often stress the importance of good commenting, two problems exist. First, it can be difficult to actually define these best practices, and second, it can be difficult to grade or assess students’ application of such methods/practices. This BoF focuses on capturing for dissemination a concrete list of code commenting best practices used by the BoF attendees as they teach their classes.}
\end{longtable}
\begin{longtable}[l]{@{}l@{}l@{}r}
    \parbox[t]{1in}{\sffamily\large\textbf{BOF}} & 
    \parbox[t]{3in}{\sffamily\raggedright\large\textbf{Design of a Computer Security Teaching and Research Laboratory}} & 
    \parbox[t]{1in}{\sffamily\raggedleft\large\textbf{306C}} \\
% row 2    
    Chair: & 
    jeffrey duffany \textit{universidad del turabo}  \\[0.5em]
% row 3
    Participants: & 
    \multicolumn{2}{@{}l}{\parbox{3.75in}{alfredo cruz politechnic university of puerto rico }} \\[2em]
% row 4
    \multicolumn{3}{@{}p{5in}}{\small To engage students and enhance the learning process a certain amout of hands-on experience is desirable to supplement the theory portion of computer security-related courses.  This includes courses in information assurance, database security, network security, computer
security, computer forensics among others. This BOF will include the opinion of professors that are actually delivering these courses to
graduate and undergraduate students. They will tell us what kind of hardware and software is needed to develop a computer security lab
or to enhance a classroom environment, with an emphasis on free and open source software, operating systems and the use of virtual
machines to perform virus research.}
\end{longtable}
\begin{longtable}[l]{@{}l@{}l@{}r}
    \parbox[t]{1in}{\sffamily\large\textbf{BOF}} & 
    \parbox[t]{3in}{\sffamily\raggedright\large\textbf{Student ICTD Research and Service Learning Abroad}} & 
    \parbox[t]{1in}{\sffamily\raggedleft\large\textbf{307}} \\
% row 2    
    Chair: & 
    Joseph Mertz \textit{Carnegie Mellon University}  \\[0.5em]
% row 3
    Participants: & 
    \multicolumn{2}{@{}l}{\parbox{3.75in}{Ralph Morelli Trinity College; Ruth Anderson University of Washington }} \\[2em]
% row 4
    \multicolumn{3}{@{}p{5in}}{\small This BOF is a chance for information sharing among faculty interested in involving students in ICTD research and/or service learning toward cultural and economic development globally.

It takes a lot to get students out into the field.  Challenges include developing partnerships, negotiating agreements, vetting the safety of destinations, identifying sources of funding, navigating the logistics of immunizations, visas, accommodations and flights to less-traveled places, reassuring parents as to the wisdom of their child's participation, managing development partner expectations, advising students' activities, and many more. This BOF will provide a venue for sharing experiences, information, and identifying potential new collaborations.}
\end{longtable}
\begin{longtable}[l]{@{}l@{}l@{}r}
    \parbox[t]{1in}{\sffamily\large\textbf{BOF}} & 
    \parbox[t]{3in}{\sffamily\raggedright\large\textbf{Imaging College Educators}} & 
    \parbox[t]{1in}{\sffamily\raggedleft\large\textbf{Marriott University A}} \\
% row 2    
    Chair: & 
    Jerod Weinman \textit{Grinnell College}  \\[0.5em]
% row 3
    Participants: & 
    \multicolumn{2}{@{}l}{\parbox{3.75in}{Ellen Walker Hiram College }} \\[2em]
% row 4
    \multicolumn{3}{@{}p{5in}}{\small Within computing, the imaging field includes computer vision, image understanding, and image processing. While much research and teaching is done at the graduate level, the typical imaging educator at an undergraduate institution is the only specialist in his or her department. This BOF brings together educators who currently teach imaging courses or may be interested in expanding curricular offerings. We will emphasize sharing best practices, ideas, and resources as well as building a network for continued cooperation. Discussion topics may include course organization, assignments and projects, and lecture aids or other materials. Our network will include a mailing list for participants to ask questions and share ideas about imaging pedagogy and other means of sharing course materials.}
\end{longtable}
\begin{longtable}[l]{@{}l@{}l@{}r}
    \parbox[t]{1in}{\sffamily\large\textbf{BOF}} & 
    \parbox[t]{3in}{\sffamily\raggedright\large\textbf{Let's Talk Social Media}} & 
    \parbox[t]{1in}{\sffamily\raggedleft\large\textbf{Marriott University B}} \\
% row 2    
    Chair: & 
    Kimberly Voll \textit{University of British Columbia}  \\[0.5em]
% row 3
    Participants: & 
    \multicolumn{2}{@{}l}{\parbox{3.75in}{ }} \\[2em]
% row 4
    \multicolumn{3}{@{}p{5in}}{\small Our students have Facebook, G+, and even Twitter accounts as a matter of course, and are used to rich, highly integrated environments. In contrast, CS education is via themed modalities: lectures, textbooks, labs, discussions, et cetera, that share no active or social connection (you cannot +1 a lecture, for example, share a passage of a text with a classmate, or pull up a view that truly integrates a course and its community). But we now have the technology to create learning environments that share the same rich, multimedia experience as the popular social media sites. What should this look like? How do we start? What have you tried? We’ll open with a brief overview of the leading social media tools for those unfamiliar, then proceed straight to an open discussion.}
\end{longtable}
\begin{longtable}[l]{@{}l@{}l@{}r}
    \parbox[t]{1in}{\sffamily\large\textbf{BOF}} & 
    \parbox[t]{3in}{\sffamily\raggedright\large\textbf{Program by Design: TeachScheme/ReachJava}} & 
    \parbox[t]{1in}{\sffamily\raggedleft\large\textbf{Marriott University C}} \\
% row 2    
    Chair: & 
    Viera Proulx \textit{Northeastern University}  \\[0.5em]
% row 3
    Participants: & 
    \multicolumn{2}{@{}l}{\parbox{3.75in}{Stephen Bloch Adelphi University }} \\[2em]
% row 4
    \multicolumn{3}{@{}p{5in}}{\small Program by Design is a new name for the comprehensive introduction to programming at all levels that began with TeachScheme/ReachJava. This unconventional introductory computing curriculum covers both functional and the object- oriented program design in a systematic design-based style, enforcing test-first design from the beginning. The Bootstrap curriculum makes programming and algebra exciting for children ages 11-15. Special libraries support the design of interactive graphics-based games, musical explorations, client-server and mobile computing.
We invite you to come and meet those who have used the curriculum, learn about new additions, libraries, bring in your experiences with the curriculum, show your projects, or ask questions about how it works and how you can use it.}
\end{longtable}
\begin{longtable}[l]{@{}l@{}l@{}r}
    \parbox[t]{1in}{\sffamily\large\textbf{BOF}} & 
    \parbox[t]{3in}{\sffamily\raggedright\large\textbf{CSTA Chapters: Supporting your local computer science educators}} & 
    \parbox[t]{1in}{\sffamily\raggedleft\large\textbf{Marriott Chancellor}} \\
% row 2    
    Chair: & 
    Frances P. Trees \textit{Rutgers, The State University of New Jersey}  \\[0.5em]
% row 3
    Participants: & 
    \multicolumn{2}{@{}l}{\parbox{3.75in}{Helen Hu Westminster College; Chinma Uche Greater Hartford Academy of Mathematics and Science }} \\[2em]
% row 4
    \multicolumn{3}{@{}p{5in}}{\small As part of its commitment to developing a strong community of computer science educators, the Computer Science Teachers Association (CSTA) supports the development of regional CSTA chapters. A CSTA chapter is a local branch of CSTA designed to facilitate discussion of local issues, provision of member services at the local level, and to promote CSTA membership on the national level.  This BOF will provide a platform for the discussion of CSTA chapter formation and for the sharing of successful chapter activities.}
\end{longtable}
\begin{longtable}[l]{@{}l@{}l@{}r}
    \parbox[t]{1in}{\sffamily\large\textbf{BOF}} & 
    \parbox[t]{3in}{\sffamily\raggedright\large\textbf{Revitalizing Computing Camp and Outreach:  How Do We Engage Teenagers in “Cool” Technology?}} & 
    \parbox[t]{1in}{\sffamily\raggedleft\large\textbf{Marriott Alumni}} \\
% row 2    
    Chair: & 
    Kristine Nagel \textit{Georgia Gwinnett College}  \\[0.5em]
% row 3
    Participants: & 
    \multicolumn{2}{@{}l}{\parbox{3.75in}{Evelyn Brannock Georgia Gwinnett College; Robert Lutz Georgia Gwinnett College }} \\[2em]
% row 4
    \multicolumn{3}{@{}p{5in}}{\small Tech Camps are popular outreach tools to interest teens in computing programs and technology careers. One of the biggest obstacles is how to make Tech Camp “cool” and inviting for teenagers. How do we grab the attention of students to enroll? Once at camp, how do we engage teens with computing as a creative tool with relevancy to their lives?  It is summer; subject areas must be entertaining and relevant. Can we stay ahead of the tech-savvy teens with our budget constraints? Robots and storytelling have long been used; how do we innovate and spark interest, throughout the year? The purpose of this BOF is to share ideas, such as App Inventor for Android to create apps, including text messaging, encouraging students to incorporate their own creative graphics, and using tablet devices.}
\end{longtable}
\vspace{0.5em}
\noindent\rule{5in}{0.02cm}
\vspace{0.5em}
\cfoot{\colorbox[gray]{0.45}{\color{white}\textsf{Thursday 18:10 - 19:00}}}
\noindent
\framebox[5in][c]{{\Large\sffamily\textbf{Thursday,  18:10 to 19:00}}}
\begin{longtable}[l]{@{}l@{}l@{}r}
    \parbox[t]{1in}{\sffamily\large\textbf{BOF}} & 
    \parbox[t]{3in}{\sffamily\raggedright\large\textbf{Active eTextbooks for CS: What Should They Be?}} & 
    \parbox[t]{1in}{\sffamily\raggedleft\large\textbf{201}} \\
% row 2    
    Chair: & 
    Cliff Shaffer \textit{Virginia Tech}  \\[0.5em]
% row 3
    Participants: & 
    \multicolumn{2}{@{}l}{\parbox{3.75in}{ }} \\[2em]
% row 4
    \multicolumn{3}{@{}p{5in}}{\small What should the textbook of tomorrow look like in a world of ubiquitous access to computing? Hypertextbooks have proved difficult to create and been fundamentally passive experiences. Commercial eBooks are merely books printed on an electronic screen instead of paper. New technologies such as HTML5 make it feasible to develop interactive applications that integrate with web services to provide a rich, pedagogically effective learning environment compatible with a range of computing platforms. We seek to generate discussion by participants to describe what they hope to see in online textbooks in the near future, and what resources and support would be required for them to adopt such a thing into their own courses.}
\end{longtable}
\begin{longtable}[l]{@{}l@{}l@{}r}
    \parbox[t]{1in}{\sffamily\large\textbf{BOF}} & 
    \parbox[t]{3in}{\sffamily\raggedright\large\textbf{Enriching Computing Instruction with Studio-Based Learning}} & 
    \parbox[t]{1in}{\sffamily\raggedleft\large\textbf{205}} \\
% row 2    
    Chair: & 
    N. Hari Narayanan \textit{Auburn University}  \\[0.5em]
% row 3
    Participants: & 
    \multicolumn{2}{@{}l}{\parbox{3.75in}{Martha Crosby University of Hawaii at Manoa; Dean Hendrix Auburn University; Christopher Hundhausen Washington State University }} \\[2em]
% row 4
    \multicolumn{3}{@{}p{5in}}{\small This BOF is related to the Special Session Transforming the CS Classroom with Studio-Based Learning (SBL). SBL promotes learning in a collaborative context by having students construct, present, review and refine their work with the guidance of peers and teachers. A team of CS educators and education experts have been implementing and evaluating SBL in CS courses over the past five years. The BOF will introduce SBL to the SIGCSE audience, and engage them in a discussion of the potential of, evidence for, and practical advice regarding SBL as an instructional approach that can motivate as well as teach students. Discussions will include "war stories" from teachers who have adopted the approach in their courses and hands-on activities to help participants apply SBL to their courses.}
\end{longtable}
\begin{longtable}[l]{@{}l@{}l@{}r}
    \parbox[t]{1in}{\sffamily\large\textbf{BOF}} & 
    \parbox[t]{3in}{\sffamily\raggedright\large\textbf{AP CS A - Sharing teaching strategies and curricular ideas}} & 
    \parbox[t]{1in}{\sffamily\raggedleft\large\textbf{206}} \\
% row 2    
    Chair: & 
    Lester Wainwright \textit{Charlottesville High School}  \\[0.5em]
% row 3
    Participants: & 
    \multicolumn{2}{@{}l}{\parbox{3.75in}{Renee Ciezki Estrella Mountain Community College; Robert Glen Martin TAG Magnet High School }} \\[2em]
% row 4
    \multicolumn{3}{@{}p{5in}}{\small This BOF will provide an opportunity for high school and college faculty to discuss the AP CS A curriculum and to explore possibilities for collaborations and outreach activities between high schools and colleges.}
\end{longtable}
\begin{longtable}[l]{@{}l@{}l@{}r}
    \parbox[t]{1in}{\sffamily\large\textbf{BOF}} & 
    \parbox[t]{3in}{\sffamily\raggedright\large\textbf{Regional Celebrations of Women in Computing (WiC) - Best Practices}} & 
    \parbox[t]{1in}{\sffamily\raggedleft\large\textbf{301AB}} \\
% row 2    
    Chair: & 
    Jodi Tims \textit{Baldwin-Wallace College}  \\[0.5em]
% row 3
    Participants: & 
    \multicolumn{2}{@{}l}{\parbox{3.75in}{Ellen Walker Hiram College; Rachelle Kristof Hippler Bowling Green State University Firelands College }} \\[2em]
% row 4
    \multicolumn{3}{@{}p{5in}}{\small Regional celebrations are locally organized, professional conferences modeled after the Grace Hopper Celebration of Women in Computing (GHC). This BOF allows people who have organized or would like to organize such a conference to get together to share successes and challenges.  Attendees that have hosted a regional celebration are invited to bring a un-poster (i.e. 8.5 x 11 flyer, 30 copies) that highlights their conference features and/or shares lessons learned. The leaders plan to divide the time between the 5 major areas of conference planning:  program, sponsorship, publicity/communications, registration, and site/logistics.}
\end{longtable}
\begin{longtable}[l]{@{}l@{}l@{}r}
    \parbox[t]{1in}{\sffamily\large\textbf{BOF}} & 
    \parbox[t]{3in}{\sffamily\raggedright\large\textbf{Hacking and the Security Curriculum}} & 
    \parbox[t]{1in}{\sffamily\raggedleft\large\textbf{302A}} \\
% row 2    
    Chair: & 
    Richard Weiss \textit{The Evergreen State College}  \\[0.5em]
% row 3
    Participants: & 
    \multicolumn{2}{@{}l}{\parbox{3.75in}{Michael Locasto University of Calgary; Jens Mache Lewis \& Clark College }} \\[2em]
% row 4
    \multicolumn{3}{@{}p{5in}}{\small Incorporating information security into the undergraduate curriculum continues to be a topic of interest to SIGCSE attendees. The purpose of this BOF is to help sustain the existing community of educators and researchers interested in bringing ethical hacking skills and an understanding of security into the classroom and relating these topics to the foundations of Computer Science. We would like to bring our colleagues together to share pedagogical practices, stories of hacking and how to use them to inspire our students and communicate complex concepts in computer science and security. We also plan to discuss our own experiences, practices and ongoing efforts (e.g., our infosec teaching experiences, the SISMAT program, EDURange and the dissemination of infosec interactive exercises).}
\end{longtable}
\begin{longtable}[l]{@{}l@{}l@{}r}
    \parbox[t]{1in}{\sffamily\large\textbf{BOF}} & 
    \parbox[t]{3in}{\sffamily\raggedright\large\textbf{Flipping the Classroom}} & 
    \parbox[t]{1in}{\sffamily\raggedleft\large\textbf{302B}} \\
% row 2    
    Chair: & 
    Barry Brown \textit{Sierra College}  \\[0.5em]
% row 3
    Participants: & 
    \multicolumn{2}{@{}l}{\parbox{3.75in}{ }} \\[2em]
% row 4
    \multicolumn{3}{@{}p{5in}}{\small In a flipped classroom, students watch or listen to the lecture at home and do homework in the classroom. The classroom becomes much more interactive and the educator has ample opportunity to provide individualized guidance when it's most needed. The watch-at-home content can include recorded lectures, demonstration videos, adaptive quizzes, or anything in between. Come share your experiences developing "flip" material, learn from others what's involved, and find out whether it's working to improve success and retention.}
\end{longtable}
\begin{longtable}[l]{@{}l@{}l@{}r}
    \parbox[t]{1in}{\sffamily\large\textbf{BOF}} & 
    \parbox[t]{3in}{\sffamily\raggedright\large\textbf{Using Social Networks to Engage Computer Science Students}} & 
    \parbox[t]{1in}{\sffamily\raggedleft\large\textbf{302C}} \\
% row 2    
    Chair: & 
    Semmy Purewal \textit{University of North Carolina at Asheville}  \\[0.5em]
% row 3
    Participants: & 
    \multicolumn{2}{@{}l}{\parbox{3.75in}{Owen Astrachan Duke University; David Brown Pellissippi State Community College; Jeffrey Forbes Duke University }} \\[2em]
% row 4
    \multicolumn{3}{@{}p{5in}}{\small Social Networking continues to be a popular past-time among high school and college students. In this birds of a feather session, we will share ideas on integrating social networking topics into computer science courses at the introductory and non-major levels. Additionally we will discuss approaches to integrating social network programming into upper level courses. Finally we will attempt to address the following questions: will social networking draw new students into the computing disciplines the way that video games did in the previous generation? Will it attract new types of students with different expectations? Is social networking just a fad that will have no effect on Computer Science programs? Or is social networking a topic that is better left to other academic disciplines?}
\end{longtable}
\begin{longtable}[l]{@{}l@{}l@{}r}
    \parbox[t]{1in}{\sffamily\large\textbf{BOF}} & 
    \parbox[t]{3in}{\sffamily\raggedright\large\textbf{Digital Humanities: Reaching Out to the Other Culture}} & 
    \parbox[t]{1in}{\sffamily\raggedleft\large\textbf{305A}} \\
% row 2    
    Chair: & 
    Robert Beck \textit{Villanova University}  \\[0.5em]
% row 3
    Participants: & 
    \multicolumn{2}{@{}l}{\parbox{3.75in}{ }} \\[2em]
% row 4
    \multicolumn{3}{@{}p{5in}}{\small This discussion will connect instructors who are reaching out to their colleagues in the humanities to discover areas of collaboration. It focuses on what these disciplines have to contribute to our knowledge of computing and how computational thinking informs these disciplines. One goal is to lay the foundation for a more general program of study in digital humanities that would reach students who would like to see how computing could enhance their work in history, literature, anthropology, or philosophy, for example.}
\end{longtable}
\begin{longtable}[l]{@{}l@{}l@{}r}
    \parbox[t]{1in}{\sffamily\large\textbf{BOF}} & 
    \parbox[t]{3in}{\sffamily\raggedright\large\textbf{A Multimedia and Liberal Arts Approach to a First Course in Programming and its Crossover Potential for Computer Science and the Arts}} & 
    \parbox[t]{1in}{\sffamily\raggedleft\large\textbf{305B}} \\
% row 2    
    Chair: & 
    Trish Cornez \textit{University of Redlands}  \\[0.5em]
% row 3
    Participants: & 
    \multicolumn{2}{@{}l}{\parbox{3.75in}{Richard Cornez University of Redlands }} \\[2em]
% row 4
    \multicolumn{3}{@{}p{5in}}{\small Students are acculturated in a visual, interactive, and interdisciplinary world. 
This BOF will provide a platform for a discussion on how multimedia can be integrated in a CS1 course.Discussions will focus on attributes of conventional and unconventional first languages and explore a liberal arts approach to integrate disciplines both scientific and artistic. We envision discussions relevant to:
•Mathematicians visualizing processes using multimedia and algorithms.
•Physicists using game programming to deconstruct and explore physical environments and re-assembling them as virtual worlds.
•Computer scientists and behavioral scientists collaborating on responsive systems to explore philosophical underpinnings of media.
•Musicians and computer scientists creating computational art.}
\end{longtable}
\begin{longtable}[l]{@{}l@{}l@{}r}
    \parbox[t]{1in}{\sffamily\large\textbf{BOF}} & 
    \parbox[t]{3in}{\sffamily\raggedright\large\textbf{Teaching with App Inventor for Android}} & 
    \parbox[t]{1in}{\sffamily\raggedleft\large\textbf{306A}} \\
% row 2    
    Chair: & 
    Jeff Gray \textit{University of Alabama}  \\[0.5em]
% row 3
    Participants: & 
    \multicolumn{2}{@{}l}{\parbox{3.75in}{Harold Abelson MIT; Ralph Morelli Trinity College; Jeff Gray University of Alabama; Chinma Uche Greater Hartford Academy of Math and Science }} \\[2em]
% row 4
    \multicolumn{3}{@{}p{5in}}{\small App Inventor for Android is a visual blocks language for building mobile apps. Like Scratch, the language’s drag-and-drop blocks interface significantly lowers the barrier to entry. Beginners can immediately build apps that interface with mobile technology (e.g., GPS, Text-to-speech, SMS Texting) and build apps that have a real-world impact. In this BoF, hosted by App Inventor creator Hal Abelson and experienced teachers and authors, we’ll discuss the language, its future in K-12 and university education, and its new home at the MIT Center for Mobile Learning.}
\end{longtable}
\begin{longtable}[l]{@{}l@{}l@{}r}
    \parbox[t]{1in}{\sffamily\large\textbf{BOF}} & 
    \parbox[t]{3in}{\sffamily\raggedright\large\textbf{Technology that Educators of Computing Hail (TECH): Come, share your favorites!}} & 
    \parbox[t]{1in}{\sffamily\raggedleft\large\textbf{306B}} \\
% row 2    
    Chair: & 
    Daniel D. Garcia \textit{UC Berkeley}  \\[0.5em]
% row 3
    Participants: & 
    \multicolumn{2}{@{}l}{\parbox{3.75in}{Luke Segars UC Berkeley }} \\[2em]
% row 4
    \multicolumn{3}{@{}p{5in}}{\small The pace of technology for use in computing education is staggering.  In the last five years, the following tools / websites have completely transformed our teaching: Piazza, Google Docs, YouTube, Doodle and whenisgood.net, Skype and Google Hangout, and Khan Academy among others.  Hardware has also played a part – we love our Zoom H2 digital voice recorder (for recording CD-quality lecture audio), Blue Yeti USB mike (for audio/videoconferences), and iClickers (for engaging students in class).

Do you wish you could easily share your favorites?  Want to find out what the others know that you don’t?  Have a tool you’ve built and want to get some users?  Come to this BOF!  We’ll also show the TECH website we’ve built that attempts to collect all of these tools in one place.}
\end{longtable}
\begin{longtable}[l]{@{}l@{}l@{}r}
    \parbox[t]{1in}{\sffamily\large\textbf{BOF}} & 
    \parbox[t]{3in}{\sffamily\raggedright\large\textbf{Motivating CS1/2 Students with the Android Platform}} & 
    \parbox[t]{1in}{\sffamily\raggedleft\large\textbf{306C}} \\
% row 2    
    Chair: & 
    John Lewis \textit{Virginia Tech}  \\[0.5em]
% row 3
    Participants: & 
    \multicolumn{2}{@{}l}{\parbox{3.75in}{Anthony Allevato Virginia Tech; Stephen H. Edwards Virginia Tech }} \\[2em]
% row 4
    \multicolumn{3}{@{}p{5in}}{\small The use of Android in computing courses is growing. Students find it engaging because they can develop Java apps for mobile devices. Android also offers challenges in the classroom, especially in CS1 and CS2. As a professional- level platform, it uses design idioms that may require students to learn advanced language features earlier. It also adds logistical complications to setting up projects and development tools. Existing approaches to software testing and automated grading need adaptation. This BOF is for sharing assignments, resources, techniques, and experiences with others, focusing on issues that arise when balancing the teaching of fundamental concepts with the complexities required to accomplish basic tasks on the Android platform.}
\end{longtable}
\begin{longtable}[l]{@{}l@{}l@{}r}
    \parbox[t]{1in}{\sffamily\large\textbf{BOF}} & 
    \parbox[t]{3in}{\sffamily\raggedright\large\textbf{Interdisciplinary Database Collaborations}} & 
    \parbox[t]{1in}{\sffamily\raggedleft\large\textbf{307}} \\
% row 2    
    Chair: & 
    Suzanne Dietrich \textit{Arizona State University}  \\[0.5em]
% row 3
    Participants: & 
    \multicolumn{2}{@{}l}{\parbox{3.75in}{Don Goelman Villanova University }} \\[2em]
% row 4
    \multicolumn{3}{@{}p{5in}}{\small Databases play a major role across many disciplines for the storage and retrieval of information. Many database educators are establishing collaborations with colleagues representing a diverse spectrum of interests, for both research and pedagogical purposes. Further, the range of cooperating disciplines is expanding, as evidenced by the emergence of new fields such as computational journalism, as well as by the proliferation of discipline-specific dialects of XML. The goal of this Birds-of-a-Feather session is to bring database educators together to share their experiences on interdisciplinary collaborations in an open dialogue that is fostered by this format.}
\end{longtable}
\begin{longtable}[l]{@{}l@{}l@{}r}
    \parbox[t]{1in}{\sffamily\large\textbf{BOF}} & 
    \parbox[t]{3in}{\sffamily\raggedright\large\textbf{Google Summer of Code and Google Code-in BoF}} & 
    \parbox[t]{1in}{\sffamily\raggedleft\large\textbf{Marriott University A}} \\
% row 2    
    Chair: & 
    Carol Smith \textit{Google, Inc.}  \\[0.5em]
% row 3
    Participants: & 
    \multicolumn{2}{@{}l}{\parbox{3.75in}{ }} \\[2em]
% row 4
    \multicolumn{3}{@{}p{5in}}{\small Google Summer of Code is the outreach program aimed at getting university students involved in a 3-month online internship working in open source software development. 

Google Code-in is the contest aimed at involving 13-18 year olds in open source software development, documentation translation, outreach, research, and more. I will be discussing both programs at this BoF and encouraging students and teachers to get involved. 

We'll open the forum for discussion amongst the attendees about how to participate, how to get the word out, and answer any questions they may have.}
\end{longtable}
\begin{longtable}[l]{@{}l@{}l@{}r}
    \parbox[t]{1in}{\sffamily\large\textbf{BOF}} & 
    \parbox[t]{3in}{\sffamily\raggedright\large\textbf{Building Partnerships Across the CS Education Spectrum}} & 
    \parbox[t]{1in}{\sffamily\raggedleft\large\textbf{Marriott University B}} \\
% row 2    
    Chair: & 
    Chris Stephenson \textit{Computer Science Teachers Association}  \\[0.5em]
% row 3
    Participants: & 
    \multicolumn{2}{@{}l}{\parbox{3.75in}{Steve Cooper Stanford University; Don Yanek Northside College Prep High School; Jeff Gray University of Alabama }} \\[2em]
% row 4
    \multicolumn{3}{@{}p{5in}}{\small Over the last five years, CSTA has built a solid outreach and teacher support network through the work of its chapters and Leadership Cohort. This network has also become a major source of active partnerships between K-12 teachers, their schools, and colleagues from colleges, universities, and industry. The goal of this BOF is to provide concrete examples and suggestions for SIGCSE members interested in building these kinds of partnerships.}
\end{longtable}
\begin{longtable}[l]{@{}l@{}l@{}r}
    \parbox[t]{1in}{\sffamily\large\textbf{BOF}} & 
    \parbox[t]{3in}{\sffamily\raggedright\large\textbf{Engaging The Community With Mobile App Projects}} & 
    \parbox[t]{1in}{\sffamily\raggedleft\large\textbf{Marriott University C}} \\
% row 2    
    Chair: & 
    William Turkett \textit{Wake Forest University}  \\[0.5em]
% row 3
    Participants: & 
    \multicolumn{2}{@{}l}{\parbox{3.75in}{Paul Pauca Wake Forest University; Joel Hollingsworth Elon University }} \\[2em]
% row 4
    \multicolumn{3}{@{}p{5in}}{\small As the popularity of mobile devices surges, more and more organizations are looking to exploit the novel interaction methods of mobile devices to re-deploy legacy software or to develop innovative new  applications. Many organizations are looking to nearby universities for expertise in this area. At the same time, mobile computing has become increasingly integrated within courses in CS departments. Historically, capstone courses and other advanced electives have resulted in the production of non-trivial software artifacts. This  BOF will provide a platform for discussion of how the use of mobile app platforms in such courses can allow for the development of meaningful software projects that engage with and give back to the community and provide rich opportunities for service learning.}
\end{longtable}
\begin{longtable}[l]{@{}l@{}l@{}r}
    \parbox[t]{1in}{\sffamily\large\textbf{BOF}} & 
    \parbox[t]{3in}{\sffamily\raggedright\large\textbf{Have Class, Will Travel}} & 
    \parbox[t]{1in}{\sffamily\raggedleft\large\textbf{Marriott Chancellor}} \\
% row 2    
    Chair: & 
    Paige Meeker \textit{Presbyterian College}  \\[0.5em]
% row 3
    Participants: & 
    \multicolumn{2}{@{}l}{\parbox{3.75in}{ }} \\[2em]
% row 4
    \multicolumn{3}{@{}p{5in}}{\small At many schools, various disciplines offer travel courses (to other lands or to locations within the USA) to give students an experiential component to their learning. How can we introduce such courses to computer science departments? This BOF will provide a time of sharing ideas for such courses and welcomes discussion of travel courses that have been successfully taught. In addition to normal course preparation, these courses also involve travel arrangements, payment schedules, and careful scheduling to provide maximum benefit to the student. Our group will share ideas for locations of travel, topics of courses, and collaboration with other disciplines, as well as the additional overhead such a course entails, such as cost/payment schedule, insurance, itinerary, safety, etc.}
\end{longtable}
\begin{longtable}[l]{@{}l@{}l@{}r}
    \parbox[t]{1in}{\sffamily\large\textbf{BOF}} & 
    \parbox[t]{3in}{\sffamily\raggedright\large\textbf{Integration of Experiential learning and teaching: -Beyond the walls of the classroom, techniques, challenges and merits.}} & 
    \parbox[t]{1in}{\sffamily\raggedleft\large\textbf{Marriott Alumni}} \\
% row 2    
    Chair: & 
    Arshia Khan \textit{The College of St. Scholastica}  \\[0.5em]
% row 3
    Participants: & 
    \multicolumn{2}{@{}l}{\parbox{3.75in}{Tamara Lichtenberg Center for Healthcare Innovation- The College of St. Scholastica; John Woosley Southeastern Louisiana University; Rishika Dhody The College of St. Scholastica; Joel Pouale The College of St. Scholastica }} \\[2em]
% row 4
    \multicolumn{3}{@{}p{5in}}{\small Integration of experiential learning is critical in the field of computer science. With technology evolving over night, job requirements are extremely volatile. Educators have a challenging task of staying abreast with the technology and market needs while self learning the new technologies. One solution is to rely on the businesses for input on what should be taught and using them to extend the learning into the real world through experiential learning(not just internships). Talking points:
•Filling gaps between academia and industry
•Faculty can share their methods of experiential learning. 
•Applying the practical skills to theoretical knowledge- turning theory into practice
•Opportunities to bring real world clients in the classroom}
\end{longtable}
\vspace{0.5em}
\noindent\rule{5in}{0.02cm}
\vspace{0.5em}
\cfoot{\colorbox[gray]{0.45}{\color{white}\textsf{Thursday 19:00 - 20:00}}}
\noindent
\framebox[5in][c]{{\Large\sffamily\textbf{Thursday,  19:00 to 20:00}}}
\begin{longtable}[l]{@{}p{1in}@{}p{3in}@{}r}
    {\sffamily\large\textbf{None}} & 
    {\sffamily\large\textbf{Reception}} & 
    {\sffamily\large\textbf{Registration Foyer}} \\
\end{longtable}    
\vspace{0.5em}
\noindent\rule{5in}{0.02cm}
\vspace{0.5em}
\addcontentsline{toc}{subsection}{Friday}
\cfoot{\colorbox[gray]{0.45}{\color{white}\textsf{Friday 07:15 - 08:15}}}
\noindent
\framebox[5in][c]{{\Large\sffamily\textbf{Friday,  7:15 to 8:15}}}
\begin{longtable}[l]{@{}p{1in}@{}p{3in}@{}r}
    {\sffamily\large\textbf{Ancillary Event}} & 
    {\sffamily\large\textbf{Alice Breakfast}} & 
    {\sffamily\large\textbf{302B}} \\
\end{longtable}    
\vspace{0.5em}
\noindent\rule{5in}{0.02cm}
\vspace{0.5em}
\cfoot{\colorbox[gray]{0.45}{\color{white}\textsf{Friday 08:30 - 10:00}}}
\noindent
\framebox[5in][c]{{\Large\sffamily\textbf{Friday,  8:30 to 10:00}}}
\begin{longtable}[l]{@{}p{1in}@{}p{3in}@{}r}
    {\sffamily\large\textbf{Plenary Session}} & 
    {\sffamily\large\textbf{Plenary SessionKeynote Speaker: Hal Abelson}} & 
    {\sffamily\large\textbf{Ballroom AB}} \\
\end{longtable}    
\vspace{0.5em}
\noindent\rule{5in}{0.02cm}
\vspace{0.5em}
\cfoot{\colorbox[gray]{0.45}{\color{white}\textsf{Friday 10:00 - 17:00}}}
\noindent
\framebox[5in][c]{{\Large\sffamily\textbf{Friday,  10:00 to 17:00}}}
\begin{longtable}[l]{@{}p{1in}@{}p{3in}@{}r}
    {\sffamily\large\textbf{Exhibits}} & 
    {\sffamily\large\textbf{Exhibits}} & 
    {\sffamily\large\textbf{Exhibit Hall A}} \\
\end{longtable}    
\vspace{0.5em}
\noindent\rule{5in}{0.02cm}
\vspace{0.5em}
\cfoot{\colorbox[gray]{0.45}{\color{white}\textsf{Friday 10:00 - 10:45}}}
\noindent
\framebox[5in][c]{{\Large\sffamily\textbf{Friday,  10:00 to 10:45}}}
\begin{longtable}[l]{@{}p{1in}@{}p{3in}@{}r}
    {\sffamily\large\textbf{None}} & 
    {\sffamily\large\textbf{Break and Exhibits}} & 
    {\sffamily\large\textbf{Exhibit Hall A}} \\
\end{longtable}    
\vspace{0.5em}
\noindent\rule{5in}{0.02cm}
\vspace{0.5em}
\cfoot{\colorbox[gray]{0.45}{\color{white}\textsf{Friday 10:00 - 11:30}}}
\noindent
\framebox[5in][c]{{\Large\sffamily\textbf{Friday,  10:00 to 11:30}}}
\begin{longtable}[l]{@{}p{1in}@{}p{3in}@{}r}
    {\sffamily\large\textbf{Project Showcase}} & 
    {\sffamily\large\textbf{NSF Showcase \#3}} & 
    {\sffamily\large\textbf{Exhibit Hall A}} \\
\end{longtable}    
\vspace{0.5em}
\noindent\rule{5in}{0.02cm}
\vspace{0.5em}
\cfoot{\colorbox[gray]{0.45}{\color{white}\textsf{Friday 10:00 - 12:00}}}
\noindent
\framebox[5in][c]{{\Large\sffamily\textbf{Friday,  10:00 to 12:00}}}
\newpage
\begin{longtable}{@{}p{0.75in}@{}p{3.25in}@{}r}
   {\sffamily\large\textbf{POSTER}} &
   {\raggedright\sffamily\large\textbf{Poster Session I}} & 
   {\sffamily\large\textbf{Exhibit Hall A }} \\
%row 2
   Chair:  & 
   {\raggedright Dummy TestData \textit{First Record of Database}} & \\ \\
\multicolumn{3}{@{}p{5in}}{\sffamily\raggedright\textbf{Using reflection to enhance feedback for automated grading}} \\
\multicolumn{3}{@{}p{5in}}{\raggedright Carl Alphonce, \textit{University at Buffalo}; Joseph LeGasse, \textit{Meritain Health}} \\ \\
\multicolumn{3}{@{}p{5in}}{\sffamily\raggedright\textbf{The Cross-Curriculum Mobile Computing Labware for CS}} \\
\multicolumn{3}{@{}p{5in}}{\raggedright Liang Hong, \textit{Tennessee State University}; Kai Qian and Dan Lo, \textit{Southern Polytechnic State University}; Yi Pan, Yanqing Zhang and Xiaolin Hu, \textit{Georgia State University}} \\ \\
\multicolumn{3}{@{}p{5in}}{\sffamily\raggedright\textbf{Merging Healthcare and Technology:  A Multi-disciplinary Health Information Technology (HIT) Curriculum}} \\
\multicolumn{3}{@{}p{5in}}{\raggedright Elizabeth Howard, Donna Evans and Marilyn Anderson, \textit{Miami University - Middletwon}; Jill Courte, \textit{Miami University - Hamilton}} \\ \\
\multicolumn{3}{@{}p{5in}}{\sffamily\raggedright\textbf{An Integrated Introduction to Network Protocols and Cryptography to High School Students}} \\
\multicolumn{3}{@{}p{5in}}{\raggedright William Mongan, \textit{Drexel University}} \\ \\
\multicolumn{3}{@{}p{5in}}{\sffamily\raggedright\textbf{A PC Robot for Learning Computer Vision and Advanced Programming}} \\
\multicolumn{3}{@{}p{5in}}{\raggedright Xuzhou Chen and Nadimpalli V.R. Mahadev, \textit{Fitchburg State University}} \\ \\
\multicolumn{3}{@{}p{5in}}{\sffamily\raggedright\textbf{Girls Gather for Computer Science (G2CS)}} \\
\multicolumn{3}{@{}p{5in}}{\raggedright Shereen Khoja, Juliet Brosing, Camille Wainwright and Jeffrey Barlow, \textit{Pacific University}} \\ \\
\multicolumn{3}{@{}p{5in}}{\sffamily\raggedright\textbf{Debuggems to Assess Student Learning in E-Textiles}} \\
\multicolumn{3}{@{}p{5in}}{\raggedright Deborah Fields, Kristin Searle, Yasmin Kafai and Hannah Min, \textit{University of Pennsylvania}} \\ \\
\multicolumn{3}{@{}p{5in}}{\sffamily\raggedright\textbf{Mediascripting – Teaching Introductory CS by Through Interactive Graphics Scripting}} \\
\multicolumn{3}{@{}p{5in}}{\raggedright Samuel Rebelsky, Janet Davis and Jerod Weinman, \textit{Grinnell College}} \\ \\
\multicolumn{3}{@{}p{5in}}{\sffamily\raggedright\textbf{Do Faculty Recognize the Difference Between Computer Science and Information Technology?  A Survey of Liberal Arts Faculty}} \\
\multicolumn{3}{@{}p{5in}}{\raggedright Jaime Spacco and Hannah Fidoten, \textit{Knox College}} \\ \\
\multicolumn{3}{@{}p{5in}}{\sffamily\raggedright\textbf{Interdisciplinary Travel Courses in Computer Science}} \\
\multicolumn{3}{@{}p{5in}}{\raggedright Paige Meeker, \textit{Presbyterian College}} \\ \\
\multicolumn{3}{@{}p{5in}}{\sffamily\raggedright\textbf{User type clustering to refine search and browse for educational resources}} \\
\multicolumn{3}{@{}p{5in}}{\raggedright Monika Akbar and Clifford A. Shaffer, \textit{Virginia Tech}} \\ \\
\multicolumn{3}{@{}p{5in}}{\sffamily\raggedright\textbf{A Comprehensive CS Curriculum Revision, Implementation and Analysis}} \\
\multicolumn{3}{@{}p{5in}}{\raggedright Steven Huss-Lederman, \textit{Beloit College}} \\ \\
\multicolumn{3}{@{}p{5in}}{\sffamily\raggedright\textbf{Developing an Interdisciplinary Health Informatics Security and Privacy Program}} \\
\multicolumn{3}{@{}p{5in}}{\raggedright Xiaohong Yuan, Jinsheng Xu, Hong Wang and Kossi Edoh, \textit{North Carolina A\&T State University}} \\ \\
\multicolumn{3}{@{}p{5in}}{\sffamily\raggedright\textbf{A Team Software Development Course Featuring iPad Programming}} \\
\multicolumn{3}{@{}p{5in}}{\raggedright Robert England, \textit{Transylvania University}} \\ \\
\multicolumn{3}{@{}p{5in}}{\sffamily\raggedright\textbf{The Role of Belonging in Computer Science Student Engagement}} \\
\multicolumn{3}{@{}p{5in}}{\raggedright Nanette Veilleux, \textit{Simmons College}; Rebecca Bates, \textit{Computer Science, Minnesota State University, Mankato}; Cheryl Allendoerfer, Diane Jones and Joy Crawford, \textit{University of Washington}} \\ \\
\multicolumn{3}{@{}p{5in}}{\sffamily\raggedright\textbf{Streamlining Project Setup in Eclipse for Both Time-Constrained and Large-Scale Assignments}} \\
\multicolumn{3}{@{}p{5in}}{\raggedright Ellen Boyd and Anthony Allevato, \textit{Virginia Tech}} \\ \\
\multicolumn{3}{@{}p{5in}}{\sffamily\raggedright\textbf{A Customizable Platform for Classroom Collaboration Using Mobile Devices}} \\
\multicolumn{3}{@{}p{5in}}{\raggedright Stephen Hughes, Ben Schafer, Aaron Mangel and Sean Fredericksen, \textit{University of Northern Iowa}} \\ \\
\multicolumn{3}{@{}p{5in}}{\sffamily\raggedright\textbf{Explaining the Dynamic Structure and Behavior of Java Programs using a Visual Debugger}} \\
\multicolumn{3}{@{}p{5in}}{\raggedright Demian Lessa and Bharat Jayaraman, \textit{SUNY at Buffalo}} \\ \\
\multicolumn{3}{@{}p{5in}}{\sffamily\raggedright\textbf{Using FPGA Systems Across the Computer Science Curriculum}} \\
\multicolumn{3}{@{}p{5in}}{\raggedright D. Brian Larkins, H. Erin Rickard and William M. Jones, \textit{Coastal Carolina University}} \\ \\
\multicolumn{3}{@{}p{5in}}{\sffamily\raggedright\textbf{Maximizing Content Learning for Deaf Students and English as a Second Language Students}} \\
\multicolumn{3}{@{}p{5in}}{\raggedright Raja Kushalnagar and Joseph Stanislow, \textit{Rochester Institute of Technology}} \\ \\
\multicolumn{3}{@{}p{5in}}{\sffamily\raggedright\textbf{All-In-One Virtualized Laboratory}} \\
\multicolumn{3}{@{}p{5in}}{\raggedright Shamsi Moussavi and Giuseppe Sena, \textit{MassBay Community College}} \\ \\
\multicolumn{3}{@{}p{5in}}{\sffamily\raggedright\textbf{Recursive Thinkers and Doers in CS1}} \\
\multicolumn{3}{@{}p{5in}}{\raggedright Suzanne Menzel and Joseph Cottam, \textit{Indiana University}} \\ \\
\multicolumn{3}{@{}p{5in}}{\sffamily\raggedright\textbf{Computing in Context: Video Scenarios for Recognizing and Utilizing Basic Computing Constructs}} \\
\multicolumn{3}{@{}p{5in}}{\raggedright Madalene Spezialetti, \textit{Trinity College}} \\ \\
\multicolumn{3}{@{}p{5in}}{\sffamily\raggedright\textbf{Programming board-game strategies in the introductory CS sequence}} \\
\multicolumn{3}{@{}p{5in}}{\raggedright Ivona Bezakova, James Heliotis, Sean Strout, Adam Oest and Paul Solt, \textit{Rochester Institute of Technology}} \\ \\
\end{longtable}


\vspace{0.5em}
\noindent\rule{5in}{0.02cm}
\vspace{0.5em}
\cfoot{\colorbox[gray]{0.45}{\color{white}\textsf{Friday 10:00 - 17:00}}}
\noindent
\framebox[5in][c]{{\Large\sffamily\textbf{Friday,  10:00 to 17:00}}}
\begin{longtable}[l]{@{}p{1in}@{}p{3in}@{}r}
    {\sffamily\large\textbf{Social}} & 
    {\sffamily\large\textbf{K-12 Teachers Room}} & 
    {\sffamily\large\textbf{202}} \\
\end{longtable}    
\begin{longtable}[l]{@{}p{1in}@{}p{3in}@{}r}
    {\sffamily\large\textbf{Social}} & 
    {\sffamily\large\textbf{CS Education Research Room}} & 
    {\sffamily\large\textbf{203}} \\
\end{longtable}    
\vspace{0.5em}
\noindent\rule{5in}{0.02cm}
\vspace{0.5em}
\cfoot{\colorbox[gray]{0.45}{\color{white}\textsf{Friday 10:45 - 12:00}}}
\noindent
\framebox[5in][c]{{\Large\sffamily\textbf{Friday,  10:45 to 12:00}}}
\begin{longtable}[l]{@{}l@{}l@{}r}
    \parbox[t]{1in}{\sffamily\large\textbf{PANEL}} & 
    \parbox[t]{3in}{\sffamily\raggedright\large\textbf{Teaching Mathematical Reasoning Across the Curriculum}} & 
    \parbox[t]{1in}{\sffamily\raggedleft\large\textbf{301AB}} \\
% row 2    
    Chair: & 
    Joan Krone \textit{Denison University}  \\[0.5em]
% row 3
    Participants: & 
    \multicolumn{2}{@{}l}{\parbox{3.75in}{Doug Baldwin, \textit{SUNY Geneseo}; Jeff Carver, \textit{University of Alabama}; Joseph Hollingsworth, \textit{Indiana University Southeast}; Amruth Kumar, \textit{Ramapo College of New Jersey} }} \\[2em]
% row 4
    \multicolumn{3}{@{}p{5in}}{\small We discuss ways in which the panel members have incorporated mathematical reasoning into a variety of courses, encouraging and supporting students to apply and enhance their reasoning skills in productive ways across the entire CS curriculum.}
\end{longtable}
\begin{longtable}[l]{@{}l@{}l@{}r}
    \parbox[t]{1in}{\sffamily\large\textbf{PANEL}} & 
    \parbox[t]{3in}{\sffamily\raggedright\large\textbf{Teaching HS Computer Science as if the Rest of the World Existed}} & 
    \parbox[t]{1in}{\sffamily\raggedleft\large\textbf{305B}} \\
% row 2    
    Chair: & 
    Scott Portnoff, \textit{Downtown Magnets High School, Los Angeles}  \\[0.5em]
% row 3
    Participants: & 
    \multicolumn{2}{@{}l}{\parbox{3.75in}{ }} \\[2em]
% row 4
    \multicolumn{3}{@{}p{5in}}{\small This session discusses the design, implementation and rationale for a HS pre-APCS curriculum of Interdisciplinary Central-Problem-Based (ICPB) units that model real-world applications. In a typical multi-week unit, students begin by solving a problem using a complex software application, such as SDSC Biology Workbench.  Students then build a small-scale version of the program, focusing on 1 or 2 algorithms, using Processing, Excel, BYOB or Alice. This approach affords students both context and practical potential for their work.  Unit topics come from the fields of Astronomy (Galileo), Bioinformatics (Evolution), Molecular Modeling (DNA Double Helix), Political Science (Women's Suffrage/ Electoral Process), Environmental Science, Music, and Holocaust Studies (Hollerith Machine Technology).}
\end{longtable}
\begin{longtable}[l]{@{}l@{}l@{}r}
    \parbox[t]{1in}{\sffamily\large\textbf{PANEL}} & 
    \parbox[t]{3in}{\sffamily\raggedright\large\textbf{Funding the Challenges in Computing}} & 
    \parbox[t]{1in}{\sffamily\raggedleft\large\textbf{306C}} \\
% row 2    
    Chair: & 
    Guy-Alain Amoussou, \textit{NSF}  \\[0.5em]
% row 3
    Participants: & 
    \multicolumn{2}{@{}l}{\parbox{3.75in}{Scott Grissom, \textit{Grand Valley State University} }} \\[2em]
% row 4
    \multicolumn{3}{@{}p{5in}}{\small What are the perceived challenges by the computing education and research communities? After small groups discuss this question, we will compare and contrast these perceived challenges to the current trend in proposals submitted and funded by the National Science Foundation’s (NSF) Transforming Undergraduate Education in STEM (TUES) program. The intention is to create awareness for all involved and to broaden the range of proposals submitted to NSF.}
\end{longtable}
\newpage
\begin{longtable}{@{}p{0.75in}@{}p{3.25in}@{}r}
   {\sffamily\large\textbf{PAPERS}} &
   {\raggedright\sffamily\large\textbf{CS1:  New Ideas}} & 
   {\sffamily\large\textbf{302A }} \\
%row 2
   Chair:  & 
   {\raggedright Lori Carter \textit{Point Loma Nazarene University}} & \\ \\
{\sffamily \large 10:45}& 
\multicolumn{2}{@{}p{3.75in}}{\sffamily\raggedright\textbf{Creative Coding and Visual Portfolios for CS1}} \\
& \multicolumn{2}{@{}p{3.75in}}{\raggedright Dianna Xu and Deepak Kumar, \textit{Bryn Mawr College}; Ira Greenberg, \textit{Southern Methodist University}} \\ \\
\multicolumn{3}{@{}p{5in}}{\small In this paper, we present the design and development of a new approach to teaching the college-level introductory computing course (CS1) using the context of art and creative coding. Over the course of a semester, students create a portfolio of aesthetic visual designs that employ basic computing structures typically taught in traditional CS1 courses using the Processing programming language. The goal of this approach is to bring the excitement, creativity, and innovation fostered by the context of creative coding. We also present results from a comparative study involving two offerings of the new course at two different institutions. Additionally, we compare our results with another successful approach that uses personal robots to teach CS1.} \\ \\
{\sffamily \large 11:10}& 
\multicolumn{2}{@{}p{3.75in}}{\sffamily\raggedright\textbf{Stepping Up to Integrative Questions on CS1 Exams}} \\
& \multicolumn{2}{@{}p{3.75in}}{\raggedright Daniel Zingaro and Michelle Craig, \textit{University of Toronto}; Andrew Petersen, \textit{University of Toronto Mississauga}} \\ \\
\multicolumn{3}{@{}p{5in}}{\small We explore the use of sequences of small code writing questions ("concept questions") designed to incrementally evaluate single programming concepts. We report on student performance on a CS1 final examination that included a traditional code-writing question and four corresponding concept questions. We find that the concept questions are significant predictors of performance on both the corresponding code-writing question and the final exam as a whole. We argue that concept questions provide more accurate formative feedback and simplify marking by reducing the number of variants that must be considered. An analysis of the student responses categorized by the students' previous programming experience suggests inexperienced students have the most to gain from the use of concept questions.} \\ \\
{\sffamily \large 11:35}& 
\multicolumn{2}{@{}p{3.75in}}{\sffamily\raggedright\textbf{Using Reflective Blogs for Pedagogical Feedback in CS1}} \\
& \multicolumn{2}{@{}p{3.75in}}{\raggedright Jeffrey Stone, \textit{Pennsylvania State University}} \\ \\
\multicolumn{3}{@{}p{5in}}{\small The use of weekly, reflective student blogs can be one method for collecting ongoing feedback about a CS1 course. Reflective blogs permit a continuous feedback loop that can be used for both formative and summative assessment of pedagogical innovations. This paper reports on a two-year qualitative study involving the use of reflective blogging in six sections of two CS1 courses. Reflective blogs were used as a low stakes, non-intimidating vehicle whereby concerns, requests, and other course-related issues could be voiced by students. The posts were used as an assessment and feedback mechanism for pedagogical transformation of the participating courses. This study demonstrates that reflective blogs in CS1 can be a useful tool for faculty course development.} \\ \\
\end{longtable}


\newpage
\begin{longtable}{@{}p{0.75in}@{}p{3.25in}@{}r}
   {\sffamily\large\textbf{PAPERS}} &
   {\raggedright\sffamily\large\textbf{Team Work}} & 
   {\sffamily\large\textbf{302B }} \\
%row 2
   Chair:  & 
   {\raggedright Jody Paul \textit{Metropolitan State College of Denver}} & \\ \\
{\sffamily \large 10:45}& 
\multicolumn{2}{@{}p{3.75in}}{\sffamily\raggedright\textbf{Participation patterns in student teams}} \\
& \multicolumn{2}{@{}p{3.75in}}{\raggedright Vreda Pieterse, Lisa Thompson and Linda Marshall, \textit{University of Pretoria}; Dina Venter, \textit{Olrac-SPS (formerly SPSS South Africa)}} \\ \\
\multicolumn{3}{@{}p{5in}}{\small We describe a process for teaching teamwork in a Software Engineering module. Our objective was to create opportunities for our students to experience some problems of working in a group before they formed teams in which they had to work for the rest of the year.  

The process entails expecting students to work on well defined assignments for short periods in teams where risk factors were induced. Through experiencing these short bursts of team tribulation students are prepared to handle difficult events and situations in their teams.  

We describe the design and implementation of this process.  We report on changes in the levels of participation of the students during the process.  We offer our explanation of possible factors that may have caused the observed variations.} \\ \\
{\sffamily \large 11:10}& 
\multicolumn{2}{@{}p{3.75in}}{\sffamily\raggedright\textbf{Application of Non-programming Focused Treisman-style Workshops in Introductory Computer Science}} \\
& \multicolumn{2}{@{}p{3.75in}}{\raggedright Lindsay Jamieson, Alan Jamieson and Angela Johnson, \textit{St. Mary's College of Maryland}} \\ \\
\multicolumn{3}{@{}p{5in}}{\small In the 60s and 70s, Uri Treisman developed a specific style of workshops to encourage the retention of underrepresented minority students in Calculus courses. Since that time, workshops based on the Treisman model have been successful across the US and have resulted in more underrepresented minority students successfully completing Calculus. Some attempts have been made to translate the Treisman model to CS1, but all previous attempts have been focused on programming skills.  However, one of the student assumptions that deter underrepresented minorities from attempting a major or minor in CS is that a computer scientist is a solitary programmer. In this paper, we discuss a specific two year pilot program of non-programming focused Treisman-style workshops in conjunction with a CS1 course.} \\ \\
{\sffamily \large 11:35}& 
\multicolumn{2}{@{}p{3.75in}}{\sffamily\raggedright\textbf{Collaboration Across the Curriculum: A Disciplined Approach to Developing Team Skills}} \\
& \multicolumn{2}{@{}p{3.75in}}{\raggedright Matthew Lang and Ben Coleman, \textit{Moravian College}} \\ \\
\multicolumn{3}{@{}p{5in}}{\small Increasing the communication and collaborative skills of computer science
students has been a priority in the community for some time.  We present our
philosophy, collaboration across the curriculum, which moves beyond existing
individual courses or course units to teach collaboration skills in a pervasive
manner.  In our approach, concepts are introduced and skills are developed
throughout the computer science curriculum---from CS1 to a capstone experience.
Students are provided with opportunities to exercise skills in reflective environments
that eventually mirror real-world experiences, and technical course content is
not compromised.

We argue for this system and provide details about how collaboration
across the curriculum is accomplished at a small liberal arts college.} \\ \\
\end{longtable}


\newpage
\begin{longtable}{@{}p{0.75in}@{}p{3.25in}@{}r}
   {\sffamily\large\textbf{PAPERS}} &
   {\raggedright\sffamily\large\textbf{Summer Experiences}} & 
   {\sffamily\large\textbf{306A }} \\
%row 2
   Chair:  & 
   {\raggedright Kinnis Gosha \textit{Morehouse College}} & \\ \\
{\sffamily \large 10:45}& 
\multicolumn{2}{@{}p{3.75in}}{\sffamily\raggedright\textbf{App Inventor for Android: Report from a Summer Camp}} \\
& \multicolumn{2}{@{}p{3.75in}}{\raggedright Krishnendu Roy, \textit{Valdosta State University}} \\ \\
\multicolumn{3}{@{}p{5in}}{\small Google's App Inventor for Android (AIA) is the newest visual programming language designed to introduce students to programming through creation of mobile apps. AIA opens up the world of mobile apps to novice programmers. Success stories of using AIA to introduce college students to programming exist. 

We used AIA in computing summer camps for high school students that we offer at our university. This paper is an experience report about using AIA in our camps. We provide a detailed description of designing our camps with AIA including the process of selecting and setting-up an Android device and instructional materials that we designed and made available to everyone. We also share evaluation results of using AIA in our camps and our impression of AIA as a programming-introduction tool.} \\ \\
{\sffamily \large 11:10}& 
\multicolumn{2}{@{}p{3.75in}}{\sffamily\raggedright\textbf{Sustainable and Effective Computing Summer Camps}} \\
& \multicolumn{2}{@{}p{3.75in}}{\raggedright Barbara Ericson, \textit{Georgia Institute of Technology}; Tom McKlin, \textit{The Findings Group}} \\ \\
\multicolumn{3}{@{}p{5in}}{\small Summer camps are a popular form of outreach for colleges and universities. But, it isn't enough to offer computing summer camps and hope students like them.  The camps should be effective by some measure, such as broadening participation by underrepresented groups and/or increasing learning. Summer camps should also be sustainable, so that institutions can continue to offer them regularly.  The summer camps at Georgia Tech have evolved to the point where they are sustainable and effective. This paper presents the rationale for our camps, the business model that makes them sustainable, and the evaluation results that demonstrate positive attitude changes and increases in learning.} \\ \\
{\sffamily \large 11:35}& 
\multicolumn{2}{@{}p{3.75in}}{\sffamily\raggedright\textbf{A Summer Science Experience with Computer Graphics for Secondary Students}} \\
& \multicolumn{2}{@{}p{3.75in}}{\raggedright Timothy Davis, \textit{Clemson University}} \\ \\
\multicolumn{3}{@{}p{5in}}{\small This paper describes the principles, implementation, and results of a weeklong summer science course for junior high and high school students interested in computer science.  To motivate and foster interest and creativity in students, while providing adequate complexity to introduce programming concepts and techniques, we used programming projects in computer graphics as the main learning tool.  Included in our discussion are experiences across three course offerings, as well as detailed course assignments.} \\ \\
\end{longtable}


\newpage
\begin{longtable}{@{}p{0.75in}@{}p{3.25in}@{}r}
   {\sffamily\large\textbf{PAPERS}} &
   {\raggedright\sffamily\large\textbf{Software Engineering}} & 
   {\sffamily\large\textbf{306B }} \\
%row 2
   Chair:  & 
   {\raggedright Ariel Ortiz \textit{Tecnologico de Monterrey, Campus Estado de Mexico}} & \\ \\
{\sffamily \large 10:45}& 
\multicolumn{2}{@{}p{3.75in}}{\sffamily\raggedright\textbf{Integrating UX with Scrum in an Undergraduate Software Development Project}} \\
& \multicolumn{2}{@{}p{3.75in}}{\raggedright Janet Davis, Chase Felker and Radka Slamova, \textit{Grinnell College}} \\ \\
\multicolumn{3}{@{}p{5in}}{\small We report our experiences using the Scrum agile software development method in an undergraduate user-centered web development project. Our chief contributions are to report on using Scrum in a summer research setting as distinct from academic-year coursework and to consider the integration of Scrum and user experience (UX) development methods in a non-professional, learning environment. Our experience with combining Scrum and UX was positive: this methodology gave our project a clear structure, kept us motivated, and focused us on developing a usable final product. We discuss our adaptations of Scrum to UX development and to the summer research setting, along with challenges we faced and lessons learned, to inform students and faculty who wish to apply such methods in future projects.} \\ \\
{\sffamily \large 11:10}& 
\multicolumn{2}{@{}p{3.75in}}{\sffamily\raggedright\textbf{Using WReSTT in SE Courses: An Empirical Study}} \\
& \multicolumn{2}{@{}p{3.75in}}{\raggedright Peter J. Clarke, Jairo Pava, Debra Davis and Frank Hernandez, \textit{Florida International University}; Tariq M. King, \textit{North Dakota State University}} \\ \\
\multicolumn{3}{@{}p{5in}}{\small There continues to be a lack of adequate training for students in software testing techniques and tools at most academic institutions. Several educators and researchers have investigated innovative approaches to integrate testing into programming and software engineering (SE) courses with some success.  The main problem is getting other educators to adopt their approaches and getting students to continue to use the techniques.   

In this paper we present a study that evaluates a non-intrusive approach to integrating software testing techniques and tools in SE courses using a Web-Based Repository of Software Testing Tools (WReSTT).  The results of the study show that students who use WReSTT in the classroom can improve their understanding and use of testing techniques and tools.} \\ \\
{\sffamily \large 11:35}& 
\multicolumn{2}{@{}p{3.75in}}{\sffamily\raggedright\textbf{Understanding the Tenets of Agile Software Engineering: Lecturing, Exploration and Critical Thinking}} \\
& \multicolumn{2}{@{}p{3.75in}}{\raggedright Shvetha Soundararajan, Amine Chigani and Arthur James, \textit{Virginia Tech}} \\ \\
\multicolumn{3}{@{}p{5in}}{\small In our quest to develop better software products, it is imperative that we strive to learn and understand the application of agile methods to the software development enterprise. Unfortunately, students have only limited exposure to the agile philosophy, principles and practices at the graduate and undergraduate levels of education. In an effort to address this concern, we offered an advanced graduate-level course entitled “Agile Software Engineering” in the Department of Computer Science at Virginia Tech. The primary objectives of the course were to introduce the values and principles and practices underlying the agile philosophy, and to do so in an atmosphere that encourages debate and critical thinking. This paper describes our experiences during the offering of that course.} \\ \\
\end{longtable}


\begin{longtable}[l]{@{}p{1in}@{}p{3in}@{}r}
    {\sffamily\large\textbf{SupporterSession}} & 
    {\sffamily\large\textbf{TBA}} & 
    {\sffamily\large\textbf{302C}} \\
\end{longtable}    
\begin{longtable}[l]{@{}p{1in}@{}p{3in}@{}r}
    {\sffamily\large\textbf{SupporterSession}} & 
    {\sffamily\large\textbf{Supporter Session: Intel}} & 
    {\sffamily\large\textbf{305A}} \\
\end{longtable}    
\vspace{0.5em}
\noindent\rule{5in}{0.02cm}
\vspace{0.5em}
\cfoot{\colorbox[gray]{0.45}{\color{white}\textsf{Friday 12:00 - 13:45}}}
\noindent
\framebox[5in][c]{{\Large\sffamily\textbf{Friday,  12:00 to 13:45}}}
\begin{longtable}[l]{@{}p{1in}@{}p{3in}@{}r}
    {\sffamily\large\textbf{None}} & 
    {\sffamily\large\textbf{Lunch Break}} & 
    {\sffamily\large\textbf{On your own}} \\
\end{longtable}    
\vspace{0.5em}
\noindent\rule{5in}{0.02cm}
\vspace{0.5em}
\cfoot{\colorbox[gray]{0.45}{\color{white}\textsf{Friday 12:10 - 13:35}}}
\noindent
\framebox[5in][c]{{\Large\sffamily\textbf{Friday,  12:10 to 13:35}}}
\begin{longtable}[l]{@{}p{1in}@{}p{3in}@{}r}
    {\sffamily\large\textbf{Ancillary Event}} & 
    {\sffamily\large\textbf{Snap! Lunch}} & 
    {\sffamily\large\textbf{301AB}} \\
\end{longtable}    
\begin{longtable}[l]{@{}p{1in}@{}p{3in}@{}r}
    {\sffamily\large\textbf{Ancillary Event}} & 
    {\sffamily\large\textbf{UPE National Meeting}} & 
    {\sffamily\large\textbf{302A}} \\
\end{longtable}    
\vspace{0.5em}
\noindent\rule{5in}{0.02cm}
\vspace{0.5em}
\cfoot{\colorbox[gray]{0.45}{\color{white}\textsf{Friday 13:45 - 15:00}}}
\noindent
\framebox[5in][c]{{\Large\sffamily\textbf{Friday,  13:45 to 15:00}}}
\begin{longtable}[l]{@{}l@{}l@{}r}
    \parbox[t]{1in}{\sffamily\large\textbf{PANEL}} & 
    \parbox[t]{3in}{\sffamily\raggedright\large\textbf{CS Principles:  Piloting a National Course}} & 
    \parbox[t]{1in}{\sffamily\raggedleft\large\textbf{301AB}} \\
% row 2    
    Chair: & 
    Owen Astrachan, \textit{Duke University}  \\[0.5em]
% row 3
    Participants: & 
    \multicolumn{2}{@{}l}{\parbox{3.75in}{Ralph Morelli, \textit{Trinity College}; Dwight Barnette, \textit{Virginia Tech}; Jeff Gray, \textit{University of Alabama}; Chinma Uche, \textit{Hartford Academy of Math and Science} }} \\[2em]
% row 4
    \multicolumn{3}{@{}p{5in}}{\small The CS Principles course has been designed to be taught nationwide at both the secondary and post-secondary levels. As part of this joint NSF/College Board project, 11 high schools are partnered with 10 colleges to teach the course, be part of a national initiative to test assessment items, and to help validate the curriculum framework that is the basis for the course and project. 
This special session is a report of the second stage of pilot that is designed to lead to a national standard and a new, additional AP exam in the next five years. This course will not replace the traditional, CS1-oriented AP exam, but will be a new national introduction to Computer Science.}
\end{longtable}
\begin{longtable}[l]{@{}l@{}l@{}r}
    \parbox[t]{1in}{\sffamily\large\textbf{PANEL}} & 
    \parbox[t]{3in}{\sffamily\raggedright\large\textbf{Fun, Phone, and the Future - Microsoft XNA Game Studio, Windows Phone, and Kinect SDK}} & 
    \parbox[t]{1in}{\sffamily\raggedleft\large\textbf{305B}} \\
% row 2    
    Chair: & 
    Pat Yongpradit, \textit{Springbrook High School}  \\[0.5em]
% row 3
    Participants: & 
    \multicolumn{2}{@{}l}{\parbox{3.75in}{ }} \\[2em]
% row 4
    \multicolumn{3}{@{}p{5in}}{\small Microsoft XNA Game Studio and C\# provide the basis of an advanced high school or introductory post-high school game development computer science course. Game development is serious computer science. The curriculum tools enable students to create games, simulations, and applications for the PC, Xbox 360, Windows Phone, and Kinect that expands students’ skills in complex logic, object oriented programming (OOP), advanced algorithms, and data structures. See and participate in demonstrations of student projects from the new Game Development with XNA course curriculum.}
\end{longtable}
\begin{longtable}[l]{@{}l@{}l@{}r}
    \parbox[t]{1in}{\sffamily\large\textbf{PANEL}} & 
    \parbox[t]{3in}{\sffamily\raggedright\large\textbf{Building an Open, Large-Scale Research Repository of Initial Programming Student Behavior}} & 
    \parbox[t]{1in}{\sffamily\raggedleft\large\textbf{306C}} \\
% row 2    
    Chair: & 
    Michael Kölling, \textit{University of Kent}  \\[0.5em]
% row 3
    Participants: & 
    \multicolumn{2}{@{}l}{\parbox{3.75in}{Ian Utting, \textit{University of Kent} }} \\[2em]
% row 4
    \multicolumn{3}{@{}p{5in}}{\small Many initiatives in improving learning of programming are based on gut instinct or localised experience. Gathering data as a basis for interventions, especially on a large scale, is hard. The BlueJ environment is being instrumented to collect data useful to a variety of educational programming researchers. BlueJ is ideally placed to collect such data: Users number in the millions, situated all over the world. This volume and diversity is unique in the history of such investigations and presents a significant opportunity for researchers. The data will be open to interested research groups, which will enable a wide variety of investigations that were previously impractical.
This session presents work to date and solicits input from researchers about the design of the data collection.}
\end{longtable}
\newpage
\begin{longtable}{@{}p{0.75in}@{}p{3.25in}@{}r}
   {\sffamily\large\textbf{PAPERS}} &
   {\raggedright\sffamily\large\textbf{Collaborative Learning}} & 
   {\sffamily\large\textbf{302A }} \\
%row 2
   Chair:  & 
   {\raggedright Adrian German \textit{Indiana University School of Informatics and Computing}} & \\ \\
{\sffamily \large 13:45}& 
\multicolumn{2}{@{}p{3.75in}}{\sffamily\raggedright\textbf{Assigning Student Programming Pairs Based on their Mental Model Consistency: An Initial Investigation}} \\
& \multicolumn{2}{@{}p{3.75in}}{\raggedright Alex Radermacher, Gursimran Walia and Richard Rummelt, \textit{North Dakota State University}} \\ \\
\multicolumn{3}{@{}p{5in}}{\small Pair Programming has been shown to be beneficial to student learning. This paper reports results of research investigating the effectiveness of pairing students based on their mental models. Prior research has found a correlation between mental model consistency and performance in computer programming courses. Students’ mental models helps to provide insights into how students approach problem solving and may indicate how to effectively pair students to improve their programming ability and learning. Results indicate that mental model consistency is a predictor of student success in an introductory programming course. Future goals of this research are to fully evaluate all pairing arrangements and to produce tests to evaluate mental model consistency for other computer science concepts.} \\ \\
{\sffamily \large 14:10}& 
\multicolumn{2}{@{}p{3.75in}}{\sffamily\raggedright\textbf{Group Whiteboards and Modeler/Customer Teams: Getting Closer to Industrial-Style Collaboration in a Classroom}} \\
& \multicolumn{2}{@{}p{3.75in}}{\raggedright David Socha, \textit{University of Washington Bothell}} \\ \\
\multicolumn{3}{@{}p{5in}}{\small This paper reports on two simple innovations that help create a more authentic and engaging modeling experience in an undergraduate analysis and design course: (a) having each team of students act both as modelers for another team, and as customers for another team, and (b) providing each team with their own whiteboard. The results from their use throughout the course, and for a single use of the whiteboards in a Computing Technology and Public Policy course, were quite positive. They resulted in a qualitatively different experience noticeable both to the instructors and the students. While some students were initially reluctant to use the whiteboards, by the end of the course most students were enthusiastic about their use.} \\ \\
{\sffamily \large 14:35}& 
\multicolumn{2}{@{}p{3.75in}}{\sffamily\raggedright\textbf{Is There Service in Computing Service Learning?}} \\
& \multicolumn{2}{@{}p{3.75in}}{\raggedright Randy Connolly, \textit{Mount Royal University}} \\ \\
\multicolumn{3}{@{}p{5in}}{\small While service learning projects in post-secondary computing can achieve important disciplinary outcomes for the students, the benefit of these projects for the service recipients and their community has been under-examined. This paper argues that since these projects are meant to benefit both student donors and community recipients, we must examine more carefully how computing service projects interact with all the social actors affected by the projects. Taking such an approach will require recognizing that ICT by itself will not increase democracy, equality, or any other social good; indeed some service learning projects may actually do more harm than good. The paper concludes with some sample computer learning projects that are oriented towards achieving true service for the recipients.} \\ \\
\end{longtable}


\newpage
\begin{longtable}{@{}p{0.75in}@{}p{3.25in}@{}r}
   {\sffamily\large\textbf{PAPERS}} &
   {\raggedright\sffamily\large\textbf{Curriculum Issues}} & 
   {\sffamily\large\textbf{302B }} \\
%row 2
   Chair:  & 
   {\raggedright Colleen Lewis \textit{University of California, Berkeley}} & \\ \\
{\sffamily \large 13:45}& 
\multicolumn{2}{@{}p{3.75in}}{\sffamily\raggedright\textbf{Computer Science in NZ High Schools: The first year of the new standards}} \\
& \multicolumn{2}{@{}p{3.75in}}{\raggedright Tim Bell, \textit{University of Canterbury}; Peter Andreae, \textit{Victoria University of Wellington}; Anthony Robins, \textit{University of Otago}} \\ \\
\multicolumn{3}{@{}p{5in}}{\small Computer science became available as a nationally assessed topic in NZ schools for the first time in 2011. We review the introduction of computer science as a formal topic, including the level of adoption, issues that have arisen in the process of introducing it, and work that has been undertaken to address those issues.} \\ \\
{\sffamily \large 14:10}& 
\multicolumn{2}{@{}p{3.75in}}{\sffamily\raggedright\textbf{Web Science: expanding the notion of Computer Science}} \\
& \multicolumn{2}{@{}p{3.75in}}{\raggedright Su White, \textit{University of Southampton}; Michalis Vafopoulos, \textit{Aristotle University of Thessaloniki}} \\ \\
\multicolumn{3}{@{}p{5in}}{\small Academic disciplines which experience rapid change face problems maintaining teaching programs. Web Science:‘the science of decentralized information systems’ is fundamentally interdisciplinary, encompassing the technologies and engineering of the Web alongside associated emerging human, social and organizational practices. As work on the Computer Curricula 2013 is underway, it seems timely to ask what place Web Science may have in the curriculum landscape. This paper discusses the role and place of Web Science in the computing disciplines. It provides an account of work towards defining a curriculum for Web Science utilizing novel methods to support and elaborate curriculum definition and review. The findings of a desk study of existing related curriculum recommendations are presented.} \\ \\
{\sffamily \large 14:35}& 
\multicolumn{2}{@{}p{3.75in}}{\sffamily\raggedright\textbf{Educating the Educator Through Computation: What GIS Can Do For Computer Science}} \\
& \multicolumn{2}{@{}p{3.75in}}{\raggedright John Barr and Ali Erkan, \textit{Ithaca College}} \\ \\
\multicolumn{3}{@{}p{5in}}{\small We designed a system where non-computational faculty members (along with undergraduates) enroll in an introductory, multidisciplinary, open source Geographic Information System (GIS) course to experience integrative learning as students. The faculty participants are subsequently required to integrate their newly acquired expertise with their own disciplinary teaching and research; the necessary time commitment is compensated by a three-credit teaching load reallocation. Our hypothesis is that increasing the general faculty's appreciation of computation (in the context of integrative learning) is an indirect yet effective and scalable way to reach a wider group of students to convey our fundamental disciplinary message: computing is more than programming and computing empowers people.} \\ \\
\end{longtable}


\newpage
\begin{longtable}{@{}p{0.75in}@{}p{3.25in}@{}r}
   {\sffamily\large\textbf{PAPERS}} &
   {\raggedright\sffamily\large\textbf{Active Learning I}} & 
   {\sffamily\large\textbf{306A }} \\
%row 2
   Chair:  & 
   {\raggedright Robert England \textit{Transylvania University}} & \\ \\
{\sffamily \large 13:45}& 
\multicolumn{2}{@{}p{3.75in}}{\sffamily\raggedright\textbf{An Experience Report: On The Use Of Multimedia Pre-Instruction And Just-In-Time Teaching In A CS1 Course}} \\
& \multicolumn{2}{@{}p{3.75in}}{\raggedright Paul Carter, \textit{University of British Columbia}} \\ \\
\multicolumn{3}{@{}p{5in}}{\small We describe an experience using online multimedia instruction and just-in-time teaching in an introductory programming course. Survey data has shown that students are strongly in favour of the approach. A series of screencasts was developed to replace the traditional lecture component of the course. Students were asked to review a small number of screencasts before each class and were assessed on their comprehension at the start of class using a series of “clicker”questions. A just-in-time mini-lecture was provided in response to the initial assessment, on an as-needed basis. The remaining class time was devoted to small-group exercises.} \\ \\
{\sffamily \large 14:10}& 
\multicolumn{2}{@{}p{3.75in}}{\sffamily\raggedright\textbf{Using JiTT in a Database Course}} \\
& \multicolumn{2}{@{}p{3.75in}}{\raggedright Alexandra Martinez, \textit{Universidad de Costa Rica}} \\ \\
\multicolumn{3}{@{}p{5in}}{\small This paper describes our experience using the Just-in-Time Teaching (JiTT) technique in an undergraduate database course for computer science majors during two semesters. JiTT was implemented by giving the students reading assignments and asking them to complete web-based reading tests the day before class, so that the instructor could detect weaknesses in students' understanding of the material and adjust the lesson plan just in time for the next day class. Based on surveys as well as on exams and course grades, we found a significant improvement on the students interest in the course and learning of the material.} \\ \\
{\sffamily \large 14:35}& 
\multicolumn{2}{@{}p{3.75in}}{\sffamily\raggedright\textbf{Process Oriented Guided Inquiry Learning (POGIL) for Computer Science}} \\
& \multicolumn{2}{@{}p{3.75in}}{\raggedright Clifton Kussmaul, \textit{Muhlenberg College}} \\ \\
\multicolumn{3}{@{}p{5in}}{\small This paper describes an ongoing project to develop activities for computer science (CS) using process oriented guided inquiry learning (POGIL). First, it reviews relevant background on effective learning and POGIL, compares POGIL to other forms of active learning, and describes the potential of POGIL for CS. Second, it describes a sample POGIL activity, including the structure and contents, student and facilitator actions during the activity, and how activities are designed. Third, it summarizes current progress and plans for a NSF TUES project to development POGIL materials for CS. Finally, it discusses student feedback and future directions.} \\ \\
\end{longtable}


\newpage
\begin{longtable}{@{}p{0.75in}@{}p{3.25in}@{}r}
   {\sffamily\large\textbf{PAPERS}} &
   {\raggedright\sffamily\large\textbf{Communication Skills}} & 
   {\sffamily\large\textbf{306B }} \\
%row 2
   Chair:  & 
   {\raggedright James Early \textit{SUNY Oswego}} & \\ \\
{\sffamily \large 13:45}& 
\multicolumn{2}{@{}p{3.75in}}{\sffamily\raggedright\textbf{Integrating Communication Skills into the Computer Science Curriculum}} \\
& \multicolumn{2}{@{}p{3.75in}}{\raggedright Katrina Falkner and Nickolas Falkner, \textit{University of Adelaide}} \\ \\
\multicolumn{3}{@{}p{5in}}{\small Computer Science majors must be able to communicate effectively.  There is considerable work in the area of communication skills development, positioned in terms of curriculum guidelines for effective communication skills development, and example communication skills activities. However, this research is deficient in detailed, contextualised methodologies and frameworks for the development of communication skills within the Computer Science curriculum.  We present a new methodology, building upon well established theoretical frameworks, designed to assist academics in the development of communication skills activities integrated with discipline content across the curriculum.} \\ \\
{\sffamily \large 14:10}& 
\multicolumn{2}{@{}p{3.75in}}{\sffamily\raggedright\textbf{'Explain in Plain English' Questions: Implications for Teaching}} \\
& \multicolumn{2}{@{}p{3.75in}}{\raggedright Laurie Murphy, \textit{Pacific Lutheran University}; Renée McCauley, \textit{College of Charleston}; Sue Fitzgerald, \textit{Metropolitan State University}} \\ \\
\multicolumn{3}{@{}p{5in}}{\small This paper reports on a replication of work by Corney, Lister and Teague [3] who performed a longitudinal study of novice programmers, looking for relationships between ability to 'explain in plain English' the meaning of a code segment and success in writing code later in the semester.  The study extends the work of Corney, Lister and Teague by qualitatively evaluating ‘explain in plain English’ responses to gain a deeper understanding of student misconceptions.  Statistical results from this study are similar to those of Corney, Lister and Teague.  Results highlight students’ fragile knowledge and suggest the need for assessment and instruction of basic concepts later into the term than instructors are likely to expect.} \\ \\
{\sffamily \large 14:35}& 
\multicolumn{2}{@{}p{3.75in}}{\sffamily\raggedright\textbf{The impact of question generation activities on performance}} \\
& \multicolumn{2}{@{}p{3.75in}}{\raggedright Andrew Luxton-Reilly, Daniel Bertinshaw, Paul Denny, Beryl Plimmer and Robert Sheehan, \textit{The University of Auckland}} \\ \\
\multicolumn{3}{@{}p{5in}}{\small Recent interest in student-centric pedagogies have resulted in the development of numerous tools that support student-generated questions.  Previous evaluations of such tools have reported strong correlations between student participation and exam performance, yet the level of student engagement with other learning activities in the course is a potential confounding factor.  We show such correlations may be explained by other factors, and we undertake a deeper analysis that reveals evidence of the positive impact question-generation activities have on student performance.} \\ \\
\end{longtable}


\begin{longtable}[l]{@{}p{1in}@{}p{3in}@{}r}
    {\sffamily\large\textbf{SupporterSession}} & 
    {\sffamily\large\textbf{TBA}} & 
    {\sffamily\large\textbf{302C}} \\
\end{longtable}    
\begin{longtable}[l]{@{}p{1in}@{}p{3in}@{}r}
    {\sffamily\large\textbf{SupporterSession}} & 
    {\sffamily\large\textbf{Supporter Session: GoogleThe MIT Center for Mobile Learning and the Future of App Inventor}} & 
    {\sffamily\large\textbf{305A}} \\
\end{longtable}    
\vspace{0.5em}
\noindent\rule{5in}{0.02cm}
\vspace{0.5em}
\cfoot{\colorbox[gray]{0.45}{\color{white}\textsf{Friday 15:00 - 15:45}}}
\noindent
\framebox[5in][c]{{\Large\sffamily\textbf{Friday,  15:00 to 15:45}}}
\begin{longtable}[l]{@{}p{1in}@{}p{3in}@{}r}
    {\sffamily\large\textbf{None}} & 
    {\sffamily\large\textbf{Break and Exhibits}} & 
    {\sffamily\large\textbf{Exhibit Hall A}} \\
\end{longtable}    
\vspace{0.5em}
\noindent\rule{5in}{0.02cm}
\vspace{0.5em}
\cfoot{\colorbox[gray]{0.45}{\color{white}\textsf{Friday 15:00 - 16:30}}}
\noindent
\framebox[5in][c]{{\Large\sffamily\textbf{Friday,  15:00 to 16:30}}}
\begin{longtable}[l]{@{}p{1in}@{}p{3in}@{}r}
    {\sffamily\large\textbf{Project Showcase}} & 
    {\sffamily\large\textbf{NSF Showcase \#4}} & 
    {\sffamily\large\textbf{Exhibit Hall A}} \\
\end{longtable}    
\vspace{0.5em}
\noindent\rule{5in}{0.02cm}
\vspace{0.5em}
\cfoot{\colorbox[gray]{0.45}{\color{white}\textsf{Friday 15:00 - 17:00}}}
\noindent
\framebox[5in][c]{{\Large\sffamily\textbf{Friday,  15:00 to 17:00}}}
\newpage
\begin{longtable}{@{}p{0.75in}@{}p{3.25in}@{}r}
   {\sffamily\large\textbf{POSTER}} &
   {\raggedright\sffamily\large\textbf{Poster Session II}} & 
   {\sffamily\large\textbf{Exhibit Hall A }} \\
%row 2
   Chair:  & 
   {\raggedright Dummy TestData \textit{First Record of Database}} & \\ \\
\multicolumn{3}{@{}p{5in}}{\sffamily\raggedright\textbf{Implementing and Assessing a Blended CS1 Course}} \\
\multicolumn{3}{@{}p{5in}}{\raggedright John Wright, \textit{Juniata College}} \\ \\
\multicolumn{3}{@{}p{5in}}{\sffamily\raggedright\textbf{Designing with Projects in Mind: An Approach for Creating Authentic (and Manageable) Programming Projects}} \\
\multicolumn{3}{@{}p{5in}}{\raggedright Scott Turner, \textit{UNC Pembroke}} \\ \\
\multicolumn{3}{@{}p{5in}}{\sffamily\raggedright\textbf{Integrating Elementary Computational Modeling and Programming Principles}} \\
\multicolumn{3}{@{}p{5in}}{\raggedright Jose Garrido, \textit{Kennesaw State University}} \\ \\
\multicolumn{3}{@{}p{5in}}{\sffamily\raggedright\textbf{RoboLIFT: Simple GUI-Based Unit Testing of Student-Written Android Applications}} \\
\multicolumn{3}{@{}p{5in}}{\raggedright Anthony Allevato and Stephen Edwards, \textit{Virginia Tech}} \\ \\
\multicolumn{3}{@{}p{5in}}{\sffamily\raggedright\textbf{OpenDSA: A Creative Commons Active-eBook}} \\
\multicolumn{3}{@{}p{5in}}{\raggedright Eric Fouh, Maoyuan Sun and Clifford Shaffer, \textit{Virginia Tech}} \\ \\
\multicolumn{3}{@{}p{5in}}{\sffamily\raggedright\textbf{Active Learning in Computer Science Education Using Meta-Cognition}} \\
\multicolumn{3}{@{}p{5in}}{\raggedright Murali Mani and Quamrul Mazumder, \textit{University of Michigan, Flint}} \\ \\
\multicolumn{3}{@{}p{5in}}{\sffamily\raggedright\textbf{Dynamic Programming Across the CS Curriculum}} \\
\multicolumn{3}{@{}p{5in}}{\raggedright Yana Kortsarts, \textit{Widener University}; Vasily Kolchenko, \textit{New York City College of Technology  The City University of New York}} \\ \\
\multicolumn{3}{@{}p{5in}}{\sffamily\raggedright\textbf{50 Ways to be a FOSSer:  Simple ways to involve students \& faculty}} \\
\multicolumn{3}{@{}p{5in}}{\raggedright Clifton Kussmaul, \textit{Muhlenberg College}; Heidi Ellis, \textit{Western New England University}; Greg Hislop, \textit{Drexel University}} \\ \\
\multicolumn{3}{@{}p{5in}}{\sffamily\raggedright\textbf{Teaching Computer Science and programming concepts using LEGO NXT and TETRIX Robotics, and Computer Science Unplugged activities}} \\
\multicolumn{3}{@{}p{5in}}{\raggedright Daniela Marghitu, Taha Ben Brahim and John Weaver, \textit{Auburn University}} \\ \\
\multicolumn{3}{@{}p{5in}}{\sffamily\raggedright\textbf{Using POGIL to Teach Students To Be Better Problem Solvers}} \\
\multicolumn{3}{@{}p{5in}}{\raggedright Helen Hu, \textit{Westminster College}} \\ \\
\multicolumn{3}{@{}p{5in}}{\sffamily\raggedright\textbf{Developing a Gaming Concentration in the Computer Science Curriculum  at an HBCU}} \\
\multicolumn{3}{@{}p{5in}}{\raggedright Jinghua Zhang and Elva Jones, \textit{Winston-Salem State University}} \\ \\
\multicolumn{3}{@{}p{5in}}{\sffamily\raggedright\textbf{OSSIE: An Open Source Software Defined Radio (SDR) Toolset for Education and Research}} \\
\multicolumn{3}{@{}p{5in}}{\raggedright Jason Snyder, \textit{Virginia Tech}} \\ \\
\multicolumn{3}{@{}p{5in}}{\sffamily\raggedright\textbf{Implementing a Communication Intensive Core Course in the CS Curriculum:  A Survey of Methods}} \\
\multicolumn{3}{@{}p{5in}}{\raggedright Jean French, \textit{Coastal Carolina University}} \\ \\
\multicolumn{3}{@{}p{5in}}{\sffamily\raggedright\textbf{The Reflective Mentor: Charting Undergraduates' Responses to Computer Science Service Learning}} \\
\multicolumn{3}{@{}p{5in}}{\raggedright Quinn Burke, Yasmin Kafai, Jean Griffin, Rita Powell, Michele Grab, Susan Davidson and Joseph Sun, \textit{University of Pennsylvania}} \\ \\
\multicolumn{3}{@{}p{5in}}{\sffamily\raggedright\textbf{Teaching Cryptography Using Hands-on Labs}} \\
\multicolumn{3}{@{}p{5in}}{\raggedright Li Yang, \textit{University of Tennessee at Chattanooga}; Joseph Kizza, \textit{Univeristy of Tennessee at Chattanooga}; Andy Wang, \textit{Southern Polytechnic State University}; Chung-Han Chen, \textit{Tuskegee University}} \\ \\
\multicolumn{3}{@{}p{5in}}{\sffamily\raggedright\textbf{From Drawing to Programming Attracting Middle-School Students to Programming through Self-Disclosing Code}} \\
\multicolumn{3}{@{}p{5in}}{\raggedright Jennelle Nystrom, Pelle Hall, Andrew Hirakawa and Samuel Rebelsky, \textit{Grinnell College}} \\ \\
\multicolumn{3}{@{}p{5in}}{\sffamily\raggedright\textbf{Proposed Revisions to the Social and Professional Knowledge Area for CS2013}} \\
\multicolumn{3}{@{}p{5in}}{\raggedright Carol Spradling, \textit{Northwest Missouri State University}; Florence Appel, \textit{Saint Xavier University}; Elizabeth Hawthorne, \textit{Union County College}} \\ \\
\multicolumn{3}{@{}p{5in}}{\sffamily\raggedright\textbf{A Better API for Java Reflection}} \\
\multicolumn{3}{@{}p{5in}}{\raggedright Zalia Shams, \textit{Virginia Tech}} \\ \\
\multicolumn{3}{@{}p{5in}}{\sffamily\raggedright\textbf{Hands-on Labs for a Mini-Course on Mobile Application Development}} \\
\multicolumn{3}{@{}p{5in}}{\raggedright Qusay H. Mahmoud, Nicholas Mair, Younis Mohamed and Sunny Dhillon, \textit{University of Guelph}} \\ \\
\multicolumn{3}{@{}p{5in}}{\sffamily\raggedright\textbf{CEOHP Evaluation, Evolution, and Archival}} \\
\multicolumn{3}{@{}p{5in}}{\raggedright Vicki Almstrum, \textit{Strayer University}; Barbara Owens, \textit{Southwestern University}; Mary Last, \textit{CEOHP}; Deepa Muralidhar, \textit{North Gwinnett High School}} \\ \\
\multicolumn{3}{@{}p{5in}}{\sffamily\raggedright\textbf{CodeTrainer Teacher Authoring System: Facilitating User-Created Content in an Intelligent Tutoring System}} \\
\multicolumn{3}{@{}p{5in}}{\raggedright Christy McGuire, Thomas Harris and Jonathan Steinhart, \textit{Tutor Technologies, Inc}; Leigh Ann Sudol-DeLyser, \textit{Carnegie Mellon University}} \\ \\
\multicolumn{3}{@{}p{5in}}{\sffamily\raggedright\textbf{Comparing Feature Sets within Visual and Command Line Environments and their effect on Novice Programming}} \\
\multicolumn{3}{@{}p{5in}}{\raggedright Edward Dillon, Jr., Monica Anderson-Herzog and Marcus Brown, \textit{University of Alabama}} \\ \\
\multicolumn{3}{@{}p{5in}}{\sffamily\raggedright\textbf{Exploring Connected Worlds}} \\
\multicolumn{3}{@{}p{5in}}{\raggedright Jeffrey Forbes, \textit{Duke University}} \\ \\
\multicolumn{3}{@{}p{5in}}{\sffamily\raggedright\textbf{Teaching parallel computing with higher-level languages and compelling examples}} \\
\multicolumn{3}{@{}p{5in}}{\raggedright Jens Mache, Christopher T. Mitchell and Julian H. Dale, \textit{Lewis \& Clark College}; David P. Bunde, Casey Samoore, Sung Joo Lee and Johnathan Ebbers, \textit{Knox College}} \\ \\
\end{longtable}


\vspace{0.5em}
\noindent\rule{5in}{0.02cm}
\vspace{0.5em}
\cfoot{\colorbox[gray]{0.45}{\color{white}\textsf{Friday 15:45 - 17:00}}}
\noindent
\framebox[5in][c]{{\Large\sffamily\textbf{Friday,  15:45 to 17:00}}}
\begin{longtable}[l]{@{}l@{}l@{}r}
    \parbox[t]{1in}{\sffamily\large\textbf{PANEL}} & 
    \parbox[t]{3in}{\sffamily\raggedright\large\textbf{Understanding NSF Funding Opportunities}} & 
    \parbox[t]{1in}{\sffamily\raggedleft\large\textbf{301AB}} \\
% row 2    
    Chair: & 
    Suzanne Westbrook, \textit{National Science Foundation}  \\[0.5em]
% row 3
    Participants: & 
    \multicolumn{2}{@{}l}{\parbox{3.75in}{Victor Piotrowski, Jeff Forbes, Harriet Taylor and Mimi McClure, \textit{National Science Foundation} }} \\[2em]
% row 4
    \multicolumn{3}{@{}p{5in}}{\small This session highlights programs in the National Science Foundation’s Division of Undergraduate Education, Office of Cyberinfrastructure and Directorate of Computer and Information Science and Engineering. The focus is on providing descriptions of several programs of interest to college faculty and discussing the requirements and guidelines for programs in these areas. It includes a description of the proposal and review processes as well as strategies for writing competitive proposals. Participants are encouraged to discuss procedural issues with the presenters.}
\end{longtable}
\begin{longtable}[l]{@{}l@{}l@{}r}
    \parbox[t]{1in}{\sffamily\large\textbf{PANEL}} & 
    \parbox[t]{3in}{\sffamily\raggedright\large\textbf{Teaching Outside the Text}} & 
    \parbox[t]{1in}{\sffamily\raggedleft\large\textbf{305B}} \\
% row 2    
    Chair: & 
    Lester Wainwright \textit{Charlottesville High School}  \\[0.5em]
% row 3
    Participants: & 
    \multicolumn{2}{@{}l}{\parbox{3.75in}{Renee Ciezki, \textit{Estrella Mountain Community College}; Barbara Ericson, \textit{Georgia Institute of Technology}; Glen Martin, \textit{TAG Magnet High School} }} \\[2em]
% row 4
    \multicolumn{3}{@{}p{5in}}{\small We know that students bring diverse experiences and an assortment of learning styles into our classrooms.  We greet them and hand out a syllabus listing the required book(s).  One size does not fit all when it comes to textbooks.  In this session, participants will discover teaching activities that can be used to supplement any text:   hands-on, interesting and fun activities that help students understand CS topics. Members of the AP Computer Science-A Development Committee will share these resources and lead a discussion of proven strategies and lesson ideas for teaching outside the textbook.}
\end{longtable}
\begin{longtable}[l]{@{}l@{}l@{}r}
    \parbox[t]{1in}{\sffamily\large\textbf{PANEL}} & 
    \parbox[t]{3in}{\sffamily\raggedright\large\textbf{Computing Engineering Review Task Force Report}} & 
    \parbox[t]{1in}{\sffamily\raggedleft\large\textbf{306C}} \\
% row 2    
    Chair: & 
    John Impagliazzo, \textit{Hofstra University}  \\[0.5em]
% row 3
    Participants: & 
    \multicolumn{2}{@{}l}{\parbox{3.75in}{Susan Conry, \textit{Clarkson University}; Eric Durant, \textit{Milwaukee School of Engineering}; Andrew McGettrick, \textit{University of Strathclyde}; Timothy Wilson, \textit{Embry-Riddle Aeronautical University}; Mitch Thornton, \textit{Southern Methodist University} }} \\[2em]
% row 4
    \multicolumn{3}{@{}p{5in}}{\small The ACM and the IEEE Computer Society created the CE2004 Review Task Force (RTF) and charged it with the task of reviewing and determining the extent to which the CE2004 document required revisions. The RTF recommended keeping the structure and the vast majority of the content of the original CE2004 document. It also recommended that contemporary topics should be strengthened or added while de-emphasizing other topics. Additionally, the RTF recommended that the two societies form a joint special-purpose committee to update and edit the earlier document and to seek input and review from the computer engineering industrial and academic communities. The presentation will provide insights in the RTF findings and thoughts on how a future computer engineering report might evolve.}
\end{longtable}
\newpage
\begin{longtable}{@{}p{0.75in}@{}p{3.25in}@{}r}
   {\sffamily\large\textbf{PAPERS}} &
   {\raggedright\sffamily\large\textbf{Projects}} & 
   {\sffamily\large\textbf{302A }} \\
%row 2
   Chair:  & 
   {\raggedright Jeff Gray \textit{University of Alabma}} & \\ \\
{\sffamily \large 15:45}& 
\multicolumn{2}{@{}p{3.75in}}{\sffamily\raggedright\textbf{Social Sensitivity and Classroom Team Projects: An Empirical Investigation}} \\
& \multicolumn{2}{@{}p{3.75in}}{\raggedright Lisa Bender, Gursimran Walia, Krishna Kambhampaty and Kendall E. Nygard, \textit{North Dakota State University}; Travis E. Nygard, \textit{Ripon College}} \\ \\
\multicolumn{3}{@{}p{5in}}{\small Team work is the norm in major development projects and industry is continually striving to improve team effectiveness.  Researchers have established that teams with high levels of social sensitivity tend to perform well when completing a variety of specific collaborative tasks. Our claim is that, the social sensitivity can be a key component in predicting the performance of teams that carry out major projects. This paper reports the results from an empirical study that investigates whether social sensitivity is correlated with the performance of student teams on large semester-long projects. The overall result supports our claim. It suggests, therefore, that educators in computer-related disciplines should take the concept of social sensitivity seriously as an aid to productivity.} \\ \\
{\sffamily \large 16:10}& 
\multicolumn{2}{@{}p{3.75in}}{\sffamily\raggedright\textbf{Taming Complexity in Large Scale Systems Projects}} \\
& \multicolumn{2}{@{}p{3.75in}}{\raggedright Shimon Schocken, \textit{IDC Herzliya}} \\ \\
\multicolumn{3}{@{}p{5in}}{\small Engaging students in large software development projects is an important objective, since it exposes design and programming challenges that come to play only with scale. Alas, large scale projects can be monstrously complex – to the extent of being infeasible in academic settings. We describe a framework and a set of principles that enable students to develop large scale systems – e.g. a complete hardware platform or a compiler – in several semester weeks.} \\ \\
{\sffamily \large 16:35}& 
\multicolumn{2}{@{}p{3.75in}}{\sffamily\raggedright\textbf{An Approach for Evaluating FOSS Projects for Student Participation}} \\
& \multicolumn{2}{@{}p{3.75in}}{\raggedright Heidi Ellis, \textit{Western New England University}; Michelle Purcell and Gregory Hislop, \textit{Drexel University}} \\ \\
\multicolumn{3}{@{}p{5in}}{\small Free and Open Source Software (FOSS) offers a transparent development environment and community in which to involve students. Students can learn much about software development and professionalism by contributing to an on-going project. However, the number of FOSS projects is very large and there is a wide range of size, complexity, domains, and communities, making selection of an ideal project for students difficult. This paper addresses the need for guidance when selecting a FOSS project for student involvement by presenting an approach for FOSS project selection based on clearly identified criteria. The approach is based on several years of experience involving students in FOSS projects.} \\ \\
\end{longtable}


\newpage
\begin{longtable}{@{}p{0.75in}@{}p{3.25in}@{}r}
   {\sffamily\large\textbf{PAPERS}} &
   {\raggedright\sffamily\large\textbf{Alice and Scratch}} & 
   {\sffamily\large\textbf{302B }} \\
%row 2
   Chair:  & 
   {\raggedright Kelly Powers \textit{Advanced Math \& Science Academy Charter School}} & \\ \\
{\sffamily \large 15:45}& 
\multicolumn{2}{@{}p{3.75in}}{\sffamily\raggedright\textbf{Integrating Computing into Middle School Disciplines Through Projects}} \\
& \multicolumn{2}{@{}p{3.75in}}{\raggedright Susan Rodger, Melissa Dalis, Peggy Li, Liz Liang and Wenhui Zhang, \textit{Duke University}; Chitra Gadwal, \textit{UMBC}; Francine Wolfe, \textit{Benedictine College}} \\ \\
\multicolumn{3}{@{}p{5in}}{\small For four years we have been integrating computing into a variety of middle 
school disciplines via the Alice programing language.  This paper describes our efforts over
the past two years in creating model projects for students to build in all
disciplines, and our most recent focus on science and mathematics
projects. For science we have introduced experiments in Alice and the tools
needed for them. In mathematics we have created projects to increase their
understanding of programming and to use the projects to increase their 
understanding of mathematics. We also discuss our efforts in workshops to 
teach K-12 teachers Alice and an analysis of the teachers' lesson plans and 
worlds developed in the most recent workshop.} \\ \\
{\sffamily \large 16:10}& 
\multicolumn{2}{@{}p{3.75in}}{\sffamily\raggedright\textbf{Children Learning Computer Science Concepts via Alice Game-Programming}} \\
& \multicolumn{2}{@{}p{3.75in}}{\raggedright Linda Werner, \textit{University of California, Santa Cruz}; Shannon Campe and Jill Denner, \textit{ETR Associates}} \\ \\
\multicolumn{3}{@{}p{5in}}{\small Programming environments that incorporate drag-and-drop methods and many pre-defined objects and operations are being widely used in K-12 settings. But can students as young as those in middle school learn complex computer science concepts using these programming environments when computer science is not the focus of the course? In this paper, we describe a semester-long game-programming course where 325 middle school students used Alice. We report on our analysis of 225 final games where we measured the frequency of successful execution of programming constructs. Our results show that many games exhibit successful uses of high level computer science concepts such as student-created abstractions, concurrent execution, and event handlers.} \\ \\
{\sffamily \large 16:35}& 
\multicolumn{2}{@{}p{3.75in}}{\sffamily\raggedright\textbf{The Writers’ Workshop for Youth Programmers: Digital Storytelling with Scratch in Middle School Classrooms}} \\
& \multicolumn{2}{@{}p{3.75in}}{\raggedright Quinn Burke and Yasmin B. Kafai, \textit{University of Pennsylvania}} \\ \\
\multicolumn{3}{@{}p{5in}}{\small This study investigates the potential to introduce basic programming concepts to middle school children within the context of a classroom writing-workshop. In this paper we describe how students drafted, revised, and published their own digital stories using the introductory programming language Scratch and in the process learned fundamental CS concepts as well as the wider connection between programming and writing as interrelated processes of composition.} \\ \\
\end{longtable}


\newpage
\begin{longtable}{@{}p{0.75in}@{}p{3.25in}@{}r}
   {\sffamily\large\textbf{PAPERS}} &
   {\raggedright\sffamily\large\textbf{Active Learning II}} & 
   {\sffamily\large\textbf{306A }} \\
%row 2
   Chair:  & 
   {\raggedright Douglas Kranch \textit{North Central State College}} & \\ \\
{\sffamily \large 15:45}& 
\multicolumn{2}{@{}p{3.75in}}{\sffamily\raggedright\textbf{A Software Craftsman’s approach to Data Structures}} \\
& \multicolumn{2}{@{}p{3.75in}}{\raggedright Arto Vihavainen, Matti Luukkainen and Thomas Vikberg, \textit{University of Helsinki}} \\ \\
\multicolumn{3}{@{}p{5in}}{\small Data Structures (CS2) courses and course books do not usually put much emphasis in the process of how a data structure is engineered or invented. Instead, algorithms are readily given, and the main focus is in the mathematical complexity analysis of the algorithms. We present an alternative approach on presenting data structures using worked examples, i.e., by explicitly displaying the process that leads to the invention and creation of a data stucture and its algorithms. Our approach is heavily backed up by some of the best programming practices advocated by the Agile and Software Craftmanship communities and it brings the often mathematically oriented CS2 course closer to modern software engineering and practical problem solving, without a need for compromise in proofs and analysis.} \\ \\
{\sffamily \large 16:10}& 
\multicolumn{2}{@{}p{3.75in}}{\sffamily\raggedright\textbf{Jutge.org: An Educational Programming Judge}} \\
& \multicolumn{2}{@{}p{3.75in}}{\raggedright Petit Jordi and Roura Salvador, \textit{Universitat Politècnica de Catalunya}; Giménez Omer, \textit{Google}} \\ \\
\multicolumn{3}{@{}p{5in}}{\small Jutge.org is an educational programming judge where students can solve more than 800 problems using 22 programming languages. The verdict of their solutions is computed using exhaustive test sets run under time, memory and security restrictions. By contrast to many popular online judges, Jutge.org is designed for students and 
instructors: On one hand, the problem repository is mainly aimed to beginners, with a clear organization and gradding. On the other 
hand, the system is designed as a virtual learning environment where instructors can administer their own courses, manage their roster of students and tutors, add problems, attach documents, create lists of problems, assignments, contests and exams. This paper presents Jutge.org and offers some case studies of courses using it.} \\ \\
{\sffamily \large 16:35}& 
\multicolumn{2}{@{}p{3.75in}}{\sffamily\raggedright\textbf{Integrating Formal Verification in an Online Judge for e-Learning Logic Circuit Design}} \\
& \multicolumn{2}{@{}p{3.75in}}{\raggedright Javier De San Pedro, Josep Carmona, Jordi Cortadella and Jordi Petit, \textit{Universitat Politècnica de Catalunya}} \\ \\
\multicolumn{3}{@{}p{5in}}{\small This paper investigates the use of formal verification techniques to create online judges that can assist in teaching logic circuit design. 
Formal verification not only contributes to give an exact assessment about correctness, but also saves the instructor the tedious task of designing test cases.
The paper explains how formal verification has been integrated in an online judge. It also describes the courseware created for a course on logic circuits and the successful experience of using it in a one-week summer course with students from secondary and high school.} \\ \\
\end{longtable}


\newpage
\begin{longtable}{@{}p{0.75in}@{}p{3.25in}@{}r}
   {\sffamily\large\textbf{PAPERS}} &
   {\raggedright\sffamily\large\textbf{Non-majors}} & 
   {\sffamily\large\textbf{306B }} \\
%row 2
   Chair:  & 
   {\raggedright Derek Schuurman \textit{Redeemer University College}} & \\ \\
{\sffamily \large 15:45}& 
\multicolumn{2}{@{}p{3.75in}}{\sffamily\raggedright\textbf{Computing For STEM Majors:  Enhancing Non CS Majors’ Computing Skills}} \\
& \multicolumn{2}{@{}p{3.75in}}{\raggedright Joel Adams and Randall Pruim, \textit{Calvin College}} \\ \\
\multicolumn{3}{@{}p{5in}}{\small One of the challenges facing the U.S. technological workforce is that fewer college graduates are being prepared for computing careers.  Besides trying to attract more CS majors, another approach is to (i) design a computing curriculum that appeals to students and faculty from non-CS disciplines, (ii) use special scholarships to attract students to that curriculum, and (iii) sponsor faculty development workshops for non-CS departments.  In this paper, we detail this approach, using a new introductory course oriented to science majors, and scholarships funded by the National Science Foundation Scholarships for Science, Technology, Engineering, and Mathematics (NSF S-STEM) program.  We also present several success stories that this approach has produced in its first two years.} \\ \\
{\sffamily \large 16:10}& 
\multicolumn{2}{@{}p{3.75in}}{\sffamily\raggedright\textbf{Operations Research: Broadening Computer Science In A Liberal Arts College}} \\
& \multicolumn{2}{@{}p{3.75in}}{\raggedright Barbara Anthony, \textit{Southwestern University}} \\ \\
\multicolumn{3}{@{}p{5in}}{\small Operations research, while not traditionally taught at many small or liberal arts colleges, can be a significant asset to the offerings of a computer science department. Often seen as a discipline at the intersection of mathematics, computer science, business, and engineering, it has great interdisciplinary potential and practical appeal, allowing for recruitment of students who may not consider taking a CS0 or CS1 course. Offering this course not only benefited computer science majors who appreciated the applications and different perspectives, but it provided a means for the department to serve a wider population, increased interdisciplinary education, and resulted in a filled-to-capacity upper-level course in computer science for the first time in recent memory.} \\ \\
{\sffamily \large 16:35}& 
\multicolumn{2}{@{}p{3.75in}}{\sffamily\raggedright\textbf{Beyond Competency: A Context-Driven CS0 Course}} \\
& \multicolumn{2}{@{}p{3.75in}}{\raggedright Jeff Cramer and Bill Toll, \textit{Taylor University}} \\ \\
\multicolumn{3}{@{}p{5in}}{\small In the process of revising our general education course we attempted to answer the question “What should a graduate of a liberal arts university understand about computational technology?” University students may know more about narrow areas of technology but the true impact on their lives cannot be understood without an appreciation for the nature and limitations of the technology.  This paper presents a set of assumptions about the impact of technology on individuals and society and describes elements of a computing context designed to enable students to critically evaluate the technology that has such an impact on their lives.  Assessment of the approach indicates that students are more aware of the impact of technology and the importance of an understanding of the technology.} \\ \\
\end{longtable}


\begin{longtable}[l]{@{}p{1in}@{}p{3in}@{}r}
    {\sffamily\large\textbf{SupporterSession}} & 
    {\sffamily\large\textbf{TBA}} & 
    {\sffamily\large\textbf{302C}} \\
\end{longtable}    
\begin{longtable}[l]{@{}p{1in}@{}p{3in}@{}r}
    {\sffamily\large\textbf{SupporterSession}} & 
    {\sffamily\large\textbf{Supporter Session: MicrosoftCloud in a Classroom:  Faculty Experiences}} & 
    {\sffamily\large\textbf{305A}} \\
\end{longtable}    
\vspace{0.5em}
\noindent\rule{5in}{0.02cm}
\vspace{0.5em}
\cfoot{\colorbox[gray]{0.45}{\color{white}\textsf{Friday 17:10 - 17:55}}}
\noindent
\framebox[5in][c]{{\Large\sffamily\textbf{Friday,  17:10 to 17:55}}}
\begin{longtable}[l]{@{}p{1in}@{}p{3in}@{}r}
    {\sffamily\large\textbf{Pre-Symposium}} & 
    {\sffamily\large\textbf{SIGCSE Business Meeting}} & 
    {\sffamily\large\textbf{302A}} \\
\end{longtable}    
\vspace{0.5em}
\noindent\rule{5in}{0.02cm}
\vspace{0.5em}
\cfoot{\colorbox[gray]{0.45}{\color{white}\textsf{Friday 18:00 - 18:45}}}
\noindent
\framebox[5in][c]{{\Large\sffamily\textbf{Friday,  18:00 to 18:45}}}
\begin{longtable}[l]{@{}p{1in}@{}p{3in}@{}r}
    {\sffamily\large\textbf{Ancillary Event}} & 
    {\sffamily\large\textbf{Going Greenfoot}} & 
    {\sffamily\large\textbf{302C}} \\
\end{longtable}    
\begin{longtable}[l]{@{}p{1in}@{}p{3in}@{}r}
    {\sffamily\large\textbf{Business Meeting}} & 
    {\sffamily\large\textbf{CCSC Business Meeting}} & 
    {\sffamily\large\textbf{305B}} \\
\end{longtable}    
\vspace{0.5em}
\noindent\rule{5in}{0.02cm}
\vspace{0.5em}
\cfoot{\colorbox[gray]{0.45}{\color{white}\textsf{Friday 19:00 - 22:00}}}
\noindent
\framebox[5in][c]{{\Large\sffamily\textbf{Friday,  19:00 to 22:00}}}
\begin{longtable}[l]{@{}l@{}l@{}r}
    \parbox[t]{0.25in}{\sffamily\large\textbf{16.}} & 
    \parbox[t]{3.75in}{\raggedright\sffamily\large\textbf{Intellectual Property Law Basics for Computer Science Instructors}} & 
    {\sffamily\large\textbf{205}} \\[1.5em]
% row 3
    \multicolumn{3}{@{}l}{\parbox{5in}{David G. Kay, \textit{UC Irvine} }} \\[1.5em]
% row 4
    \multicolumn{3}{@{}p{5in}}{\small Increasingly the practice of 
computing involves legal issues.  
Patenting algorithms, domain 
name poaching, downloading 
music, and "re-using" HTML and 
graphics from web sites all raise 
questions of intellectual 
property (IP) law (which includes 
patents, copyrights, trade 
secrets, and trademarks).  In the 
classroom, computer science 
educators often confront 
questions that have legal 
ramifications. The presenter, 
who is both a computer 
scientist and a lawyer, will 
introduce the basics of 
intellectual property law to give 
instructors a framework for 
recognizing the issues, 
answering students' questions, 
debunking the most egregious 
misconceptions about IP, and 
understanding generally how 
the law and computing interact.  
All CS educators are welcome; 
no computer is required.}
\end{longtable}
\begin{longtable}[l]{@{}l@{}l@{}r}
    \parbox[t]{0.25in}{\sffamily\large\textbf{17.}} & 
    \parbox[t]{3.75in}{\raggedright\sffamily\large\textbf{Teaching and Learning Computing via Social Gaming with Pex4Fun}} & 
    {\sffamily\large\textbf{301A}} \\[1.5em]
% row 3
    \multicolumn{3}{@{}l}{\parbox{5in}{Nikolai Tillmann, Jonathan de Halleux and Judith Bishop, \textit{Microsoft Research}; Tao Xie, \textit{North Carolina State University} }} \\[1.5em]
% row 4
    \multicolumn{3}{@{}p{5in}}{\small Pex4Fun (pexforfun.com) is a 
web-based serious gaming 
environment for teaching 
computing at many levels, from 
high school all the way through 
graduate courses. Unique to the 
Pex4Fun experience is a cloud-
based program evaluation 
engine based on dynamic 
symbolic execution and SMT-
solving, which provides 
customized feedback to the 
student and automated grading 
for the teacher. Thus, Pex4Fun 
connects teachers, curriculum 
authors, and students in a 
social experience, tracking and 
streaming progress updates in 
real time. This workshop 
involves creating and teaching 
course materials at Pex4Fun. 
Participants should bring a 
laptop computer. The intended 
audience includes all levels of 
CS educators who are interested 
in integrating educational 
technology in their teaching 
environments.}
\end{longtable}
\begin{longtable}[l]{@{}l@{}l@{}r}
    \parbox[t]{0.25in}{\sffamily\large\textbf{18.}} & 
    \parbox[t]{3.75in}{\raggedright\sffamily\large\textbf{Welcome to Makerland: A First Cultural Immersion into Open Source Communities}} & 
    {\sffamily\large\textbf{301B}} \\[1.5em]
% row 3
    \multicolumn{3}{@{}l}{\parbox{5in}{Mel Chua, \textit{Purdue University}; Sebastian Dziallas, \textit{Franklin W. Olin College of Engineering}; Heidi Ellis, \textit{Western New England University}; Greg Hislop, \textit{Drexel University}; Karl Wurst, \textit{Worcester State University} }} \\[1.5em]
% row 4
    \multicolumn{3}{@{}p{5in}}{\small Participating in free and open 
source (FOSS) software 
communities provides students 
with authentic learning while 
supplying instructors with a 
wide variety of educational 
opportunities including coding, 
testing, documentation, 
professionalism and more. 
However, instructors may be 
unfamiliar with how FOSS 
communities work and therefore 
may be reluctant to involve 
students. This workshop is a 
subset of material used in Red 
Hat's Professors' Open Source 
Summer Experience 
(http://communityleadershiptea
m.org/posse), now in its third 
year of successfully providing a 
ramp to FOSS projects for 
instructors. These instructors 
have demonstrated success in 
involving their students in FOSS 
communities where students 
have contributed code, interface 
design, and more. Laptop 
Required.}
\end{longtable}
\begin{longtable}[l]{@{}l@{}l@{}r}
    \parbox[t]{0.25in}{\sffamily\large\textbf{19.}} & 
    \parbox[t]{3.75in}{\raggedright\sffamily\large\textbf{Computational Art and Creative Coding: Teaching CS1 with Processing}} & 
    {\sffamily\large\textbf{302A}} \\[1.5em]
% row 3
    \multicolumn{3}{@{}l}{\parbox{5in}{Dianna Xu and Deepak Kumar, \textit{Bryn Mawr College}; Ira Greenberg, \textit{Southern Methodist University} }} \\[1.5em]
% row 4
    \multicolumn{3}{@{}p{5in}}{\small This workshop showcases a new 
approach to teaching CS1 using 
computational art as a context. 
Participants will be introduced 
to the Processing programming 
language and environment, 
designed for the construction of 
2D and 3D visual forms. Its IDE 
is light-weight, but well-suited 
for the rapid proto-typing 
needed for dynamic visual work. 
We hope to bring the 
excitement, creativity, and 
innovation fostered by 
Processing into the computer 
science education community. 
Instructors of all experience 
levels are welcome. Hands-on 
portion of the workshop will 
enable participants to explore 
Processing and create visual 
effects on the fly. Course 
materials and handouts 
detailing the software and 
teaching 
resources will be given out.
Laptop Required.}
\end{longtable}
\begin{longtable}[l]{@{}l@{}l@{}r}
    \parbox[t]{0.25in}{\sffamily\large\textbf{20.}} & 
    \parbox[t]{3.75in}{\raggedright\sffamily\large\textbf{AP CS Principles and The Beauty and Joy of Computing Curriculum}} & 
    {\sffamily\large\textbf{302B}} \\[1.5em]
% row 3
    \multicolumn{3}{@{}l}{\parbox{5in}{Daniel Garcia, Brian Harvey, Nathaniel Titterton and Luke Segars, \textit{UC Berkeley}; Tiffany Barnes, \textit{University of North Carolina, Charlotte;}; Eugene Lemon, \textit{Ralph J Bunche High School}; Sean Morris, \textit{Albany High School}; Josh Paley, \textit{Henry M. Gunn High School} }} \\[1.5em]
% row 4
    \multicolumn{3}{@{}p{5in}}{\small The Beauty and Joy of 
Computing (BJC) is an 
introductory computer science 
curriculum developed at UC 
Berkeley (and adapted at UNC
Charlotte), intended for high 
school juniors through 
university non-majors. It was 
used in two of the five initial 
pilot programs for the AP CS 
Principles course being 
developed by the College Board 
and the NSF. Our overall goal is 
to support the CS10K project by 
preparing instructors to teach 
the AP CS Principles course 
through the BJC curriculum.
We will share our 
experiences as instructors of the 
course at the university and 
high school level, provide a 
glimpse into a typical week of 
the course, and share details of 
NSF-funded summer 
professional development 
opportunities.
Laptop Required.}
\end{longtable}
\begin{longtable}[l]{@{}l@{}l@{}r}
    \parbox[t]{0.25in}{\sffamily\large\textbf{21.}} & 
    \parbox[t]{3.75in}{\raggedright\sffamily\large\textbf{Peer Instruction in the CS Classroom: A Hands-On Introduction}} & 
    {\sffamily\large\textbf{302C}} \\[1.5em]
% row 3
    \multicolumn{3}{@{}l}{\parbox{5in}{Daniel Zingaro, \textit{University of Toronto}; Cynthia Bailey-Lee and Beth Simon, \textit{University of California, San Diego}; John Glick, \textit{University of San Diego}; Leo Porter, \textit{Skidmore College} }} \\[1.5em]
% row 4
    \multicolumn{3}{@{}p{5in}}{\small We introduce participants to 
Peer Instruction (PI): an active 
learning technique applicable to 
the teaching of many subjects, 
including CS. In PI, Students 
work together to exchange 
perspectives and answer 
challenging conceptual 
questions, and are supported by 
short teaching segments. We 
will introduce and motivate PI, 
demonstrate its use  in 
combination with a clicker 
system, and show that PI is 
much more than the use of 
clickers.

Participants will work in groups 
to develop new PI questions 
addressing challenges to their 
students' learning, and discuss 
numerous pedagogical benefits 
conferred through PI.

Instructors interested in 
increasing engagement in any 
CS course may attend. 
Participants are encouraged to 
bring current lecture materials. 
Laptop optional.}
\end{longtable}
\begin{longtable}[l]{@{}l@{}l@{}r}
    \parbox[t]{0.25in}{\sffamily\large\textbf{22.}} & 
    \parbox[t]{3.75in}{\raggedright\sffamily\large\textbf{Incorporating Software Architecture in the Computer Science Curriculum}} & 
    {\sffamily\large\textbf{307}} \\[1.5em]
% row 3
    \multicolumn{3}{@{}l}{\parbox{5in}{Martin Barrett, \textit{East Tennessee State University}; Steve Chenoweth, \textit{Rose-Hulman Institute of Technology}; Larry Jones, \textit{Software Engineering Institute}; Amine Chigani, \textit{Virginia Tech}; Ayse Bener, \textit{Ryerson University}; Mei-Huei Tang, \textit{Gannon University} }} \\[1.5em]
% row 4
    \multicolumn{3}{@{}p{5in}}{\small This workshop introduces and 
incorporates 
software architecture concepts 
into CS and SE curricula.  
Participants will learn 
techniques used in industry to 
specify quality attributes critical 
to architecture and use those 
attributes to drive the system 
structure using common 
architectural styles.  Exercises 
will demonstrate pedagogical 
uses of the 
techniques in CS and SE classes. 
Sample computer science 
curricula with courses that 
integrate workshop material will 
be presented. Presenters will 
lead a brainstorming session to 
help participants develop 
practical methods for using the 
material in their courses. 
Participants will become part of 
a community of educators 
sharing educational resources in 
software architecture.
Laptop Optional.}
\end{longtable}
\begin{longtable}[l]{@{}l@{}l@{}r}
    \parbox[t]{0.25in}{\sffamily\large\textbf{23.}} & 
    \parbox[t]{3.75in}{\raggedright\sffamily\large\textbf{Parallelism and Concurrency for Data-Structures \& Algorithms Courses}} & 
    {\sffamily\large\textbf{305A}} \\[1.5em]
% row 3
    \multicolumn{3}{@{}l}{\parbox{5in}{Robert Chesebrough, \textit{Intel Corporation}; Johnnie Baker, \textit{Kent State University} }} \\[1.5em]
% row 4
    \multicolumn{3}{@{}p{5in}}{\small This workshop is inspired by 
Dan Grossman’s SIGCSE 2011 
workshop on Data Abstractions.  
We review C/C++ conversions 
of the original Java-based 
materials and will include 
material from the Parallel 
Algorithms course at Kent State.  
The workshop will appeal to 
data-structure and algorithms 
course instructors. Workshop 
topics will include divide and 
conquer approaches, work 
sharing concepts, and a scoped 
locking scheme in OpenMP for 
C++ classes. This material is 
driven via core data-structure 
examples (queues, sorting, 
reductions, etc.) and using a 
Fork/Join Framework found in 
OpenMP and Intel® Cilk Plus and 
Intel® Threading Building 
Blocks.  Participants will write 
parallel programs and test them 
on the Intel® Many-core 
Testing Lab. Laptop Required.}
\end{longtable}
\begin{longtable}[l]{@{}l@{}l@{}r}
    \parbox[t]{0.25in}{\sffamily\large\textbf{24.}} & 
    \parbox[t]{3.75in}{\raggedright\sffamily\large\textbf{ARTSI Robotics Roadshow-in-a-Box: Turnkey Solution for Providing Robotics Workshops to Middle and High School Students}} & 
    {\sffamily\large\textbf{305B}} \\[1.5em]
% row 3
    \multicolumn{3}{@{}l}{\parbox{5in}{Monica Anderson, \textit{The University of Alabama}; Dave Touretzky, \textit{Carnegie Mellon}; Chutima Boonthum-Denecke, \textit{Hampton University} }} \\[1.5em]
% row 4
    \multicolumn{3}{@{}p{5in}}{\small In this half-day tutorial, we will 
introduce the ARTSI “Robotics 
Roadshow-in-a-Box (RRIB)”, a 
single point resource for those 
getting started in 
robotics outreach. The RRIB is a 
kit which contains robots, 
software 
and prepared materials for 
providing robotics workshops 
for middle and high school 
students that focuses on 
showing computer scientists as 
problem solvers and not just 
programmers through activities 
with a larger context.  The RRIB 
fills a need for materials that 
are accessible to those who may 
have limited knowledge of 
robotics or limited experience in 
middle school outreach, 
whether that is undergraduate 
students or faculty researchers 
who might have limited 
outreach experience or 
preparation time. Laptop 
Required.}
\end{longtable}
\begin{longtable}[l]{@{}l@{}l@{}r}
    \parbox[t]{0.25in}{\sffamily\large\textbf{25.}} & 
    \parbox[t]{3.75in}{\raggedright\sffamily\large\textbf{Program by Design: From Animations to Data Structures}} & 
    {\sffamily\large\textbf{306A}} \\[1.5em]
% row 3
    \multicolumn{3}{@{}l}{\parbox{5in}{Kathi Fisler, \textit{WPI}; Stephen Bloch, \textit{Adelphi University} }} \\[1.5em]
% row 4
    \multicolumn{3}{@{}p{5in}}{\small We present the Program by 
Design introductory CS 
curriculum through the lenses 
of graphics, animations, 
algebra, and data structures.  
Animations programming is 
popular for CS1, but many such 
curricula lack clean paths into 
CS2.  Program by Design is 
different.  Using and reinforcing 
concepts from algebra, students 
learn to write animations 
(including standard topics such 
as model/view separation and 
event-handling), then move 
seamlessly into working with 
structured data, lists, trees, and 
objects.  The curriculum 
emphasizes design, testing, and 
writing maintainable programs, 
without losing the engagement 
of animations.  The workshop 
uses lectures and hands-on 
exercises to provide high-
school and college teachers an 
overview of the approach.  See 
www.programbydesign.org.   
LAPTOP OPTIONAL}
\end{longtable}
\begin{longtable}[l]{@{}l@{}l@{}r}
    \parbox[t]{0.25in}{\sffamily\large\textbf{26.}} & 
    \parbox[t]{3.75in}{\raggedright\sffamily\large\textbf{CS Outreach with App Inventor}} & 
    {\sffamily\large\textbf{306B}} \\[1.5em]
% row 3
    \multicolumn{3}{@{}l}{\parbox{5in}{Michelle Friend, \textit{Stanford University}; Jeff Gray, \textit{University of Alabama} }} \\[1.5em]
% row 4
    \multicolumn{3}{@{}p{5in}}{\small Mobile phone programming can 
provide teens an authentic and 
engaging hook into computer 
science. With App Inventor, 
developed by Google and moved 
to MIT, programming Android 
apps is as easy as clicking 
blocks together. App Inventor 
has been used successfully in 
after school programs, 
roadshows, summer camps, 
teacher workshops, and 
computer science classrooms 
from middle school through 
college. Participants will get an 
overview of App Inventor 
including project ideas and 
sample student code, hear 
outreach planning suggestions, 
write programs, develop 
outreach plans, and see how the 
Java Bridge helps transition from 
App Inventor to Java. Even the 
most time-stretched professor 
or teacher can encourage 
students in computer science 
with App Inventor. Laptop 
Required}
\end{longtable}
\begin{longtable}[l]{@{}l@{}l@{}r}
    \parbox[t]{0.25in}{\sffamily\large\textbf{27.}} & 
    \parbox[t]{3.75in}{\raggedright\sffamily\large\textbf{Making Mathematical Reasoning Fun: Tool-Assisted, Collaborative Techniques}} & 
    {\sffamily\large\textbf{306C}} \\[1.5em]
% row 3
    \multicolumn{3}{@{}l}{\parbox{5in}{Jason Hallstrom and Murali Sitaraman, \textit{School of Computing, Clemson University}; Joe Hollingsworth, \textit{Computer Science, Indiana University Southeast}; Joan Krone, \textit{Mathematics and Computer Science, Denison University} }} \\[1.5em]
% row 4
    \multicolumn{3}{@{}p{5in}}{\small Is it possible to excite students 
about learning the mathematical 
principles that underly high-
quality software? Can we teach 
them to apply these principles 
using modern software tools? 
Can this be accomplished 
without displacing existing 
content? Yes, with the right 
pedagogical principles, teaching 
tools, and classroom exercises. 
This hands-on laboratory will 
introduce a set of principles, 
tools, and exercises that have 
proven to work. By adopting one 
content module at a time, 
educators will better prepare 
students to reason rigorously 
about the software they develop 
and maintain. Fees for this 
workshop will be covered for a 
limited number of attendees 
through an NSF award; limited 
travel support is also available.
Laptop Required.}
\end{longtable}
\vspace{0.5em}
\noindent\rule{5in}{0.02cm}
\vspace{0.5em}
\addcontentsline{toc}{subsection}{Saturday}
\cfoot{\colorbox[gray]{0.45}{\color{white}\textsf{Saturday 08:30 - 09:45}}}
\noindent
\framebox[5in][c]{{\Large\sffamily\textbf{Saturday,  8:30 to 9:45}}}
\begin{longtable}[l]{@{}l@{}l@{}r}
    \parbox[t]{1in}{\sffamily\large\textbf{PANEL}} & 
    \parbox[t]{3in}{\sffamily\raggedright\large\textbf{Nifty Assignments}} & 
    \parbox[t]{1in}{\sffamily\raggedleft\large\textbf{301AB}} \\
% row 2    
    Chair: & 
    Nick Parlante, \textit{Stanford University}  \\[0.5em]
% row 3
    Participants: & 
    \multicolumn{2}{@{}l}{\parbox{3.75in}{Julie Zelenski, \textit{Stanford University}; Daniel Zingaro, \textit{University of Toronto}; Kevin Wayne, \textit{Princeton University}; Joshua Guerin, \textit{University of Kentucky}; Stephen Davies, \textit{University of Mary Washington}; Dave O'Hallaron, \textit{Carnegie Mellon University} }} \\[2em]
% row 4
    \multicolumn{3}{@{}p{5in}}{\small I can worry about the strategy of my syllabus, and I can fret over my lectures. Nonetheless, I am always struck that what my students really learn and enjoy in the course depends very much on the assignments. Great assignments are hard to dream up and time-consuming to develop. With that in mind, the Nifty Assignments session is all about promoting and sharing the ideas and concrete materials of successful assignments.}
\end{longtable}
\begin{longtable}[l]{@{}l@{}l@{}r}
    \parbox[t]{1in}{\sffamily\large\textbf{PANEL}} & 
    \parbox[t]{3in}{\sffamily\raggedright\large\textbf{Update on the CS Principles Project}} & 
    \parbox[t]{1in}{\sffamily\raggedleft\large\textbf{305B}} \\
% row 2    
    Chair: & 
    Amy Briggs, \textit{Middlebury College}  \\[0.5em]
% row 3
    Participants: & 
    \multicolumn{2}{@{}l}{\parbox{3.75in}{Owen Astrachan, \textit{Duke University}; Jan Cuny, \textit{National Science Foundation}; Lien Diaz, \textit{College Board}; Chris Stephenson, \textit{Computer Science Teachers Association} }} \\[2em]
% row 4
    \multicolumn{3}{@{}p{5in}}{\small The CS Principles Project is a collaborative effort to develop a new introductory course in computer science, accessible to all students. Computer Science educators at all levels have worked together on the development of the new curriculum under the direction of the College Board with support from the National Science Foundation. This special session provides an opportunity for the CS Principles project leaders to report on recent updates and new directions, and to engage in discussion on all aspects of the project with SIGCSE participants.}
\end{longtable}
\begin{longtable}[l]{@{}l@{}l@{}r}
    \parbox[t]{1in}{\sffamily\large\textbf{PANEL}} & 
    \parbox[t]{3in}{\sffamily\raggedright\large\textbf{Implementing Evidence-Based Practices makes a Difference in Female Undergraduate Enrollments}} & 
    \parbox[t]{1in}{\sffamily\raggedleft\large\textbf{306C}} \\
% row 2    
    Chair: & 
    Wendy DuBow \textit{University of Colorado}  \\[0.5em]
% row 3
    Participants: & 
    \multicolumn{2}{@{}l}{\parbox{3.75in}{Wendy DuBow, \textit{University of Colorado}; Elizabeth Litzler, \textit{University of Washington}; Maureen Biggers, \textit{Indiana University}; Mike Erlinger, \textit{Harvey Mudd College} }} \\[2em]
% row 4
    \multicolumn{3}{@{}p{5in}}{\small While many departments are aware of promising and best practices for recruiting and retaining female students in undergraduate computing majors, there seems to be a drive to try novel approaches instead of evidence-based approaches. Developing a diverse student body requires active recruitment, inclusive pedagogy, meaningful curriculum, evaluation of progress, as well as student, faculty and institutional support. Given the intrinsic challenges of enacting change, departments could make it easier on themselves - and likely achieve better results - if they intentionally and systematically used practices that have been shown to be effective. This panel will present the rationale for implementing evidence-based practices and share the successes some departments have achieved by doing so.}
\end{longtable}
\vspace{0.5em}
\noindent\rule{5in}{0.02cm}
\vspace{0.5em}
\cfoot{\colorbox[gray]{0.45}{\color{white}\textsf{Saturday 08:30 - 10:10}}}
\noindent
\framebox[5in][c]{{\Large\sffamily\textbf{Saturday,  8:30 to 10:10}}}
\newpage
\begin{longtable}{@{}p{0.75in}@{}p{3.25in}@{}r}
   {\sffamily\large\textbf{PAPERS}} &
   {\raggedright\sffamily\large\textbf{High School Collaborations}} & 
   {\sffamily\large\textbf{302A }} \\
%row 2
   Chair:  & 
   {\raggedright Tim Bell \textit{University of Canterbury}} & \\ \\
{\sffamily \large 8:30}& 
\multicolumn{2}{@{}p{3.75in}}{\sffamily\raggedright\textbf{Life Two Years After a Game Programming Course: Longitudinal Viewpoints on K-12 Outreach}} \\
& \multicolumn{2}{@{}p{3.75in}}{\raggedright Antti-Jussi Lakanen, Ville Isomöttönen and Vesa Lappalainen, \textit{Department of Mathematical Information Technology, University of Jyvaskyla}} \\ \\
\multicolumn{3}{@{}p{5in}}{\small In our faculty we have run week-long K-12 game programming courses now for three summers. In this paper we investigate what programming-related activities students do after they take a course, and what factors in the students' background relate to post-course programming. We also investigate a possible change in the students' interest towards higher education science studies. We find that most students continue programming after the course and that their interest towards science studies keeps increasing. In student background we observed some indicative trends, but did not find reliable explaining factors related to post-course programming or increased interest towards science studies.} \\ \\
{\sffamily \large 8:55}& 
\multicolumn{2}{@{}p{3.75in}}{\sffamily\raggedright\textbf{Reflections on Outreach Programs in CS Classes: Learning Objectives for “Unplugged” Activities}} \\
& \multicolumn{2}{@{}p{3.75in}}{\raggedright Renate Thies, \textit{Cusanus-Gymnasium Erkelenz and Technische Universität Dortmund}; Jan Vahrenhold, \textit{Technische Universität Dortmund}} \\ \\
\multicolumn{3}{@{}p{5in}}{\small To provide a unified view of any scientific field, outreach programs need to realistically portray the subject in question. Consequently, topics and methods actually taught in Computer Science courses should to be touched upon in Computer Science outreach programs or, conversely, elements from successful Computer Science outreach programs should be used to enrich established courses in Computer Science.

We follow up on the latter aspect and extract and classify learning objectives from the activities of the well-received Computer Science Unplugged program. Based upon this classification, we comment on where and to which extent these activities can be used to enrich teaching Computer Science in secondary education.} \\ \\
{\sffamily \large 9:20}& 
\multicolumn{2}{@{}p{3.75in}}{\sffamily\raggedright\textbf{Weaving a Tapestry Satellite Workshop to Support HS CS Teachers in Attracting and Engaging Students}} \\
& \multicolumn{2}{@{}p{3.75in}}{\raggedright Ambareen Siraj, Martha Kosa and Summer Olmstead, \textit{Tennessee Tech University}} \\ \\
\multicolumn{3}{@{}p{5in}}{\small In this paper, we describe the Tennessee Tech University (TTU) Tapestry Workshop for high school (HS) teachers. The Tapestry Workshop initiative, a collaborative partnership between TTU, the University of Virginia (UVA) and HS teachers to share strategies, practices, and innovative ideas for teaching Computer Science (CS) effectively. This three-day professional development workshop utilized informational, technical, networking, activity-based, and discussion-oriented sessions geared towards attracting and engaging CS students. The workshop was a worthwhile professional development activity for both the organizers and attendees and contributed towards initiation of HS CS program locally.} \\ \\
{\sffamily \large 9:45}& 
\multicolumn{2}{@{}p{3.75in}}{\sffamily\raggedright\textbf{Who AM I? Understanding High School Computer Science Teachers’ Professional Identity}} \\
& \multicolumn{2}{@{}p{3.75in}}{\raggedright Lijun Ni and Mark Guzdial, \textit{Georgia Institute of Technology}} \\ \\
\multicolumn{3}{@{}p{5in}}{\small We need committed, quality CS teachers for quality secondary computing education. Teacher education literature suggests that teachers’ sense of commitment and (other aspects of) teaching profession is tightly linked with their teacher identity. However, the current educational system does not provide typical contexts for teachers to build a sense of identity as CS teachers. This study is intended to gain an initial understanding of CS teachers’ perceptions about their professional identity and potential factors that might contribute to these perceptions. Our findings indicate that current HS teachers teaching CS courses do not necessarily identify themselves as CS teachers. They have different perceptions related to CS teaching. Four kinds of factors can contribute to these perceptions.} \\ \\
\end{longtable}


\newpage
\begin{longtable}{@{}p{0.75in}@{}p{3.25in}@{}r}
   {\sffamily\large\textbf{PAPERS}} &
   {\raggedright\sffamily\large\textbf{Parallelism and Concurrency}} & 
   {\sffamily\large\textbf{302B }} \\
%row 2
   Chair:  & 
   {\raggedright Jodi Tims \textit{Baldwin-Wallace College}} & \\ \\
{\sffamily \large 8:30}& 
\multicolumn{2}{@{}p{3.75in}}{\sffamily\raggedright\textbf{Introducing Parallelism and Concurrency in the Data Structures Course}} \\
& \multicolumn{2}{@{}p{3.75in}}{\raggedright Ruth E. Anderson and Dan Grossman, \textit{University of Washington - Seattle}} \\ \\
\multicolumn{3}{@{}p{5in}}{\small We report on our experience integrating a 3-week introduction to multithreading in a required data structures course for 2nd-year computer science majors. We emphasize a distinction between parallelism and concurrency that teaches students to use extra processors effectively and enforce mutual exclusion correctly. The material fits naturally in the data structures course by having the same mix of algorithms, programming, and asymptotic analysis as the rest of the course. Our department has used this unit for 1.5 years and we report feedback from students, multiple instructors for the course, and students in a later course that uses threads. We developed a full set of course materials that have been adapted for use by instructors in various courses at five other institutions so far.} \\ \\
{\sffamily \large 8:55}& 
\multicolumn{2}{@{}p{3.75in}}{\sffamily\raggedright\textbf{Exploring Concurrency Using The Parallel Analysis Tool}} \\
& \multicolumn{2}{@{}p{3.75in}}{\raggedright Brian Rague, \textit{Weber State University}} \\ \\
\multicolumn{3}{@{}p{5in}}{\small One area of investigation that has become increasingly important across all levels of CS instruction is parallel computing.  This paper describes the initial version of the Parallel Analysis Tool (PAT), a pedagogical tool designed to assist undergraduate students in visualizing concurrency and effectively connecting parallel processing to coding strategies. The PAT is a complete Java development environment, with an emphasis on (1) helping students to identify appropriate code locations where parallelization can be applied and (2) allowing students to subsequently examine the practical performance tradeoffs of these parallelization decisions in a laboratory setting. The Parallel Quotient supports the analysis of the relative benefits of employing various parallel programming strategies.} \\ \\
{\sffamily \large 9:20}& 
\multicolumn{2}{@{}p{3.75in}}{\sffamily\raggedright\textbf{Virtual Clusters for Parallel and Distributed Education}} \\
& \multicolumn{2}{@{}p{3.75in}}{\raggedright Elizabeth Shoop, Eric Biggers, Malcom Kane, Devry Lin and Maura Warner, \textit{Macalester College}; Richard Brown, \textit{St. Olaf College}} \\ \\
\multicolumn{3}{@{}p{5in}}{\small The reality of multicore machines as a standard and the prevalence of distributed cloud computing has signaled a need for parallel and distributed computing to become integrated into the computer science curriculum.  At the same time, operating system virtualization has become a common technique with open standard tools available to any practitioners.  Virtual machines (VMs) installed on available computer lab resources can be used to simulate high-performance cluster computing environments.  This paper describes two such virtual clusters in use at small colleges, reports on their effectiveness, and provides information about how to obtain the VMs for use in an educational lab setting.} \\ \\
{\sffamily \large 9:45}& 
\multicolumn{2}{@{}p{3.75in}}{\sffamily\raggedright\textbf{Cross Teaching Parallelism and Ray Tracing: A Project-based Approach to Teaching Applied Parallel Computing}} \\
& \multicolumn{2}{@{}p{3.75in}}{\raggedright Chris Lupo and Zoe Wood, \textit{Cal Poly State University}; Christine Victorino, \textit{University of California, Santa Barbara}} \\ \\
\multicolumn{3}{@{}p{5in}}{\small This paper describes the integration of two undergraduate computer science courses to enhance student learning in parallel computing concepts. In this cross teaching experience we structured the integration of the two courses such that students studying parallel computing worked with students studying advanced rendering for approximately 30\% of the quarter long courses. Working in teams, both groups of students saw the application of parallelization to an existing software project early in the curriculum of both courses. Motivating projects and performance gains are discussed, as well as student survey data on the effectiveness of the learning outcomes. Performance and survey data indicate a positive gain from the cross teaching experience.} \\ \\
\end{longtable}


\newpage
\begin{longtable}{@{}p{0.75in}@{}p{3.25in}@{}r}
   {\sffamily\large\textbf{PAPERS}} &
   {\raggedright\sffamily\large\textbf{Mobile Computing}} & 
   {\sffamily\large\textbf{306A }} \\
%row 2
   Chair:  & 
   {\raggedright Cyndi Rader \textit{Colorado School of Mines}} & \\ \\
{\sffamily \large 8:30}& 
\multicolumn{2}{@{}p{3.75in}}{\sffamily\raggedright\textbf{Cabana: A Cross-platform Mobile Development System}} \\
& \multicolumn{2}{@{}p{3.75in}}{\raggedright Paul E. Dickson, \textit{Hampshire College}} \\ \\
\multicolumn{3}{@{}p{5in}}{\small Mobile application development is a hot topic in computer science education, and debate rages over which platform to develop on and what software to use for development. Cabana is a web-based application designed to enable development on multiple mobile platforms and to make application development easier. It uses an approach to application programming based on a wiring diagram that is supplemented with the ability to program directly using JavaScript. It is an ideal choice for application development in introductory computer science courses and for upper-level courses where the focus is on application design and not application programming. This paper introduces Cabana and describes its use in two different computer science courses.} \\ \\
{\sffamily \large 8:55}& 
\multicolumn{2}{@{}p{3.75in}}{\sffamily\raggedright\textbf{Mobile Apps for the Greater Good: A Socially Relevant Approach to Software Engineering}} \\
& \multicolumn{2}{@{}p{3.75in}}{\raggedright Victor Pauca, \textit{Wake Forest University}; Richard Guy, \textit{University of Toronto}} \\ \\
\multicolumn{3}{@{}p{5in}}{\small Socially relevant computing has recently been proposed as a way to reinvigorate interest in computer science. By appealing to students' interest in helping others, it aims to give students life-changing experiential learning not typically achieved in the classroom, while providing software that benefits society at large. 
For the last two years, we have been using mobile device programming, agile methods, and real-world, socially relevant projects for teaching software engineering in a liberal arts Computer Science curricula. We report on teaching methods, student experiences, and products  delivered by this approach. In particular, one of these products, Verbal Victor, is now a commercial and social entrepreneurship success in assistive technology for communication disabilities.} \\ \\
{\sffamily \large 9:20}& 
\multicolumn{2}{@{}p{3.75in}}{\sffamily\raggedright\textbf{Using Mobile Phone Programming to Teach Java and Advanced Programming to Computer Scientists}} \\
& \multicolumn{2}{@{}p{3.75in}}{\raggedright Derek Riley, \textit{University of Wisconsin- Parkside}} \\ \\
\multicolumn{3}{@{}p{5in}}{\small In this work the approach employing the Android mobile phone platform in an upper division computer science course to teach Java programming and other advanced computer science topics is presented.  Mobile phones are growing influences in the computing market, but their strengths and popularity are rarely exploited in computer science classrooms.  
The aim of the course is to harness this enthusiasm
to improve fluency in the Java language to afford an opportunity to learn how to work on large, complex projects and to enhance the students’ preparedness for the job market.  The ideas presented in this work could be adapted for improving learning in many courses across the computing curriculum.} \\ \\
{\sffamily \large 9:45}& 
\multicolumn{2}{@{}p{3.75in}}{\sffamily\raggedright\textbf{RoboLIFT: Engaging CS2 Students with Testable, Automatically Evaluated Android Applications}} \\
& \multicolumn{2}{@{}p{3.75in}}{\raggedright Anthony Allevato and Stephen H. Edwards, \textit{Virginia Tech}} \\ \\
\multicolumn{3}{@{}p{5in}}{\small Making computer science assignments interesting and relevant is a constant challenge for instructors of introductory courses. Android has become popular in these courses to take advantage of the increasing popularity of smartphones and mobile “apps.” This has been shown to increase student engagement but it is only the first step, and we must continue to provide support for teaching methodologies that we have used in the past, such as test-driven development and automated assessment. We have developed RoboLIFT, a library that makes unit testing of Android applications approachable for students. Furthermore, by supporting existing automated grading techniques, we are able to sustain large student enrollments, and we evaluate the effects that using Android has had on student performance.} \\ \\
\end{longtable}


\newpage
\begin{longtable}{@{}p{0.75in}@{}p{3.25in}@{}r}
   {\sffamily\large\textbf{PAPERS}} &
   {\raggedright\sffamily\large\textbf{Visualization}} & 
   {\sffamily\large\textbf{306B }} \\
%row 2
   Chair:  & 
   {\raggedright Demian Lessa \textit{State University of New York at Buffalo}} & \\ \\
{\sffamily \large 8:30}& 
\multicolumn{2}{@{}p{3.75in}}{\sffamily\raggedright\textbf{Highway Data and Map Visualizations for Educational Use}} \\
& \multicolumn{2}{@{}p{3.75in}}{\raggedright James Teresco, \textit{Siena College}} \\ \\
\multicolumn{3}{@{}p{5in}}{\small It is often a challenge to find interesting and appropriate data sets to use as examples to demonstrate graph data structures and algorithms.  The data should include examples small enough to work through manually, but some large enough to demonstrate important behaviors of a structure or algorithm.  It should be freely available in a convenient format and have some real-world relevance.  Visualization of the data and results computed from it is helpful.
This paper describes a collection of graph data sets generated from the Clinched Highway Mapping Project's highway data and some examples of their use in courses.  The source data, the process used to convert the data into a more useful format, some examples of its use, and a visualization tool using the Google Maps API, are described.} \\ \\
{\sffamily \large 8:55}& 
\multicolumn{2}{@{}p{3.75in}}{\sffamily\raggedright\textbf{Experiments with Algorithm Visualization Tool Development}} \\
& \multicolumn{2}{@{}p{3.75in}}{\raggedright Michael Orsega, \textit{University of West Georgia}; Bradley Vander Zanden and Christopher Skinner, \textit{University of Tennessee}} \\ \\
\multicolumn{3}{@{}p{5in}}{\small This paper presents the initial stages of a teaching tool named iSketchmate, intended for instructor use during lecture. iSketchmate allows users to create and manipulate splay trees through an animated GUI. It improves upon existing tools by providing (1) the ability to begin with any user-defined tree, (2) a history mechanism so tree operations can be repeated or changed, and (3) finer-grained animation within each operation so instructors may give further descriptions at intermediate steps within any given operation. Experiments showed iSketchmate users could produce significantly more diagrams and these diagrams were significantly more accurate than those made with pencil and paper.} \\ \\
{\sffamily \large 9:20}& 
\multicolumn{2}{@{}p{3.75in}}{\sffamily\raggedright\textbf{CSTutor:  A Pen-Based Tool for Visualizing Data Structures}} \\
& \multicolumn{2}{@{}p{3.75in}}{\raggedright Sarah Buchanan, Brandon Ochs and Joseph LaViola, \textit{University of Central Florida}} \\ \\
\multicolumn{3}{@{}p{5in}}{\small We present CSTutor, a pen-based application for data structure visualization which allows the user to manipulate data structures through the recognition of handwritten symbols and gestures as well as edit the corresponding code. The UI consists of a sketching area where the user can draw a data structure in a way that is as natural as pen and paper.  Running in parallel with the visualization is a code view window where the user can make changes to the source code and add functions which manipulate the data structure on the canvas in real time.  We  also  present  the results  of a perceived  usefulness survey.  The results of the study indicate that the majority of students would find CSTutor helpful for learning data structures.} \\ \\
{\sffamily \large 9:45}& 
\multicolumn{2}{@{}p{3.75in}}{\sffamily\raggedright\textbf{ECVisual: A Visualization Tool for Elliptic Curve Based Ciphers}} \\
& \multicolumn{2}{@{}p{3.75in}}{\raggedright Jean Mayo, Jun Tao, Jun Ma, Melissa Keranen and Ching-Kuang Shene, \textit{Michigan Technological University}} \\ \\
\multicolumn{3}{@{}p{5in}}{\small This paper describes a visualization tool ECvisual that helps students understand and instructors teach elliptic curve based ciphers. This tool permits a user to visualize elliptic curves over the real field and over a finite field of prime order, perform arithmetic operations, do encryption and decryption, and convert plaintext to a point. The demo mode of ECvisual can be used for classroom presentation and self-study. With the practice mode, a user may go through steps in finite field computations, encryption, decryption and plaintext conversion. She may compute, and then check, the answer to each operation herself. The opportunity for self-study provides an instructor greater flexibility  in selecting a lecture pace for this detail-filled material. Classroom evaluation was positive.} \\ \\
\end{longtable}


\begin{longtable}[l]{@{}p{1in}@{}p{3in}@{}r}
    {\sffamily\large\textbf{Presentations}} & 
    {\sffamily\large\textbf{Student Research Competition - Graduate}} & 
    {\sffamily\large\textbf{302C}} \\
\end{longtable}    
\begin{longtable}[l]{@{}p{1in}@{}p{3in}@{}r}
    {\sffamily\large\textbf{Presentations}} & 
    {\sffamily\large\textbf{Student Research Competition - Undergraduate}} & 
    {\sffamily\large\textbf{305A}} \\
\end{longtable}    
\vspace{0.5em}
\noindent\rule{5in}{0.02cm}
\vspace{0.5em}
\cfoot{\colorbox[gray]{0.45}{\color{white}\textsf{Saturday 09:00 - 12:00}}}
\noindent
\framebox[5in][c]{{\Large\sffamily\textbf{Saturday,  9:00 to 12:00}}}
\begin{longtable}[l]{@{}p{1in}@{}p{3in}@{}r}
    {\sffamily\large\textbf{Social}} & 
    {\sffamily\large\textbf{K-12 Teachers Room}} & 
    {\sffamily\large\textbf{202}} \\
\end{longtable}    
\begin{longtable}[l]{@{}p{1in}@{}p{3in}@{}r}
    {\sffamily\large\textbf{Social}} & 
    {\sffamily\large\textbf{CS Education Research Room}} & 
    {\sffamily\large\textbf{203}} \\
\end{longtable}    
\vspace{0.5em}
\noindent\rule{5in}{0.02cm}
\vspace{0.5em}
\cfoot{\colorbox[gray]{0.45}{\color{white}\textsf{Saturday 09:30 - 12:00}}}
\noindent
\framebox[5in][c]{{\Large\sffamily\textbf{Saturday,  9:30 to 12:00}}}
\begin{longtable}[l]{@{}p{1in}@{}p{3in}@{}r}
    {\sffamily\large\textbf{Exhibits}} & 
    {\sffamily\large\textbf{Exhibits}} & 
    {\sffamily\large\textbf{Exhibit Hall A}} \\
\end{longtable}    
\vspace{0.5em}
\noindent\rule{5in}{0.02cm}
\vspace{0.5em}
\cfoot{\colorbox[gray]{0.45}{\color{white}\textsf{Saturday 10:00 - 11:30}}}
\noindent
\framebox[5in][c]{{\Large\sffamily\textbf{Saturday,  10:00 to 11:30}}}
\begin{longtable}[l]{@{}p{1in}@{}p{3in}@{}r}
    {\sffamily\large\textbf{Project Showcase}} & 
    {\sffamily\large\textbf{NSF Showcase \#5}} & 
    {\sffamily\large\textbf{Exhibit Hall A}} \\
\end{longtable}    
\vspace{0.5em}
\noindent\rule{5in}{0.02cm}
\vspace{0.5em}
\cfoot{\colorbox[gray]{0.45}{\color{white}\textsf{Saturday 10:10 - 10:55}}}
\noindent
\framebox[5in][c]{{\Large\sffamily\textbf{Saturday,  10:10 to 10:55}}}
\begin{longtable}[l]{@{}p{1in}@{}p{3in}@{}r}
    {\sffamily\large\textbf{None}} & 
    {\sffamily\large\textbf{Break and Exhibits}} & 
    {\sffamily\large\textbf{Exhibit Hall A}} \\
\end{longtable}    
\vspace{0.5em}
\noindent\rule{5in}{0.02cm}
\vspace{0.5em}
\cfoot{\colorbox[gray]{0.45}{\color{white}\textsf{Saturday 10:55 - 12:10}}}
\noindent
\framebox[5in][c]{{\Large\sffamily\textbf{Saturday,  10:55 to 12:10}}}
\begin{longtable}[l]{@{}l@{}l@{}r}
    \parbox[t]{1in}{\sffamily\large\textbf{PANEL}} & 
    \parbox[t]{3in}{\sffamily\raggedright\large\textbf{Rediscovering the Passion, Beauty, Joy, and Awe:  Making Computing Fun Again}} & 
    \parbox[t]{1in}{\sffamily\raggedleft\large\textbf{301AB}} \\
% row 2    
    Chair: & 
    Daniel Garcia \textit{UC Berkeley}  \\[0.5em]
% row 3
    Participants: & 
    \multicolumn{2}{@{}l}{\parbox{3.75in}{Barbara Ericson, \textit{Georgia Institute of Technology}; Joanna Goode, \textit{University of Oregon}; Colleen Lewis, \textit{UC Berkeley} }} \\[2em]
% row 4
    \multicolumn{3}{@{}p{5in}}{\small In his SIGCSE 2007 keynote, Grady Booch exhorted us to share the “passion, beauty, joy and awe” (PBJA) of computing. This led to a series of SIGCSE sessions that provided a forum for sharing:
• What we’ve done: Highlighting successful PBJA initiatives the presenters have undertaken or seen and wish to celebrate
• What we should do (curriculum): Pointing out where our curriculum is lacking in PBJA, and how to fix it
• How we should do it (pedagogy): Sharing how a change in attitude / focus / etc. can make strides to improving PBJA
This year we’ve invited 3 educators who have worked tirelessly to broaden participation of computing to underrepresented groups.  The hope with this panel is to be able to explore best practices in outreach, in terms of extolling the PBJA of computing.}
\end{longtable}
\begin{longtable}[l]{@{}l@{}l@{}r}
    \parbox[t]{1in}{\sffamily\large\textbf{PANEL}} & 
    \parbox[t]{3in}{\sffamily\raggedright\large\textbf{Promoting Student-Centered Learning with POGIL}} & 
    \parbox[t]{1in}{\sffamily\raggedleft\large\textbf{305B}} \\
% row 2    
    Chair: & 
    Helen Hu, \textit{Westminster College}  \\[0.5em]
% row 3
    Participants: & 
    \multicolumn{2}{@{}l}{\parbox{3.75in}{Clifton Kussmaul, \textit{Muhlenberg College} }} \\[2em]
% row 4
    \multicolumn{3}{@{}p{5in}}{\small POGIL (Process Oriented Guided Inquiry Learning) is a type of learning based on the principle that students learn more when they construct their own understanding. Rather than teaching by telling, POGIL instructors provide activities that guide students to discover concepts on their own. Students work in groups, encouraging them to discuss their findings with their peers. Not only do students learn the material better, but the process of discovery teaches them to be better problem solvers. This special session will provide SIGCSE attendees the opportunity to experience a POGIL activity. The presenters will share guided inquiry activities. We will discuss ways that POGIL may be used to transform computer science classes at all levels, from small schools to large universities.}
\end{longtable}
\begin{longtable}[l]{@{}l@{}l@{}r}
    \parbox[t]{1in}{\sffamily\large\textbf{PANEL}} & 
    \parbox[t]{3in}{\sffamily\raggedright\large\textbf{Teaching Secure Coding - Report from Summit on Education in Secure Software}} & 
    \parbox[t]{1in}{\sffamily\raggedleft\large\textbf{306C}} \\
% row 2    
    Chair: & 
    Blair Taylor, \textit{Towson University}  \\[0.5em]
% row 3
    Participants: & 
    \multicolumn{2}{@{}l}{\parbox{3.75in}{Matt Bishop, \textit{University of California Davis}; Diana Burley, \textit{George Washington University}; Steve Cooper, \textit{Stanford University}; Ron Dodge, \textit{United States Military Academy} }} \\[2em]
% row 4
    \multicolumn{3}{@{}p{5in}}{\small To help education design curricula that integrates principles and practices of secure programming, the National Science Foundation Directorates of Computer and Information Science and Engineering (CISE) and Education and Human Resources (EHR) jointly sponsored the Summit on Education in Secure Software (SESS), held in Washington, DC in October, 2010. The goal of the summit was to develop roadmaps showing how best to educate students and current professionals on robust, secure programming concepts and practices, and to identify both the resources required and the problems that had to be overcome. The goal of this session is to share some of the key findings and challenges identified by the summit and to actively engage the community in the discussions.}
\end{longtable}
\newpage
\begin{longtable}{@{}p{0.75in}@{}p{3.25in}@{}r}
   {\sffamily\large\textbf{PAPERS}} &
   {\raggedright\sffamily\large\textbf{Attracting Majors}} & 
   {\sffamily\large\textbf{302A }} \\
%row 2
   Chair:  & 
   {\raggedright Nanette Veilleux \textit{Simmons College}} & \\ \\
{\sffamily \large 10:55}& 
\multicolumn{2}{@{}p{3.75in}}{\sffamily\raggedright\textbf{Starting with ubicomp:   using the SenseBoard to introduce computing}} \\
& \multicolumn{2}{@{}p{3.75in}}{\raggedright Marian Petre, Mike Richards and Arosha Bandara, \textit{The Open University}} \\ \\
\multicolumn{3}{@{}p{5in}}{\small In this paper, we describe a new undergraduate module for novice students conducted entirely through distance learning: My Digital Life (TU100). The course has been designed to lower the barriers to creating programs that interact with the world; TU100’s materials have been designed to excite, encourage, reassure and support learners who explore the novel topic of ubiquitous computing through playful experimentation. It introduces the fundamentals of computing by giving students the capability for programming a device, the SenseBoard, which has built-in input/output and sensors.  Programming is done in Sense, an extension of Scratch, which scaffolds programming and reduces the syntax burden.} \\ \\
{\sffamily \large 11:20}& 
\multicolumn{2}{@{}p{3.75in}}{\sffamily\raggedright\textbf{Improving First-Year Success and Retention through Interest-Based CS0 Courses}} \\
& \multicolumn{2}{@{}p{3.75in}}{\raggedright Michael Haungs, Christopher Clark, John Clements and David Janzen, \textit{California Polytechnic State Unviersity, San Luis Obispo}} \\ \\
\multicolumn{3}{@{}p{5in}}{\small Many computer science programs suffer from low student retention rates. At Cal Poly, academic performance and retention rates among first-year computer science students are among the lowest on campus.
     In order to remedy this, we have developed a new CS0 course featuring different "tracks'' that students can choose from. This allows students to learn the basics of programming, teamwork, and college-level study in a domain that is of personal interest. In addition, the course relies on classic Project-based Learning (PBL) approaches as well as a focus on both academic and non-academic factors shown to increase student retention.
     Initial assessment demonstrates positive results in the form of
increased academic performance in post CS0 courses and student retention.} \\ \\
{\sffamily \large 11:45}& 
\multicolumn{2}{@{}p{3.75in}}{\sffamily\raggedright\textbf{Reshaping The Image Of Computer Science In Only Fifteen Minutes (Of Class) A Week}} \\
& \multicolumn{2}{@{}p{3.75in}}{\raggedright Sara Sprenkle, \textit{Washington and Lee University}; Shannon Duvall, \textit{Elon University}} \\ \\
\multicolumn{3}{@{}p{5in}}{\small Low undergraduate enrollments in computer science will not meet the future demand of employers.  Some reasons for the low enrollments are computer science's nerdy image, lack of understanding of the field, and low motivation for learning programming.  We propose to change the image of computer science by exposing students to applications of computing and its impact on their lives through reading and discussing recent news articles in 15 minutes of class.  We call this component of our courses the Broader Issues in computer science.} \\ \\
\end{longtable}


\newpage
\begin{longtable}{@{}p{0.75in}@{}p{3.25in}@{}r}
   {\sffamily\large\textbf{PAPERS}} &
   {\raggedright\sffamily\large\textbf{OS and Distributed Computing}} & 
   {\sffamily\large\textbf{302B }} \\
%row 2
   Chair:  & 
   {\raggedright William Mongan \textit{Drexel University}} & \\ \\
{\sffamily \large 10:55}& 
\multicolumn{2}{@{}p{3.75in}}{\sffamily\raggedright\textbf{Experiences Teaching MapReduce in the Cloud}} \\
& \multicolumn{2}{@{}p{3.75in}}{\raggedright Ariel Rabkin, Charles Reiss, Randy Katz and David Patterson, \textit{UC Berkeley}} \\ \\
\multicolumn{3}{@{}p{5in}}{\small We describe our experiences teaching MapReduce in a large undergraduate lecture course using public cloud services. Using the cloud, every student could carry out scalability benchmarking assignments on realistic hardware, which would have been impossible otherwise. Over two semesters, over 500 students took our course. We believe this is the first large-scale demonstration that it is feasible to use pay-as-you-go billing in the Cloud for a large undergraduate course. Modest instructor effort was sufficient to prevent students from overspending. Average per-pupil expenses in the Cloud were under \$45, less than half our available grant funding. Students were excited by the assignment: 90\% said they thought it should be retained in future course offerings.} \\ \\
{\sffamily \large 11:20}& 
\multicolumn{2}{@{}p{3.75in}}{\sffamily\raggedright\textbf{Developing Microlabs Using Google Web Toolkit}} \\
& \multicolumn{2}{@{}p{3.75in}}{\raggedright Barry Kurtz, James Fenwick and Philip Meznar, \textit{Appalachian State University}} \\ \\
\multicolumn{3}{@{}p{5in}}{\small Closed labs provide hands-on experience in a supervised setting.  Microlabs extend this approach into lecture with very short hands-on activities occurring in lecture.  Programming microlabs were developed for a distributed computing course.  This paper describes our logical microlabs where students solve conceptual problems that do not involve programming.  These two microlab approaches are integrated into the Microlab Learning Cycle.  Microlab activities should be usable with a wide variety of computing devices, including tablets. After experimenting with different development environments we have adopted the Google Web Toolkit. After presenting the current status of our activities, we discuss future directions for microlab development.} \\ \\
{\sffamily \large 11:45}& 
\multicolumn{2}{@{}p{3.75in}}{\sffamily\raggedright\textbf{Teaching Operating Systems Using Android}} \\
& \multicolumn{2}{@{}p{3.75in}}{\raggedright Jeremy Andrus and Jason Nieh, \textit{Columbia University}} \\ \\
\multicolumn{3}{@{}p{5in}}{\small The computing landscape is shifting towards mobile and embedded devices. To learn about operating systems, it is increasingly important for students to gain hands-on kernel programming experience in these environments, which are quite different from desktops and servers. We present our work to teach operating systems by leveraging Android, an open, commercially supported software platform increasingly used on mobile and embedded devices. We introduce a series of 5 Android kernel programming projects, and an Android virtual lab which gives students hands-on Android experience with minimal computing infrastructure. We used these projects and virtual lab to teach an introductory operating systems course. Over 80\% of students surveyed enjoyed applying operating systems concepts to Android.} \\ \\
\end{longtable}


\newpage
\begin{longtable}{@{}p{0.75in}@{}p{3.25in}@{}r}
   {\sffamily\large\textbf{PAPERS}} &
   {\raggedright\sffamily\large\textbf{Curricular Innovations and Research}} & 
   {\sffamily\large\textbf{306A }} \\
%row 2
   Chair:  & 
   {\raggedright Michael Hewner \textit{Duke University}} & \\ \\
{\sffamily \large 10:55}& 
\multicolumn{2}{@{}p{3.75in}}{\sffamily\raggedright\textbf{Open Educational Resources in Computer Science Teaching}} \\
& \multicolumn{2}{@{}p{3.75in}}{\raggedright Christo Dichev and Darina Dicheva, \textit{Winston Salem State University}} \\ \\
\multicolumn{3}{@{}p{5in}}{\small Open content and open access to resources are important factors in the innovation of Computer Science education. This paper presents a study aimed at gaining an understanding of the needs of Computer Science educators in terms of Open Educational Resources (OER): what kind of resources they need, when they need them, how they use them, and what are the barriers and the enablers for using OER The results of the study are compared and analyzed in the context of the popular OER sites. The work contributes to the research on OER utilization and discovery.} \\ \\
{\sffamily \large 11:20}& 
\multicolumn{2}{@{}p{3.75in}}{\sffamily\raggedright\textbf{Emergent Themes in a UI Design Hybrid-Studio Course}} \\
& \multicolumn{2}{@{}p{3.75in}}{\raggedright Yolanda Reimer, \textit{University of Montana}; Katherine Cennamo, \textit{Virginia Tech}; Sarah Douglas, \textit{University of Oregon}} \\ \\
\multicolumn{3}{@{}p{5in}}{\small The goal of our research and teaching collaboration has been to learn more about how key aspects of pedagogy commonly incorporated in architecture and industrial design classes might positively impact the teaching of user interface (UI) design within a standard computer science curriculum. Toward that end, we studied a number of studio design courses, developed a set of curriculum guidelines, and analyzed the effectiveness of these guidelines as implemented in a UI design course. We discovered three emergent themes: 1) students need early and constant reminders that design is an iterative process involving user feedback and testing; 2) instructor modeling is critical; and 3) technology needs to be carefully managed at critical junctures throughout the class.} \\ \\
{\sffamily \large 11:45}& 
\multicolumn{2}{@{}p{3.75in}}{\sffamily\raggedright\textbf{A Multilevel, Multidimensional Undergraduate Course and Lab Experience on Embedded Multimedia Systems}} \\
& \multicolumn{2}{@{}p{3.75in}}{\raggedright Dimitrios Charalampidis, \textit{University of New Orleans}; James Haralambides, \textit{Barry University}} \\ \\
\multicolumn{3}{@{}p{5in}}{\small Traditional curricular structures can be fragmented in that course inter\&\#8208;relationships or links between theories, methodologies, and practices, are not immediately recognized by the students. The completion of the course puzzle and the integration of course knowledge usually become evident only after graduation. This paper describes a course/lab implementation that offered students a unique opportunity to experience the full spectrum of course elements, namely, fundamentals of theory, algorithmic/hardware design and simulation, and implementation and testing on FPGAs all within a single framework. The course/lab design is a collaborative effort between the U. of New Orleans and Barry U. to ensure that the course/lab can be implemented successfully in diverse environments.} \\ \\
\end{longtable}


\newpage
\begin{longtable}{@{}p{0.75in}@{}p{3.25in}@{}r}
   {\sffamily\large\textbf{PAPERS}} &
   {\raggedright\sffamily\large\textbf{CS Education Research}} & 
   {\sffamily\large\textbf{306B }} \\
%row 2
   Chair:  & 
   {\raggedright Yana Kortsarts \textit{Widener University, Chester}} & \\ \\
{\sffamily \large 10:55}& 
\multicolumn{2}{@{}p{3.75in}}{\sffamily\raggedright\textbf{Effective Closed Labs in Early CS Courses: Lessons from Seven Terms of Action Research}} \\
& \multicolumn{2}{@{}p{3.75in}}{\raggedright Elizabeth Patitsas, \textit{University of Toronto}; Steve Wolfman, \textit{University of British Columbia}} \\ \\
\multicolumn{3}{@{}p{5in}}{\small We report on best practices we have established to teach first-year computer science students in closed laboratories, founded on over three years of action research in a large introductory discrete mathematics and digital logic course. Our practices have resulted in statistically significant improvements in student and teaching assistant perception of the labs. Specifically, we discuss our practices of streamlining labs to reduce load on students that is extraneous to the lab's learning goals; establishing a positive first impression for students and TAs in the early weeks of the term; and effectively managing the teaching staff, including weekly preparation meetings for TAs using and a gradual, iterative curriculum development cycle that engages all stakeholders in the course.} \\ \\
{\sffamily \large 11:20}& 
\multicolumn{2}{@{}p{3.75in}}{\sffamily\raggedright\textbf{What Do Students Learn About Programming  From Game, Music Video, And Storytelling Projects?}} \\
& \multicolumn{2}{@{}p{3.75in}}{\raggedright Joel Adams and Andrew Webster, \textit{Calvin College}} \\ \\
\multicolumn{3}{@{}p{5in}}{\small Drag-and-drop learning environments like Alice and Scratch eliminate syntax errors, making them attractive as ways to introduce programming concepts to students.  Having had students create games, music videos, and storytelling projects, we began to wonder: What programming constructs do students actually use and hence learn well enough to be able to apply when creating different kinds of projects?  We conducted a quantitative analysis of a collection of over 300 different student projects, and found significant differences in how frequently the students creating those projects used variables, if statements, loops, and dialog constructs.} \\ \\
{\sffamily \large 11:45}& 
\multicolumn{2}{@{}p{3.75in}}{\sffamily\raggedright\textbf{Bayesian Network Analysis of Computer Science Grade Distributions}} \\
& \multicolumn{2}{@{}p{3.75in}}{\raggedright Adam Anthony and Mitchell Raney, \textit{Baldwin-Wallace College}} \\ \\
\multicolumn{3}{@{}p{5in}}{\small Time to completion is a major factor in determining the total cost of a college degree.  In an effort to reduce the number of students taking more than four years to complete a degree, we propose the use of Bayesian networks to predict student grades, given past performance prerequisite courses.  This is an intuitive approach because the necessary structure of any Bayesian network must be a directed acyclic graph, which is also the case for prerequisite graphs.  We demonstrate that building a Bayesian network directly from the prerequisite graph results in effective predictions, and demonstrate a few applications of the resulting network in areas of identifying struggling students and deciding upon which courses a department should allocate tutoring resources.} \\ \\
\end{longtable}


\begin{longtable}[l]{@{}p{1in}@{}p{3in}@{}r}
    {\sffamily\large\textbf{SupporterSession}} & 
    {\sffamily\large\textbf{TBA}} & 
    {\sffamily\large\textbf{302C}} \\
\end{longtable}    
\begin{longtable}[l]{@{}p{1in}@{}p{3in}@{}r}
    {\sffamily\large\textbf{SupporterSession}} & 
    {\sffamily\large\textbf{TBA}} & 
    {\sffamily\large\textbf{305A}} \\
\end{longtable}    
\vspace{0.5em}
\noindent\rule{5in}{0.02cm}
\vspace{0.5em}
\cfoot{\colorbox[gray]{0.45}{\color{white}\textsf{Saturday 12:30 - 14:30}}}
\noindent
\framebox[5in][c]{{\Large\sffamily\textbf{Saturday,  12:30 to 14:30}}}
\begin{longtable}[l]{@{}p{1in}@{}p{3in}@{}r}
    {\sffamily\large\textbf{Plenary Session}} & 
    {\sffamily\large\textbf{SIGCSE LuncheonKeynote speaker: Fernanda Viégas and Martin Wattenberg}} & 
    {\sffamily\large\textbf{Ballroom AB}} \\
\end{longtable}    
\vspace{0.5em}
\noindent\rule{5in}{0.02cm}
\vspace{0.5em}
\cfoot{\colorbox[gray]{0.45}{\color{white}\textsf{Saturday 15:00 - 18:00}}}
\noindent
\framebox[5in][c]{{\Large\sffamily\textbf{Saturday,  15:00 to 18:00}}}
\begin{longtable}[l]{@{}l@{}l@{}r}
    \parbox[t]{0.25in}{\sffamily\large\textbf{28.}} & 
    \parbox[t]{3.75in}{\raggedright\sffamily\large\textbf{Snap! (Build Your Own Blocks)}} & 
    {\sffamily\large\textbf{301A}} \\[1.5em]
% row 3
    \multicolumn{3}{@{}l}{\parbox{5in}{Brian Harvey, Daniel Garcia and Luke Segars, \textit{University of California, Berkeley}; Josh Paley, \textit{Henry M. Gunn High School} }} \\[1.5em]
% row 4
    \multicolumn{3}{@{}p{5in}}{\small This workshop is for high school 
and college teachers of general-
interest ("CS 0") computer 
science courses.  It presents the 
programming environment used 
in two of the five initial AP CS 
Principles pilot courses.

Snap! (Build Your Own Blocks) is 
a free, graphical, drag-and-
drop extension to the Scratch 
programming language.  
Scratch, designed for 8-14 year 
olds, models programs as 
"scripts" without names, 
arguments, or return values. 
Snap! supports older learners 
(14-20) by adding named 
procedures (thus recursion), 
procedures as data (thus higher 
order functions) structured lists, 
and sprites as first class objects 
with inheritance.

Participants will learn Snap! 
through discussion, 
programming exercises, and 
exploration.  See 
http://snap.berkeley.edu for 
details.  Laptop required.}
\end{longtable}
\begin{longtable}[l]{@{}l@{}l@{}r}
    \parbox[t]{0.25in}{\sffamily\large\textbf{29.}} & 
    \parbox[t]{3.75in}{\raggedright\sffamily\large\textbf{Circuits and Microcontrollers in Computer Organization Laboratories}} & 
    {\sffamily\large\textbf{301B}} \\[1.5em]
% row 3
    \multicolumn{3}{@{}l}{\parbox{5in}{Marge Coahran, \textit{Dickinson College}; Janet Davis, \textit{Grinnell College} }} \\[1.5em]
% row 4
    \multicolumn{3}{@{}p{5in}}{\small This workshop will introduce a 
set of hands-on laboratory 
activities appropriate for a first 
Computer Organization course. 
Participants will work with real 
equipment: first implementing 
elementary digital circuits with 
TTL logic chips, and then 
programming AVR 
microcontrollers in assembly to 
drive fun accessories such as 
LEDs and speakers. Participants 
will not take equipment home 
afterwards, but will receive parts 
lists and vendor information. 
The workshop is intended for 
educators with little electronics 
background who are interested 
in incorporating electronics 
laboratories into their courses. 
Laptops (Linux, Mac, or 
Windows) will provide the 
programming environment for 
the AVRs. Free software will be 
available before the workshop. 
Participants will work in pairs. 
Laptop recommended.}
\end{longtable}
\begin{longtable}[l]{@{}l@{}l@{}r}
    \parbox[t]{0.25in}{\sffamily\large\textbf{30.}} & 
    \parbox[t]{3.75in}{\raggedright\sffamily\large\textbf{Web Development with Python and Django}} & 
    {\sffamily\large\textbf{302A}} \\[1.5em]
% row 3
    \multicolumn{3}{@{}l}{\parbox{5in}{Ariel Ortiz, \textit{Tecnologico de Monterrey, Campus Estado de Mexico} }} \\[1.5em]
% row 4
    \multicolumn{3}{@{}p{5in}}{\small Many instructors have already 
discovered the joy of teaching 
programming using Python. 
Now it's time to take Python to 
the next level. This workshop 
will introduce Django, an open 
source Python web framework 
that saves you time and makes 
web development fun. It's aimed 
at CS instructors who want to 
teach how to build elegant web 
applications with minimal fuss. 
Django is Python's equivalent to 
the popular Ruby on Rails 
framework. Topics that will be 
covered include: setup and 
configuration, template 
language, and database 
integration through object-
relational mapping. Participants 
should have some familiarity 
with Python, HTML and SQL. 
More information at: 
http://webcem01.cem.itesm.mx
:8005/django/ Laptop Required}
\end{longtable}
\begin{longtable}[l]{@{}l@{}l@{}r}
    \parbox[t]{0.25in}{\sffamily\large\textbf{31.}} & 
    \parbox[t]{3.75in}{\raggedright\sffamily\large\textbf{Improving the Accessibility of Computing Enrichment Programs}} & 
    {\sffamily\large\textbf{302B}} \\[1.5em]
% row 3
    \multicolumn{3}{@{}l}{\parbox{5in}{Richard Ladner, \textit{University of Washington}; Karen Alkoby, \textit{Gallaudet University}; Jeff Bigham, \textit{University of Rochester}; Stephanie Ludi, \textit{Rochester Institute of Technology}; Daniela Marghitu, \textit{Auburn University}; Andreas Stefik, \textit{University of Southern Illinois, Edwardsville} }} \\[1.5em]
% row 4
    \multicolumn{3}{@{}p{5in}}{\small Many wonderful enrichment 
programs have been created to 
introduce young people to 
computing, but with little 
attention to making them 
accessible to students with 
disabilities.  In this workshop 
participants will learn from 
practitioners who have 
introduced computing and 
programming to young people 
with disabilities.  They will also 
learn first-hand from students 
with disabilities about their 
needs in learning programming. 
There will be breakout sessions 
for participants to apply what 
they have learned to improve 
existing enrichment programs 
such as Alice, Arduino, Scratch, 
Kodu, App Inventor, Greenfoot, 
Lego Mindstorms, Processing, 
and Computer Science 
Unplugged. Laptop 
Recommended.}
\end{longtable}
\begin{longtable}[l]{@{}l@{}l@{}r}
    \parbox[t]{0.25in}{\sffamily\large\textbf{32.}} & 
    \parbox[t]{3.75in}{\raggedright\sffamily\large\textbf{Enhancing Student Interest by Extending Graphics Applications}} & 
    {\sffamily\large\textbf{302C}} \\[1.5em]
% row 3
    \multicolumn{3}{@{}l}{\parbox{5in}{Samuel Rebelsky, \textit{Grinnell College} }} \\[1.5em]
% row 4
    \multicolumn{3}{@{}p{5in}}{\small Computer science teachers 
strive for new examples and 
problems to interest millenials. 
The Media Computing approach 
has proven successful in 
attracting students in contexts 
from community colleges to R1 
universities – students are 
clearly excited by writing 
programs that make images.

In this workshop, we show how 
to go a step further and have 
write scripts and plug-ins in 
Python for open-source 
graphics programs such as GIMP 
and Inkscape.  Students not only 
make images, they write filters 
and features that they can share 
with others, even non-
programmers.  E.g., students 
have written filters that 
“fractalize” vector graphics or 
that turn images into stained 
glass.  

Further information can be 
found at
http://www.cs.grinnell.edu/~reb
elsky/Workshops/SIGCSE2012/

Laptop required.}
\end{longtable}
\begin{longtable}[l]{@{}l@{}l@{}r}
    \parbox[t]{0.25in}{\sffamily\large\textbf{33.}} & 
    \parbox[t]{3.75in}{\raggedright\sffamily\large\textbf{Engage Your Students by Teaching Programming Using Only Mobile Devices with TouchDevelop}} & 
    {\sffamily\large\textbf{305A}} \\[1.5em]
% row 3
    \multicolumn{3}{@{}l}{\parbox{5in}{Nikolai Tillmann, Michal Moskal, Jonathan de Halleux and Manuel Fahndrich, \textit{Microsoft Research}; Tao Xie, \textit{North Carolina State University} }} \\[1.5em]
% row 4
    \multicolumn{3}{@{}p{5in}}{\small The world experiences a 
technology shift: Powerful and 
easy-to-use mobile devices like 
smartphones and tablets are 
becoming more prevalent than 
traditional PCs and laptops. We 
propose to reflect this new 
reality by adapting how 
programming is taught. 
Students should develop 
software directly on 
smartphones. In this workshop, 
we introduce TouchDevelop on 
Windows Phone 7, a novel 
application creation 
environment from Microsoft 
Research. Its typed, structured 
programming language is built 
around the idea of only using a 
touchscreen as the input device 
to author code. Easy access to 
the rich sensor and personal 
data available on a mobile 
device results in an engaging 
programming experience for 
students who learn 
programming by creating fun 
games and applications. Laptop 
Optional.}
\end{longtable}
\begin{longtable}[l]{@{}l@{}l@{}r}
    \parbox[t]{0.25in}{\sffamily\large\textbf{34.}} & 
    \parbox[t]{3.75in}{\raggedright\sffamily\large\textbf{CS in Parallel: Modules for Adding Parallel Computing to CS Courses, from CS2 to Theory of Computation}} & 
    {\sffamily\large\textbf{306A}} \\[1.5em]
% row 3
    \multicolumn{3}{@{}l}{\parbox{5in}{Richard Brown, \textit{St. Olaf College}; Elizabeth Shoop, \textit{Macalester College} }} \\[1.5em]
% row 4
    \multicolumn{3}{@{}p{5in}}{\small Parallel computing with more 
and more cores is here to stay.  
This workshop presents four 
independent, class-tested, 
primarily hands-on modules for 
incrementally adding parallelism 
in undergraduate CS courses, 
each requiring 1 to 3 class days 
and versatile for diverse courses 
and curricula: parallelizing loops 
and sharing memory on Intel's 
Manycore Testing lab (for a 
second CS course or for 
computer organization);  
parallel web crawler in Java or 
C++ (second CS course);  
parallel sorting (algorithms);  \&\#960;-
calculus theory for 
communicating sequential 
processes (theory of 
computation).  Workshop 
materials provided, drawn from 
CSinParallel.org.  Intended 
audience: CS instructors.  
Laptop recommended (Windows, 
Mac, Linux).}
\end{longtable}
\begin{longtable}[l]{@{}l@{}l@{}r}
    \parbox[t]{0.25in}{\sffamily\large\textbf{35.}} & 
    \parbox[t]{3.75in}{\raggedright\sffamily\large\textbf{Listening to Linked Lists: Using Multimedia to Learn Data Structures}} & 
    {\sffamily\large\textbf{306B}} \\[1.5em]
% row 3
    \multicolumn{3}{@{}l}{\parbox{5in}{Mark Guzdial and Barbara Ericson, \textit{Georgia Institute of Technology} }} \\[1.5em]
% row 4
    \multicolumn{3}{@{}p{5in}}{\small Everybody teaches linked lists, 
with homework like 
implementing duplicate, weave, 
and reverse. When those nodes 
contain strings or numbers, 
these are pretty boring 
assignments. When these nodes 
contain music (MIDI), these 
operations are composing 
music, which can then be 
played. This workshop shows 
how to use music, images, and 
sounds to teach the basic data 
structures, including linked 
lists, circular linked lists, stacks, 
queues, and trees. These pieces 
can then be tied together 
through the use of simulations 
to generate animated movies.  
We will be using Java, though 
many of the methods can also 
be used in Python. Laptop 
Recommended.}
\end{longtable}
\begin{longtable}[l]{@{}l@{}l@{}r}
    \parbox[t]{0.25in}{\sffamily\large\textbf{36.}} & 
    \parbox[t]{3.75in}{\raggedright\sffamily\large\textbf{Puzzle-Based Learning: Introducing Creative Thinking and Problem Solving for Computer Science and Engineering}} & 
    {\sffamily\large\textbf{306C}} \\[1.5em]
% row 3
    \multicolumn{3}{@{}l}{\parbox{5in}{Raja Sooriamurthi, \textit{Carnegie Mellon University}; Nick Falkner and Zbigniew Michalewicz, \textit{University of Adelaide} }} \\[1.5em]
% row 4
    \multicolumn{3}{@{}p{5in}}{\small Puzzle-based learning (PBL) is 
an emerging model of teaching 
critical thinking and problem 
solving.  Today’s market place 
needs skilled graduates capable 
of solving real problems of 
innovation in a changing 
environment. While solving 
puzzles is innately fun, 
companies such as Google and 
Yahoo also use puzzles to 
assess the creative problem 
solving skills of potential 
employees.  In this interactive 
workshop we will examine a 
range of puzzles, games, and 
general problem solving 
strategies. Participants will 
emerge with the needed 
pedagogical foundation to offer 
a full course on PBL or to 
include it as part of another 
course.   Currently 40+ 
institutions around the world 
are offering courses based on 
PBL. More details are available at 
www.PuzzleBasedLearning.edu.a
u.  Laptop optional.}
\end{longtable}
\vspace{0.5em}
\noindent\rule{5in}{0.02cm}
\vspace{0.5em}
